\documentclass[11pt]{article}
\usepackage[utf8]{inputenc}
\usepackage{polski}
\usepackage{graphicx}
\usepackage{array}
\usepackage{paralist}
\usepackage{verbatim}
\usepackage{subfig}
\usepackage{amsmath}
\usepackage{float}
\usepackage{amsthm}
\usepackage{amssymb}
\usepackage{amsfonts}
\theoremstyle{definition}
\newtheorem{zadanie}{Zadanie}
\title{1001 zadań z programowania}
\author{Igor Nowicki}

\newcommand{\fromA}{\small Ze zbioru \cite{python100}.}

\begin{document}

\maketitle
\tableofcontents
\section{Wstęp}

Zgromadzone zadania z wielu źródeł, m.in.:

\begin{itemize}
\item 100+ Python challenging programming exercises
\item Konkurs informatyczny Logia i Minilogia
\item Olimpiada Informatyczna Gimnazjalistów oraz Olimpiada Informatyczna
\end{itemize}

\section{Zadania}

\begin{zadanie}
Write a program which will find all such numbers which are divisible by 7 but are not a multiple of 5,
between 2000 and 3200 (both included).
The numbers obtained should be printed in a comma-separated sequence on a single line.

\fromA
\end{zadanie}

\begin{zadanie}
Write a program which can compute the factorial of a given numbers.
The results should be printed in a comma-separated sequence on a single line.
Suppose the following input is supplied to the program:
\begin{verbatim}
8
\end{verbatim}
Then, the output should be:
\begin{verbatim}
40320
\end{verbatim}

\fromA
\end{zadanie}

\begin{zadanie}
With a given integral number n, write a program to generate a dictionary that contains (i, i*i) such that is an integral number between 1 and n (both included). and then the program should print the dictionary.
Suppose the following input is supplied to the program:
\begin{verbatim}
8
\end{verbatim}
Then, the output should be:
\begin{verbatim}
{1: 1, 2: 4, 3: 9, 4: 16, 5: 25, 6: 36, 7: 49, 8: 64}
\end{verbatim}

\fromA
\end{zadanie}

\begin{zadanie}
Write a program which accepts a sequence of comma-separated numbers from console and generate a list and a tuple which contains every number.
Suppose the following input is supplied to the program:
\begin{verbatim}
34,67,55,33,12,98
\end{verbatim}
Then, the output should be:
\begin{verbatim}
['34', '67', '55', '33', '12', '98']
('34', '67', '55', '33', '12', '98')
\end{verbatim}

\fromA
\end{zadanie}

\begin{zadanie}
Define a class which has at least two methods:
\begin{itemize}
\item\texttt{getString}- to get a string from console input,
\item\texttt{printString}- to print the string in upper case.
\end{itemize}
Also please include simple test function to test the class methods.

\fromA
\end{zadanie}

\begin{zadanie}
Write a program that calculates and prints the value according to the given formula:
\begin{verbatim}
Q = Square root of [(2 * C * D)/H]
\end{verbatim}
Following are the fixed values of C and H:
\begin{itemize}
\item C is 50. 
\item H is 30.
\item D is the variable whose values should be input to your program in a comma-separated sequence.
\end{itemize}
Example
Let us assume the following comma separated input sequence is given to the program:
\begin{verbatim}
100,150,180
\end{verbatim}
The output of the program should be:
\begin{verbatim}
18,22,24
\end{verbatim}

\fromA
\end{zadanie}

\begin{zadanie}
Write a program which takes 2 digits, X,Y as input and generates a 2-dimensional array. The element value in the i-th row and j-th column of the array should be i*j.
Note: i=0,1.., X-1; j=0,1,¡­Y-1.

Example
Suppose the following inputs are given to the program:
\begin{verbatim}
3,5
\end{verbatim}
Then, the output of the program should be:
\begin{verbatim}
[[0, 0, 0, 0, 0], [0, 1, 2, 3, 4], [0, 2, 4, 6, 8]] 
\end{verbatim}

\fromA
\end{zadanie}

\begin{zadanie}
Write a program that accepts a comma separated sequence of words as input and prints the words in a comma-separated sequence after sorting them alphabetically.
Suppose the following input is supplied to the program:
\begin{verbatim}
without,hello,bag,world
\end{verbatim}
Then, the output should be:
\begin{verbatim}
bag,hello,without,world
\end{verbatim}

\fromA
\end{zadanie}

\begin{zadanie}
Write a program that accepts sequence of lines as input and prints the lines after making all characters in the sentence capitalized.
Suppose the following input is supplied to the program:
Hello world
Practice makes perfect
Then, the output should be:
HELLO WORLD
PRACTICE MAKES PERFECT

\fromA
\end{zadanie}

\begin{zadanie}
Write a program that accepts a sequence of whitespace separated words as input and prints the words after removing all duplicate words and sorting them alphanumerically.
Suppose the following input is supplied to the program:
hello world and practice makes perfect and hello world again
Then, the output should be:
again and hello makes perfect practice world

\fromA
\end{zadanie}
\begin{zadanie}
Write a program which accepts a sequence of comma separated 4 digit binary numbers as its input and then check whether they are divisible by 5 or not. The numbers that are divisible by 5 are to be printed in a comma separated sequence.

\fromA
\end{zadanie}
\begin{zadanie}
Write a program, which will find all such numbers between 1000 and 3000 (both included) such that each digit of the number is an even number.
The numbers obtained should be printed in a comma-separated sequence on a single line.

\fromA
\end{zadanie}
\begin{zadanie}
Write a program that accepts a sentence and calculate the number of letters and digits.
Suppose the following input is supplied to the program:
hello world! 123
Then, the output should be:
LETTERS 10
DIGITS 3

\fromA
\end{zadanie}
\begin{zadanie}
Write a program that accepts a sentence and calculate the number of upper case letters and lower case letters.
Suppose the following input is supplied to the program:
Hello world!
Then, the output should be:
UPPER CASE 1
LOWER CASE 9

\fromA
\end{zadanie}
\begin{zadanie}
Write a program that computes the value of a+aa+aaa+aaaa with a given digit as the value of a.
Suppose the following input is supplied to the program:
9
Then, the output should be:
11106

\fromA
\end{zadanie}
\begin{zadanie}
Use a list comprehension to square each odd number in a list. The list is input by a sequence of comma-separated numbers.
Suppose the following input is supplied to the program:
1,2,3,4,5,6,7,8,9
Then, the output should be:
1,3,5,7,9

\fromA
\end{zadanie}
\begin{zadanie}
Write a program that computes the net amount of a bank account based a transaction log from console input. The transaction log format is shown as following:
D 100
W 200
D means deposit while W means withdrawal.
Suppose the following input is supplied to the program:
D 300
D 300
W 200
D 100
Then, the output should be:
500

\fromA
\end{zadanie}
\begin{zadanie}
A website requires the users to input username and password to register. Write a program to check the validity of password input by users.
Following are the criteria for checking the password:
1. At least 1 letter between [a-z]
2. At least 1 number between [0-9]
1. At least 1 letter between [A-Z]
3. At least 1 character from [\#@]
4. Minimum length of transaction password: 6
5. Maximum length of transaction password: 12
Your program should accept a sequence of comma separated passwords and will check them according to the above criteria. Passwords that match the criteria are to be printed, each separated by a comma.
Example
If the following passwords are given as input to the program:
ABd1234@1,a F\#,2w3E*,2We3345
Then, the output of the program should be:
ABd1234@1

\fromA
\end{zadanie}
\begin{zadanie}
You are required to write a program to sort the (name, age, height) tuples by ascending order where name is string, age and height are numbers. The tuples are input by console. The sort criteria is:
1: Sort based on name;
2: Then sort based on age;
3: Then sort by score.
The priority is that name > age > score.
If the following tuples are given as input to the program:
Tom,19,80
John,20,90
Jony,17,91
Jony,17,93
Json,21,85
Then, the output of the program should be:
\begin{verbatim}
[('John', '20', '90'), ('Jony', '17', '91'), ('Jony', '17', '93'), ('Json', '21', '85'), ('Tom', '19', '80')]
\end{verbatim}

\fromA
\end{zadanie}
\begin{zadanie}
Define a class with a generator which can iterate the numbers, which are divisible by 7, between a given range 0 and n.

\fromA
\end{zadanie}
\begin{zadanie}
A robot moves in a plane starting from the original point (0,0). The robot can move toward UP, DOWN, LEFT and RIGHT with a given steps. The trace of robot movement is shown as the following:
\begin{verbatim}
UP 5
DOWN 3
LEFT 3
RIGHT 2
\end{verbatim}
The numbers after the direction are steps. Please write a program to compute the distance from current position after a sequence of movement and original point. If the distance is a float, then just print the nearest integer.
Example:
If the following tuples are given as input to the program:
\begin{verbatim}
UP 5
DOWN 3
LEFT 3
RIGHT 2
\end{verbatim}
Then, the output of the program should be:
2

Hints:
In case of input data being supplied to the question, it should be assumed to be a console input.


\fromA
\end{zadanie}
\begin{zadanie}
Question 22
Level 3

Question:
Write a program to compute the frequency of the words from the input. The output should output after sorting the key alphanumerically. 
Suppose the following input is supplied to the program:
New to Python or choosing between Python 2 and Python 3? Read Python 2 or Python 3.
Then, the output should be:
\begin{verbatim}
2:2
3.:1
3?:1
New:1
Python:5
Read:1
and:1
between:1
choosing:1
or:2
to:1
\end{verbatim}

Hints

In case of input data being supplied to the question, it should be assumed to be a console input.

\fromA
\end{zadanie}
\begin{zadanie}
Write a method which can calculate square value of number

\fromA
\end{zadanie}
\begin{zadanie}
Python has many built-in functions, and if you do not know how to use it, you can read document online or find some books. But Python has a built-in document function for every built-in functions.
    Please write a program to print some Python built-in functions documents, such as abs(), int(), raw\
    input()
    And add document for your own function

\fromA
\end{zadanie}
\begin{zadanie}
Define a class, which have a class parameter and have a same instance parameter.

\fromA
\end{zadanie}
\begin{zadanie}
Define a function which can compute the sum of two numbers.

\fromA
\end{zadanie}
\begin{zadanie}
Define a function that can convert a integer into a string and print it in console.

\fromA
\end{zadanie}
\begin{zadanie}
Define a function that can convert a integer into a string and print it in console.

\fromA
\end{zadanie}
\begin{zadanie}
Define a function that can receive two integral numbers in string form and compute their sum and then print it in console.

\fromA
\end{zadanie}
\begin{zadanie}
Define a function that can accept two strings as input and concatenate them and then print it in console.

\fromA
\end{zadanie}
\begin{zadanie}
Define a function that can accept two strings as input and print the string with maximum length in console. If two strings have the same length, then the function should print al l strings line by line.

\fromA
\end{zadanie}
\begin{zadanie}
Define a function that can accept an integer number as input and print the "It is an even number" if the number is even, otherwise print "It is an odd number".

\fromA
\end{zadanie}
\begin{zadanie}
Define a function which can print a dictionary where the keys are numbers between 1 and 3 (both included) and the values are square of keys.

\fromA
\end{zadanie}
\begin{zadanie}
Define a function which can print a dictionary where the keys are numbers between 1 and 20 (both included) and the values are square of keys.

\fromA
\end{zadanie}
\begin{zadanie}
Define a function which can generate a dictionary where the keys are numbers between 1 and 20 (both included) and the values are square of keys. The function should just print the values only.

\fromA
\end{zadanie}
\begin{zadanie}
Define a function which can generate a dictionary where the keys are numbers between 1 and 20 (both included) and the values are square of keys. The function should just print the keys only.

\fromA
\end{zadanie}
\begin{zadanie}
Define a function which can generate and print a list where the values are square of numbers between 1 and 20 (both included).

\fromA
\end{zadanie}
\begin{zadanie}
Define a function which can generate a list where the values are square of numbers between 1 and 20 (both included). Then the function needs to print the first 5 elements in the list.

\fromA
\end{zadanie}
\begin{zadanie}
Define a function which can generate a list where the values are square of numbers between 1 and 20 (both included). Then the function needs to print the last 5 elements in the list.

\fromA
\end{zadanie}
\begin{zadanie}
Define a function which can generate a list where the values are square of numbers between 1 and 20 (both included). Then the function needs to print all values except the first 5 elements in the list.

\fromA
\end{zadanie}
\begin{zadanie}
Define a function which can generate and print a tuple where the value are square of numbers between 1 and 20 (both included).

\fromA
\end{zadanie}
\begin{zadanie}
With a given tuple (1,2,3,4,5,6,7,8,9,10), write a program to print the first half values in one line and the last half values in one line.

\fromA
\end{zadanie}
\begin{zadanie}
Write a program to generate and print another tuple whose values are even numbers in the given tuple (1,2,3,4,5,6,7,8,9,10).

\fromA
\end{zadanie}
\begin{zadanie}
Write a program which accepts a string as input to print "Yes" if the string is "yes" or "YES" or "Yes", otherwise print "No".

\fromA
\end{zadanie}
\begin{zadanie}
Write a program which can filter even numbers in a list by using filter function. The list is: [1,2,3,4,5,6,7,8,9,10].

\fromA
\end{zadanie}
\begin{zadanie}
Write a program which can map() to make a list whose elements are square of elements in [1,2,3,4,5,6,7,8,9,10].

\fromA
\end{zadanie}
\begin{zadanie}
Write a program which can map() and filter() to make a list whose elements are square of even number in [1,2,3,4,5,6,7,8,9,10].

\fromA
\end{zadanie}
\begin{zadanie}
Write a program which can filter() to make a list whose elements are even number between 1 and 20 (both included).

\fromA
\end{zadanie}
\begin{zadanie}
Write a program which can map() to make a list whose elements are square of numbers between 1 and 20 (both included).

\fromA
\end{zadanie}
\begin{zadanie}
Define a class named American which has a static method called printNationality.

\fromA
\end{zadanie}
\begin{zadanie}
Define a class named American and its subclass NewYorker.

\fromA
\end{zadanie}
\begin{zadanie}
Define a class named Circle which can be constructed by a radius. The Circle class has a method which can compute the area.

\fromA
\end{zadanie}
\begin{zadanie}
Define a class named Rectangle which can be constructed by a length and width. The Rectangle class has a method which can compute the area. 

Hints:

Use def methodName(self) to define a method.


\fromA
\end{zadanie}
\begin{zadanie}
7.2

Define a class named Shape and its subclass Square. The Square class has an init function which takes a length as argument. Both classes have a area function which can print the area of the shape where Shape's area is 0 by default.

Hints:

To override a method in super class, we can define a method with the same name in the super class.


\fromA
\end{zadanie}
\begin{zadanie}
Please raise a RuntimeError exception.

Hints:

Use raise() to raise an exception.

Solution:

raise RuntimeError('something wrong')

\fromA
\end{zadanie}
\begin{zadanie}
Write a function to compute 5/0 and use try/except to catch the exceptions.

Hints:

Use try/except to catch exceptions.


\fromA
\end{zadanie}
\begin{zadanie}
Define a custom exception class which takes a string message as attribute.

Hints:

To define a custom exception, we need to define a class inherited from Exception.


\fromA
\end{zadanie}
\begin{zadanie}
Assuming that we have some email addresses in the "username@companyname.com" format, please write program to print the user name of a given email address. Both user names and company names are composed of letters only.

\fromA
\end{zadanie}
\begin{zadanie}
Assuming that we have some email addresses in the "username@companyname.com" format, please write program to print the company name of a given email address. Both user names and company names are composed of letters only.

\fromA
\end{zadanie}
\begin{zadanie}
Write a program which accepts a sequence of words separated by whitespace as input to print the words composed of digits only.

\fromA
\end{zadanie}
\begin{zadanie}
Print a unicode string "hello world".

\fromA
\end{zadanie}
\begin{zadanie}
Write a program to read an ASCII string and to convert it to a unicode string encoded by utf-8.

Hints:

Use unicode() function to convert.

print u

\fromA
\end{zadanie}
\begin{zadanie}
Write a special comment to indicate a Python source code file is in unicode.

\fromA
\end{zadanie}
\begin{zadanie}
Write a program to compute 1/2+2/3+3/4+...+n/n+1 with a given n input by console (n>0).

\fromA
\end{zadanie}
\begin{zadanie}
Write a program to compute:

\begin{verbatim}
f(n)=f(n-1)+100 when n>0
and f(0)=1
\end{verbatim}

with a given n input by console (n>0).

\fromA
\end{zadanie}
\begin{zadanie}
The Fibonacci Sequence is computed based on the following formula:

\begin{verbatim}
f(n)=0 if n=0
f(n)=1 if n=1
f(n)=f(n-1)+f(n-2) if n>1
\end{verbatim}

Please write a program to compute the value of f(n) with a given n input by console.

\fromA
\end{zadanie}
\begin{zadanie}
The Fibonacci Sequence is computed based on the following formula:

\begin{verbatim}
f(n)=0 if n=0
f(n)=1 if n=1
f(n)=f(n-1)+f(n-2) if n>1
\end{verbatim}

Please write a program using list comprehension to print the Fibonacci Sequence in comma separated form with a given n input by console.

\fromA
\end{zadanie}
\begin{zadanie}
Please write a program using generator to print the even numbers between 0 and n in comma separated form while n is input by console.

\fromA
\end{zadanie}
\begin{zadanie}
Please write a program using generator to print the numbers which can be divisible by 5 and 7 between 0 and n in comma separated form while n is input by console.

\fromA
\end{zadanie}
\begin{zadanie}
Please write assert statements to verify that every number in the list [2,4,6,8] is even.

\fromA
\end{zadanie}
\begin{zadanie}
Please write a program which accepts basic mathematic expression from console and print the evaluation result.

\fromA
\end{zadanie}
\begin{zadanie}
Please write a binary search function which searches an item in a sorted list. The function should return the index of element to be searched in the list.

\fromA
\end{zadanie}
\begin{zadanie}
Please write a binary search function which searches an item in a sorted list. The function should return the index of element to be searched in the list.

\fromA
\end{zadanie}
\begin{zadanie}
Please generate a random float where the value is between 10 and 100 using Python math module.

\fromA
\end{zadanie}
\begin{zadanie}
Please generate a random float where the value is between 5 and 95 using Python math module.

\fromA
\end{zadanie}
\begin{zadanie}
Please write a program to output a random even number between 0 and 10 inclusive using random module and list comprehension.

\fromA
\end{zadanie}
\begin{zadanie}
Please write a program to output a random number, which is divisible by 5 and 7, between 0 and 10 inclusive using random module and list comprehension.

\fromA
\end{zadanie}
\begin{zadanie}
Please write a program to generate a list with 5 random numbers between 100 and 200 inclusive.

\fromA
\end{zadanie}
\begin{zadanie}
Please write a program to randomly generate a list with 5 even numbers between 100 and 200 inclusive.

\fromA
\end{zadanie}
\begin{zadanie}
Please write a program to randomly generate a list with 5 numbers, which are divisible by 5 and 7 , between 1 and 1000 inclusive.

\fromA
\end{zadanie}
\begin{zadanie}
Please write a program to randomly print a integer number between 7 and 15 inclusive.

\fromA
\end{zadanie}
\begin{zadanie}
Please write a program to compress and decompress the string "hello world!hello world!hello world!hello world!".

\fromA
\end{zadanie}
\begin{zadanie}
Please write a program to print the running time of execution of "1+1" for 100 times.

\fromA
\end{zadanie}
\begin{zadanie}
Please write a program to shuffle and print the list [3,6,7,8].

\fromA
\end{zadanie}
\begin{zadanie}
Please write a program to shuffle and print the list [3,6,7,8].

\fromA
\end{zadanie}
\begin{zadanie}
Please write a program to generate all sentences where subject is in ["I", "You"] and verb is in ["Play", "Love"] and the object is in ["Hockey","Football"].

\fromA
\end{zadanie}
\begin{zadanie}
Please write a program to print the list after removing delete even numbers in [5,6,77,45,22,12,24].

Hints:
Use list comprehension to delete a bunch of element from a list.

\fromA
\end{zadanie}
\begin{zadanie}
By using list comprehension, please write a program to print the list after removing delete numbers which are divisible by 5 and 7 in [12,24,35,70,88,120,155].

\fromA
\end{zadanie}
\begin{zadanie}
By using list comprehension, please write a program to print the list after removing the 0th, 2nd, 4th,6th numbers in [12,24,35,70,88,120,155].

\fromA
\end{zadanie}
\begin{zadanie}
By using list comprehension, please write a program generate a 3*5*8 3D array whose each element is 0.

\fromA
\end{zadanie}
\begin{zadanie}
By using list comprehension, please write a program to print the list after removing the 0th,4th,5th numbers in [12,24,35,70,88,120,155].

\fromA
\end{zadanie}
\begin{zadanie}
By using list comprehension, please write a program to print the list after removing the value 24 in [12,24,35,24,88,120,155].

\fromA
\end{zadanie}
\begin{zadanie}
With two given lists [1,3,6,78,35,55] and [12,24,35,24,88,120,155], write a program to make a list whose elements are intersection of the above given lists.

\fromA
\end{zadanie}
\begin{zadanie}
With a given list [12,24,35,24,88,120,155,88,120,155], write a program to print this list after removing all duplicate values with original order reserved.

Hints:
Use set() to store a number of values without duplicate.

\fromA
\end{zadanie}
\begin{zadanie}
Define a class Person and its two child classes: Male and Female. All classes have a method "getGender" which can print "Male" for Male class and "Female" for Female class.

\fromA
\end{zadanie}
\begin{zadanie}
Please write a program which count and print the numbers of each character in a string input by console.

\fromA
\end{zadanie}
\begin{zadanie}
Please write a program which accepts a string from console and print it in reverse order.

\fromA
\end{zadanie}
\begin{zadanie}
Please write a program which accepts a string from console and print the characters that have even indexes.

\fromA
\end{zadanie}
\begin{zadanie}
Please write a program which prints all permutations of [1,2,3]

\fromA
\end{zadanie}
\begin{zadanie}
Write a program to solve a classic ancient Chinese puzzle: 
We count 35 heads and 94 legs among the chickens and rabbits in a farm. How many rabbits and how many chickens do we have?

Hint:
Use for loop to iterate all possible solutions.

\fromA
\end{zadanie}

\section{Zadania 2}


\section{Bibliografia}
\begin{thebibliography}{9}

\bibitem{python100}
 100+ Python challenging programming exercises
\end{thebibliography}


\end{document}