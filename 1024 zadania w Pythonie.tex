\documentclass[11pt]{article}
\usepackage[utf8]{inputenc}
\usepackage{polski}
\usepackage{graphicx}
\usepackage{array}
\usepackage{paralist}
\usepackage{verbatim}
\usepackage{subfig}
\usepackage{amsmath}
\usepackage{float}
\usepackage{amsthm}
\usepackage{amssymb}
\usepackage{pdfpages}
\usepackage{amsfonts}
%\usepackage{hyperref}
\setlength\parindent{0pt}
\theoremstyle{definition}
\newtheorem{zadanie}{Zadanie}
\maxdeadcycles=1000
\extrafloats{1000}
\title{1024 zadania w Pythonie}
\author{Igor Nowicki}

\begin{document}

\maketitle
\tableofcontents
\section{Wstęp}

Zgromadzone zadania z wielu źródeł, m.in.:

\begin{itemize}
	\item 100+ Python challenging programming exercises
	\item Konkurs informatyczny Logia i Minilogia
	\item Olimpiada Informatyczna Gimnazjalistów oraz Olimpiada Informatyczna
	\item Project Euler
\end{itemize}

\section{Zadania}

\begin{zadanie}
	Napisz program, który znajdzie wszystkie liczby z przedziału 2000 oraz 3200, które są podzielne przez 7, jednocześnie nie bedąc wielokrotnościami 5. Następnie program powinien wyświetlić sumę tych wartości na ekranie.
\end{zadanie}

\begin{zadanie}
	Stwórz funkcję \texttt{factorial(n)} która zwraca wartość silni z danej liczby naturalnej \texttt{n}. Następnie nakaż programowi wypisać na ekranie wartość silni z 8. Program powinien wyświetlić wartość 40320.
\end{zadanie}

\begin{zadanie}
Napisz program generujący słownik zawierający wartości \texttt{(i, i*i)} dla \texttt{i} z przedziału od \texttt{1} do \texttt{n}. Dla wartości 8 program powinien wyświetlić na ekranie następującą wiadomość:

\begin{verbatim}
{1: 1, 2: 4, 3: 9, 4: 16, 5: 25, 6: 36, 7: 49, 8: 64}
\end{verbatim}

\end{zadanie}

\begin{zadanie}
	Napisz program który przyjmuje z konsoli ciąg liczb oddzielonych przecinkami. Program powinien wygenerować listę oraz krotkę zawierające każdą z tych liczb.
	
	\bigskip
	Dla danych wejściowych:
	
	\begin{verbatim}
		34,67,55,33,12,98
	\end{verbatim}
	
	wiadomość wyjściowa powinna być następująca:
	
	\begin{verbatim}
		['34', '67', '55', '33', '12', '98']
		('34', '67', '55', '33', '12', '98')
	\end{verbatim}
\end{zadanie}

\begin{zadanie}
Stwórz klasę posiadającą przynajmniej dwie metody:
\begin{itemize}
	\item\texttt{get\_string}- metoda uzyskująca ciąg znaków ze standardowego wejścia,
	\item\texttt{print\_string}- metoda drukująca dany ciąg znaków wielkimi literami.
\end{itemize}
Dodatkowo, przygotuj prostą funkcję testową sprawdzającą obydwie metody klasy.
\end{zadanie}

\begin{zadanie}
Napisz program obliczający i wyświetlający na ekranie wartość wyliczaną na podstawie podanego wzoru:

$$Q = \sqrt{(2 \cdot C \cdot D)/H}$$

Poniżej podane są ustalone wartości $C$ oraz $H$:

\begin{itemize}
	\item C ma wartość 50.
	\item H ma wartość 30.
	\item D jest wartością podawaną przez użytkownika poprzez standardowe wejście.
\end{itemize}

Przykład: Załóżmy że do programu zostały podane następujące wartości:
\begin{verbatim}
100,150,180
\end{verbatim}

Wynik wyświetlony na ekranie powinien być następujący:
\begin{verbatim}
18,22,24
\end{verbatim}
\end{zadanie}

\begin{zadanie}
Napisz program który przyjmuje dwie cyfry, \texttt{x, y} ze standardowego wejścia i tworzy dwuwymiarową tablicę. Wartość elementu  w i-tym wierszu i j-tej kolumnie powinna być równa \texttt{i*j}. Uwaga: \texttt{i = 0, 1, 2, ..., x-1}, \texttt{j = 0, 1, 2, ..., y-1},

Przykład: załóżmy że do programu podano następujące dane wejściowe:
\begin{verbatim}
3,5
\end{verbatim}

Wtedy, program powinien wyświetlić następującą wiadomość:
\begin{verbatim}
[[0, 0, 0, 0, 0], [0, 1, 2, 3, 4], [0, 2, 4, 6, 8]] 
\end{verbatim}	
\end{zadanie}

\begin{zadanie}
Napisz program który przyjmuje słowa oddzielone przecinkami na standardowym wejściu i drukuje te same słowa po sortowaniu w kolejności alfabetycznej.

Przykład: założmy, że następujące dane zostały przekazane do programu:

\begin{verbatim}
without,hello,bag,world
\end{verbatim}
Wtedy, dane wyjściowe powinny być następujące:
\begin{verbatim}
bag,hello,without,world
\end{verbatim}
\end{zadanie}

\begin{zadanie}
Napisz program który przyjmuje ciąg linii na wejściu i drukuje je z powrotem po zmianie wszystkich znaków na wielkie litery.

Przykład: 

\begin{verbatim}
	Hello world
	Practice makes perfect
\end{verbatim}
Wtedy, zwrócona wiadomość powinna być następująca:
\begin{verbatim}
HELLO WORLD
PRACTICE MAKES PERFECT
\end{verbatim}
\end{zadanie}

\begin{zadanie}
Napisz program którzy przyjmuje ciąg słów rozdzielonych spacjami na wejściu i zwraca słowa z usuniętymi powtórzeniami, posortowane alfabetycznie.

Przykład: załóżmy, że następujące dane wejściowe zostały dostarczone do programu:
\begin{verbatim}
hello world and practice makes perfect and hello world again
\end{verbatim}
Wtedy, na wyjściu powinna być wyświetlona wiadomość:
\begin{verbatim}
again and hello makes perfect practice world
\end{verbatim}
\end{zadanie}
\begin{zadanie}
Napisz program który przyjmuje ciąg 4-cyfrowych liczb binarnych oddzielonych przecinkami, a następnie sprawdza czy są podzielne przez 5. Liczby podzielne przez 5, oddzielone przecinkami, są wyświetlane na ekranie.
\end{zadanie}
\begin{zadanie}
Napisz program który znajdzie wszystkie liczby pomiędzy 1000 i 3000 (włączając) takie, że każda cyfra jest liczbą parzystą. Uzyskane liczby rozdzielone przecinkami powinny być wydrukowane na ekranie jako informacja zwrotna.
\end{zadanie}

\begin{zadanie}
Stwórz funkcje \texttt{expand(s)} oraz \texttt{shorten(s)} które działają w następujący sposób:

\begin{itemize}
\item \texttt{expand(s)} przekształca każdy ciąg znaków postaci \texttt{`a2b3c1`} do rozszerzonej postaci: \texttt{`aabbbc`},
\item \texttt{shorten(s)} przekształca dowolny ciąg znaków do postaci skróconej. Przykład: \texttt{`aaabbccc`} jest przekształcane do postaci \texttt{`a3b2c3`}.
\end{itemize}

\end{zadanie}

\begin{zadanie}
Write a program that accepts a sentence and calculate the number of letters and digits.
Suppose the following input is supplied to the program:
hello world! 123
Then, the output should be:
LETTERS 10
DIGITS 3
\end{zadanie}
\begin{zadanie}
	Write a program that accepts a sentence and calculate the number of upper case letters and lower case letters.
	Suppose the following input is supplied to the program:
	Hello world!
	Then, the output should be:
	UPPER CASE 1
	LOWER CASE 9

	
\end{zadanie}
\begin{zadanie}
	Write a program that computes the value of a+aa+aaa+aaaa with a given digit as the value of a.
	Suppose the following input is supplied to the program:
	9
	Then, the output should be:
	11106	
\end{zadanie}
\begin{zadanie}
	Use a list comprehension to square each odd number in a list. The list is input by a sequence of comma-separated numbers.
	Suppose the following input is supplied to the program:
	1,2,3,4,5,6,7,8,9
	Then, the output should be:
	1,3,5,7,9

	
\end{zadanie}
\begin{zadanie}
	Write a program that computes the net amount of a bank account based a transaction log from console input. The transaction log format is shown as following:
	D 100
	W 200
	D means deposit while W means withdrawal.
	Suppose the following input is supplied to the program:
	D 300
	D 300
	W 200
	D 100
	Then, the output should be:
	500

	
\end{zadanie}
\begin{zadanie}
	A website requires the users to input username and password to register. Write a program to check the validity of password input by users.
	Following are the criteria for checking the password:
	1. At least 1 letter between [a-z]
	2. At least 1 number between [0-9]
	1. At least 1 letter between [A-Z]
	3. At least 1 character from [\#@]
	4. Minimum length of transaction password: 6
	5. Maximum length of transaction password: 12
	Your program should accept a sequence of comma separated passwords and will check them according to the above criteria. Passwords that match the criteria are to be printed, each separated by a comma.
	Example
	If the following passwords are given as input to the program:
	ABd1234@1,a F\#,2w3E*,2We3345
	Then, the output of the program should be:
	ABd1234@1

	
\end{zadanie}


\begin{comment}

\begin{zadanie}
	You are required to write a program to sort the (name, age, height) tuples by ascending order where name is string, age and height are numbers. The tuples are input by console. The sort criteria is:
	1: Sort based on name;
	2: Then sort based on age;
	3: Then sort by score.
	The priority is that name > age > score.
	If the following tuples are given as input to the program:
	Tom,19,80
	John,20,90
	Jony,17,91
	Jony,17,93
	Json,21,85
	Then, the output of the program should be:
	\begin{verbatim}
[('John', '20', '90'), ('Jony', '17', '91'), ('Jony', '17', '93'), ('Json', '21', '85'), ('Tom', '19', '80')]
\end{verbatim}

	
\end{zadanie}
\begin{zadanie}
	Define a class with a generator which can iterate the numbers, which are divisible by 7, between a given range 0 and n.

	
\end{zadanie}
\begin{zadanie}
	A robot moves in a plane starting from the original point (0,0). The robot can move toward UP, DOWN, LEFT and RIGHT with a given steps. The trace of robot movement is shown as the following:
	\begin{verbatim}
UP 5
DOWN 3
LEFT 3
RIGHT 2
\end{verbatim}
	The numbers after the direction are steps. Please write a program to compute the distance from current position after a sequence of movement and original point. If the distance is a float, then just print the nearest integer.
	Example:
	If the following tuples are given as input to the program:
	\begin{verbatim}
UP 5
DOWN 3
LEFT 3
RIGHT 2
\end{verbatim}
	Then, the output of the program should be:
	2

	Hints:
	In case of input data being supplied to the question, it should be assumed to be a console input.


	
\end{zadanie}
\begin{zadanie}
	Question 22
	Level 3

	Question:
	Write a program to compute the frequency of the words from the input. The output should output after sorting the key alphanumerically.
	Suppose the following input is supplied to the program:
	New to Python or choosing between Python 2 and Python 3? Read Python 2 or Python 3.
	Then, the output should be:
	\begin{verbatim}
2:2
3.:1
3?:1
New:1
Python:5
Read:1
and:1
between:1
choosing:1
or:2
to:1
\end{verbatim}

	Hints

	In case of input data being supplied to the question, it should be assumed to be a console input.

	
\end{zadanie}
\begin{zadanie}
	Write a method which can calculate square value of number

	
\end{zadanie}
\begin{zadanie}
	Python has many built-in functions, and if you do not know how to use it, you can read document online or find some books. But Python has a built-in document function for every built-in functions.
	Please write a program to print some Python built-in functions documents, such as abs(), int(), raw\
	input()
	And add document for your own function

	
\end{zadanie}
\begin{zadanie}
	Define a class, which have a class parameter and have a same instance parameter.

	
\end{zadanie}
\begin{zadanie}
	Define a function which can compute the sum of two numbers.

	
\end{zadanie}
\begin{zadanie}
	Define a function that can convert a integer into a string and print it in console.

	
\end{zadanie}
\begin{zadanie}
	Define a function that can convert a integer into a string and print it in console.

	
\end{zadanie}
\begin{zadanie}
	Define a function that can receive two integral numbers in string form and compute their sum and then print it in console.

	
\end{zadanie}
\begin{zadanie}
	Define a function that can accept two strings as input and concatenate them and then print it in console.

	
\end{zadanie}
\begin{zadanie}
	Define a function that can accept two strings as input and print the string with maximum length in console. If two strings have the same length, then the function should print al l strings line by line.

	
\end{zadanie}
\begin{zadanie}
	Define a function that can accept an integer number as input and print the "It is an even number" if the number is even, otherwise print "It is an odd number".

	
\end{zadanie}
\begin{zadanie}
	Define a function which can print a dictionary where the keys are numbers between 1 and 3 (both included) and the values are square of keys.

	
\end{zadanie}
\begin{zadanie}
	Define a function which can print a dictionary where the keys are numbers between 1 and 20 (both included) and the values are square of keys.

	
\end{zadanie}
\begin{zadanie}
	Define a function which can generate a dictionary where the keys are numbers between 1 and 20 (both included) and the values are square of keys. The function should just print the values only.

	
\end{zadanie}
\begin{zadanie}
	Define a function which can generate a dictionary where the keys are numbers between 1 and 20 (both included) and the values are square of keys. The function should just print the keys only.

	
\end{zadanie}
\begin{zadanie}
	Define a function which can generate and print a list where the values are square of numbers between 1 and 20 (both included).

	
\end{zadanie}
\begin{zadanie}
	Define a function which can generate a list where the values are square of numbers between 1 and 20 (both included). Then the function needs to print the first 5 elements in the list.

	
\end{zadanie}
\begin{zadanie}
	Define a function which can generate a list where the values are square of numbers between 1 and 20 (both included). Then the function needs to print the last 5 elements in the list.

	
\end{zadanie}
\begin{zadanie}
	Define a function which can generate a list where the values are square of numbers between 1 and 20 (both included). Then the function needs to print all values except the first 5 elements in the list.

	
\end{zadanie}
\begin{zadanie}
	Define a function which can generate and print a tuple where the value are square of numbers between 1 and 20 (both included).

	
\end{zadanie}
\begin{zadanie}
	With a given tuple (1,2,3,4,5,6,7,8,9,10), write a program to print the first half values in one line and the last half values in one line.

	
\end{zadanie}
\begin{zadanie}
	Write a program to generate and print another tuple whose values are even numbers in the given tuple (1,2,3,4,5,6,7,8,9,10).

	
\end{zadanie}
\begin{zadanie}
	Write a program which accepts a string as input to print "Yes" if the string is "yes" or "YES" or "Yes", otherwise print "No".

	
\end{zadanie}
\begin{zadanie}
	Write a program which can filter even numbers in a list by using filter function. The list is: [1,2,3,4,5,6,7,8,9,10].

	
\end{zadanie}
\begin{zadanie}
	Write a program which can map() to make a list whose elements are square of elements in [1,2,3,4,5,6,7,8,9,10].

	
\end{zadanie}
\begin{zadanie}
	Write a program which can map() and filter() to make a list whose elements are square of even number in [1,2,3,4,5,6,7,8,9,10].

	
\end{zadanie}
\begin{zadanie}
	Write a program which can filter() to make a list whose elements are even number between 1 and 20 (both included).

	
\end{zadanie}
\begin{zadanie}
	Write a program which can map() to make a list whose elements are square of numbers between 1 and 20 (both included).

	
\end{zadanie}
\begin{zadanie}
	Define a class named American which has a static method called printNationality.

	
\end{zadanie}
\begin{zadanie}
	Define a class named American and its subclass NewYorker.

	
\end{zadanie}
\begin{zadanie}
	Define a class named Circle which can be constructed by a radius. The Circle class has a method which can compute the area.

	
\end{zadanie}
\begin{zadanie}
	Define a class named Rectangle which can be constructed by a length and width. The Rectangle class has a method which can compute the area.

	Hints:

	Use def methodName(self) to define a method.


	
\end{zadanie}
\begin{zadanie}
	7.2

	Define a class named Shape and its subclass Square. The Square class has an init function which takes a length as argument. Both classes have a area function which can print the area of the shape where Shape's area is 0 by default.

	Hints:

	To override a method in super class, we can define a method with the same name in the super class.


	
\end{zadanie}
\begin{zadanie}
	Please raise a RuntimeError exception.

	Hints:

	Use raise() to raise an exception.

	Solution:

	raise RuntimeError('something wrong')

	
\end{zadanie}
\begin{zadanie}
	Write a function to compute 5/0 and use try/except to catch the exceptions.

	Hints:

	Use try/except to catch exceptions.


	
\end{zadanie}
\begin{zadanie}
	Define a custom exception class which takes a string message as attribute.

	Hints:

	To define a custom exception, we need to define a class inherited from Exception.


	
\end{zadanie}
\begin{zadanie}
	Assuming that we have some email addresses in the "username@companyname.com" format, please write program to print the user name of a given email address. Both user names and company names are composed of letters only.

	
\end{zadanie}
\begin{zadanie}
	Assuming that we have some email addresses in the "username@companyname.com" format, please write program to print the company name of a given email address. Both user names and company names are composed of letters only.

	
\end{zadanie}
\begin{zadanie}
	Write a program which accepts a sequence of words separated by whitespace as input to print the words composed of digits only.

	
\end{zadanie}
\begin{zadanie}
	Print a unicode string "hello world".

	
\end{zadanie}
\begin{zadanie}
	Write a program to read an ASCII string and to convert it to a unicode string encoded by utf-8.

	Hints:

	Use unicode() function to convert.

	print u

	
\end{zadanie}
\begin{zadanie}
	Write a special comment to indicate a Python source code file is in unicode.

	
\end{zadanie}
\begin{zadanie}
	Write a program to compute 1/2+2/3+3/4+...+n/n+1 with a given n input by console (n>0).

	
\end{zadanie}
\begin{zadanie}
	Write a program to compute:

	\begin{verbatim}
f(n)=f(n-1)+100 when n>0
and f(0)=1
\end{verbatim}

	with a given n input by console (n>0).

	
\end{zadanie}
\begin{zadanie}
	The Fibonacci Sequence is computed based on the following formula:

	\begin{verbatim}
f(n)=0 if n=0
f(n)=1 if n=1
f(n)=f(n-1)+f(n-2) if n>1
\end{verbatim}

	Please write a program to compute the value of f(n) with a given n input by console.

	
\end{zadanie}
\begin{zadanie}
	The Fibonacci Sequence is computed based on the following formula:

	\begin{verbatim}
f(n)=0 if n=0
f(n)=1 if n=1
f(n)=f(n-1)+f(n-2) if n>1
\end{verbatim}

	Please write a program using list comprehension to print the Fibonacci Sequence in comma separated form with a given n input by console.

	
\end{zadanie}
\begin{zadanie}
	Please write a program using generator to print the even numbers between 0 and n in comma separated form while n is input by console.

	
\end{zadanie}
\begin{zadanie}
	Please write a program using generator to print the numbers which can be divisible by 5 and 7 between 0 and n in comma separated form while n is input by console.

	
\end{zadanie}
\begin{zadanie}
	Please write assert statements to verify that every number in the list [2,4,6,8] is even.

	
\end{zadanie}
\begin{zadanie}
	Please write a program which accepts basic mathematic expression from console and print the evaluation result.

	
\end{zadanie}
\begin{zadanie}
	Please write a binary search function which searches an item in a sorted list. The function should return the index of element to be searched in the list.

	
\end{zadanie}
\begin{zadanie}
	Please write a binary search function which searches an item in a sorted list. The function should return the index of element to be searched in the list.

	
\end{zadanie}
\begin{zadanie}
	Please generate a random float where the value is between 10 and 100 using Python math module.

	
\end{zadanie}
\begin{zadanie}
	Please generate a random float where the value is between 5 and 95 using Python math module.

	
\end{zadanie}
\begin{zadanie}
	Please write a program to output a random even number between 0 and 10 inclusive using random module and list comprehension.

	
\end{zadanie}
\begin{zadanie}
	Please write a program to output a random number, which is divisible by 5 and 7, between 0 and 10 inclusive using random module and list comprehension.

	
\end{zadanie}
\begin{zadanie}
	Please write a program to generate a list with 5 random numbers between 100 and 200 inclusive.

	
\end{zadanie}
\begin{zadanie}
	Please write a program to randomly generate a list with 5 even numbers between 100 and 200 inclusive.

	
\end{zadanie}
\begin{zadanie}
	Please write a program to randomly generate a list with 5 numbers, which are divisible by 5 and 7 , between 1 and 1000 inclusive.

	
\end{zadanie}
\begin{zadanie}
	Please write a program to randomly print a integer number between 7 and 15 inclusive.

	
\end{zadanie}
\begin{zadanie}
	Please write a program to compress and decompress the string "hello world!hello world!hello world!hello world!".

	
\end{zadanie}
\begin{zadanie}
	Please write a program to print the running time of execution of "1+1" for 100 times.

	
\end{zadanie}
\begin{zadanie}
	Please write a program to shuffle and print the list [3,6,7,8].

	
\end{zadanie}
\begin{zadanie}
	Please write a program to shuffle and print the list [3,6,7,8].

	
\end{zadanie}
\begin{zadanie}
	Please write a program to generate all sentences where subject is in ["I", "You"] and verb is in ["Play", "Love"] and the object is in ["Hockey","Football"].

	
\end{zadanie}
\begin{zadanie}
	Please write a program to print the list after removing delete even numbers in [5,6,77,45,22,12,24].

	Hints:
	Use list comprehension to delete a bunch of element from a list.

	
\end{zadanie}
\begin{zadanie}
	By using list comprehension, please write a program to print the list after removing delete numbers which are divisible by 5 and 7 in [12,24,35,70,88,120,155].

	
\end{zadanie}
\begin{zadanie}
	By using list comprehension, please write a program to print the list after removing the 0th, 2nd, 4th,6th numbers in [12,24,35,70,88,120,155].

	
\end{zadanie}
\begin{zadanie}
	By using list comprehension, please write a program generate a 3*5*8 3D array whose each element is 0.

	
\end{zadanie}
\begin{zadanie}
	By using list comprehension, please write a program to print the list after removing the 0th,4th,5th numbers in [12,24,35,70,88,120,155].

	
\end{zadanie}
\begin{zadanie}
	By using list comprehension, please write a program to print the list after removing the value 24 in [12,24,35,24,88,120,155].

	
\end{zadanie}
\begin{zadanie}
	With two given lists [1,3,6,78,35,55] and [12,24,35,24,88,120,155], write a program to make a list whose elements are intersection of the above given lists.

	
\end{zadanie}
\begin{zadanie}
	With a given list [12,24,35,24,88,120,155,88,120,155], write a program to print this list after removing all duplicate values with original order reserved.

	Hints:
	Use set() to store a number of values without duplicate.

	
\end{zadanie}
\begin{zadanie}
	Define a class Person and its two child classes: Male and Female. All classes have a method "getGender" which can print "Male" for Male class and "Female" for Female class.

	
\end{zadanie}
\begin{zadanie}
	Please write a program which count and print the numbers of each character in a string input by console.

	
\end{zadanie}
\begin{zadanie}
	Please write a program which accepts a string from console and print it in reverse order.

	
\end{zadanie}
\begin{zadanie}
	Please write a program which accepts a string from console and print the characters that have even indexes.

	
\end{zadanie}
\begin{zadanie}
	Please write a program which prints all permutations of [1,2,3]

	
\end{zadanie}
\begin{zadanie}
	Write a program to solve a classic ancient Chinese puzzle:
	We count 35 heads and 94 legs among the chickens and rabbits in a farm. How many rabbits and how many chickens do we have?

	Hint:
	Use for loop to iterate all possible solutions.

	
\end{zadanie}

\section{Zadania 4}

\begin{zadanie}

	[An editor is available at the bottom of the page to write and execute the scripts.]

	1. Write a Python program to print the following string in a specific format (see the output).
	Sample String : "Twinkle, twinkle, little star, How I wonder what you are! Up above the world so high, Like a diamond in the sky. Twinkle, twinkle, little star, How I wonder what you are" Output :

	Twinkle, twinkle, little star,
	How I wonder what you are!
	Up above the world so high,
	Like a diamond in the sky.
	Twinkle, twinkle, little star,
	How I wonder what you are


\end{zadanie}

\begin{zadanie}


	2. Write a Python program to get the Python version you are using.

\end{zadanie}

\begin{zadanie}


	3. Write a Python program to display the current date and time.
	Sample Output :
	Current date and time :
	2014-07-05 14:34:14

\end{zadanie}

\begin{zadanie}


	4. Write a Python program which accepts the radius of a circle from the user and compute the area.
	Sample Output :
	r = 1.1
	Area = 3.8013271108436504

\end{zadanie}

\begin{zadanie}


	5. Write a Python program which accepts the user's first and last name and print them in reverse order with a space between them.

\end{zadanie}

\begin{zadanie}


	6. Write a Python program which accepts a sequence of comma-separated numbers from user and generate a list and a tuple with those numbers.
	Sample data : 3, 5, 7, 23
	Output :
	List : ['3', ' 5', ' 7', ' 23']
	Tuple : ('3', ' 5', ' 7', ' 23')

\end{zadanie}

\begin{zadanie}


	7. Write a Python program to accept a filename from the user and print the extension of that.
	Sample filename : abc.java
	Output : java

\end{zadanie}

\begin{zadanie}


	8. Write a Python program to display the first and last colors from the following list.
	\begin{verbatim}
color_list = ["Red","Green","White" ,"Black"]
\end{verbatim}
\end{zadanie}

\begin{zadanie}


	9. Write a Python program to display the examination schedule. (extract the date from \texttt{exam\_st\_date}).

	\begin{verbatim}
exam_st_date = (11, 12, 2014)
\end{verbatim}
	Sample Output : The examination will start from : 11 / 12 / 2014

\end{zadanie}

\begin{zadanie}


	10. Write a Python program that accepts an integer (n) and computes the value of n+nn+nnn.
	Sample value of n is 5
	Expected Result : 615

\end{zadanie}

\begin{zadanie}


	11. Write a Python program to print the documents (syntax, description etc.) of Python built-in function(s).
	Sample function : abs()
	Expected Result :
	abs(number) -> number
	Return the absolute value of the argument.

\end{zadanie}

\begin{zadanie}


	12. Write a Python program to print the calendar of a given month and year.
	Note : Use 'calendar' module.

\end{zadanie}

\begin{zadanie}


	13. Write a Python program to print the following here document.
	Sample string :
	a string that you "don't" have to escape
	This
	is a ....... multi-line
	heredoc string --------> example

\end{zadanie}

\begin{zadanie}


	14. Write a Python program to calculate number of days between two dates.
	Sample dates : (2014, 7, 2), (2014, 7, 11)
	Expected output : 9 days

\end{zadanie}

\begin{zadanie}


	15. Write a Python program to get the volume of a sphere with radius 6.

\end{zadanie}

\begin{zadanie}


	16. Write a Python program to get the difference between a given number and 17, if the number is greater than 17 return double the absolute difference.

\end{zadanie}

\begin{zadanie}


	17. Write a Python program to test whether a number is within 100 of 1000 or 2000.

\end{zadanie}

\begin{zadanie}


	18. Write a Python program to calculate the sum of three given numbers, if the values are equal then return three times of their sum.

\end{zadanie}

\begin{zadanie}


	19. Write a Python program to get a new string from a given string where "Is" has been added to the front. If the given string already begins with "Is" then return the string unchanged.

\end{zadanie}

\begin{zadanie}


	20. Write a Python program to get a string which is n (non-negative integer) copies of a given string.

\end{zadanie}

\begin{zadanie}


	21. Write a Python program to find whether a given number (accept from the user) is even or odd, print out an appropriate message to the user.

\end{zadanie}

\begin{zadanie}


	22. Write a Python program to count the number 4 in a given list.

\end{zadanie}

\begin{zadanie}


	23. Write a Python program to get the n (non-negative integer) copies of the first 2 characters of a given string. Return the n copies of the whole string if the length is less than 2.

\end{zadanie}

\begin{zadanie}


	24. Write a Python program to test whether a passed letter is a vowel or not.

\end{zadanie}

\begin{zadanie}


	25. Write a Python program to check whether a specified value is contained in a group of values.
	Test Data :
	3 -> [1, 5, 8, 3] : True
	-1 -> [1, 5, 8, 3] : False


\end{zadanie}

\begin{zadanie}


	26. Write a Python program to create a histogram from a given list of integers.

\end{zadanie}

\begin{zadanie}


	27. Write a Python program to concatenate all elements in a list into a string and return it.

\end{zadanie}

\begin{zadanie}


	28. Write a Python program to print all even numbers from a given numbers list in the same order and stop the printing if any numbers that come after 237 in the sequence.
	Sample numbers list :

	numbers = [
	386, 462, 47, 418, 907, 344, 236, 375, 823, 566, 597, 978, 328, 615, 953, 345,
	399, 162, 758, 219, 918, 237, 412, 566, 826, 248, 866, 950, 626, 949, 687, 217,
	815, 67, 104, 58, 512, 24, 892, 894, 767, 553, 81, 379, 843, 831, 445, 742, 717,
	958,743, 527
	]


\end{zadanie}

\begin{zadanie}


	29. Write a Python program to print out a set containing all the colors from \texttt{color\_list\_1} which are not present in \texttt{color\_list\_2}.
	\begin{verbatim}
Test Data :
color_list_1 = set(["White", "Black", "Red"])
color_list_2 = set(["Red", "Green"])
Expected Output :
{'Black', 'White'}
\end{verbatim}

\end{zadanie}

\begin{zadanie}


	30. Write a Python program that will accept the base and height of a triangle and compute the area.

\end{zadanie}

\begin{zadanie}


	31. Write a Python program to compute the greatest common divisor (GCD) of two positive integers.

\end{zadanie}

\begin{zadanie}


	32. Write a Python program to get the least common multiple (LCM) of two positive integers.

\end{zadanie}

\begin{zadanie}


	33. Write a Python program to sum of three given integers. However, if two values are equal sum will be zero.

\end{zadanie}

\begin{zadanie}


	34. Write a Python program to sum of two given integers. However, if the sum is between 15 to 20 it will return 20.

\end{zadanie}

\begin{zadanie}


	35. Write a Python program that will return true if the two given integer values are equal or their sum or difference is 5.

\end{zadanie}

\begin{zadanie}


	36. Write a Python program to add two objects if both objects are an integer type.

\end{zadanie}

\begin{zadanie}


	37. Write a Python program to display your details like name, age, address in three different lines.

\end{zadanie}

\begin{zadanie}


	38. Write a Python program to solve (x + y) * (x + y).
	\begin{verbatim}
Test Data : x = 4, y = 3
Expected Output : (4 + 3) ^ 2) = 49
\end{verbatim}
\end{zadanie}

\begin{zadanie}


	39. Write a Python program to compute the future value of a specified principal amount, rate of interest, and a number of years.
	Test Data : amt = 10000, int = 3.5, years = 7
	Expected Output : 12722.79

\end{zadanie}

\begin{zadanie}


	40. Write a Python program to compute the distance between the points (x1, y1) and (x2, y2).

\end{zadanie}

\begin{zadanie}


	41. Write a Python program to check whether a file exists.

\end{zadanie}

\begin{zadanie}


	42. Write a Python program to determine if a Python shell is executing in 32bit or 64bit mode on OS.

\end{zadanie}

\begin{zadanie}


	43. Write a Python program to get OS name, platform and release information.

\end{zadanie}

\begin{zadanie}


	44. Write a Python program to locate Python site-packages.

\end{zadanie}

\begin{zadanie}


	45. Write a python program to call an external command in Python.

\end{zadanie}

\begin{zadanie}


	46. Write a python program to get the path and name of the file that is currently executing.

\end{zadanie}

\begin{zadanie}


	47. Write a Python program to find out the number of CPUs using.

\end{zadanie}

\begin{zadanie}


	48. Write a Python program to parse a string to Float or Integer.

\end{zadanie}

\begin{zadanie}


	49. Write a Python program to list all files in a directory in Python.

\end{zadanie}

\begin{zadanie}


	50. Write a Python program to print without newline or space.

\end{zadanie}

\begin{zadanie}


	51. Write a Python program to determine profiling of Python programs.
	Note: A profile is a set of statistics that describes how often and for how long various parts of the program executed. These statistics can be formatted into reports via the pstats module.

\end{zadanie}

\begin{zadanie}


	52. Write a Python program to print to stderr.

\end{zadanie}

\begin{zadanie}


	53. Write a python program to access environment variables.

\end{zadanie}

\begin{zadanie}


	54. Write a Python program to get the current username

\end{zadanie}

\begin{zadanie}


	55. Write a Python to find local IP addresses using Python's stdlib

\end{zadanie}

\begin{zadanie}


	56. Write a Python program to get height and width of the console window.

\end{zadanie}

\begin{zadanie}


	57. Write a program to get execution time for a Python method.

\end{zadanie}

\begin{zadanie}


	58. Write a python program to sum of the first n positive integers.

\end{zadanie}

\begin{zadanie}


	59. Write a Python program to convert height (in feet and inches) to centimeters.

\end{zadanie}

\begin{zadanie}


	60. Write a Python program to calculate the hypotenuse of a right angled triangle.

\end{zadanie}

\begin{zadanie}


	61. Write a Python program to convert the distance (in feet) to inches, yards, and miles.

\end{zadanie}

\begin{zadanie}


	62. Write a Python program to convert all units of time into seconds.

\end{zadanie}

\begin{zadanie}


	63. Write a Python program to get an absolute file path.

\end{zadanie}

\begin{zadanie}


	64. Write a Python program to get file creation and modification date/times.

\end{zadanie}

\begin{zadanie}


	65. Write a Python program to convert seconds to day, hour, minutes and seconds.

\end{zadanie}

\begin{zadanie}


	66. Write a Python program to calculate body mass index.

\end{zadanie}

\begin{zadanie}


	67. Write a Python program to convert pressure in kilopascals to pounds per square inch, a millimeter of mercury (mmHg) and atmosphere pressure.

\end{zadanie}

\begin{zadanie}


	68. Write a Python program to calculate the sum of the digits in an integer.

\end{zadanie}

\begin{zadanie}


	69. Write a Python program to sort three integers without using conditional statements and loops.

\end{zadanie}

\begin{zadanie}


	70. Write a Python program to sort files by date.

\end{zadanie}

\begin{zadanie}


	71. Write a Python program to get a directory listing, sorted by creation date.

\end{zadanie}

\begin{zadanie}


	72. Write a Python program to get the details of math module.

\end{zadanie}

\begin{zadanie}


	73. Write a Python program to calculate midpoints of a line.

\end{zadanie}

\begin{zadanie}


	74. Write a Python program to hash a word.

\end{zadanie}

\begin{zadanie}


	75. Write a Python program to get the copyright information.

\end{zadanie}

\begin{zadanie}


	76. Write a Python program to get the command-line arguments (name of the script, the number of arguments, arguments) passed to a script.

\end{zadanie}

\begin{zadanie}


	77. Write a Python program to test whether the system is a big-endian platform or little-endian platform.

\end{zadanie}

\begin{zadanie}


	78. Write a Python program to find the available built-in modules.

\end{zadanie}

\begin{zadanie}


	79. Write a Python program to get the size of an object in bytes.

\end{zadanie}

\begin{zadanie}


	80. Write a Python program to get the current value of the recursion limit.

\end{zadanie}

\begin{zadanie}


	81. Write a Python program to concatenate N strings.

\end{zadanie}

\begin{zadanie}


	82. Write a Python program to calculate the sum over a container.

\end{zadanie}

\begin{zadanie}


	83. Write a Python program to test whether all numbers of a list is greater than a certain number.

\end{zadanie}

\begin{zadanie}


	84. Write a Python program to count the number occurrence of a specific character in a string.

\end{zadanie}

\begin{zadanie}


	85. Write a Python program to check if a file path is a file or a directory.

\end{zadanie}

\begin{zadanie}


	86. Write a Python program to get the ASCII value of a character.

\end{zadanie}

\begin{zadanie}


	87. Write a Python program to get the size of a file.

\end{zadanie}

\begin{zadanie}


	88. Given variables x=30 and y=20, write a Python program to print t "30+20=50".

\end{zadanie}

\begin{zadanie}


	89. Write a Python program to perform an action if a condition is true.
	Given a variable name, if the value is 1, display the string "First day of a Month!" and do nothing if the value is not equal.

\end{zadanie}

\begin{zadanie}


	90. Write a Python program to create a copy of its own source code.

\end{zadanie}

\begin{zadanie}


	91. Write a Python program to swap two variables.

\end{zadanie}

\begin{zadanie}


	92. Write a Python program to define a string containing special characters in various forms.

\end{zadanie}

\begin{zadanie}


	93. Write a Python program to get the identity of an object.

\end{zadanie}

\begin{zadanie}


	94. Write a Python program to convert a byte string to a list of integers.

\end{zadanie}

\begin{zadanie}


	95. Write a Python program to check if a string is numeric.

\end{zadanie}

\begin{zadanie}


	96. Write a Python program to print the current call stack.

\end{zadanie}

\begin{zadanie}


	97. Write a Python program to list the special variables used within the language.

\end{zadanie}

\begin{zadanie}


	98. Write a Python program to get the system time.

	Note : The system time is important for debugging, network information, random number seeds, or something as simple as program performance.

\end{zadanie}

\begin{zadanie}


	99. Write a Python program to clear the screen or terminal.

\end{zadanie}

\begin{zadanie}


	100. Write a Python program to get the name of the host on which the routine is running.

\end{zadanie}

\begin{zadanie}


	101. Write a Python program to access and print a URL's content to the console.

\end{zadanie}

\begin{zadanie}


	102. Write a Python program to get system command output.

\end{zadanie}

\begin{zadanie}


	103. Write a Python program to extract the filename from a given path.

\end{zadanie}

\begin{zadanie}


	104. Write a Python program to get the effective group id, effective user id, real group id, a list of supplemental group ids associated with the current process.
	Note: Availability: Unix.

\end{zadanie}

\begin{zadanie}


	105. Write a Python program to get the users environment.

\end{zadanie}

\begin{zadanie}


	106. Write a Python program to divide a path on the extension separator.

\end{zadanie}

\begin{zadanie}


	107. Write a Python program to retrieve file properties.

\end{zadanie}

\begin{zadanie}


	108. Write a Python program to find path refers to a file or directory when you encounter a path name.

\end{zadanie}

\begin{zadanie}


	109. Write a Python program to check if a number is positive, negative or zero.

\end{zadanie}

\begin{zadanie}


	110. Write a Python program to get numbers divisible by fifteen from a list using an anonymous function.

\end{zadanie}

\begin{zadanie}


	111. Write a Python program to make file lists from current directory using a wildcard.

\end{zadanie}

\begin{zadanie}


	112. Write a Python program to remove the first item from a specified list.

\end{zadanie}

\begin{zadanie}


	113. Write a Python program to input a number, if it is not a number generate an error message.

\end{zadanie}

\begin{zadanie}


	114. Write a Python program to filter the positive numbers from a list.

\end{zadanie}

\begin{zadanie}


	115. Write a Python program to compute the product of a list of integers (without using for loop).

\end{zadanie}

\begin{zadanie}


	116. Write a Python program to print Unicode characters.

\end{zadanie}

\begin{zadanie}


	117. Write a Python program to prove that two string variables of same value point same memory location.

\end{zadanie}

\begin{zadanie}


	118. Write a Python program to create a bytearray from a list.

\end{zadanie}

\begin{zadanie}


	119. Write a Python program to display a floating number in specified numbers.

\end{zadanie}

\begin{zadanie}


	120. Write a Python program to format a specified string to limit the number of characters to 6.

\end{zadanie}

\begin{zadanie}


	121. Write a Python program to determine if variable is defined or not.

\end{zadanie}

\begin{zadanie}


	122. Write a Python program to empty a variable without destroying it.

	Sample data: n=20
	d = {"x":200}
	Expected Output : 0
		{}


\end{zadanie}

\begin{zadanie}


	123. Write a Python program to determine the largest and smallest integers, longs, floats.

\end{zadanie}

\begin{zadanie}


	124. Write a Python program to check if multiple variables have the same value.

\end{zadanie}

\begin{zadanie}


	125. Write a Python program to sum of all counts in a collections?

\end{zadanie}

\begin{zadanie}


	126. Write a Python program to get the actual module object for a given object.

\end{zadanie}

\begin{zadanie}


	127. Write a Python program to check if an integer fits in 64 bits.

\end{zadanie}

\begin{zadanie}


	128. Write a Python program to check if lowercase letters exist in a string.

\end{zadanie}

\begin{zadanie}


	129. Write a Python program to add leading zeroes to a string.

\end{zadanie}

\begin{zadanie}


	130. Write a Python program to use double quotes to display strings.

\end{zadanie}

\begin{zadanie}


	131. Write a Python program to split a variable length string into variables.

\end{zadanie}

\begin{zadanie}


	132. Write a Python program to list home directory without absolute path.

\end{zadanie}

\begin{zadanie}


	133. Write a Python program to calculate the time runs (difference between start and current time) of a program.

\end{zadanie}

\begin{zadanie}


	134. Write a Python program to input two integers in a single line.

\end{zadanie}

\begin{zadanie}


	135. Write a Python program to print a variable without spaces between values.
	Sample value : x =30
	Expected output : Value of x is "30"

\end{zadanie}

\begin{zadanie}


	136. Write a Python program to find files and skip directories of a given directory.

\end{zadanie}

\begin{zadanie}


	137. Write a Python program to extract single key-value pair of a dictionary in variables.

\end{zadanie}

\begin{zadanie}


	138. Write a Python program to convert true to 1 and false to 0.

\end{zadanie}

\begin{zadanie}


	139. Write a Python program to valid a IP address.

\end{zadanie}

\begin{zadanie}


	140. Write a Python program to convert an integer to binary keep leading zeros.
	Sample data : 50
	Expected output : 00001100, 0000001100

\end{zadanie}

\begin{zadanie}


	141. Write a python program to convert decimal to hexadecimal.
	Sample decimal number: 30, 4
	Expected output: 1e, 04

\end{zadanie}

\begin{zadanie}


	142. Write a Python program to find the operating system name, platform and platform release date.
	Operating system name:
	posix
	Platform name:
	Linux
	Platform release:
	4.4.0-47-generic

\end{zadanie}

\begin{zadanie}


	143. Write a Python program to determine if the python shell is executing in 32bit or 64bit mode on operating system.

\end{zadanie}

\begin{zadanie}


	144. Write a Python program to check if variable is of integer or string.

\end{zadanie}

\begin{zadanie}


	145. Write a Python program to test if a variable is a list or tuple or a set.

\end{zadanie}

\begin{zadanie}


	146. Write a Python program to find the location of Python module sources.
	Operating system name:
	posix
	Platform name:
	Linux
	Platform release:
	4.4.0-47-generic

\end{zadanie}

\begin{zadanie}


	147. Write a Python function to check whether a number is divisible by another number. Accept two integers values form the user.

\end{zadanie}

\begin{zadanie}


	148. Write a Python function to find the maximum and minimum numbers from a sequence of numbers.
	Note: Do not use built-in functions.

\end{zadanie}

\begin{zadanie}


	149. Write a Python function that takes a positive integer and returns the sum of the cube of all the positive integers smaller than the specified number.

\end{zadanie}

\begin{zadanie}
	150. Write a Python function to find a distinct pair of numbers whose product is odd from a sequence of integer values.
\end{zadanie}

\section{Zadania 5}
\begin{zadanie}
	1. Write a Python function that takes a sequence of numbers and determines if all the numbers are different from each other.

\end{zadanie}

\begin{zadanie}


	2. Write a Python program to create all possible strings by using 'a', 'e', 'i', 'o', 'u'. Use the characters exactly once.

\end{zadanie}

\begin{zadanie}


	3. Write a Python program to remove and print every third number from a list of numbers until the list becomes empty.

\end{zadanie}

\begin{zadanie}


	4. Write a Python program to find unique triplets whose three elements gives the sum of zero from an array of n integers.

\end{zadanie}

\begin{zadanie}


	5. Write a Python program to create the combinations of 3 digit combo.

\end{zadanie}

\begin{zadanie}


	6. Write a Python program to print a long text, convert the string to a list and print all the words and their frequencies.

\end{zadanie}

\begin{zadanie}


	7. Write a Python program to count the number of each character of a given text of a text file.

\end{zadanie}

\begin{zadanie}


	8. Write a Python program to get the top stories from Google news.

\end{zadanie}

\begin{zadanie}


	9. Write a Python program to get a list of locally installed Python modules.

\end{zadanie}

\begin{zadanie}


	10. Write a Python program to display some information about the OS where the script is running.

\end{zadanie}

\begin{zadanie}


	11. Write a Python program to check the sum of three elements (each from an array) from three arrays is equal to a target value. Print all those three-element combinations.
	Sample data:
	/*
	X = [10, 20, 20, 20]
	Y = [10, 20, 30, 40]
	Z = [10, 30, 40, 20]
	target = 70
	*/

\end{zadanie}

\begin{zadanie}


	12. Write a Python program to create all possible permutations from a given collection of distinct numbers.

\end{zadanie}

\begin{zadanie}


	13. Write a Python program to get all possible two digit letter combinations from a digit (1 to 9) string.

	\begin{verbatim}
string_maps = {
"1": "abc",
"2": "def",
"3": "ghi",
"4": "jkl",
"5": "mno",
"6": "pqrs",
"7": "tuv",
"8": "wxy",
"9": "z"
}
\end{verbatim}

\end{zadanie}

\begin{zadanie}


	14. Write a Python program to add two positive integers without using the '+' operator.
	Note: Use bit wise operations to add two numbers.

\end{zadanie}

\begin{zadanie}


	15. Write a Python program to check the priority of the four operators (+, -, *, /).

\end{zadanie}

\begin{zadanie}


	16. Write a Python program to get the third side of right angled triangle from two given sides.

\end{zadanie}

\begin{zadanie}


	17. Write a Python program to get all strobogrammatic numbers that are of length n.
	A strobogrammatic number is a number whose numeral is rotationally symmetric, so that it appears the same when rotated 180 degrees. In other words, the numeral looks the same right-side up and upside down (e.g., 69, 96, 1001).
	For example,
	Given n = 2, return ["11", "69", "88", "96"].
	Given n = 3, return ['818', '111', '916', '619', '808', '101', '906', '609', '888', '181', '986', '689']

\end{zadanie}

\begin{zadanie}


	18. Write a Python program to find the median among three given numbers.

\end{zadanie}

\begin{zadanie}


	19. Write a Python program to find the value of n where n degrees of number 2 are written sequentially in a line without spaces.

\end{zadanie}

\begin{zadanie}


	20. Write a Python program to find the number of zeros at the end of a factorial of a given positive number.
	Range of the number(n): (1 = n = 2*109).

\end{zadanie}

\begin{zadanie}


	21. Write a Python program to find the number of notes (Sample of notes: 10, 20, 50, 100, 200 and 500 ) against an given amount.
	Range - Number of notes(n) : n (1 = n = 1000000).

\end{zadanie}

\begin{zadanie}


	22. Write a Python program to create a sequence where the first four members of the sequence are equal to one, and each successive term of the sequence is equal to the sum of the four previous ones. Find the Nth member of the sequence.

\end{zadanie}

\begin{zadanie}


	23. Write a Python program that accept a positive number and subtract from this number the sum of its digits and so on. Continues this operation until the number is positive.

\end{zadanie}

\begin{zadanie}


	24. Write a Python program to find the number of divisors of a given integer is even or odd.

\end{zadanie}

\begin{zadanie}


	25. Write a Python program to find the digits which are absent in a given mobile number.

\end{zadanie}

\begin{zadanie}


	26. Write a Python program to compute the summation of the absolute difference of all distinct pairs in an given array (non-decreasing order).
	Sample array: [1, 2, 3]
	Then all the distinct pairs will be:
	1 2
	1 3
	2 3

\end{zadanie}

\begin{zadanie}


	27. Write a Python program to find the type of the progression (arithmetic progression/geometric progression) and the next successive member of a given three successive members of a sequence.
	According to Wikipedia, an arithmetic progression (AP) is a sequence of numbers such that the difference of any two successive members of the sequence is a constant. For instance, the sequence 3, 5, 7, 9, 11, 13, . . . is an arithmetic progression with common difference 2. For this problem, we will limit ourselves to arithmetic progression whose common difference is a non-zero integer.
	On the other hand, a geometric progression (GP) is a sequence of numbers where each term after the first is found by multiplying the previous one by a fixed non-zero number called the common ratio. For example, the sequence 2, 6, 18, 54, . . . is a geometric progression with common ratio 3. For this problem, we will limit ourselves to geometric progression whose common ratio is a non-zero integer.

\end{zadanie}

\begin{zadanie}


	28. Write a Python program to print the length of the series and the series from the given 3rd term, 3rd last term and the sum of a series.
	Input data:
	3rd term - 3
	3rd last term - 118 55
	Sum of the series - 91

\end{zadanie}

\begin{zadanie}


	29. Write a Python program to find common divisors between two numbers in a given pair.

\end{zadanie}

\begin{zadanie}


	30. Write a Python program to reverse the digits of a given number and add it to the original, If the sum is not a palindrome repeat this procedure.
	Note: A palindrome is a word, number, or other sequence of characters which reads the same backward as forward, such as madam or racecar.

\end{zadanie}

\begin{zadanie}


	31. Write a Python program to count the number of carry operations for each of a set of addition problems.
	According to Wikipedia " In elementary arithmetic, a carry is a digit that is transferred from one column of digits to another column of more significant digits. It is part of the standard algorithm to add numbers together by starting with the rightmost digits and working to the left. For example, when 6 and 7 are added to make 13, the "3" is written to the same column and the "1" is carried to the left".

\end{zadanie}

\begin{zadanie}


	32. Write a python program to find heights of the top three building in descending order from eight given buildings.
	Input:
	0 = height of building (integer) = 10,000
	Input the heights of eight buildings:
	25
	35
	15
	16
	30
	45
	37
	39
	Heights of the top three buildings:
	45
	39
	37

\end{zadanie}

\begin{zadanie}


	33. Write a Python program to compute the digit number of sum of two given integers.
	Input:
	Each test case consists of two non-negative integers x and y which are separated by a space in a line.
	0 = x, y = 1,000,000
	Input two integers(a b):
	5 7
	Sum of two integers a and b.:
	2

\end{zadanie}

\begin{zadanie}


	34. Write a Python program to check whether three given lengths (integers) of three sides form a right triangle. Print "Yes" if the given sides form a right triangle otherwise print "No".
	Input:
	Integers separated by a single space.
	1 = length of the side = 1,000
	Input three integers(sides of a triangle)
	8 6 7
	No

\end{zadanie}

\begin{zadanie}


	35. Write a Python program which solve the equation:
	ax+by=c
	dx+ey=f
	Print the values of x, y where a, b, c, d, e and f are given.
	Input:
	a,b,c,d,e,f separated by a single space.
	(-1,000 = a,b,c,d,e,f = 1,000)
	Input the value of a, b, c, d, e, f:
	5 8 6 7 9 4
	Values of x and y:
	-2.000 2.000

\end{zadanie}

\begin{zadanie}


	36. Write a Python program to compute the amount of the debt in n months. The borrowing amount is $100,000 and the loan adds 5% interest of the debt and rounds it to the nearest 1,000 above month by month. 
		Input:
		An integer n (0 = n = 100)
		Input number of months:
		7 Amount of debt: $144000

\end{zadanie}

\begin{zadanie}


	37. Write a Python program which reads an integer n and find the number of combinations of a,b,c and d (0 = a,b,c,d = 9) where (a + b + c + d) will be equal to n.
	Input:
	n (1 = n = 50)
	Input the number(n):
	15
	Number of combinations: 592

\end{zadanie}

\begin{zadanie}


	38. Write a Python program to print the number of prime numbers which are less than or equal to an given integer.
	Input:
	n (1 = n = 999,999)
	Input the number(n):
	35
	Number of prime numbers which are less than or equal to n.: 11

\end{zadanie}

\begin{zadanie}


	39. Write a program to compute the radius and the central coordinate (x, y) of a circle which is constructed by three given points on the plane surface.
	Input:
	x1, y1, x2, y2, x3, y3 separated by a single space.
	Input three coordinate of the circle:
	9 3 6 8 3 6
	Radius of the said circle:
	3.358
	Central coordinate (x, y) of the circle:
	6.071 4.643

\end{zadanie}

\begin{zadanie}


	40. Write a Python program to check if a point (x,y) is in a triangle or not. There is a triangle formed by three points.
	Input:
	x1,y1,x2,y2,x3,y3,xp,yp separated by a single space.
	Input three coordinate of the circle:
	9 3 6 8 3 6
	Radius of the said circle:
	3.358
	Central coordinate (x, y) of the circle:
	6.071 4.643

\end{zadanie}

\begin{zadanie}


	41. Write a Python program to compute and print sum of two given integers (more than or equal to zero). If given integers or the sum have more than 80 digits, print "overflow".
	Input first integer:
	25
	Input second integer:
	22
	Sum of the two integers: 47

\end{zadanie}

\begin{zadanie}


	42. Write a Python program that accepts six numbers as input and sorts them in descending order.
	Input:
	Input consists of six numbers n1, n2, n3, n4, n5, n6 (-100000 = n1, n2, n3, n4, n5, n6 = 100000). The six numbers are separated by a space.
	Input six integers:
	15 30 25 14 35 40
	After sorting the said integers:
	40 35 30 25 15 14

\end{zadanie}

\begin{zadanie}


	43. Write a Python program to test whether two lines PQ and RS are parallel. The four points are P(x1, y1), Q(x2, y2), R(x3, y3), S(x4, y4).
	Input:
	x1,y1,x2,y2,x3,y3,xp,yp separated by a single space
	Input x1,y1,x2,y2,x3,y3,xp,yp:
	2 5 6 4 8 3 9 7
	PQ and RS are not parallel

\end{zadanie}

\begin{zadanie}


	44. Write a Python program to find the maximum sum of a contiguous subsequence from a given sequence of numbers a1, a2, a3, ... an. A subsequence of one element is also a continuous subsequence.
	Input:
	You can assume that 1 = n = 5000 and -100000 = ai = 100000.
	Input numbers are separated by a space.
	Input 0 to exit.
	Input number of sequence of numbers you want to input (0 to exit):
	3
	Input numbers:
	2
	4
	6
	Maximum sum of the said contiguous subsequence:
	12 Input number of sequence of numbers you want to input (0 to exit):
	0

\end{zadanie}

\begin{zadanie}


	45. There are two circles C1 with radius r1, central coordinate (x1, y1) and C2 with radius r2 and central coordinate (x2, y2).

	Write a Python program to test the followings -

	"C2 is in C1" if C2 is in C1
	"C1 is in C2" if C1 is in C2
	"Circumference of C1 and C2 intersect" if circumference of C1 and C2 intersect, and
	"C1 and C2 do not overlap" if C1 and C2 do not overlap.

	Input:
	Input numbers (real numbers) are separated by a space.
	Input x1, y1, r1, x2, y2, r2:
	5 6 4 8 7 9
	C1 is in C2

\end{zadanie}

\begin{zadanie}


	46. Write a Python program to that reads a date (from 2016/1/1 to 2016/12/31) and prints the day of the date. Jan. 1, 2016, is Friday. Note that 2016 is a leap year.
	Input:
	Two integers m and d separated by a single space in a line, m ,d represent the month and the day.
	Input month and date (separated by a single space):
	5 15
	Name of the date: Sunday

\end{zadanie}

\begin{zadanie}


	47. Write a Python program which reads a text (only alphabetical characters and spaces.) and prints two words. The first one is the word which is arise most frequently in the text. The second one is the word which has the maximum number of letters.

	Note: A word is a sequence of letters which is separated by the spaces.
	Input:
	A text is given in a line with following condition:
	a. The number of letters in the text is less than or equal to 1000.
	b. The number of letters in a word is less than or equal to 32.
	c. There is only one word which is arise most frequently in given text.
	d. There is only one word which has the maximum number of letters in given text.
	Input text: Thank you for your comment and your participation.
	Output: your participation.

\end{zadanie}

\begin{zadanie}


	48. Write a Python program that reads n digits (given) chosen from 0 to 9 and prints the number of combinations where the sum of the digits equals to another given number (s). Do not use the same digits in a combination.
	Input:
	Two integers as number of combinations and their sum by a single space in a line. Input 0 0 to exit.
	Input number of combinations and sum, input 0 0 to exit:
	5 6
	2 4
	0 0
	2

\end{zadanie}

\begin{zadanie}


	49. Write a Python program which reads the two adjoined sides and the diagonal of a parallelogram and check whether the parallelogram is a rectangle or a rhombus.
	According to Wikipedia-
	parallelograms: In Euclidean geometry, a parallelogram is a simple (non-self-intersecting) quadrilateral with two pairs of parallel sides. The opposite or facing sides of a parallelogram are of equal length and the opposite angles of a parallelogram are of equal measure.
	rectangles: In Euclidean plane geometry, a rectangle is a quadrilateral with four right angles. It can also be defined as an equiangular quadrilateral, since equiangular means that all of its angles are equal
	$(360^\circ/4 = 90^\circ)$.

	It can also be defined as a parallelogram containing a right angle.
	rhombus: In plane Euclidean geometry, a rhombus (plural rhombi or rhombuses) is a simple (non-self-intersecting) quadrilateral whose four sides all have the same length. Another name is equilateral quadrilateral, since equilateral means that all of its sides are equal in length. The rhombus is often called a diamond, after the diamonds suit in playing cards which resembles the projection of an octahedral diamond, or a lozenge, though the former sometimes refers specifically to a rhombus with a $60^\circ$ angle, and the latter sometimes refers specifically to a rhombus with a $45^\circ$ angle.
	Input:
	Two adjoined sides and the diagonal.
	1 = ai, bi, ci = 1000, ai + bi > ci
	Input two adjoined sides and the diagonal of a parallelogram (comma separated):
	3,4,5
	This is a rectangle.

\end{zadanie}

\begin{zadanie}


	50. Write a Python program to replace a string "Python" with "Java" and "Java" with "Python" in a given string.
	Input:
	English letters (including single byte alphanumeric characters, blanks, symbols) are given on one line. The length of the input character string is 1000 or less.
	Input a text with two words 'Python' and 'Java'
	Python is popular than Java
	Java is popular than Python

\end{zadanie}

\begin{zadanie}


	51. Write a Python program to find the difference between the largest integer and the smallest integer which are created by 8 numbers from 0 to 9. The number that can be rearranged shall start with 0 as in 00135668.
	Input:
	Input an integer created by 8 numbers from 0 to 9.:
	2345
	Difference between the largest and the smallest integer from the given integer:
	3087

\end{zadanie}

\begin{zadanie}


	52. Write a Python program to compute the sum of first n given prime numbers.
	Input:
	n ( n = 10000). Input 0 to exit the program.
	Input a number (n=10000) to compute the sum:(0 to exit)
	25
	Sum of first 25 prime numbers:
	1060

\end{zadanie}

\begin{zadanie}


	53. Write a Python program that accept a even number (>=4, Goldbach number) from the user and create a combinations that express the given number as a sum of two prime numbers. Print the number of combinations.
	Goldbach number: A Goldbach number is a positive even integer that can be expressed as the sum of two odd primes.[4] Since four is the only even number greater than two that requires the even prime 2 in order to be written as the sum of two primes, another form of the statement of Goldbach's conjecture is that all even integers greater than 4 are Goldbach numbers.
	The expression of a given even number as a sum of two primes is called a Goldbach partition of that number. The following are examples of Goldbach partitions for some even numbers:
	6 = 3 + 3
	8 = 3 + 5
	10 = 3 + 7 = 5 + 5
	12 = 7 + 5
	...
	100 = 3 + 97 = 11 + 89 = 17 + 83 = 29 + 71 = 41 + 59 = 47 + 53
	Input an even number (0 to exit):
	100
	Number of combinations:
	6

\end{zadanie}

\begin{zadanie}


	54. if you draw a straight line on a plane, the plane is divided into two regions. For example, if you pull two straight lines in parallel, you get three areas, and if you draw vertically one to the other you get 4 areas.
	Write a Python program to create maximum number of regions obtained by drawing n given straight lines.
	Input:
	(1 = n = 10,000)
	Input number of straight lines (o to exit):
	5
	Number of regions:
	16

\end{zadanie}

\begin{zadanie}


	55. There are four different points on a plane, P(xp,yp), Q(xq, yq), R(xr, yr) and S(xs, ys). Write a Python program to test AB and CD are orthogonal or not.
	Input:
	xp,yp, xq, yq, xr, yr, xs and ys are -100 to 100 respectively and each value can be up to 5 digits after the decimal point It is given as a real number including the number of. Output:
	Output AB and CD are not orthogonal! or AB and CD are orthogonal!.

\end{zadanie}

\begin{zadanie}


	56. Write a Python program to sum of all numerical values (positive integers) embedded in a sentence.
	Write a Python program to create maximum number of regions obtained by drawing n given straight lines.
	Input:
	Sentences with positive integers are given over multiple lines. Each line is a character string containing one-byte alphanumeric characters, symbols, spaces, or an empty line. However the input is 80 characters or less per line and the sum is 10,000 or less.
	Input some text and numeric values ( to exit):
	Sum of the numeric values: 80
	None
	Input some text and numeric values ( to exit):
	Sum of the numeric values: 17
	None
	Input some text and numeric values ( to exit):
	Sum of the numeric values: 10
	None

\end{zadanie}

\begin{zadanie}


	57. There are 10 vertical and horizontal squares on a plane. Each square is painted blue and green. Blue represents the sea, and green represents the land. When two green squares are in contact with the top and bottom, or right and left, they are said to be ground. The area created by only one green square is called "island". For example, there are five islands in the figure below.
	Write a Python program to read the mass data and find the number of islands.
	Input:
	Input 10 rows of 10 numbers representing green squares (island) as 1 and blue squares (sea) as zeros
	1100000111
	1000000111
	0000000111
	0010001000
	0000011100
	0000111110
	0001111111
	1000111110
	1100011100
	1110001000
	Number of islands:
	5

\end{zadanie}

\begin{zadanie}


	58. When character are consecutive in a string , it is possible to shorten the character string by replacing the character with a certain rule. For example, in the case of the character string YYYYY, if it is expressed as \# 5 Y, it is compressed by one character.
	Write a Python program to restore the original string by entering the compressed string with this rule. However, the \# character does not appear in the restored character string.
	Note: The original sentences are uppercase letters, lowercase letters, numbers, symbols, less than 100 letters, and consecutive letters are not more than 9 letters.
	Input:
	The restored character string for each character on one line.
	\begin{verbatim}
Original text: XY#6Z1#4023
XYZZZZZZ1000023
Original text: #39+1=1#30
999+1=1000
\end{verbatim}
\end{zadanie}

\begin{zadanie}


	59. A convex polygon is a simple polygon in which no line segment between two points on the boundary ever goes outside the polygon. Equivalently, it is a simple polygon whose interior is a convex set. In a convex polygon, all interior angles are less than or equal to 180 degrees, while in a strictly convex polygon all interior angles are strictly less than 180 degrees.
	Write a Python program that compute the area of the polygon . The vertices have the names vertex 1, vertex 2, vertex 3, ... vertex n according to the order of edge connections
	Note: The original sentences are uppercase letters, lowercase letters, numbers, symbols, less than 100 letters, and consecutive letters are not more than 9 letters.
	Input:
	Input is given in the following format.
	x1 , y1
	x2 , y2
	:
	xn , yn
	xi , yi are real numbers representing the x and y coordinates of vertex i , respectively.
	Input the coordinates (ctrl+d to exit):
	1.0, 0.0
	0.0, 0.0
	1.0, 1.0
	2.0, 0.0
	-1.0, 1.0
	Area of the polygon;
	1.50000000.

\end{zadanie}

\begin{zadanie}


	60. Internet search engine giant, such as Google accepts web pages around the world and classify them, creating a huge database. The search engines also analyze the search keywords entered by the user and create inquiries for database search. In both cases, complicated processing is carried out in order to realize efficient retrieval, but basics are all cutting out words from sentences.
	Write a Python program to cut out words of 3 to 6 characters length from a given sentence not more than 1024 characters.
	Input:
	English sentences consisting of delimiters and alphanumeric characters are given on one line.
	Input a sentence (1024 characters. max.)
	The quick brown fox
	3 to 6 characters length of words:
	The quick brown fox

\end{zadanie}

\begin{zadanie}


	61. Arrange integers (0 to 99) as narrow hilltop, as illustrated in Figure 1. Reading such data representing huge, when starting from the top and proceeding according to the next rule to the bottom. Write a Python program that compute the maximum value of the sum of the passing integers.
	Input:
	A series of integers separated by commas are given in diamonds. No spaces are included in each line. The input example corresponds to Figure 1. The number of lines of data is less than 100 lines.
	Output:
	The maximum value of the sum of integers passing according to the rule on one line.
	Input the numbers (ctrl+d to exit):
	8
	4, 9
	9, 2, 1
	3, 8, 5, 5
	5, 6, 3, 7, 6
	3, 8, 5, 5
	9, 2, 1
	4, 9
	8
	Maximum value of the sum of integers passing according to the rule on one line.
	64

\end{zadanie}

\begin{zadanie}


	62. Write a Python program to find the number of combinations that satisfy p + q + r + s = n where n is a given number <= 4000 and p, q, r, s in the range of 0 to 1000.
	Input a positive integer: (ctrl+d to exit)
	252
	Number of combinations of a,b,c,d: 2731135

\end{zadanie}

\begin{zadanie}


	63. Write a Python program which adds up columns and rows of given table as shown in the specified figure.
	Input number of rows/columns (0 to exit)
	4
	Input cell value:
	25 69 51 26
	68 35 29 54
	54 57 45 63
	61 68 47 59
	Result:
	25 69 51 26 171
	68 35 29 54 186
	54 57 45 63 219
	61 68 47 59 235
	208 229 172 202 811
	Input number of rows/columns (0 to exit)

\end{zadanie}

\section{Zadania 6}
\begin{zadanie}
	1. Write a Python program to calculate the length of a string.

\end{zadanie}

\begin{zadanie}


	2. Write a Python program to count the number of characters (character frequency) in a string.
	Sample String : google.com'
	Expected Result :
	\begin{verbatim}
{'o': 3, 'g': 2, '.': 1, 'e': 1, 'l': 1, 'm': 1, 'c': 1}
\end{verbatim}
\end{zadanie}

\begin{zadanie}


	3. Write a Python program to get a string made of the first 2 and the last 2 chars from a given a string. If the string length is less than 2, return instead of the empty string.
	Sample String : 'w3resource'
	Expected Result : 'w3ce'
	Sample String : 'w3'
	Expected Result : 'w3w3'
	Sample String : ' w'
	Expected Result : Empty String

\end{zadanie}

\begin{zadanie}


	4. Write a Python program to get a string from a given string where all occurrences of its first char have been changed to '$', except the first char itself.
		Sample String : 'restart'
		Expected Result : 'resta$t'

\end{zadanie}

\begin{zadanie}


	5. Write a Python program to get a single string from two given strings, separated by a space and swap the first two characters of each string.
	Sample String : 'abc', 'xyz'
	Expected Result : 'xyc abz'

\end{zadanie}

\begin{zadanie}


	6. Write a Python program to add 'ing' at the end of a given string (length should be at least 3). If the given string already ends with 'ing' then add 'ly' instead. If the string length of the given string is less than 3, leave it unchanged.
	Sample String : 'abc'
	Expected Result : 'abcing'
	Sample String : 'string'
	Expected Result : 'stringly'

\end{zadanie}

\begin{zadanie}


	7. Write a Python program to find the first appearance of the substring 'not' and 'poor' from a given string, if 'not' follows the 'poor', replace the whole 'not'...'poor' substring with 'good'. Return the resulting string.
	Sample String : 'The lyrics is not that poor!'
	'The lyrics is poor!'
	Expected Result : 'The lyrics is good!'
	'The lyrics is poor!'

\end{zadanie}

\begin{zadanie}


	8. Write a Python function that takes a list of words and returns the length of the longest one.

\end{zadanie}

\begin{zadanie}


	9. Write a Python program to remove the nth index character from a nonempty string.

\end{zadanie}

\begin{zadanie}


	10. Write a Python program to change a given string to a new string where the first and last chars have been exchanged.

\end{zadanie}

\begin{zadanie}


	11. Write a Python program to remove the characters which have odd index values of a given string.

\end{zadanie}

\begin{zadanie}


	12. Write a Python program to count the occurrences of each word in a given sentence.

\end{zadanie}

\begin{zadanie}


	13. Write a Python script that takes input from the user and displays that input back in upper and lower cases.

\end{zadanie}

\begin{zadanie}


	14. Write a Python program that accepts a comma separated sequence of words as input and prints the unique words in sorted form (alphanumerically).
	Sample Words : red, white, black, red, green, black
	Expected Result : black, green, red, white,red

\end{zadanie}

\begin{zadanie}


	15. Write a Python function to create the HTML string with tags around the word(s).
	Sample function and result :
	\begin{verbatim}
add_tags('i', 'Python') -> '<i>Python</i>'
add_tags('b', 'Python Tutorial') -> '<b>Python Tutorial </b>'
\end{verbatim}
\end{zadanie}

\begin{zadanie}


	16. Write a Python function to insert a string in the middle of a string.
	Sample function and result :
	\begin{verbatim}
insert_sting_middle('[[]]<<>>', 'Python') -> [[Python]]
insert_sting_middle('{{}}', 'PHP') -> {{PHP}}
\end{verbatim}
\end{zadanie}

\begin{zadanie}


	17. Write a Python function to get a string made of 4 copies of the last two characters of a specified string (length must be at least 2).
	Sample function and result :
	\begin{verbatim}
insert_end('Python') -> onononon
insert_end('Exercises') -> eseseses
\end{verbatim}
\end{zadanie}

\begin{zadanie}


	18. Write a Python function to get a string made of its first three characters of a specified string. If the length of the string is less than 3 then return the original string.
	Sample function and result :
	\begin{verbatim}
first_three('ipy') -> ipy
first_three('python') -> pyt
\end{verbatim}
\end{zadanie}

\begin{zadanie}


	19. Write a Python program to get the last part of a string before a specified character.
\end{zadanie}

\begin{zadanie}


	20. Write a Python function to reverses a string if it's length is a multiple of 4.
\end{zadanie}

\begin{zadanie}


	21. Write a Python function to convert a given string to all uppercase if it contains at least 2 uppercase characters in the first 4 characters.

\end{zadanie}

\begin{zadanie}


	22.Write a Python program to sort a string lexicographically.

\end{zadanie}

\begin{zadanie}


	23. Write a Python program to remove a newline in Python.

\end{zadanie}

\begin{zadanie}


	24. Write a Python program to check whether a string starts with specified characters.

\end{zadanie}

\begin{zadanie}


	25. Write a Python program to create a Caesar encryption.

	Note : In cryptography, a Caesar cipher, also known as Caesar's cipher, the shift cipher, Caesar's code or Caesar shift, is one of the simplest and most widely known encryption techniques. It is a type of substitution cipher in which each letter in the plaintext is replaced by a letter some fixed number of positions down the alphabet. For example, with a left shift of 3, D would be replaced by A, E would become B, and so on. The method is named after Julius Caesar, who used it in his private correspondence.


\end{zadanie}

\begin{zadanie}


	26. Write a Python program to display formatted text (width=50) as output.

\end{zadanie}

\begin{zadanie}


	27. Write a Python program to remove existing indentation from all of the lines in a given text.

\end{zadanie}

\begin{zadanie}


	28. Write a Python program to add a prefix text to all of the lines in a string.

\end{zadanie}

\begin{zadanie}


	29. Write a Python program to set the indentation of the first line.

\end{zadanie}

\begin{zadanie}


	30. Write a Python program to print the following floating numbers upto 2 decimal places.

\end{zadanie}

\begin{zadanie}


	31. Write a Python program to print the following floating numbers upto 2 decimal places with a sign.

\end{zadanie}

\begin{zadanie}


	32. Write a Python program to print the following floating numbers with no decimal places.

\end{zadanie}

\begin{zadanie}


	33. Write a Python program to print the following integers with zeros on the left of specified width.

\end{zadanie}

\begin{zadanie}


	34. Write a Python program to print the following integers with '*' on the right of specified width.

\end{zadanie}

\begin{zadanie}


	35. Write a Python program to display a number with a comma separator.

\end{zadanie}

\begin{zadanie}


	36. Write a Python program to format a number with a percentage.

\end{zadanie}

\begin{zadanie}


	37. Write a Python program to display a number in left, right and center aligned of width 10.

\end{zadanie}

\begin{zadanie}


	38. Write a Python program to count occurrences of a substring in a string.

\end{zadanie}

\begin{zadanie}


	39. Write a Python program to reverse a string.

\end{zadanie}

\begin{zadanie}


	40. Write a Python program to reverse words in a string.

\end{zadanie}

\begin{zadanie}


	41. Write a Python program to strip a set of characters from a string.

\end{zadanie}

\begin{zadanie}


	42. Write a Python program to count repeated characters in a string.
	Sample string: 'thequickbrownfoxjumpsoverthelazydog'
	Expected output :
	o 4
	e 3
	u 2
	h 2
	r 2
	t 2

\end{zadanie}

\begin{zadanie}


	43. Write a Python program to print the square and cube symbol in the area of a rectangle and volume of a cylinder.
	Sample output:
	The area of the rectangle is 1256.66cm2
	The volume of the cylinder is 1254.725cm3

\end{zadanie}

\begin{zadanie}


	44. Write a Python program to print the index of the character in a string.
	Sample string: w3resource
	Expected output:
	Current character w position at 0
	Current character 3 position at 1
	Current character r position at 2
	- - - - - - - - - - - - - - - - - - - - - - - - -
	Current character c position at 8
	Current character e position at 9

\end{zadanie}

\begin{zadanie}


	45. Write a Python program to check if a string contains all letters of the alphabet.

\end{zadanie}

\begin{zadanie}


	46. Write a Python program to convert a string in a list.

\end{zadanie}

\begin{zadanie}


	47. Write a Python program to lowercase first n characters in a string.

\end{zadanie}

\begin{zadanie}


	48. Write a Python program to swap comma and dot in a string.
	Sample string: "32.054,23"
	Expected Output: "32,054.23"

\end{zadanie}

\begin{zadanie}


	49. Write a Python program to count and display the vowels of a given text.

\end{zadanie}

\begin{zadanie}


	50. Write a Python program to split a string on the last occurrence of the delimiter.

\end{zadanie}

\begin{zadanie}


	51. Write a Python program to find the first non-repeating character in given string.

\end{zadanie}

\begin{zadanie}


	52. Write a Python program to print all permutations with given repetition number of characters of a given string.

\end{zadanie}

\begin{zadanie}


	53. Write a Python program to find the first repeated character in a given string.

\end{zadanie}

\begin{zadanie}


	54. Write a Python program to find the first repeated character of a given string where the index of first occurrence is smallest.

\end{zadanie}

\begin{zadanie}


	55.Write a Python program to find the first repeated word in a given string.

\end{zadanie}

\begin{zadanie}


	56. Write a Python program to find the second most repeated word in a given string.

\end{zadanie}

\begin{zadanie}


	57.Write a Python program to remove spaces from a given string.

\end{zadanie}

\begin{zadanie}


	58. Write a Python program to move spaces to the front of a given string.

\end{zadanie}

\begin{zadanie}


	59. Write a Python program to find the maximum occurring character in a given string.

\end{zadanie}

\begin{zadanie}


	60. Write a Python program to capitalize first and last letters of each word of a given string.

\end{zadanie}

\begin{zadanie}


	61. Write a Python program to remove duplicate characters of a given string.

\end{zadanie}

\begin{zadanie}


	62. Write a Python program to compute sum of digits of a given string.

\end{zadanie}

\begin{zadanie}


	63. Write a Python program to remove leading zeros from an IP address.

\end{zadanie}

\begin{zadanie}


	64. Write a Python program to find maximum length of consecutive 0's in a given binary string.

\end{zadanie}

\begin{zadanie}


	65. Write a Python program to find all the common characters in lexicographical order from two given lower case strings. If there are no common letters print "No common characters".

\end{zadanie}

\begin{zadanie}


	66. Write a Python program to make two given strings (lower case, may or may not be of the same length) anagrams removing any characters from any of the strings.

\end{zadanie}

\begin{zadanie}


	67. Write a Python program to remove all consecutive duplicates of a given string.

\end{zadanie}

\begin{zadanie}


	68. Write a Python program to create two strings from a given string. Create the first string using those character which occurs only once and create the second string which consists of multi-time occurring characters in the said string.

\end{zadanie}

\begin{zadanie}


	69. Write a Python program to find the longest common sub-string from two given strings.

\end{zadanie}

\begin{zadanie}


	70. Write a Python program to create a string from two given strings concatenating uncommon characters of the said strings.

\end{zadanie}

\begin{zadanie}


	71. Write a Python program to move all spaces to the front of a given string in single traversal.

\end{zadanie}

\begin{zadanie}


	72. Write a Python program to remove all consecutive duplicates from a given string.

\end{zadanie}

\begin{zadanie}


	73. Write a Python program to count Uppercase, Lowercase, special character and numeric values in a given string.

\end{zadanie}

\begin{zadanie}


	74. Write a Python program to find the minimum window in a given string which will contain all the characters of another given string.
	Example 1
	Input : str1 = " PRWSOERIUSFK "
	str2 = " OSU "
	Output: Minimum window is "OERIUS"

\end{zadanie}

\begin{zadanie}


	75. Write a Python program to find smallest window that contains all characters of a given string.

\end{zadanie}

\begin{zadanie}


	76. Write a Python program to count number of substrings from a given string of lowercase alphabets with exactly k distinct (given) characters.

\end{zadanie}

\begin{zadanie}


	77. Write a Python program to count number of non-empty substrings of a given string.

\end{zadanie}

\begin{zadanie}


	78. Write a Python program to count characters at same position in a given string (lower and uppercase characters) as in English alphabet.

\end{zadanie}

\begin{zadanie}


	79. Write a Python program to find smallest and largest word in a given string.

\end{zadanie}

\begin{zadanie}


	80. Write a Python program to count number of substrings with same first and last characters of a given string.

\end{zadanie}



\section{Lists}
\begin{zadanie}
	1. Write a Python program to sum all the items in a list. Go to the editor
\end{zadanie}

\begin{zadanie}


	2. Write a Python program to multiplies all the items in a list. Go to the editor


\end{zadanie}

\begin{zadanie}


	3. Write a Python program to get the largest number from a list. Go to the editor


\end{zadanie}

\begin{zadanie}


	4. Write a Python program to get the smallest number from a list. Go to the editor


\end{zadanie}

\begin{zadanie}


	5. Write a Python program to count the number of strings where the string length is 2 or more and the first and last character are same from a given list of strings. Go to the editor
	Sample List : ['abc', 'xyz', 'aba', '1221']
	Expected Result : 2


\end{zadanie}

\begin{zadanie}


	6. Write a Python program to get a list, sorted in increasing order by the last element in each tuple from a given list of non-empty tuples. Go to the editor
	Sample List : [(2, 5), (1, 2), (4, 4), (2, 3), (2, 1)]
	Expected Result : [(2, 1), (1, 2), (2, 3), (4, 4), (2, 5)]


\end{zadanie}

\begin{zadanie}


	7. Write a Python program to remove duplicates from a list. Go to the editor


\end{zadanie}

\begin{zadanie}


	8. Write a Python program to check a list is empty or not. Go to the editor


\end{zadanie}

\begin{zadanie}


	9. Write a Python program to clone or copy a list. Go to the editor


\end{zadanie}

\begin{zadanie}


	10. Write a Python program to find the list of words that are longer than n from a given list of words. Go to the editor


\end{zadanie}

\begin{zadanie}


	11. Write a Python function that takes two lists and returns True if they have at least one common member. Go to the editor


\end{zadanie}

\begin{zadanie}


	12. Write a Python program to print a specified list after removing the 0th, 4th and 5th elements. Go to the editor
	Sample List : ['Red', 'Green', 'White', 'Black', 'Pink', 'Yellow']
	Expected Output : ['Green', 'White', 'Black']


\end{zadanie}

\begin{zadanie}


	13. Write a Python program to generate a 3*4*6 3D array whose each element is *. Go to the editor


\end{zadanie}

\begin{zadanie}


	14. Write a Python program to print the numbers of a specified list after removing even numbers from it. Go to the editor


\end{zadanie}

\begin{zadanie}


	15. Write a Python program to shuffle and print a specified list. Go to the editor


\end{zadanie}

\begin{zadanie}


	16. Write a Python program to generate and print a list of first and last 5 elements where the values are square of numbers between 1 and 30 (both included). Go to the editor


\end{zadanie}

\begin{zadanie}


	17. Write a Python program to generate and print a list except for the first 5 elements, where the values are square of numbers between 1 and 30 (both included). Go to the editor


\end{zadanie}

\begin{zadanie}


	18. Write a Python program to generate all permutations of a list in Python. Go to the editor


\end{zadanie}

\begin{zadanie}


	19. Write a Python program to get the difference between the two lists. Go to the editor


\end{zadanie}

\begin{zadanie}


	20. Write a Python program access the index of a list. Go to the editor


\end{zadanie}

\begin{zadanie}


	21. Write a Python program to convert a list of characters into a string. Go to the editor


\end{zadanie}

\begin{zadanie}


	22. Write a Python program to find the index of an item in a specified list. Go to the editor


\end{zadanie}

\begin{zadanie}


	23. Write a Python program to flatten a shallow list. Go to the editor


\end{zadanie}

\begin{zadanie}


	24. Write a Python program to append a list to the second list. Go to the editor


\end{zadanie}

\begin{zadanie}


	25. Write a Python program to select an item randomly from a list. Go to the editor


\end{zadanie}

\begin{zadanie}


	26. Write a python program to check whether two lists are circularly identical. Go to the editor


\end{zadanie}

\begin{zadanie}


	27. Write a Python program to find the second smallest number in a list. Go to the editor


\end{zadanie}

\begin{zadanie}


	28. Write a Python program to find the second largest number in a list. Go to the editor


\end{zadanie}

\begin{zadanie}


	29. Write a Python program to get unique values from a list. Go to the editor


\end{zadanie}

\begin{zadanie}


	30. Write a Python program to get the frequency of the elements in a list. Go to the editor


\end{zadanie}

\begin{zadanie}


	31. Write a Python program to count the number of elements in a list within a specified range. Go to the editor


\end{zadanie}

\begin{zadanie}


	32. Write a Python program to check whether a list contains a sublist. Go to the editor


\end{zadanie}

\begin{zadanie}


	33. Write a Python program to generate all sublists of a list. Go to the editor


\end{zadanie}

\begin{zadanie}


	34. Write a Python program using Sieve of Eratosthenes method for computing primes upto a specified number. Go to the editor
	Note: In mathematics, the sieve of Eratosthenes, one of a number of prime number sieves, is a simple, ancient algorithm for finding all prime numbers up to any given limit.


\end{zadanie}

\begin{zadanie}


	35. Write a Python program to create a list by concatenating a given list which range goes from 1 to n. Go to the editor
	Sample list : ['p', 'q']
	n =5
	Sample Output : ['p1', 'q1', 'p2', 'q2', 'p3', 'q3', 'p4', 'q4', 'p5', 'q5']


\end{zadanie}

\begin{zadanie}


	36. Write a Python program to get variable unique identification number or string. Go to the editor


\end{zadanie}

\begin{zadanie}


	37. Write a Python program to find common items from two lists. Go to the editor


\end{zadanie}

\begin{zadanie}


	38. Write a Python program to change the position of every n-th value with the (n+1)th in a list. Go to the editor
	Sample list: [0,1,2,3,4,5]
	Expected Output: [1, 0, 3, 2, 5, 4]


\end{zadanie}

\begin{zadanie}


	39. Write a Python program to convert a list of multiple integers into a single integer. Go to the editor
	Sample list: [11, 33, 50]
	Expected Output: 113350


\end{zadanie}

\begin{zadanie}


	40. Write a Python program to split a list based on first character of word. Go to the editor


\end{zadanie}

\begin{zadanie}


	41. Write a Python program to create multiple lists. Go to the editor


\end{zadanie}

\begin{zadanie}


	42. Write a Python program to find missing and additional values in two lists. Go to the editor
	Sample data : Missing values in second list: b,a,c
	Additional values in second list: g,h


\end{zadanie}

\begin{zadanie}


	43. Write a Python program to split a list into different variables. Go to the editor


\end{zadanie}

\begin{zadanie}


	44. Write a Python program to generate groups of five consecutive numbers in a list. Go to the editor


\end{zadanie}

\begin{zadanie}


	45. Write a Python program to convert a pair of values into a sorted unique array. Go to the editor


\end{zadanie}

\begin{zadanie}


	46. Write a Python program to select the odd items of a list. Go to the editor


\end{zadanie}

\begin{zadanie}


	47. Write a Python program to insert an element before each element of a list. Go to the editor


\end{zadanie}

\begin{zadanie}


	48. Write a Python program to print a nested lists (each list on a new line) using the print() function. Go to the editor


\end{zadanie}

\begin{zadanie}


	49. Write a Python program to convert list to list of dictionaries. Go to the editor
	\begin{verbatim}
Sample lists: ["Black", "Red", "Maroon", "Yellow"], ["#000000", "#FF0000", "#800000", "#FFFF00"]
Expected Output: [{'color_name': 'Black', 'color_code': '#000000'}, {'color_name': 'Red', 'color_code': '#FF0000'}, {'color_name': 'Maroon', 'color_code': '#800000'}, {'color_name': 'Yellow', 'color_code': '#FFFF00'}]
\end{verbatim}

\end{zadanie}

\begin{zadanie}


	50. Write a Python program to sort a list of nested dictionaries. Go to the editor


\end{zadanie}

\begin{zadanie}


	51. Write a Python program to split a list every Nth element. Go to the editor
	Sample list: ['a', 'b', 'c', 'd', 'e', 'f', 'g', 'h', 'i', 'j', 'k', 'l', 'm', 'n']
	Expected Output: [['a', 'd', 'g', 'j', 'm'], ['b', 'e', 'h', 'k', 'n'], ['c', 'f', 'i', 'l']]


\end{zadanie}

\begin{zadanie}


	52. Write a Python program to compute the similarity between two lists. Go to the editor
	Sample data: ["red", "orange", "green", "blue", "white"], ["black", "yellow", "green", "blue"]
	Expected Output:
	Color1-Color2: ['white', 'orange', 'red']
	Color2-Color1: ['black', 'yellow']


\end{zadanie}

\begin{zadanie}


	53. Write a Python program to create a list with infinite elements. Go to the editor


\end{zadanie}

\begin{zadanie}


	54. Write a Python program to concatenate elements of a list. Go to the editor


\end{zadanie}

\begin{zadanie}


	55. Write a Python program to remove key values pairs from a list of dictionaries. Go to the editor


\end{zadanie}

\begin{zadanie}


	56. Write a Python program to convert a string to a list. Go to the editor


\end{zadanie}

\begin{zadanie}


	57. Write a Python program to check if all items of a list is equal to a given string. Go to the editor


\end{zadanie}

\begin{zadanie}


	58. Write a Python program to replace the last element in a list with another list. Go to the editor
	Sample data : [1, 3, 5, 7, 9, 10], [2, 4, 6, 8]
	Expected Output: [1, 3, 5, 7, 9, 2, 4, 6, 8]


\end{zadanie}

\begin{zadanie}


	59. Write a Python program to check if the n-th element exists in a given list. Go to the editor


\end{zadanie}

\begin{zadanie}


	60. Write a Python program to find a tuple, the smallest second index value from a list of tuples. Go to the editor


\end{zadanie}

\begin{zadanie}


	61. Write a Python program to create a list of empty dictionaries. Go to the editor


\end{zadanie}

\begin{zadanie}


	62. Write a Python program to print a list of space-separated elements. Go to the editor


\end{zadanie}

\begin{zadanie}


	63. Write a Python program to insert a given string at the beginning of all items in a list. Go to the editor
	Sample list : [1,2,3,4], string : emp
	Expected output : ['emp1', 'emp2', 'emp3', 'emp4']


\end{zadanie}

\begin{zadanie}


	64. Write a Python program to iterate over two lists simultaneously. Go to the editor


\end{zadanie}

\begin{zadanie}


	65. Write a Python program to access dictionary keys element by index. Go to the editor


\end{zadanie}

\begin{zadanie}


	66. Write a Python program to find the list in a list of lists whose sum of elements is the highest. Go to the editor
	Sample lists: [1,2,3], [4,5,6], [10,11,12], [7,8,9]
	Expected Output: [10, 11, 12]


\end{zadanie}

\begin{zadanie}


	67. Write a Python program to find all the values in a list are greater than a specified number. Go to the editor


\end{zadanie}

\begin{zadanie}


	68. Write a Python program to extend a list without append. Go to the editor
	Sample data: [10, 20, 30]
	[40, 50, 60]
	Expected output : [40, 50, 60, 10, 20, 30]


\end{zadanie}

\begin{zadanie}


	69. Write a Python program to remove duplicates from a list of lists. Go to the editor
	Sample list : [[10, 20], [40], [30, 56, 25], [10, 20], [33], [40]]
	New List : [[10, 20], [30, 56, 25], [33], [40]]


\end{zadanie}

\begin{zadanie}


	70. Write a Python program to get the depth of a dictionary. Go to the editor


\end{zadanie}

\begin{zadanie}


	71. Write a Python program to check if all dictionaries in a list are empty or not. Go to the editor
	Sample list : [{},{},{}]
	Return value : True
	Sample list : [{1,2},{},{}]
	Return value : False


\end{zadanie}




\section{Zadanie 7}

\begin{zadanie}

	1. Write a Python program to find those numbers which are divisible by 7 and multiple of 5, between 1500 and 2700 (both included). Go to the editor


\end{zadanie}

\begin{zadanie}


	2. Write a Python program to convert temperatures to and from celsius, fahrenheit. Go to the editor
	[ Formula : c/5 = f-32/9 [ where c = temperature in celsius and f = temperature in fahrenheit ]
	Expected Output :
	$60^\circ C$ is 140 in Fahrenheit
	$45^\circ F$ is 7 in Celsius


\end{zadanie}

\begin{zadanie}


	3. Write a Python program to guess a number between 1 to 9. Go to the editor
	Note : User is prompted to enter a guess. If the user guesses wrong then the prompt appears again until the guess is correct, on successful guess, user will get a "Well guessed!" message, and the program will exit.


\end{zadanie}

\begin{zadanie}


	4. Write a Python program to construct the following pattern, using a nested for loop.

	*
	* *
	* * *
	* * * *
	* * * * *
	* * * *
	* * *
	* *
	*



\end{zadanie}

\begin{zadanie}


	5. Write a Python program that accepts a word from the user and reverse it. Go to the editor


\end{zadanie}

\begin{zadanie}


	6. Write a Python program to count the number of even and odd numbers from a series of numbers. Go to the editor
	Sample numbers : numbers = (1, 2, 3, 4, 5, 6, 7, 8, 9)
	Expected Output :
	Number of even numbers : 5
	Number of odd numbers : 4


\end{zadanie}

\begin{zadanie}


	7. Write a Python program that prints each item and its corresponding type from the following list.
	Sample List : datalist = [1452, 11.23, 1+2j, True, 'w3resource', (0, -1), [5, 12], {"class":'V', "section":'A'}]


\end{zadanie}

\begin{zadanie}


	8. Write a Python program that prints all the numbers from 0 to 6 except 3 and 6.
	Note : Use 'continue' statement.
	Expected Output : 0 1 2 4 5


\end{zadanie}

\begin{zadanie}


	9. Write a Python program to get the Fibonacci series between 0 to 50. Go to the editor
	Note : The Fibonacci Sequence is the series of numbers :
	0, 1, 1, 2, 3, 5, 8, 13, 21, ....
	Every next number is found by adding up the two numbers before it.
	Expected Output : 1 1 2 3 5 8 13 21 34


\end{zadanie}

\begin{zadanie}


	10. Write a Python program which iterates the integers from 1 to 50. For multiples of three print "Fizz" instead of the number and for the multiples of five print "Buzz". For numbers which are multiples of both three and five print "FizzBuzz".
	Sample Output :
	fizzbuzz
	1
	2
	fizz
	4
	buzz


\end{zadanie}

\begin{zadanie}


	11. Write a Python program which takes two digits m (row) and n (column) as input and generates a two-dimensional array. The element value in the i-th row and j-th column of the array should be i*j. Go to the editor
	Note :
	i = 0,1.., m-1
	j = 0,1, n-1.

	Test Data : Rows = 3, Columns = 4
	Expected Result : [[0, 0, 0, 0], [0, 1, 2, 3], [0, 2, 4, 6]]


\end{zadanie}

\begin{zadanie}


	12. Write a Python program that accepts a sequence of lines (blank line to terminate) as input and prints the lines as output (all characters in lower case). Go to the editor


\end{zadanie}

\begin{zadanie}


	13. Write a Python program which accepts a sequence of comma separated 4 digit binary numbers as its input and print the numbers that are divisible by 5 in a comma separated sequence. Go to the editor
	Sample Data : 0100,0011,1010,1001,1100,1001
	Expected Output : 1010


\end{zadanie}

\begin{zadanie}


	14. Write a Python program that accepts a string and calculate the number of digits and letters. Go to the editor
	Sample Data : Python 3.2
	Expected Output :
	Letters 6
	Digits 2



\end{zadanie}

\begin{zadanie}


	15. Write a Python program to check the validity of password input by users. Go to the editor
	Validation :

	At least 1 letter between [a-z] and 1 letter between [A-Z].
	At least 1 number between [0-9].
	At least 1 character from [\$\#@].
	Minimum length 6 characters.
	Maximum length 16 characters.



\end{zadanie}

\begin{zadanie}


	16. Write a Python program to find numbers between 100 and 400 (both included) where each digit of a number is an even number. The numbers obtained should be printed in a comma-separated sequence. Go to the editor


\end{zadanie}

\begin{zadanie}


	17. Write a Python program to print alphabet pattern 'A'. Go to the editor
	Expected Output:

	***
	*   *
	*   *
	*****
	*   *
	*   *
	*   *



\end{zadanie}

\begin{zadanie}


	18. Write a Python program to print alphabet pattern 'D'. Go to the editor
	Expected Output:

	****
	*   *
	*   *
	*   *
	*   *
	*   *
	****



\end{zadanie}

\begin{zadanie}


	19. Write a Python program to print alphabet pattern 'E'. Go to the editor
	Expected Output:

	*****
	*
	*
	****
	*
	*
	*****



\end{zadanie}

\begin{zadanie}


	20. Write a Python program to print alphabet pattern 'G'. Go to the editor
	Expected Output:

	***
	*   *
	*
	* ***
	*   *
	*   *
	***



\end{zadanie}

\begin{zadanie}


	21. Write a Python program to print alphabet pattern 'L'. Go to the editor
	Expected Output:

	*
	*
	*
	*
	*
	*
	*****



\end{zadanie}

\begin{zadanie}


	22. Write a Python program to print alphabet pattern 'M'. Go to the editor
	Expected Output:

	*       *
	*       *
	* *   * *
	*   *   *
	*       *
	*       *
	*       *



\end{zadanie}

\begin{zadanie}


	23. Write a Python program to print alphabet pattern 'O'. Go to the editor
	Expected Output:

	***
	*   *
	*   *
	*   *
	*   *
	*   *
	***



\end{zadanie}

\begin{zadanie}


	24. Write a Python program to print alphabet pattern 'P'. Go to the editor
	Expected Output:

	****
	*   *
	*   *
	****
	*
	*
	*



\end{zadanie}

\begin{zadanie}


	25. Write a Python program to print alphabet pattern 'R'. Go to the editor
	Expected Output:

	****
	*   *
	*   *
	****
	* *
	*  *
	*   *



\end{zadanie}

\begin{zadanie}


	26. Write a Python program to print the following patterns. Go to the editor
	Expected Output:

	****
	*
	*
	***
	*
	*
	****

	ooooooooooooooooo
	ooooooooooooooooo
	ooooooooooooooooo
	oooo
	oooo
	oooo
	ooooooooooooooooo
	ooooooooooooooooo
	ooooooooooooooooo
	oooo
	oooo
	oooo
	ooooooooooooooooo
	ooooooooooooooooo
	ooooooooooooooooo



\end{zadanie}

\begin{zadanie}

	27. Write a Python program to print alphabet pattern 'T'. Go to the editor
	Expected Output:

	*****
	*
	*
	*
	*
	*
	*



\end{zadanie}

\begin{zadanie}


	28. Write a Python program to print alphabet pattern 'U'. Go to the editor
	Expected Output:

	*   *
	*   *
	*   *
	*   *
	*   *
	*   *
	***



\end{zadanie}

\begin{zadanie}


	29. Write a Python program to print alphabet pattern 'X'. Go to the editor
	Expected Output:

	*   *
	*   *
	* *
	*
	* *
	*   *
	*   *



\end{zadanie}

\begin{zadanie}


	30. Write a Python program to print alphabet pattern 'Z'. Go to the editor
	Expected Output:

	*******
	*
	*
	*
	*
	*
	*******



\end{zadanie}

\begin{zadanie}


	31. Write a Python program to calculate a dog's age in dog's years. Go to the editor
	Note: For the first two years, a dog year is equal to 10.5 human years. After that, each dog year equals 4 human years.
	Expected Output:

	Input a dog's age in human years: 15
	The dog's age in dog's years is 73



\end{zadanie}

\begin{zadanie}


	32. Write a Python program to check whether an alphabet is a vowel or consonant. Go to the editor
	Expected Output:

	Input a letter of the alphabet: k
	k is a consonant.



\end{zadanie}

\begin{zadanie}


	33. Write a Python program to convert month name to a number of days. Go to the editor
	Expected Output:

	List of months: January, February, March, April, May, June, July, August
	, September, October, November, December
	Input the name of Month: February
	No. of days: 28/29 days



\end{zadanie}

\begin{zadanie}


	34. Write a Python program to sum of two given integers. However, if the sum is between 15 to 20 it will return 20. Go to the editor



\end{zadanie}

\begin{zadanie}


	35. Write a Python program to check a string represent an integer or not. Go to the editor
	Expected Output:

	Input a string: Python
	The string is not an integer.



\end{zadanie}

\begin{zadanie}


	36. Write a Python program to check a triangle is equilateral, isosceles or scalene. Go to the editor
	Note :
	An equilateral triangle is a triangle in which all three sides are equal.
	A scalene triangle is a triangle that has three unequal sides.
	An isosceles triangle is a triangle with (at least) two equal sides.
	Expected Output:

	Input lengths of the triangle sides:
	x: 6
	y: 8
	z: 12
	Scalene triangle



\end{zadanie}

\begin{zadanie}


	37. Write a Python program that reads two integers representing a month and day and prints the season for that month and day. Go to the editor
	Expected Output:

	Input the month (e.g. January, February etc.): july
	Input the day: 31
	Season is autumn



\end{zadanie}

\begin{zadanie}


	38. Write a Python program to display astrological sign for given date of birth. Go to the editor
	Expected Output:

	Input birthday: 15
	Input month of birth (e.g. march, july etc): may
	Your Astrological sign is : Taurus



\end{zadanie}

\begin{zadanie}


	39. Write a Python program to display the sign of the Chinese Zodiac for given year in which you were born. Go to the editor
	Expected Output:

	Input your birth year: 1973
	Your Zodiac sign : Ox



\end{zadanie}

\begin{zadanie}


	40. Write a Python program to find the median of three values. Go to the editor
	Expected Output:

	Input first number: 15
	Input second number: 26
	Input third number: 29
	The median is 26.0



\end{zadanie}

\begin{zadanie}


	41. Write a Python program to get next day of a given date. Go to the editor
	Expected Output:

	Input a year: 2016
	Input a month [1-12]: 08
	Input a day [1-31]: 23
	The next date is [yyyy-mm-dd] 2016-8-24



\end{zadanie}

\begin{zadanie}


	42. Write a Python program to calculate the sum and average of n integer numbers (input from the user). Input 0 to finish. Go to the editor



\end{zadanie}

\begin{zadanie}


	43. Write a Python program to create the multiplication table (from 1 to 10) of a number. Go to the editor
	Expected Output:

	Input a number: 6
	6 x 1 = 6
	6 x 2 = 12
	6 x 3 = 18
	6 x 4 = 24
	6 x 5 = 30
	6 x 6 = 36
	6 x 7 = 42
	6 x 8 = 48
	6 x 9 = 54
	6 x 10 = 60



\end{zadanie}

\begin{zadanie}


	44. Write a Python program to construct the following pattern, using a nested loop number. Go to the editor
	Expected Output:

	1
	22
	333
	4444
	55555
	666666
	7777777
	88888888
	999999999



\end{zadanie}

\section{Zadanie 7}

\begin{zadanie}


	1. Write a Python function to find the Max of three numbers. Go to the editor


\end{zadanie}

\begin{zadanie}


	2. Write a Python function to sum all the numbers in a list. Go to the editor
	Sample List : (8, 2, 3, 0, 7)
	Expected Output : 20


\end{zadanie}

\begin{zadanie}


	3. Write a Python function to multiply all the numbers in a list. Go to the editor
	Sample List : (8, 2, 3, -1, 7)
	Expected Output : -336


\end{zadanie}

\begin{zadanie}


	4. Write a Python program to reverse a string. Go to the editor
	Sample String : "1234abcd"
	Expected Output : "dcba4321"


\end{zadanie}

\begin{zadanie}


	5. Write a Python function to calculate the factorial of a number (a non-negative integer). The function accepts the number as an argument. Go to the editor


\end{zadanie}

\begin{zadanie}


	6. Write a Python function to check whether a number is in a given range. Go to the editor


\end{zadanie}

\begin{zadanie}


	7. Write a Python function that accepts a string and calculate the number of upper case letters and lower case letters. Go to the editor
	Sample String : 'The quick Brow Fox'
	Expected Output :
	No. of Upper case characters : 3
	No. of Lower case Characters : 12


\end{zadanie}

\begin{zadanie}


	8. Write a Python function that takes a list and returns a new list with unique elements of the first list. Go to the editor
	Sample List : [1,2,3,3,3,3,4,5]
	Unique List : [1, 2, 3, 4, 5]


\end{zadanie}

\begin{zadanie}


	9. Write a Python function that takes a number as a parameter and check the number is prime or not. Go to the editor
	Note : A prime number (or a prime) is a natural number greater than 1 and that has no positive divisors other than 1 and itself.


\end{zadanie}

\begin{zadanie}


	10. Write a Python program to print the even numbers from a given list. Go to the editor
	Sample List : [1, 2, 3, 4, 5, 6, 7, 8, 9]
	Expected Result : [2, 4, 6, 8]


\end{zadanie}

\begin{zadanie}


	11. Write a Python function to check whether a number is perfect or not. Go to the editor
	According to Wikipedia : In number theory, a perfect number is a positive integer that is equal to the sum of its proper positive divisors, that is, the sum of its positive divisors excluding the number itself (also known as its aliquot sum). Equivalently, a perfect number is a number that is half the sum of all of its positive divisors (including itself).
	Example : The first perfect number is 6, because 1, 2, and 3 are its proper positive divisors, and 1 + 2 + 3 = 6. Equivalently, the number 6 is equal to half the sum of all its positive divisors: ( 1 + 2 + 3 + 6 ) / 2 = 6. The next perfect number is 28 = 1 + 2 + 4 + 7 + 14. This is followed by the perfect numbers 496 and 8128.


\end{zadanie}

\begin{zadanie}


	12. Write a Python function that checks whether a passed string is palindrome or not. Go to the editor
	Note: A palindrome is a word, phrase, or sequence that reads the same backward as forward, e.g., madam or nurses run.


\end{zadanie}

\begin{zadanie}


	13. Write a Python function that prints out the first n rows of Pascal's triangle. Go to the editor
	Note : Pascal's triangle is an arithmetic and geometric figure first imagined by Blaise Pascal.

	Sample Pascal's triangle :
	Pascal's triangle

	Each number is the two numbers above it added together Go to the editor


\end{zadanie}

\begin{zadanie}


	14. Write a Python function to check whether a string is a pangram or not. Go to the editor
	Note : Pangrams are words or sentences containing every letter of the alphabet at least once.
	For example : "The quick brown fox jumps over the lazy dog"


\end{zadanie}

\begin{zadanie}


	15. Write a Python program that accepts a hyphen-separated sequence of words as input and prints the words in a hyphen-separated sequence after sorting them alphabetically. Go to the editor
	Sample Items : green-red-yellow-black-white
	Expected Result : black-green-red-white-yellow


\end{zadanie}

\begin{zadanie}


	16. Write a Python function to create and print a list where the values are square of numbers between 1 and 30 (both included). Go to the editor


\end{zadanie}

\begin{zadanie}


	17. Write a Python program to make a chain of function decorators (bold, italic, underline etc.) in Python. Go to the editor


\end{zadanie}

\begin{zadanie}


	18. Write a Python program to execute a string containing Python code. Go to the editor


\end{zadanie}

\begin{zadanie}


	19. Write a Python program to access a function inside a function. Go to the editor


\end{zadanie}

\begin{zadanie}


	20. Write a Python program to detect the number of local variables declared in a function. Go to the editor


\end{zadanie}
%
%Ćwiczenie 1.
%Wiedząc, że pierwiastek n-tego stopnia z x równa się x do potęgi 1/n i wykorzystując wiedzę o użyciu liczb zespolonych w Pythonie, wylicz wartość pierwiastka drugiego stopnia z –7.
%
%
%Ćwiczenie 2.
%Używając instrukcji Pythona oblicz resztę z dzielenia 11 przez 7 i zapamiętaj wynik w zmiennej o nazwie Z. Następnie, pojedynczym poleceniem Pythona i bez użycia nawiasów, przemnóż zmienną Z przez Z+1.
%
%
%Ćwiczenie 3.
%Spowoduj pojedynczym poleceniem Pythona, by na ekranie 200-krotnie wyświetliła się wartość wyrażenia 1.23e+4+45.0 każdorazowo rozdzielona średnikiem.
%
%
%Ćwiczenie 4.
%Napisz skrypt, który zsumuje dwie liczby, nie używając znaku "+" (w całym skrypcie).
%
%
%Ćwiczenie 5.
%Napisz program "powiel.py", który wczyta od użytkownika pewien napis, a następnie wyświetli 50 kopii tego napisu, każda w osobnej linii.
%
%
%Ćwiczenie 6.
%Napisz program "pole_tr.py", który obliczy pole trójkąta, pod warunkiem że użytkownik poda wysokość i długość podstawy tego trójkąta.
%
%
%Ćwiczenie 7.
%Napisz program "odsetki.py", który obliczy stan konta za N lat, gdzie stan początkowy konta wynosi SPK, a stopa oprocentowania P % rocznie (obowiązuje roczna kapitalizacja odsetek). N, SPK i P podaje użytkownik programu.
%
%
%Ćwiczenie 8.
%Napisz program "parzyste.py", który wczyta od użytkownika liczbę całkowitą i wyświetli informację, czy jest to liczba parzysta, czy nieparzysta.
%
%
%Ćwiczenie 9.
%Napisz program "prk.py", który obliczy wszystkie pierwiastki rzeczywiste równania kwadratowego o postaci ax2+bx+c=0, gdzie a, b i c podaje użytkownik. Program powinien na początku sprawdzić, czy wprowadzone równanie jest rzeczywiście kwadratowe.
%
%
%Ćwiczenie 10.
%Napisz program "parzyste2.py", który wczyta od użytkownika liczbę całkowitą i bez użycia instrukcji "if" wyświetli informację, czy jest to liczba parzysta, czy nieparzysta.
%
%
%Ćwiczenie 11.
%Napisz program "numer.py", który zamieni wprowadzony przez użytkownika ciąg cyfr na formę tekstową:
%a) znaki nie będące cyframi mają być ignorowane
%b) konwertujemy cyfry, nie liczby, a zatem:
%- 911 to "dziewięć jeden jeden"
%- 1100 to "jeden jeden zero zero"
%
%
%Ćwiczenie 12.
%Woda zamarza przy 32 stopniach Fahrenheita, a wrze przy 212 stopniach Fahrenheita. Napisz program "stopnie.py", który wyświetli tabelę przeliczeń stopni Celsjusza na stopnie Fahrenheita w zakresie od –20 do +40 stopni Celsjusza (co 5 stopni). Pamiętaj o wyświetlaniu znaku plus/minus przy temperaturze.
%
%
%Ćwiczenie 13.
%Napisz program "oceny.py", który wczytuje od użytkownika kolejne oceny i:
%a) sprawdza czy wprowadzona ocena jest na liście dopuszczalnych na wydziale ocen (jeżeli ocena jest na liście dopuszczalnych na wydziale ocen, dodaje ją na listę otrzymanych ocen)
%b) jeżeli wciśnięto sam Enter, oznacza to koniec listy otrzymanych ocen
%c) wyświetla wyliczoną dla listy otrzymanych ocen średnią arytmetyczną.
%
%
%Ćwiczenie 14.
%Napisz program "tryg.py", który wczyta od użytkownika wielkość kąta w stopniach i wyświetli wartość czterech podstawowych funkcji trygonometrycznych (sin, cos, tg, ctg) o ile dla danego kąta jest to możliwe.
%
%
%Ćwiczenie 15.
%Napisz program "lotto.py", który wyświetli 6 losowych i nie powtarzających się liczb z zakresu od 1 do 49.
%
%
%Ćwiczenie 16.
%Napisz program "wyrazy.py", który wczyta od użytkownika pewien tekst, a następnie podzieli go na zdania (zakładamy, że jednoznacznie kropka rozdziela zdania) i dla każdego zdania wyświetli ile jest w nim wyrazów (zakładamy, że spacja oddziela wyrazy w zdaniu).
%
%
%Ćwiczenie 17.
%Zdefiniuj funkcję "geo", która dla podanych trzech parametrów: n=numer elementu ciągu, a1=wartość pierwszego elementu ciągu (domyślnie 1), q=wartość iloczynu ciągu geometrycznego (domyślnie 2) zwróci n-ty element ciągu geometrycznego.
%
%
%Ćwiczenie 18.
%Zdefiniuj funkcję "avg", która dla dowolnej liczby parametrów zwróci ich średnią arytmetyczną (lub 0 dla 0 parametrów).
%
%
%Ćwiczenie 19.
%Zdefiniuj funkcję "sylaby", która dla parametru będącego wyrazem w języku polskim, zwróci listę jego sylab. Funkcja nie musi działać perfekcyjnie dla każdego wyrazu (wyjątków itp.), ale im więcej przypadków obsłuży prawidłowo, tym lepiej.
%
%
%Ćwiczenie 20.
%Napisz program "login.py", który wczyta od użytkownika login i hasło. Jeżeli w słowniku haseł znajduje się odpowiedni login i hasło, program powinien wyświetlić „Autoryzacja pomyślna”; inaczej – „Błąd logowania”.
%
%
%Ćwiczenie 21.
%Napisz program „lista.py”, który:
%a) Jeżeli na dysku nie ma pliku „lista.txt”, wczyta od użytkownika listę studentów (imię, nazwisko, grupa) i zapisze ją do pliku „lista.txt”
%b) Jeżeli na dysku jest już plik „lista.txt”, wczyta z niego listę studentów (imię, nazwisko, grupa) , a użytkownikowi umożliwi dopisywanie nowych studentów do listy, na koniec zapisze listę z powrotem do pliku.
%
%
%Ćwiczenie 22.
%Wykorzystując funkcję sylaby z ćwiczenia III z lekcji 8, napisz program „sylab.py”, który wczyta plik tekstowy o podanej przez użytkownika nazwie i rozbije zawarty w nim tekst na sylaby.
%
%
%Ćwiczenie 23.
%Napisz program, który przeliczy wprowadzoną liczbę rzymską na jej postać dziesiętną.
%
%
%Ćwiczenia 24.
%Napisz program, który wczyta od użytkownika tekst, a następnie wyświetli go w ramce z gwiazdek,
%
% np.: HELLO STRANGER! ->
%
%*******************
%*   HELLO STRANGER!  *
%*******************
%
%
%Ćwiczenia 25.
%Napisz program, który wczyta od użytkownika liczbę X, a następnie wyświetli TRÓJKĄT prostokątny o długości przyprostokątnej X,
%
% np.: 5 ->
%
%*
%* *
%*   *
%*     *
%*****
%
%
%Ćwiczenia 26.
%
%Napisz program, który wczyta od użytkownika liczbę X, a następnie wyświetli romb o wysokości 2X,
%
% np.: 2 ->
% 
%  /\
% /  \
% \  /
%  \/
% 
% 
%Ćwiczenia 27.
%
%Napisz program, który wczyta od użytkownika liczbę X, a następnie wyświetli kwadrat z gwiazdek o długości boku X,
% 
%  np.: 4 ->
%
%****
%*    *
%*    *
%****
% 
% 
%Ćwiczenia 28.
%Napisz program, który:
%a) wczyta od użytkownika do listy R 10 elementów będących dodatnimi liczbami rzeczywistymi
%b) wyliczy dla elementów listy R sumę S wyrażeń i/ gdzie i oznacza nr elementu tablicy R
%c) wyświetli na ekranie sumę S
% 
% 
%Ćwiczenia 29.
%Napisz program, który wczyta od użytkownika tekst, a następnie podwoi w nim co drugi znak i wyświetli wynik,
%
% np.: ALA MA PSA -> ALLA  MAA PPSAA
%
%
%Ćwiczenia 30.
%Napisz program, który wczyta od użytkownika tekst, a następnie wyświetli go w dwóch linijkach naprzemiennie,
%
% np.: HELLO STRANGER! ->
%
%H L O S R N E !
%
% E L   T A G R

\section{Zadania 10}

\begin{zadanie}

	If we list all the natural numbers below 10 that are multiples of 3 or 5, we get 3, 5, 6 and 9. The sum of these multiples is 23.

	Find the sum of all the multiples of 3 or 5 below 1000.

	\href{https://projecteuler.net/problem=1}{abc}
\end{zadanie}
\begin{zadanie}
	If we list all the natural numbers below 10 that are multiples of 3 or 5, we get 3, 5, 6 and 9. The sum of these multiples is 23. Find the sum of all the multiples of 3 or 5 below 1000. \href{https://projecteuler.net/problem=1}{Zadanie 1}
\end{zadanie}
\begin{zadanie}
	Each new term in the Fibonacci sequence is generated by adding the previous two terms. By starting with 1 and 2, the first 10 terms will be: 1, 2, 3, 5, 8, 13, 21, 34, 55, 89, ... By considering the terms in the Fibonacci sequence whose values do not exceed four million, find the sum of the even-valued terms. \href{https://projecteuler.net/problem=2}{Zadanie 2}
\end{zadanie}
\begin{zadanie}
	The prime factors of 13195 are 5, 7, 13 and 29. What is the largest prime factor of the number 600851475143 ? \href{https://projecteuler.net/problem=3}{Zadanie 3}
\end{zadanie}
\begin{zadanie}
	A palindromic number reads the same both ways. The largest palindrome made from the product of two 2-digit numbers is 9009 = 91 $\times$ 99. Find the largest palindrome made from the product of two 3-digit numbers. \href{https://projecteuler.net/problem=4}{Zadanie 4}
\end{zadanie}
\begin{zadanie}
	2520 is the smallest number that can be divided by each of the numbers from 1 to 10 without any remainder. What is the smallest positive number that is evenly divisible by all of the numbers from 1 to 20? \href{https://projecteuler.net/problem=5}{Zadanie 5}
\end{zadanie}
\begin{zadanie}
	The sum of the squares of the first ten natural numbers is, The square of the sum of the first ten natural numbers is, Hence the difference between the sum of the squares of the first ten natural numbers and the square of the sum is 3025 - 385 = 2640. Find the difference between the sum of the squares of the first one hundred natural numbers and the square of the sum. \href{https://projecteuler.net/problem=6}{Zadanie 6}
\end{zadanie}
\begin{zadanie}
	By listing the first six prime numbers: 2, 3, 5, 7, 11, and 13, we can see that the 6th prime is 13. What is the 10 001st prime number? \href{https://projecteuler.net/problem=7}{Zadanie 7}
\end{zadanie}
\begin{zadanie}
	The four adjacent digits in the 1000-digit number that have the greatest product are 9 $\times$ 9 $\times$ 8 $\times$ 9 = 5832. Find the thirteen adjacent digits in the 1000-digit number that have the greatest product. What is the value of this product? \href{https://projecteuler.net/problem=8}{Zadanie 8}
\end{zadanie}
\begin{zadanie}
	A Pythagorean triplet is a set of three natural numbers, a < b < c , for which, For example, 3 2 + 4 2 = 9 + 16 = 25 = 5 2 . There exists exactly one Pythagorean triplet for which a + b + c = 1000. Find the product abc . \href{https://projecteuler.net/problem=9}{Zadanie 9}
\end{zadanie}
\begin{zadanie}
	The sum of the primes below 10 is 2 + 3 + 5 + 7 = 17. Find the sum of all the primes below two million. \href{https://projecteuler.net/problem=10}{Zadanie 10}
\end{zadanie}
\begin{zadanie}
	In the 20$\times$20 grid below, four numbers along a diagonal line have been marked in red. The product of these numbers is 26 $\times$ 63 $\times$ 78 $\times$ 14 = 1788696. What is the greatest product of four adjacent numbers in the same direction (up, down, left, right, or diagonally) in the 20$\times$20 grid? \href{https://projecteuler.net/problem=11}{Zadanie 11}
\end{zadanie}
\begin{zadanie}
	The sequence of triangle numbers is generated by adding the natural numbers. So the 7 th triangle number would be 1 + 2 + 3 + 4 + 5 + 6 + 7 = 28. The first ten terms would be: 1, 3, 6, 10, 15, 21, 28, 36, 45, 55, ... Let us list the factors of the first seven triangle numbers: We can see that 28 is the first triangle number to have over five divisors. What is the value of the first triangle number to have over five hundred divisors? \href{https://projecteuler.net/problem=12}{Zadanie 12}
\end{zadanie}
\begin{zadanie}
	Work out the first ten digits of the sum of the following one-hundred 50-digit numbers. \href{https://projecteuler.net/problem=13}{Zadanie 13}
\end{zadanie}
\begin{zadanie}
	The following iterative sequence is defined for the set of positive integers: n $\rightarrow$ n /2 ( n is even) n $\rightarrow$ 3 n + 1 ( n is odd) Using the rule above and starting with 13, we generate the following sequence: It can be seen that this sequence (starting at 13 and finishing at 1) contains 10 terms. Although it has not been proved yet (Collatz Problem), it is thought that all starting numbers finish at 1. Which starting number, under one million, produces the longest chain? NOTE: Once the chain starts the terms are allowed to go above one million. \href{https://projecteuler.net/problem=14}{Zadanie 14}
\end{zadanie}
\begin{zadanie}
	Starting in the top left corner of a 2$\times$2 grid, and only being able to move to the right and down, there are exactly 6 routes to the bottom right corner. How many such routes are there through a 20$\times$20 grid? \href{https://projecteuler.net/problem=15}{Zadanie 15}
\end{zadanie}
\begin{zadanie}
	2 15 = 32768 and the sum of its digits is 3 + 2 + 7 + 6 + 8 = 26. What is the sum of the digits of the number 2 1000 ? \href{https://projecteuler.net/problem=16}{Zadanie 16}
\end{zadanie}
\begin{zadanie}
	If the numbers 1 to 5 are written out in words: one, two, three, four, five, then there are 3 + 3 + 5 + 4 + 4 = 19 letters used in total. If all the numbers from 1 to 1000 (one thousand) inclusive were written out in words, how many letters would be used? NOTE: Do not count spaces or hyphens. For example, 342 (three hundred and forty-two) contains 23 letters and 115 (one hundred and fifteen) contains 20 letters. The use of "and" when writing out numbers is in compliance with British usage. \href{https://projecteuler.net/problem=17}{Zadanie 17}
\end{zadanie}
\begin{zadanie}
	By starting at the top of the triangle below and moving to adjacent numbers on the row below, the maximum total from top to bottom is 23. That is, 3 + 7 + 4 + 9 = 23. Find the maximum total from top to bottom of the triangle below: NOTE: As there are only 16384 routes, it is possible to solve this problem by trying every route. However, Problem 67 , is the same challenge with a triangle containing one-hundred rows; it cannot be solved by brute force, and requires a clever method! ;o) \href{https://projecteuler.net/problem=18}{Zadanie 18}
\end{zadanie}
\begin{zadanie}
	You are given the following information, but you may prefer to do some research for yourself. How many Sundays fell on the first of the month during the twentieth century (1 Jan 1901 to 31 Dec 2000)? \href{https://projecteuler.net/problem=19}{Zadanie 19}
\end{zadanie}
\begin{zadanie}
	n ! means n $\times$ ( n - 1) $\times$ ... $\times$ 3 $\times$ 2 $\times$ 1 For example, 10! = 10 $\times$ 9 $\times$ ... $\times$ 3 $\times$ 2 $\times$ 1 = 3628800, and the sum of the digits in the number 10! is 3 + 6 + 2 + 8 + 8 + 0 + 0 = 27. Find the sum of the digits in the number 100! \href{https://projecteuler.net/problem=20}{Zadanie 20}
\end{zadanie}
\begin{zadanie}
	Let d( n ) be defined as the sum of proper divisors of n (numbers less than n which divide evenly into n ). If d( a ) = b and d( b ) = a , where a $\neq$ b , then a and b are an amicable pair and each of a and b are called amicable numbers. For example, the proper divisors of 220 are 1, 2, 4, 5, 10, 11, 20, 22, 44, 55 and 110; therefore d(220) = 284. The proper divisors of 284 are 1, 2, 4, 71 and 142; so d(284) = 220. Evaluate the sum of all the amicable numbers under 10000. \href{https://projecteuler.net/problem=21}{Zadanie 21}
\end{zadanie}
\begin{zadanie}
	Using names.txt (right click and 'Save Link/Target As...'), a 46K text file containing over five-thousand first names, begin by sorting it into alphabetical order. Then working out the alphabetical value for each name, multiply this value by its alphabetical position in the list to obtain a name score. For example, when the list is sorted into alphabetical order, COLIN, which is worth 3 + 15 + 12 + 9 + 14 = 53, is the 938th name in the list. So, COLIN would obtain a score of 938 $\times$ 53 = 49714. What is the total of all the name scores in the file? \href{https://projecteuler.net/problem=22}{Zadanie 22}
\end{zadanie}
\begin{zadanie}
	A perfect number is a number for which the sum of its proper divisors is exactly equal to the number. For example, the sum of the proper divisors of 28 would be 1 + 2 + 4 + 7 + 14 = 28, which means that 28 is a perfect number. A number n is called deficient if the sum of its proper divisors is less than n and it is called abundant if this sum exceeds n . As 12 is the smallest abundant number, 1 + 2 + 3 + 4 + 6 = 16, the smallest number that can be written as the sum of two abundant numbers is 24. By mathematical analysis, it can be shown that all integers greater than 28123 can be written as the sum of two abundant numbers. However, this upper limit cannot be reduced any further by analysis even though it is known that the greatest number that cannot be expressed as the sum of two abundant numbers is less than this limit. Find the sum of all the positive integers which cannot be written as the sum of two abundant numbers. \href{https://projecteuler.net/problem=23}{Zadanie 23}
\end{zadanie}
\begin{zadanie}
	A permutation is an ordered arrangement of objects. For example, 3124 is one possible permutation of the digits 1, 2, 3 and 4. If all of the permutations are listed numerically or alphabetically, we call it lexicographic order. The lexicographic permutations of 0, 1 and 2 are: 012 021 102 120 201 210 What is the millionth lexicographic permutation of the digits 0, 1, 2, 3, 4, 5, 6, 7, 8 and 9? \href{https://projecteuler.net/problem=24}{Zadanie 24}
\end{zadanie}
\begin{zadanie}
	The Fibonacci sequence is defined by the recurrence relation: Hence the first 12 terms will be: The 12th term, F 12 , is the first term to contain three digits. What is the index of the first term in the Fibonacci sequence to contain 1000 digits? \href{https://projecteuler.net/problem=25}{Zadanie 25}
\end{zadanie}
\begin{zadanie}
	A unit fraction contains 1 in the numerator. The decimal representation of the unit fractions with denominators 2 to 10 are given: Where 0.1(6) means 0.166666..., and has a 1-digit recurring cycle. It can be seen that 1 / 7 has a 6-digit recurring cycle. Find the value of d < 1000 for which 1 / d contains the longest recurring cycle in its decimal fraction part. \href{https://projecteuler.net/problem=26}{Zadanie 26}
\end{zadanie}
\begin{zadanie}
	Euler discovered the remarkable quadratic formula: $n^2 + n + 41$ It turns out that the formula will produce 40 primes for the consecutive integer values $0 \le n \le 39$. However, when $n = 40, 40^2 + 40 + 41 = 40(40 + 1) + 41$ is divisible by 41, and certainly when $n = 41, 41^2 + 41 + 41$ is clearly divisible by 41. The incredible formula $n^2 - 79n + 1601$ was discovered, which produces 80 primes for the consecutive values $0 \le n \le 79$. The product of the coefficients, -79 and 1601, is -126479. Considering quadratics of the form: Find the product of the coefficients, $a$ and $b$, for the quadratic expression that produces the maximum number of primes for consecutive values of $n$, starting with $n = 0$. \href{https://projecteuler.net/problem=27}{Zadanie 27}
\end{zadanie}
\begin{zadanie}
	Starting with the number 1 and moving to the right in a clockwise direction a 5 by 5 spiral is formed as follows: It can be verified that the sum of the numbers on the diagonals is 101. What is the sum of the numbers on the diagonals in a 1001 by 1001 spiral formed in the same way? \href{https://projecteuler.net/problem=28}{Zadanie 28}
\end{zadanie}
\begin{zadanie}
	Consider all integer combinations of a b for 2 $\leq$ a $\leq$ 5 and 2 $\leq$ b $\leq$ 5: If they are then placed in numerical order, with any repeats removed, we get the following sequence of 15 distinct terms: 4, 8, 9, 16, 25, 27, 32, 64, 81, 125, 243, 256, 625, 1024, 3125 How many distinct terms are in the sequence generated by a b for 2 $\leq$ a $\leq$ 100 and 2 $\leq$ b $\leq$ 100? \href{https://projecteuler.net/problem=29}{Zadanie 29}
\end{zadanie}
\begin{zadanie}
	Surprisingly there are only three numbers that can be written as the sum of fourth powers of their digits: As 1 = 1 4 is not a sum it is not included. The sum of these numbers is 1634 + 8208 + 9474 = 19316. Find the sum of all the numbers that can be written as the sum of fifth powers of their digits. \href{https://projecteuler.net/problem=30}{Zadanie 30}
\end{zadanie}
\begin{zadanie}
	In England the currency is made up of pound, L, and pence, p, and there are eight coins in general circulation: It is possible to make 2 in the following way: How many different ways can 2 be made using any number of coins? \href{https://projecteuler.net/problem=31}{Zadanie 31}
\end{zadanie}
\begin{zadanie}
	We shall say that an n -digit number is pandigital if it makes use of all the digits 1 to n exactly once; for example, the 5-digit number, 15234, is 1 through 5 pandigital. The product 7254 is unusual, as the identity, 39 $\times$ 186 = 7254, containing multiplicand, multiplier, and product is 1 through 9 pandigital. Find the sum of all products whose multiplicand/multiplier/product identity can be written as a 1 through 9 pandigital. \href{https://projecteuler.net/problem=32}{Zadanie 32}
\end{zadanie}
\begin{zadanie}
	The fraction 49 / 98 is a curious fraction, as an inexperienced mathematician in attempting to simplify it may incorrectly believe that 49 / 98 = 4 / 8 , which is correct, is obtained by cancelling the 9s. We shall consider fractions like, 30 / 50 = 3 / 5 , to be trivial examples. There are exactly four non-trivial examples of this type of fraction, less than one in value, and containing two digits in the numerator and denominator. If the product of these four fractions is given in its lowest common terms, find the value of the denominator. \href{https://projecteuler.net/problem=33}{Zadanie 33}
\end{zadanie}
\begin{zadanie}
	145 is a curious number, as 1! + 4! + 5! = 1 + 24 + 120 = 145. Find the sum of all numbers which are equal to the sum of the factorial of their digits. Note: as 1! = 1 and 2! = 2 are not sums they are not included. \href{https://projecteuler.net/problem=34}{Zadanie 34}
\end{zadanie}
\begin{zadanie}
	The number, 197, is called a circular prime because all rotations of the digits: 197, 971, and 719, are themselves prime. There are thirteen such primes below 100: 2, 3, 5, 7, 11, 13, 17, 31, 37, 71, 73, 79, and 97. How many circular primes are there below one million? \href{https://projecteuler.net/problem=35}{Zadanie 35}
\end{zadanie}
\begin{zadanie}
	The decimal number, 585 = 1001001001 2 (binary), is palindromic in both bases. Find the sum of all numbers, less than one million, which are palindromic in base 10 and base 2. (Please note that the palindromic number, in either base, may not include leading zeros.) \href{https://projecteuler.net/problem=36}{Zadanie 36}
\end{zadanie}
\begin{zadanie}
	The number 3797 has an interesting property. Being prime itself, it is possible to continuously remove digits from left to right, and remain prime at each stage: 3797, 797, 97, and 7. Similarly we can work from right to left: 3797, 379, 37, and 3. Find the sum of the only eleven primes that are both truncatable from left to right and right to left. NOTE: 2, 3, 5, and 7 are not considered to be truncatable primes. \href{https://projecteuler.net/problem=37}{Zadanie 37}
\end{zadanie}
\begin{zadanie}
	Take the number 192 and multiply it by each of 1, 2, and 3: By concatenating each product we get the 1 to 9 pandigital, 192384576. We will call 192384576 the concatenated product of 192 and (1,2,3) The same can be achieved by starting with 9 and multiplying by 1, 2, 3, 4, and 5, giving the pandigital, 918273645, which is the concatenated product of 9 and (1,2,3,4,5). What is the largest 1 to 9 pandigital 9-digit number that can be formed as the concatenated product of an integer with (1,2, ... , n ) where n > 1? \href{https://projecteuler.net/problem=38}{Zadanie 38}
\end{zadanie}
\begin{zadanie}
	If p is the perimeter of a right angle triangle with integral length sides, { a , b , c }, there are exactly three solutions for p = 120. {20,48,52}, {24,45,51}, {30,40,50} For which value of p $\leq$ 1000, is the number of solutions maximised? \href{https://projecteuler.net/problem=39}{Zadanie 39}
\end{zadanie}
\begin{zadanie}
	An irrational decimal fraction is created by concatenating the positive integers: 0.12345678910 1 112131415161718192021... It can be seen that the 12 th digit of the fractional part is 1. If d n represents the n th digit of the fractional part, find the value of the following expression. d 1 $\times$ d 10 $\times$ d 100 $\times$ d 1000 $\times$ d 10000 $\times$ d 100000 $\times$ d 1000000 \href{https://projecteuler.net/problem=40}{Zadanie 40}
\end{zadanie}
\begin{zadanie}
	We shall say that an n -digit number is pandigital if it makes use of all the digits 1 to n exactly once. For example, 2143 is a 4-digit pandigital and is also prime. What is the largest n -digit pandigital prime that exists? \href{https://projecteuler.net/problem=41}{Zadanie 41}
\end{zadanie}
\begin{zadanie}
	The n th term of the sequence of triangle numbers is given by, t n = $\frac12$ n ( n +1); so the first ten triangle numbers are: 1, 3, 6, 10, 15, 21, 28, 36, 45, 55, ... By converting each letter in a word to a number corresponding to its alphabetical position and adding these values we form a word value. For example, the word value for SKY is 19 + 11 + 25 = 55 = t 10 . If the word value is a triangle number then we shall call the word a triangle word. Using words.txt (right click and 'Save Link/Target As...'), a 16K text file containing nearly two-thousand common English words, how many are triangle words? \href{https://projecteuler.net/problem=42}{Zadanie 42}
\end{zadanie}
\begin{zadanie}
	The number, 1406357289, is a 0 to 9 pandigital number because it is made up of each of the digits 0 to 9 in some order, but it also has a rather interesting sub-string divisibility property. Let d 1 be the 1 st digit, d 2 be the 2 nd digit, and so on. In this way, we note the following: Find the sum of all 0 to 9 pandigital numbers with this property. \href{https://projecteuler.net/problem=43}{Zadanie 43}
\end{zadanie}
\begin{zadanie}
	Pentagonal numbers are generated by the formula, P n = n (3 n -1)/2. The first ten pentagonal numbers are: 1, 5, 12, 22, 35, 51, 70, 92, 117, 145, ... It can be seen that P 4 + P 7 = 22 + 70 = 92 = P 8 . However, their difference, 70 - 22 = 48, is not pentagonal. Find the pair of pentagonal numbers, P j and P k , for which their sum and difference are pentagonal and D = |P k - P j | is minimised; what is the value of D? \href{https://projecteuler.net/problem=44}{Zadanie 44}
\end{zadanie}
\begin{zadanie}
	Triangle, pentagonal, and hexagonal numbers are generated by the following formulae: It can be verified that T 285 = P 165 = H 143 = 40755. Find the next triangle number that is also pentagonal and hexagonal. \href{https://projecteuler.net/problem=45}{Zadanie 45}
\end{zadanie}
\begin{zadanie}
	It was proposed by Christian Goldbach that every odd composite number can be written as the sum of a prime and twice a square. 9 = 7 + 2$\times$1 2 15 = 7 + 2$\times$2 2 21 = 3 + 2$\times$3 2 25 = 7 + 2$\times$3 2 27 = 19 + 2$\times$2 2 33 = 31 + 2$\times$1 2 It turns out that the conjecture was false. What is the smallest odd composite that cannot be written as the sum of a prime and twice a square? \href{https://projecteuler.net/problem=46}{Zadanie 46}
\end{zadanie}
\begin{zadanie}
	The first two consecutive numbers to have two distinct prime factors are: 14 = 2 $\times$ 7 15 = 3 $\times$ 5 The first three consecutive numbers to have three distinct prime factors are: 644 = 2$^2$ $\times$ 7 $\times$ 23 645 = 3 $\times$ 5 $\times$ 43 646 = 2 $\times$ 17 $\times$ 19. Find the first four consecutive integers to have four distinct prime factors each. What is the first of these numbers? \href{https://projecteuler.net/problem=47}{Zadanie 47}
\end{zadanie}
\begin{zadanie}
	The series, 1 1 + 2 2 + 3 3 + ... + 10 10 = 10405071317. Find the last ten digits of the series, 1 1 + 2 2 + 3 3 + ... + 1000 1000 . \href{https://projecteuler.net/problem=48}{Zadanie 48}
\end{zadanie}
\begin{zadanie}
	The arithmetic sequence, 1487, 4817, 8147, in which each of the terms increases by 3330, is unusual in two ways: (i) each of the three terms are prime, and, (ii) each of the 4-digit numbers are permutations of one another. There are no arithmetic sequences made up of three 1-, 2-, or 3-digit primes, exhibiting this property, but there is one other 4-digit increasing sequence. What 12-digit number do you form by concatenating the three terms in this sequence? \href{https://projecteuler.net/problem=49}{Zadanie 49}
\end{zadanie}
\begin{zadanie}
	The prime 41, can be written as the sum of six consecutive primes: This is the longest sum of consecutive primes that adds to a prime below one-hundred. The longest sum of consecutive primes below one-thousand that adds to a prime, contains 21 terms, and is equal to 953. Which prime, below one-million, can be written as the sum of the most consecutive primes? \href{https://projecteuler.net/problem=50}{Zadanie 50}
\end{zadanie}
\begin{zadanie}
	By replacing the 1 st digit of the 2-digit number *3, it turns out that six of the nine possible values: 13, 23, 43, 53, 73, and 83, are all prime. By replacing the 3 rd and 4 th digits of 56**3 with the same digit, this 5-digit number is the first example having seven primes among the ten generated numbers, yielding the family: 56003, 56113, 56333, 56443, 56663, 56773, and 56993. Consequently 56003, being the first member of this family, is the smallest prime with this property. Find the smallest prime which, by replacing part of the number (not necessarily adjacent digits) with the same digit, is part of an eight prime value family. \href{https://projecteuler.net/problem=51}{Zadanie 51}
\end{zadanie}
\begin{zadanie}
	It can be seen that the number, 125874, and its double, 251748, contain exactly the same digits, but in a different order. Find the smallest positive integer, x , such that 2 x , 3 x , 4 x , 5 x , and 6 x , contain the same digits. \href{https://projecteuler.net/problem=52}{Zadanie 52}
\end{zadanie}
\begin{zadanie}
	There are exactly ten ways of selecting three from five, 12345: 123, 124, 125, 134, 135, 145, 234, 235, 245, and 345 In combinatorics, we use the notation, $\displaystyle \binom 5 3 = 10$. In general, $\displaystyle \binom n r = \dfrac{n!}{r!(n-r)!}$, where $r \le n$, $n! = n \times (n-1) \times ... \times 3 \times 2 \times 1$, and $0! = 1$. It is not until $n = 23$, that a value exceeds one-million: $\displaystyle \binom {23} {10} = 1144066$. How many, not necessarily distinct, values of $\displaystyle \binom n r$ for $1 \le n \le 100$, are greater than one-million? \href{https://projecteuler.net/problem=53}{Zadanie 53}
\end{zadanie}
\begin{zadanie}
	In the card game poker, a hand consists of five cards and are ranked, from lowest to highest, in the following way: The cards are valued in the order: 2, 3, 4, 5, 6, 7, 8, 9, 10, Jack, Queen, King, Ace. If two players have the same ranked hands then the rank made up of the highest value wins; for example, a pair of eights beats a pair of fives (see example 1 below). But if two ranks tie, for example, both players have a pair of queens, then highest cards in each hand are compared (see example 4 below); if the highest cards tie then the next highest cards are compared, and so on. Consider the following five hands dealt to two players: The file, poker.txt , contains one-thousand random hands dealt to two players. Each line of the file contains ten cards (separated by a single space): the first five are Player 1's cards and the last five are Player 2's cards. You can assume that all hands are valid (no invalid characters or repeated cards), each player's hand is in no specific order, and in each hand there is a clear winner. How many hands does Player 1 win? \href{https://projecteuler.net/problem=54}{Zadanie 54}
\end{zadanie}
\begin{zadanie}
	If we take 47, reverse and add, 47 + 74 = 121, which is palindromic. Not all numbers produce palindromes so quickly. For example, 349 + 943 = 1292, 1292 + 2921 = 4213 4213 + 3124 = 7337 That is, 349 took three iterations to arrive at a palindrome. Although no one has proved it yet, it is thought that some numbers, like 196, never produce a palindrome. A number that never forms a palindrome through the reverse and add process is called a Lychrel number. Due to the theoretical nature of these numbers, and for the purpose of this problem, we shall assume that a number is Lychrel until proven otherwise. In addition you are given that for every number below ten-thousand, it will either (i) become a palindrome in less than fifty iterations, or, (ii) no one, with all the computing power that exists, has managed so far to map it to a palindrome. In fact, 10677 is the first number to be shown to require over fifty iterations before producing a palindrome: 4668731596684224866951378664 (53 iterations, 28-digits). Surprisingly, there are palindromic numbers that are themselves Lychrel numbers; the first example is 4994. How many Lychrel numbers are there below ten-thousand? NOTE: Wording was modified slightly on 24 April 2007 to emphasise the theoretical nature of Lychrel numbers. \href{https://projecteuler.net/problem=55}{Zadanie 55}
\end{zadanie}
\begin{zadanie}
	A googol (10 100 ) is a massive number: one followed by one-hundred zeros; 100 100 is almost unimaginably large: one followed by two-hundred zeros. Despite their size, the sum of the digits in each number is only 1. Considering natural numbers of the form, a b , where a, b < 100, what is the maximum digital sum? \href{https://projecteuler.net/problem=56}{Zadanie 56}
\end{zadanie}
\begin{zadanie}
	It is possible to show that the square root of two can be expressed as an infinite continued fraction. $\sqrt 2 =1+ \frac 1 {2+ \frac 1 {2 +\frac 1 {2+ \dots}}}$ By expanding this for the first four iterations, we get: $1 + \frac 1 2 = \frac 32 = 1.5$ $1 + \frac 1 {2 + \frac 1 2} = \frac 7 5 = 1.4$ $1 + \frac 1 {2 + \frac 1 {2+\frac 1 2}} = \frac {17}{12} = 1.41666 \dots$ $1 + \frac 1 {2 + \frac 1 {2+\frac 1 {2+\frac 1 2}}} = \frac {41}{29} = 1.41379 \dots$ The next three expansions are $\frac {99}{70}$, $\frac {239}{169}$, and $\frac {577}{408}$, but the eighth expansion, $\frac {1393}{985}$, is the first example where the number of digits in the numerator exceeds the number of digits in the denominator. In the first one-thousand expansions, how many fractions contain a numerator with more digits than the denominator? \href{https://projecteuler.net/problem=57}{Zadanie 57}
\end{zadanie}
\begin{zadanie}
	Starting with 1 and spiralling anticlockwise in the following way, a square spiral with side length 7 is formed. It is interesting to note that the odd squares lie along the bottom right diagonal, but what is more interesting is that 8 out of the 13 numbers lying along both diagonals are prime; that is, a ratio of 8/13 $\approx$ 62\%. If one complete new layer is wrapped around the spiral above, a square spiral with side length 9 will be formed. If this process is continued, what is the side length of the square spiral for which the ratio of primes along both diagonals first falls below 10\%? \href{https://projecteuler.net/problem=58}{Zadanie 58}
\end{zadanie}
\begin{zadanie}
	Each character on a computer is assigned a unique code and the preferred standard is ASCII (American Standard Code for Information Interchange). For example, uppercase A = 65, asterisk (*) = 42, and lowercase k = 107. A modern encryption method is to take a text file, convert the bytes to ASCII, then XOR each byte with a given value, taken from a secret key. The advantage with the XOR function is that using the same encryption key on the cipher text, restores the plain text; for example, 65 XOR 42 = 107, then 107 XOR 42 = 65. For unbreakable encryption, the key is the same length as the plain text message, and the key is made up of random bytes. The user would keep the encrypted message and the encryption key in different locations, and without both "halves", it is impossible to decrypt the message. Unfortunately, this method is impractical for most users, so the modified method is to use a password as a key. If the password is shorter than the message, which is likely, the key is repeated cyclically throughout the message. The balance for this method is using a sufficiently long password key for security, but short enough to be memorable. Your task has been made easy, as the encryption key consists of three lower case characters. Using p059\_cipher.txt (right click and 'Save Link/Target As...'), a file containing the encrypted ASCII codes, and the knowledge that the plain text must contain common English words, decrypt the message and find the sum of the ASCII values in the original text. \href{https://projecteuler.net/problem=59}{Zadanie 59}
\end{zadanie}
\begin{zadanie}
	The primes 3, 7, 109, and 673, are quite remarkable. By taking any two primes and concatenating them in any order the result will always be prime. For example, taking 7 and 109, both 7109 and 1097 are prime. The sum of these four primes, 792, represents the lowest sum for a set of four primes with this property. Find the lowest sum for a set of five primes for which any two primes concatenate to produce another prime. \href{https://projecteuler.net/problem=60}{Zadanie 60}
\end{zadanie}
\begin{zadanie}
	Triangle, square, pentagonal, hexagonal, heptagonal, and octagonal numbers are all figurate (polygonal) numbers and are generated by the following formulae: The ordered set of three 4-digit numbers: 8128, 2882, 8281, has three interesting properties. Find the sum of the only ordered set of six cyclic 4-digit numbers for which each polygonal type: triangle, square, pentagonal, hexagonal, heptagonal, and octagonal, is represented by a different number in the set. \href{https://projecteuler.net/problem=61}{Zadanie 61}
\end{zadanie}
\begin{zadanie}
	The cube, 41063625 (345 3 ), can be permuted to produce two other cubes: 56623104 (384 3 ) and 66430125 (405 3 ). In fact, 41063625 is the smallest cube which has exactly three permutations of its digits which are also cube. Find the smallest cube for which exactly five permutations of its digits are cube. \href{https://projecteuler.net/problem=62}{Zadanie 62}
\end{zadanie}
\begin{zadanie}
	The 5-digit number, 16807=7 5 , is also a fifth power. Similarly, the 9-digit number, 134217728=8 9 , is a ninth power. How many n -digit positive integers exist which are also an n th power? \href{https://projecteuler.net/problem=63}{Zadanie 63}
\end{zadanie}
\begin{zadanie}
	All square roots are periodic when written as continued fractions and can be written in the form: For example, let us consider $\sqrt{23}:$ If we continue we would get the following expansion: The process can be summarised as follows: $\quad \quad a\_0=4, \frac 1 {\sqrt{23}-4}=\frac {\sqrt{23}+4} 7=1+\frac {\sqrt{23}-3} 7$ $\quad \quad a\_1=1, \frac 7 {\sqrt{23}-3}=\frac {7(\sqrt{23}+3)} {14}=3+\frac {\sqrt{23}-3} 2$ $\quad \quad a\_2=3, \frac 2 {\sqrt{23}-3}=\frac {2(\sqrt{23}+3)} {14}=1+\frac {\sqrt{23}-4} 7$ $\quad \quad a\_3=1, \frac 7 {\sqrt{23}-4}=\frac {7(\sqrt{23}+4)} 7=8+\sqrt{23}-4$ $\quad \quad a\_4=8, \frac 1 {\sqrt{23}-4}=\frac {\sqrt{23}+4} 7=1+\frac {\sqrt{23}-3} 7$ $\quad \quad a\_5=1, \frac 7 {\sqrt{23}-3}=\frac {7 (\sqrt{23}+3)} {14}=3+\frac {\sqrt{23}-3} 2$ $\quad \quad a\_6=3, \frac 2 {\sqrt{23}-3}=\frac {2(\sqrt{23}+3)} {14}=1+\frac {\sqrt{23}-4} 7$ $\quad \quad a\_7=1, \frac 7 {\sqrt{23}-4}=\frac {7(\sqrt{23}+4)} {7}=8+\sqrt{23}-4$ It can be seen that the sequence is repeating. For conciseness, we use the notation $\sqrt{23}=[4;(1,3,1,8)]$, to indicate that the block (1,3,1,8) repeats indefinitely. The first ten continued fraction representations of (irrational) square roots are: $\quad \quad \sqrt{2}=[1;(2)]$, period=$1$ $\quad \quad \sqrt{3}=[1;(1,2)]$, period=$2$ $\quad \quad \sqrt{5}=[2;(4)]$, period=$1$ $\quad \quad \sqrt{6}=[2;(2,4)]$, period=$2$ $\quad \quad \sqrt{7}=[2;(1,1,1,4)]$, period=$4$ $\quad \quad \sqrt{8}=[2;(1,4)]$, period=$2$ $\quad \quad \sqrt{10}=[3;(6)]$, period=$1$ $\quad \quad \sqrt{11}=[3;(3,6)]$, period=$2$ $\quad \quad \sqrt{12}=[3;(2,6)]$, period=$2$ $\quad \quad \sqrt{13}=[3;(1,1,1,1,6)]$, period=$5$ Exactly four continued fractions, for $N \le 13$, have an odd period. How many continued fractions for $N \le 10\,000$ have an odd period? \href{https://projecteuler.net/problem=64}{Zadanie 64}
\end{zadanie}
\begin{zadanie}
	The square root of 2 can be written as an infinite continued fraction. $\sqrt{2} = 1 + \dfrac{1}{2 + \dfrac{1}{2 + \dfrac{1}{2 + \dfrac{1}{2 + ...}}}}$ The infinite continued fraction can be written, $\sqrt{2} = [1; (2)]$, $(2)$ indicates that 2 repeats ad infinitum . In a similar way, $\sqrt{23} = [4; (1, 3, 1, 8)]$. It turns out that the sequence of partial values of continued fractions for square roots provide the best rational approximations. Let us consider the convergents for $\sqrt{2}$. $
		1 + \dfrac{1}{2} = \dfrac{3}{2}\\
		1 + \dfrac{1}{2 + \dfrac{1}{2}} = \dfrac{7}{5}\\
		1 + \dfrac{1}{2 + \dfrac{1}{2 + \dfrac{1}{2}}} = \dfrac{17}{12}\\
		1 + \dfrac{1}{2 + \dfrac{1}{2 + \dfrac{1}{2 + \dfrac{1}{2}}}} = \dfrac{41}{29}
	$ Hence the sequence of the first ten convergents for $\sqrt{2}$ are: $1, \dfrac{3}{2}, \dfrac{7}{5}, \dfrac{17}{12}, \dfrac{41}{29}, \dfrac{99}{70}, \dfrac{239}{169}, \dfrac{577}{408}, \dfrac{1393}{985}, \dfrac{3363}{2378}, ...$ What is most surprising is that the important mathematical constant, $e = [2; 1, 2, 1, 1, 4, 1, 1, 6, 1, ... , 1, 2k, 1, ...]$. The first ten terms in the sequence of convergents for e are: $2, 3, \dfrac{8}{3}, \dfrac{11}{4}, \dfrac{19}{7}, \dfrac{87}{32}, \dfrac{106}{39}, \dfrac{193}{71}, \dfrac{1264}{465}, \dfrac{1457}{536}, ...$ The sum of digits in the numerator of the 10 th convergent is $1 + 4 + 5 + 7 = 17$. Find the sum of digits in the numerator of the 100 th convergent of the continued fraction for $e$. \href{https://projecteuler.net/problem=65}{Zadanie 65}
\end{zadanie}
\begin{zadanie}
	Consider quadratic Diophantine equations of the form: x 2  D y 2 = 1 For example, when D=13, the minimal solution in x is 649 2  13$\times$180 2 = 1. It can be assumed that there are no solutions in positive integers when D is square. By finding minimal solutions in x for D = {2, 3, 5, 6, 7}, we obtain the following: 3 2  2$\times$2 2 = 1 2 2  3$\times$1 2 = 1 9 2  5$\times$4 2 = 1 5 2  6$\times$2 2 = 1 8 2  7$\times$3 2 = 1 Hence, by considering minimal solutions in x for D $\leq$ 7, the largest x is obtained when D=5. Find the value of D $\leq$ 1000 in minimal solutions of x for which the largest value of x is obtained. \href{https://projecteuler.net/problem=66}{Zadanie 66}
\end{zadanie}
\begin{zadanie}
	By starting at the top of the triangle below and moving to adjacent numbers on the row below, the maximum total from top to bottom is 23. That is, 3 + 7 + 4 + 9 = 23. Find the maximum total from top to bottom in triangle.txt (right click and 'Save Link/Target As...'), a 15K text file containing a triangle with one-hundred rows. NOTE: This is a much more difficult version of Problem 18 . It is not possible to try every route to solve this problem, as there are 2 99 altogether! If you could check one trillion (10 12 ) routes every second it would take over twenty billion years to check them all. There is an efficient algorithm to solve it. ;o) \href{https://projecteuler.net/problem=67}{Zadanie 67}
\end{zadanie}
\begin{zadanie}
	Consider the following "magic" 3-gon ring, filled with the numbers 1 to 6, and each line adding to nine. Working clockwise , and starting from the group of three with the numerically lowest external node (4,3,2 in this example), each solution can be described uniquely. For example, the above solution can be described by the set: 4,3,2; 6,2,1; 5,1,3. It is possible to complete the ring with four different totals: 9, 10, 11, and 12. There are eight solutions in total. By concatenating each group it is possible to form 9-digit strings; the maximum string for a 3-gon ring is 432621513. Using the numbers 1 to 10, and depending on arrangements, it is possible to form 16- and 17-digit strings. What is the maximum 16-digit string for a "magic" 5-gon ring? \href{https://projecteuler.net/problem=68}{Zadanie 68}
\end{zadanie}
\begin{zadanie}
	Euler's Totient function, $\phi$( n ) [sometimes called the phi function], is used to determine the number of numbers less than n which are relatively prime to n . For example, as 1, 2, 4, 5, 7, and 8, are all less than nine and relatively prime to nine, $\phi$(9)=6. It can be seen that n =6 produces a maximum n /$\phi$( n ) for n $\leq$ 10. Find the value of n $\leq$ 1,000,000 for which n /$\phi$( n ) is a maximum. \href{https://projecteuler.net/problem=69}{Zadanie 69}
\end{zadanie}
\begin{zadanie}
	Euler's Totient function, $\phi$( n ) [sometimes called the phi function], is used to determine the number of positive numbers less than or equal to n which are relatively prime to n . For example, as 1, 2, 4, 5, 7, and 8, are all less than nine and relatively prime to nine, $\phi$(9)=6. The number 1 is considered to be relatively prime to every positive number, so $\phi$(1)=1. Interestingly, $\phi$(87109)=79180, and it can be seen that 87109 is a permutation of 79180. Find the value of n , 1 < n < 10 7 , for which $\phi$( n ) is a permutation of n and the ratio n /$\phi$( n ) produces a minimum. \href{https://projecteuler.net/problem=70}{Zadanie 70}
\end{zadanie}
\begin{zadanie}
	Consider the fraction, n/d , where n and d are positive integers. If n < d and HCF( n,d )=1, it is called a reduced proper fraction. If we list the set of reduced proper fractions for d $\leq$ 8 in ascending order of size, we get: 1/8, 1/7, 1/6, 1/5, 1/4, 2/7, 1/3, 3/8, 2/5 , 3/7, 1/2, 4/7, 3/5, 5/8, 2/3, 5/7, 3/4, 4/5, 5/6, 6/7, 7/8 It can be seen that 2/5 is the fraction immediately to the left of 3/7. By listing the set of reduced proper fractions for d $\leq$ 1,000,000 in ascending order of size, find the numerator of the fraction immediately to the left of 3/7. \href{https://projecteuler.net/problem=71}{Zadanie 71}
\end{zadanie}
\begin{zadanie}
	Consider the fraction, n/d , where n and d are positive integers. If n < d and HCF( n,d )=1, it is called a reduced proper fraction. If we list the set of reduced proper fractions for d $\leq$ 8 in ascending order of size, we get: 1/8, 1/7, 1/6, 1/5, 1/4, 2/7, 1/3, 3/8, 2/5, 3/7, 1/2, 4/7, 3/5, 5/8, 2/3, 5/7, 3/4, 4/5, 5/6, 6/7, 7/8 It can be seen that there are 21 elements in this set. How many elements would be contained in the set of reduced proper fractions for d $\leq$ 1,000,000? \href{https://projecteuler.net/problem=72}{Zadanie 72}
\end{zadanie}
\begin{zadanie}
	Consider the fraction, n/d , where n and d are positive integers. If n < d and HCF( n,d )=1, it is called a reduced proper fraction. If we list the set of reduced proper fractions for d $\leq$ 8 in ascending order of size, we get: 1/8, 1/7, 1/6, 1/5, 1/4, 2/7, 1/3, 3/8, 2/5, 3/7 , 1/2, 4/7, 3/5, 5/8, 2/3, 5/7, 3/4, 4/5, 5/6, 6/7, 7/8 It can be seen that there are 3 fractions between 1/3 and 1/2. How many fractions lie between 1/3 and 1/2 in the sorted set of reduced proper fractions for d $\leq$ 12,000? \href{https://projecteuler.net/problem=73}{Zadanie 73}
\end{zadanie}
\begin{zadanie}
	The number 145 is well known for the property that the sum of the factorial of its digits is equal to 145: 1! + 4! + 5! = 1 + 24 + 120 = 145 Perhaps less well known is 169, in that it produces the longest chain of numbers that link back to 169; it turns out that there are only three such loops that exist: 169 $\rightarrow$ 363601 $\rightarrow$ 1454 $\rightarrow$ 169 871 $\rightarrow$ 45361 $\rightarrow$ 871 872 $\rightarrow$ 45362 $\rightarrow$ 872 It is not difficult to prove that EVERY starting number will eventually get stuck in a loop. For example, 69 $\rightarrow$ 363600 $\rightarrow$ 1454 $\rightarrow$ 169 $\rightarrow$ 363601 ($\rightarrow$ 1454) 78 $\rightarrow$ 45360 $\rightarrow$ 871 $\rightarrow$ 45361 ($\rightarrow$ 871) 540 $\rightarrow$ 145 ($\rightarrow$ 145) Starting with 69 produces a chain of five non-repeating terms, but the longest non-repeating chain with a starting number below one million is sixty terms. How many chains, with a starting number below one million, contain exactly sixty non-repeating terms? \href{https://projecteuler.net/problem=74}{Zadanie 74}
\end{zadanie}
\begin{zadanie}
	It turns out that 12 cm is the smallest length of wire that can be bent to form an integer sided right angle triangle in exactly one way, but there are many more examples. 12 cm : (3,4,5) 24 cm : (6,8,10) 30 cm : (5,12,13) 36 cm : (9,12,15) 40 cm : (8,15,17) 48 cm : (12,16,20) In contrast, some lengths of wire, like 20 cm, cannot be bent to form an integer sided right angle triangle, and other lengths allow more than one solution to be found; for example, using 120 cm it is possible to form exactly three different integer sided right angle triangles. 120 cm : (30,40,50), (20,48,52), (24,45,51) Given that L is the length of the wire, for how many values of L $\leq$ 1,500,000 can exactly one integer sided right angle triangle be formed? \href{https://projecteuler.net/problem=75}{Zadanie 75}
\end{zadanie}
\begin{zadanie}
	It is possible to write five as a sum in exactly six different ways: 4 + 1 3 + 2 3 + 1 + 1 2 + 2 + 1 2 + 1 + 1 + 1 1 + 1 + 1 + 1 + 1 How many different ways can one hundred be written as a sum of at least two positive integers? \href{https://projecteuler.net/problem=76}{Zadanie 76}
\end{zadanie}
\begin{zadanie}
	It is possible to write ten as the sum of primes in exactly five different ways: 7 + 3 5 + 5 5 + 3 + 2 3 + 3 + 2 + 2 2 + 2 + 2 + 2 + 2 What is the first value which can be written as the sum of primes in over five thousand different ways? \href{https://projecteuler.net/problem=77}{Zadanie 77}
\end{zadanie}
\begin{zadanie}
	Let p( n ) represent the number of different ways in which n coins can be separated into piles. For example, five coins can be separated into piles in exactly seven different ways, so p(5)=7. Find the least value of n for which p( n ) is divisible by one million. \href{https://projecteuler.net/problem=78}{Zadanie 78}
\end{zadanie}
\begin{zadanie}
	A common security method used for online banking is to ask the user for three random characters from a passcode. For example, if the passcode was 531278, they may ask for the 2nd, 3rd, and 5th characters; the expected reply would be: 317. The text file, keylog.txt , contains fifty successful login attempts. Given that the three characters are always asked for in order, analyse the file so as to determine the shortest possible secret passcode of unknown length. \href{https://projecteuler.net/problem=79}{Zadanie 79}
\end{zadanie}
\begin{zadanie}
	It is well known that if the square root of a natural number is not an integer, then it is irrational. The decimal expansion of such square roots is infinite without any repeating pattern at all. The square root of two is 1.41421356237309504880..., and the digital sum of the first one hundred decimal digits is 475. For the first one hundred natural numbers, find the total of the digital sums of the first one hundred decimal digits for all the irrational square roots. \href{https://projecteuler.net/problem=80}{Zadanie 80}
\end{zadanie}
\begin{zadanie}
	In the 5 by 5 matrix below, the minimal path sum from the top left to the bottom right, by only moving to the right and down , is indicated in bold red and is equal to 2427. Find the minimal path sum, in matrix.txt (right click and "Save Link/Target As..."), a 31K text file containing a 80 by 80 matrix, from the top left to the bottom right by only moving right and down. \href{https://projecteuler.net/problem=81}{Zadanie 81}
\end{zadanie}
\begin{zadanie}
	NOTE: This problem is a more challenging version of Problem 81 . The minimal path sum in the 5 by 5 matrix below, by starting in any cell in the left column and finishing in any cell in the right column, and only moving up, down, and right, is indicated in red and bold; the sum is equal to 994. Find the minimal path sum, in matrix.txt (right click and "Save Link/Target As..."), a 31K text file containing a 80 by 80 matrix, from the left column to the right column. \href{https://projecteuler.net/problem=82}{Zadanie 82}
\end{zadanie}
\begin{zadanie}
	NOTE: This problem is a significantly more challenging version of Problem 81 . In the 5 by 5 matrix below, the minimal path sum from the top left to the bottom right, by moving left, right, up, and down, is indicated in bold red and is equal to 2297. Find the minimal path sum, in matrix.txt (right click and "Save Link/Target As..."), a 31K text file containing a 80 by 80 matrix, from the top left to the bottom right by moving left, right, up, and down. \href{https://projecteuler.net/problem=83}{Zadanie 83}
\end{zadanie}
\begin{zadanie}
	In the game, Monopoly , the standard board is set up in the following way: A player starts on the GO square and adds the scores on two 6-sided dice to determine the number of squares they advance in a clockwise direction. Without any further rules we would expect to visit each square with equal probability: 2.5\%. However, landing on G2J (Go To Jail), CC (community chest), and CH (chance) changes this distribution. In addition to G2J, and one card from each of CC and CH, that orders the player to go directly to jail, if a player rolls three consecutive doubles, they do not advance the result of their 3rd roll. Instead they proceed directly to jail. At the beginning of the game, the CC and CH cards are shuffled. When a player lands on CC or CH they take a card from the top of the respective pile and, after following the instructions, it is returned to the bottom of the pile. There are sixteen cards in each pile, but for the purpose of this problem we are only concerned with cards that order a movement; any instruction not concerned with movement will be ignored and the player will remain on the CC/CH square. The heart of this problem concerns the likelihood of visiting a particular square. That is, the probability of finishing at that square after a roll. For this reason it should be clear that, with the exception of G2J for which the probability of finishing on it is zero, the CH squares will have the lowest probabilities, as 5/8 request a movement to another square, and it is the final square that the player finishes at on each roll that we are interested in. We shall make no distinction between "Just Visiting" and being sent to JAIL, and we shall also ignore the rule about requiring a double to "get out of jail", assuming that they pay to get out on their next turn. By starting at GO and numbering the squares sequentially from 00 to 39 we can concatenate these two-digit numbers to produce strings that correspond with sets of squares. Statistically it can be shown that the three most popular squares, in order, are JAIL (6.24\%) = Square 10, E3 (3.18\%) = Square 24, and GO (3.09\%) = Square 00. So these three most popular squares can be listed with the six-digit modal string: 102400. If, instead of using two 6-sided dice, two 4-sided dice are used, find the six-digit modal string. \href{https://projecteuler.net/problem=84}{Zadanie 84}
\end{zadanie}
\begin{zadanie}
	By counting carefully it can be seen that a rectangular grid measuring 3 by 2 contains eighteen rectangles: Although there exists no rectangular grid that contains exactly two million rectangles, find the area of the grid with the nearest solution. \href{https://projecteuler.net/problem=85}{Zadanie 85}
\end{zadanie}
\begin{zadanie}
	A spider, S, sits in one corner of a cuboid room, measuring 6 by 5 by 3, and a fly, F, sits in the opposite corner. By travelling on the surfaces of the room the shortest "straight line" distance from S to F is 10 and the path is shown on the diagram. However, there are up to three "shortest" path candidates for any given cuboid and the shortest route doesn't always have integer length. It can be shown that there are exactly 2060 distinct cuboids, ignoring rotations, with integer dimensions, up to a maximum size of M by M by M, for which the shortest route has integer length when M = 100. This is the least value of M for which the number of solutions first exceeds two thousand; the number of solutions when M = 99 is 1975. Find the least value of M such that the number of solutions first exceeds one million. \href{https://projecteuler.net/problem=86}{Zadanie 86}
\end{zadanie}
\begin{zadanie}
	The smallest number expressible as the sum of a prime square, prime cube, and prime fourth power is 28. In fact, there are exactly four numbers below fifty that can be expressed in such a way: 28 = 2 2 + 2 3 + 2 4 33 = 3 2 + 2 3 + 2 4 49 = 5 2 + 2 3 + 2 4 47 = 2 2 + 3 3 + 2 4 How many numbers below fifty million can be expressed as the sum of a prime square, prime cube, and prime fourth power? \href{https://projecteuler.net/problem=87}{Zadanie 87}
\end{zadanie}
\begin{zadanie}
	A natural number, N, that can be written as the sum and product of a given set of at least two natural numbers, { a 1 , a 2 , ... , a k } is called a product-sum number: N = a 1 + a 2 + ... + a k = a 1 $\times$ a 2 $\times$ ... $\times$ a k . For example, 6 = 1 + 2 + 3 = 1 $\times$ 2 $\times$ 3. For a given set of size, k , we shall call the smallest N with this property a minimal product-sum number. The minimal product-sum numbers for sets of size, k = 2, 3, 4, 5, and 6 are as follows. k =2: 4 = 2 $\times$ 2 = 2 + 2 k =3: 6 = 1 $\times$ 2 $\times$ 3 = 1 + 2 + 3 k =4: 8 = 1 $\times$ 1 $\times$ 2 $\times$ 4 = 1 + 1 + 2 + 4 k =5: 8 = 1 $\times$ 1 $\times$ 2 $\times$ 2 $\times$ 2 = 1 + 1 + 2 + 2 + 2 k =6: 12 = 1 $\times$ 1 $\times$ 1 $\times$ 1 $\times$ 2 $\times$ 6 = 1 + 1 + 1 + 1 + 2 + 6 Hence for 2$\leq$ k $\leq$6, the sum of all the minimal product-sum numbers is 4+6+8+12 = 30; note that 8 is only counted once in the sum. In fact, as the complete set of minimal product-sum numbers for 2$\leq$ k $\leq$12 is {4, 6, 8, 12, 15, 16}, the sum is 61. What is the sum of all the minimal product-sum numbers for 2$\leq$ k $\leq$12000? \href{https://projecteuler.net/problem=88}{Zadanie 88}
\end{zadanie}
\begin{zadanie}
	For a number written in Roman numerals to be considered valid there are basic rules which must be followed. Even though the rules allow some numbers to be expressed in more than one way there is always a "best" way of writing a particular number. For example, it would appear that there are at least six ways of writing the number sixteen: However, according to the rules only an XIIIIII and an XVI are valid, and the last example is considered to be the most efficient, as it uses the least number of numerals. an XIIIIII an XVI The 11K text file, roman.txt (right click and 'Save Link/Target As...'), contains one thousand numbers written in valid, but not necessarily minimal, Roman numerals; see About... Roman Numerals for the definitive rules for this problem. Find the number of characters saved by writing each of these in their minimal form. Note: You can assume that all the Roman numerals in the file contain no more than four consecutive identical units. \href{https://projecteuler.net/problem=89}{Zadanie 89}
\end{zadanie}
\begin{zadanie}
	Each of the six faces on a cube has a different digit (0 to 9) written on it; the same is done to a second cube. By placing the two cubes side-by-side in different positions we can form a variety of 2-digit numbers. For example, the square number 64 could be formed: In fact, by carefully choosing the digits on both cubes it is possible to display all of the square numbers below one-hundred: 01, 04, 09, 16, 25, 36, 49, 64, and 81. For example, one way this can be achieved is by placing {0, 5, 6, 7, 8, 9} on one cube and {1, 2, 3, 4, 8, 9} on the other cube. However, for this problem we shall allow the 6 or 9 to be turned upside-down so that an arrangement like {0, 5, 6, 7, 8, 9} and {1, 2, 3, 4, 6, 7} allows for all nine square numbers to be displayed; otherwise it would be impossible to obtain 09. In determining a distinct arrangement we are interested in the digits on each cube, not the order. {1, 2, 3, 4, 5, 6} is equivalent to {3, 6, 4, 1, 2, 5} {1, 2, 3, 4, 5, 6} is distinct from {1, 2, 3, 4, 5, 9} But because we are allowing 6 and 9 to be reversed, the two distinct sets in the last example both represent the extended set {1, 2, 3, 4, 5, 6, 9} for the purpose of forming 2-digit numbers. How many distinct arrangements of the two cubes allow for all of the square numbers to be displayed? \href{https://projecteuler.net/problem=90}{Zadanie 90}
\end{zadanie}
\begin{zadanie}
	The points P ( x 1 , y 1 ) and Q ( x 2 , y 2 ) are plotted at integer co-ordinates and are joined to the origin, O(0,0), to form $\Delta$OPQ. There are exactly fourteen triangles containing a right angle that can be formed when each co-ordinate lies between 0 and 2 inclusive; that is, 0 $\leq$ x 1 , y 1 , x 2 , y 2 $\leq$ 2. Given that 0 $\leq$ x 1 , y 1 , x 2 , y 2 $\leq$ 50, how many right triangles can be formed? \href{https://projecteuler.net/problem=91}{Zadanie 91}
\end{zadanie}
\begin{zadanie}
	A number chain is created by continuously adding the square of the digits in a number to form a new number until it has been seen before. For example, 44 $\rightarrow$ 32 $\rightarrow$ 13 $\rightarrow$ 10 $\rightarrow$ 1 $\rightarrow$ 1 85 $\rightarrow$ 89 $\rightarrow$ 145 $\rightarrow$ 42 $\rightarrow$ 20 $\rightarrow$ 4 $\rightarrow$ 16 $\rightarrow$ 37 $\rightarrow$ 58 $\rightarrow$ 89 Therefore any chain that arrives at 1 or 89 will become stuck in an endless loop. What is most amazing is that EVERY starting number will eventually arrive at 1 or 89. How many starting numbers below ten million will arrive at 89? \href{https://projecteuler.net/problem=92}{Zadanie 92}
\end{zadanie}
\begin{zadanie}
	By using each of the digits from the set, {1, 2, 3, 4}, exactly once, and making use of the four arithmetic operations (+, -, *, /) and brackets/parentheses, it is possible to form different positive integer targets. For example, Note that concatenations of the digits, like 12 + 34, are not allowed. Using the set, {1, 2, 3, 4}, it is possible to obtain thirty-one different target numbers of which 36 is the maximum, and each of the numbers 1 to 28 can be obtained before encountering the first non-expressible number. Find the set of four distinct digits, a < b < c < d , for which the longest set of consecutive positive integers, 1 to n , can be obtained, giving your answer as a string: abcd . \href{https://projecteuler.net/problem=93}{Zadanie 93}
\end{zadanie}
\begin{zadanie}
	It is easily proved that no equilateral triangle exists with integral length sides and integral area. However, the almost equilateral triangle 5-5-6 has an area of 12 square units. We shall define an almost equilateral triangle to be a triangle for which two sides are equal and the third differs by no more than one unit. Find the sum of the perimeters of all almost equilateral triangles with integral side lengths and area and whose perimeters do not exceed one billion (1,000,000,000). \href{https://projecteuler.net/problem=94}{Zadanie 94}
\end{zadanie}
\begin{zadanie}
	The proper divisors of a number are all the divisors excluding the number itself. For example, the proper divisors of 28 are 1, 2, 4, 7, and 14. As the sum of these divisors is equal to 28, we call it a perfect number. Interestingly the sum of the proper divisors of 220 is 284 and the sum of the proper divisors of 284 is 220, forming a chain of two numbers. For this reason, 220 and 284 are called an amicable pair. Perhaps less well known are longer chains. For example, starting with 12496, we form a chain of five numbers: 12496 $\rightarrow$ 14288 $\rightarrow$ 15472 $\rightarrow$ 14536 $\rightarrow$ 14264 ($\rightarrow$ 12496 $\rightarrow$ ...) Since this chain returns to its starting point, it is called an amicable chain. Find the smallest member of the longest amicable chain with no element exceeding one million. \href{https://projecteuler.net/problem=95}{Zadanie 95}
\end{zadanie}
\begin{zadanie}
	Su Doku (Japanese meaning number place ) is the name given to a popular puzzle concept. Its origin is unclear, but credit must be attributed to Leonhard Euler who invented a similar, and much more difficult, puzzle idea called Latin Squares. The objective of Su Doku puzzles, however, is to replace the blanks (or zeros) in a 9 by 9 grid in such that each row, column, and 3 by 3 box contains each of the digits 1 to 9. Below is an example of a typical starting puzzle grid and its solution grid. \href{https://projecteuler.net/problem=96}{Zadanie 96}
\end{zadanie}
\begin{zadanie}
	The first known prime found to exceed one million digits was discovered in 1999, and is a Mersenne prime of the form 2 6972593 -1; it contains exactly 2,098,960 digits. Subsequently other Mersenne primes, of the form 2 p -1, have been found which contain more digits. However, in 2004 there was found a massive non-Mersenne prime which contains 2,357,207 digits: 28433$\times$2 7830457 +1. Find the last ten digits of this prime number. \href{https://projecteuler.net/problem=97}{Zadanie 97}
\end{zadanie}
\begin{zadanie}
	By replacing each of the letters in the word CARE with 1, 2, 9, and 6 respectively, we form a square number: 1296 = 36 2 . What is remarkable is that, by using the same digital substitutions, the anagram, RACE, also forms a square number: 9216 = 96 2 . We shall call CARE (and RACE) a square anagram word pair and specify further that leading zeroes are not permitted, neither may a different letter have the same digital value as another letter. Using words.txt (right click and 'Save Link/Target As...'), a 16K text file containing nearly two-thousand common English words, find all the square anagram word pairs (a palindromic word is NOT considered to be an anagram of itself). What is the largest square number formed by any member of such a pair? NOTE: All anagrams formed must be contained in the given text file. \href{https://projecteuler.net/problem=98}{Zadanie 98}
\end{zadanie}
\begin{zadanie}
	Comparing two numbers written in index form like 2 11 and 3 7 is not difficult, as any calculator would confirm that 2 11 = 2048 < 3 7 = 2187. However, confirming that 632382 518061 > 519432 525806 would be much more difficult, as both numbers contain over three million digits. Using base\_exp.txt (right click and 'Save Link/Target As...'), a 22K text file containing one thousand lines with a base/exponent pair on each line, determine which line number has the greatest numerical value. NOTE: The first two lines in the file represent the numbers in the example given above. \href{https://projecteuler.net/problem=99}{Zadanie 99}
\end{zadanie}
\begin{zadanie}
	If a box contains twenty-one coloured discs, composed of fifteen blue discs and six red discs, and two discs were taken at random, it can be seen that the probability of taking two blue discs, P(BB) = (15/21)$\times$(14/20) = 1/2. The next such arrangement, for which there is exactly 50\% chance of taking two blue discs at random, is a box containing eighty-five blue discs and thirty-five red discs. By finding the first arrangement to contain over 10 12 = 1,000,000,000,000 discs in total, determine the number of blue discs that the box would contain. \href{https://projecteuler.net/problem=100}{Zadanie 100}
\end{zadanie}
\begin{zadanie}
	If we are presented with the first k terms of a sequence it is impossible to say with certainty the value of the next term, as there are infinitely many polynomial functions that can model the sequence. As an example, let us consider the sequence of cube numbers. This is defined by the generating function, u n = n 3 : 1, 8, 27, 64, 125, 216, ... Suppose we were only given the first two terms of this sequence. Working on the principle that "simple is best" we should assume a linear relationship and predict the next term to be 15 (common difference 7). Even if we were presented with the first three terms, by the same principle of simplicity, a quadratic relationship should be assumed. We shall define OP( k , n ) to be the n th term of the optimum polynomial generating function for the first k terms of a sequence. It should be clear that OP( k , n ) will accurately generate the terms of the sequence for n $\leq$ k , and potentially the first incorrect term (FIT) will be OP( k , k +1); in which case we shall call it a bad OP (BOP). As a basis, if we were only given the first term of sequence, it would be most sensible to assume constancy; that is, for n $\geq$ 2, OP(1, n ) = u 1 . Hence we obtain the following OPs for the cubic sequence: Clearly no BOPs exist for k $\geq$ 4. By considering the sum of FITs generated by the BOPs (indicated in red above), we obtain 1 + 15 + 58 = 74. Consider the following tenth degree polynomial generating function: u n = 1 - n + n 2 - n 3 + n 4 - n 5 + n 6 - n 7 + n 8 - n 9 + n 10 Find the sum of FITs for the BOPs. \href{https://projecteuler.net/problem=101}{Zadanie 101}
\end{zadanie}
\begin{zadanie}
	Three distinct points are plotted at random on a Cartesian plane, for which -1000 $\leq$ x , y $\leq$ 1000, such that a triangle is formed. Consider the following two triangles: A(-340,495), B(-153,-910), C(835,-947) X(-175,41), Y(-421,-714), Z(574,-645) It can be verified that triangle ABC contains the origin, whereas triangle XYZ does not. Using triangles.txt (right click and 'Save Link/Target As...'), a 27K text file containing the co-ordinates of one thousand "random" triangles, find the number of triangles for which the interior contains the origin. NOTE: The first two examples in the file represent the triangles in the example given above. \href{https://projecteuler.net/problem=102}{Zadanie 102}
\end{zadanie}
\begin{zadanie}
	Let S(A) represent the sum of elements in set A of size n . We shall call it a special sum set if for any two non-empty disjoint subsets, B and C, the following properties are true: If S(A) is minimised for a given n , we shall call it an optimum special sum set. The first five optimum special sum sets are given below. n = 1: {1} n = 2: {1, 2} n = 3: {2, 3, 4} n = 4: {3, 5, 6, 7} n = 5: {6, 9, 11, 12, 13} It seems that for a given optimum set, A = { a 1 , a 2 , ... , a n }, the next optimum set is of the form B = { b , a 1 + b , a 2 + b , ... , a n + b }, where b is the "middle" element on the previous row. By applying this "rule" we would expect the optimum set for n = 6 to be A = {11, 17, 20, 22, 23, 24}, with S(A) = 117. However, this is not the optimum set, as we have merely applied an algorithm to provide a near optimum set. The optimum set for n = 6 is A = {11, 18, 19, 20, 22, 25}, with S(A) = 115 and corresponding set string: 111819202225. Given that A is an optimum special sum set for n = 7, find its set string. NOTE: This problem is related to Problem 105 and Problem 106 . \href{https://projecteuler.net/problem=103}{Zadanie 103}
\end{zadanie}
\begin{zadanie}
	The Fibonacci sequence is defined by the recurrence relation: It turns out that F 541 , which contains 113 digits, is the first Fibonacci number for which the last nine digits are 1-9 pandigital (contain all the digits 1 to 9, but not necessarily in order). And F 2749 , which contains 575 digits, is the first Fibonacci number for which the first nine digits are 1-9 pandigital. Given that F k is the first Fibonacci number for which the first nine digits AND the last nine digits are 1-9 pandigital, find k . \href{https://projecteuler.net/problem=104}{Zadanie 104}
\end{zadanie}
\begin{zadanie}
	Let S(A) represent the sum of elements in set A of size n . We shall call it a special sum set if for any two non-empty disjoint subsets, B and C, the following properties are true: For example, {81, 88, 75, 42, 87, 84, 86, 65} is not a special sum set because 65 + 87 + 88 = 75 + 81 + 84, whereas {157, 150, 164, 119, 79, 159, 161, 139, 158} satisfies both rules for all possible subset pair combinations and S(A) = 1286. Using sets.txt (right click and "Save Link/Target As..."), a 4K text file with one-hundred sets containing seven to twelve elements (the two examples given above are the first two sets in the file), identify all the special sum sets, A 1 , A 2 , ..., A k , and find the value of S(A 1 ) + S(A 2 ) + ... + S(A k ). NOTE: This problem is related to Problem 103 and Problem 106 . \href{https://projecteuler.net/problem=105}{Zadanie 105}
\end{zadanie}
\begin{zadanie}
	Let S(A) represent the sum of elements in set A of size n . We shall call it a special sum set if for any two non-empty disjoint subsets, B and C, the following properties are true: For this problem we shall assume that a given set contains n strictly increasing elements and it already satisfies the second rule. Surprisingly, out of the 25 possible subset pairs that can be obtained from a set for which n = 4, only 1 of these pairs need to be tested for equality (first rule). Similarly, when n = 7, only 70 out of the 966 subset pairs need to be tested. For n = 12, how many of the 261625 subset pairs that can be obtained need to be tested for equality? NOTE: This problem is related to Problem 103 and Problem 105 . \href{https://projecteuler.net/problem=106}{Zadanie 106}
\end{zadanie}
\begin{zadanie}
	The following undirected network consists of seven vertices and twelve edges with a total weight of 243. The same network can be represented by the matrix below. However, it is possible to optimise the network by removing some edges and still ensure that all points on the network remain connected. The network which achieves the maximum saving is shown below. It has a weight of 93, representing a saving of 243 - 93 = 150 from the original network. Using network.txt (right click and 'Save Link/Target As...'), a 6K text file containing a network with forty vertices, and given in matrix form, find the maximum saving which can be achieved by removing redundant edges whilst ensuring that the network remains connected. \href{https://projecteuler.net/problem=107}{Zadanie 107}
\end{zadanie}
\begin{zadanie}
	In the following equation x , y , and n are positive integers. For n = 4 there are exactly three distinct solutions: What is the least value of n for which the number of distinct solutions exceeds one-thousand? NOTE: This problem is an easier version of Problem 110 ; it is strongly advised that you solve this one first. \href{https://projecteuler.net/problem=108}{Zadanie 108}
\end{zadanie}
\begin{zadanie}
	In the game of darts a player throws three darts at a target board which is split into twenty equal sized sections numbered one to twenty. The score of a dart is determined by the number of the region that the dart lands in. A dart landing outside the red/green outer ring scores zero. The black and cream regions inside this ring represent single scores. However, the red/green outer ring and middle ring score double and treble scores respectively. At the centre of the board are two concentric circles called the bull region, or bulls-eye. The outer bull is worth 25 points and the inner bull is a double, worth 50 points. There are many variations of rules but in the most popular game the players will begin with a score 301 or 501 and the first player to reduce their running total to zero is a winner. However, it is normal to play a "doubles out" system, which means that the player must land a double (including the double bulls-eye at the centre of the board) on their final dart to win; any other dart that would reduce their running total to one or lower means the score for that set of three darts is "bust". When a player is able to finish on their current score it is called a "checkout" and the highest checkout is 170: T20 T20 D25 (two treble 20s and double bull). There are exactly eleven distinct ways to checkout on a score of 6: Note that D1 D2 is considered different to D2 D1 as they finish on different doubles. However, the combination S1 T1 D1 is considered the same as T1 S1 D1. In addition we shall not include misses in considering combinations; for example, D3 is the same as 0 D3 and 0 0 D3. Incredibly there are 42336 distinct ways of checking out in total. How many distinct ways can a player checkout with a score less than 100? \href{https://projecteuler.net/problem=109}{Zadanie 109}
\end{zadanie}
\begin{zadanie}
	In the following equation x , y , and n are positive integers. It can be verified that when n = 1260 there are 113 distinct solutions and this is the least value of n for which the total number of distinct solutions exceeds one hundred. What is the least value of n for which the number of distinct solutions exceeds four million? NOTE: This problem is a much more difficult version of Problem 108 and as it is well beyond the limitations of a brute force approach it requires a clever implementation. \href{https://projecteuler.net/problem=110}{Zadanie 110}
\end{zadanie}
\begin{zadanie}
	Considering 4-digit primes containing repeated digits it is clear that they cannot all be the same: 1111 is divisible by 11, 2222 is divisible by 22, and so on. But there are nine 4-digit primes containing three ones: 1117, 1151, 1171, 1181, 1511, 1811, 2111, 4111, 8111 We shall say that M( n , d ) represents the maximum number of repeated digits for an n -digit prime where d is the repeated digit, N( n , d ) represents the number of such primes, and S( n , d ) represents the sum of these primes. So M(4, 1) = 3 is the maximum number of repeated digits for a 4-digit prime where one is the repeated digit, there are N(4, 1) = 9 such primes, and the sum of these primes is S(4, 1) = 22275. It turns out that for d = 0, it is only possible to have M(4, 0) = 2 repeated digits, but there are N(4, 0) = 13 such cases. In the same way we obtain the following results for 4-digit primes. For d = 0 to 9, the sum of all S(4, d ) is 273700. Find the sum of all S(10, d ). \href{https://projecteuler.net/problem=111}{Zadanie 111}
\end{zadanie}
\begin{zadanie}
	Working from left-to-right if no digit is exceeded by the digit to its left it is called an increasing number; for example, 134468. Similarly if no digit is exceeded by the digit to its right it is called a decreasing number; for example, 66420. We shall call a positive integer that is neither increasing nor decreasing a "bouncy" number; for example, 155349. Clearly there cannot be any bouncy numbers below one-hundred, but just over half of the numbers below one-thousand (525) are bouncy. In fact, the least number for which the proportion of bouncy numbers first reaches 50% is 538. Surprisingly, bouncy numbers become more and more common and by the time we reach 21780 the proportion of bouncy numbers is equal to 90%. Find the least number for which the proportion of bouncy numbers is exactly 99%. \href{https://projecteuler.net/problem=112}{Zadanie 112}
\end{zadanie}
\begin{zadanie}
	Working from left-to-right if no digit is exceeded by the digit to its left it is called an increasing number; for example, 134468. Similarly if no digit is exceeded by the digit to its right it is called a decreasing number; for example, 66420. We shall call a positive integer that is neither increasing nor decreasing a "bouncy" number; for example, 155349. As n increases, the proportion of bouncy numbers below n increases such that there are only 12951 numbers below one-million that are not bouncy and only 277032 non-bouncy numbers below 10 10 . How many numbers below a googol (10 100 ) are not bouncy? \href{https://projecteuler.net/problem=113}{Zadanie 113}
\end{zadanie}
\begin{zadanie}
	A row measuring seven units in length has red blocks with a minimum length of three units placed on it, such that any two red blocks (which are allowed to be different lengths) are separated by at least one grey square. There are exactly seventeen ways of doing this. How many ways can a row measuring fifty units in length be filled? NOTE: Although the example above does not lend itself to the possibility, in general it is permitted to mix block sizes. For example, on a row measuring eight units in length you could use red (3), grey (1), and red (4). \href{https://projecteuler.net/problem=114}{Zadanie 114}
\end{zadanie}
\begin{zadanie}
	NOTE: This is a more difficult version of Problem 114 . A row measuring n units in length has red blocks with a minimum length of m units placed on it, such that any two red blocks (which are allowed to be different lengths) are separated by at least one black square. Let the fill-count function, F( m , n ), represent the number of ways that a row can be filled. For example, F(3, 29) = 673135 and F(3, 30) = 1089155. That is, for m = 3, it can be seen that n = 30 is the smallest value for which the fill-count function first exceeds one million. In the same way, for m = 10, it can be verified that F(10, 56) = 880711 and F(10, 57) = 1148904, so n = 57 is the least value for which the fill-count function first exceeds one million. For m = 50, find the least value of n for which the fill-count function first exceeds one million. \href{https://projecteuler.net/problem=115}{Zadanie 115}
\end{zadanie}
\begin{zadanie}
	A row of five grey square tiles is to have a number of its tiles replaced with coloured oblong tiles chosen from red (length two), green (length three), or blue (length four). If red tiles are chosen there are exactly seven ways this can be done. If green tiles are chosen there are three ways. And if blue tiles are chosen there are two ways. Assuming that colours cannot be mixed there are 7 + 3 + 2 = 12 ways of replacing the grey tiles in a row measuring five units in length. How many different ways can the grey tiles in a row measuring fifty units in length be replaced if colours cannot be mixed and at least one coloured tile must be used? NOTE: This is related to Problem 117 . \href{https://projecteuler.net/problem=116}{Zadanie 116}
\end{zadanie}
\begin{zadanie}
	Using a combination of grey square tiles and oblong tiles chosen from: red tiles (measuring two units), green tiles (measuring three units), and blue tiles (measuring four units), it is possible to tile a row measuring five units in length in exactly fifteen different ways. How many ways can a row measuring fifty units in length be tiled? NOTE: This is related to Problem 116 . \href{https://projecteuler.net/problem=117}{Zadanie 117}
\end{zadanie}
\begin{zadanie}
	Using all of the digits 1 through 9 and concatenating them freely to form decimal integers, different sets can be formed. Interestingly with the set {2,5,47,89,631}, all of the elements belonging to it are prime. How many distinct sets containing each of the digits one through nine exactly once contain only prime elements? \href{https://projecteuler.net/problem=118}{Zadanie 118}
\end{zadanie}
\begin{zadanie}
	The number 512 is interesting because it is equal to the sum of its digits raised to some power: 5 + 1 + 2 = 8, and 8 3 = 512. Another example of a number with this property is 614656 = 28 4 . We shall define a n to be the n th term of this sequence and insist that a number must contain at least two digits to have a sum. You are given that a 2 = 512 and a 10 = 614656. Find a 30 . \href{https://projecteuler.net/problem=119}{Zadanie 119}
\end{zadanie}
\begin{zadanie}
	Let r be the remainder when ( a -1) n + ( a +1) n is divided by a 2 . For example, if a = 7 and n = 3, then r = 42: 6 3 + 8 3 = 728 $\equiv$ 42 mod 49. And as n varies, so too will r , but for a = 7 it turns out that r max = 42. For 3 $\leq$ a $\leq$ 1000, find an $\Sigma$ r max . \href{https://projecteuler.net/problem=120}{Zadanie 120}
\end{zadanie}
\begin{zadanie}
	A bag contains one red disc and one blue disc. In a game of chance a player takes a disc at random and its colour is noted. After each turn the disc is returned to the bag, an extra red disc is added, and another disc is taken at random. The player pays 1 to play and wins if they have taken more blue discs than red discs at the end of the game. If the game is played for four turns, the probability of a player winning is exactly 11/120, and so the maximum prize fund the banker should allocate for winning in this game would be 10 before they would expect to incur a loss. Note that any payout will be a whole number of pounds and also includes the original 1 paid to play the game, so in the example given the player actually wins 9. Find the maximum prize fund that should be allocated to a single game in which fifteen turns are played. \href{https://projecteuler.net/problem=121}{Zadanie 121}
\end{zadanie}
\begin{zadanie}
	The most naive way of computing n 15 requires fourteen multiplications: n $\times$ n $\times$ ... $\times$ n = n 15 But using a "binary" method you can compute it in six multiplications: n $\times$ n = n 2 n 2 $\times$ n 2 = n 4 n 4 $\times$ n 4 = n 8 n 8 $\times$ n 4 = n 12 n 12 $\times$ n 2 = n 14 n 14 $\times$ n = n 15 However it is yet possible to compute it in only five multiplications: n $\times$ n = n 2 n 2 $\times$ n = n 3 n 3 $\times$ n 3 = n 6 n 6 $\times$ n 6 = n 12 n 12 $\times$ n 3 = n 15 We shall define m( k ) to be the minimum number of multiplications to compute n k ; for example m(15) = 5. For 1 $\leq$ k $\leq$ 200, find an $\Sigma$ m( k ). \href{https://projecteuler.net/problem=122}{Zadanie 122}
\end{zadanie}
\begin{zadanie}
	Let p n be the n th prime: 2, 3, 5, 7, 11, ..., and let r be the remainder when ( p n -1) n + ( p n +1) n is divided by p n 2 . For example, when n = 3, p 3 = 5, and 4 3 + 6 3 = 280 $\equiv$ 5 mod 25. The least value of n for which the remainder first exceeds 10 9 is 7037. Find the least value of n for which the remainder first exceeds 10 10 . \href{https://projecteuler.net/problem=123}{Zadanie 123}
\end{zadanie}
\begin{zadanie}
	The radical of n , rad( n ), is the product of the distinct prime factors of n . For example, 504 = 2 3 $\times$ 3 2 $\times$ 7, so rad(504) = 2 $\times$ 3 $\times$ 7 = 42. If we calculate rad( n ) for 1 $\leq$ n $\leq$ 10, then sort them on rad( n ), and sorting on n if the radical values are equal, we get: Let E( k ) be the k th element in the sorted n column; for example, E(4) = 8 and E(6) = 9. If rad( n ) is sorted for 1 $\leq$ n $\leq$ 100000, find E(10000). \href{https://projecteuler.net/problem=124}{Zadanie 124}
\end{zadanie}
\begin{zadanie}
	The palindromic number 595 is interesting because it can be written as the sum of consecutive squares: 6 2 + 7 2 + 8 2 + 9 2 + 10 2 + 11 2 + 12 2 . There are exactly eleven palindromes below one-thousand that can be written as consecutive square sums, and the sum of these palindromes is 4164. Note that 1 = 0 2 + 1 2 has not been included as this problem is concerned with the squares of positive integers. Find the sum of all the numbers less than 10 8 that are both palindromic and can be written as the sum of consecutive squares. \href{https://projecteuler.net/problem=125}{Zadanie 125}
\end{zadanie}
\begin{zadanie}
	The minimum number of cubes to cover every visible face on a cuboid measuring 3x2x1 is twenty-two. If we then add a second layer to this solid it would require forty-six cubes to cover every visible face, the third layer would require seventy-eight cubes, and the fourth layer would require one-hundred and eighteen cubes to cover every visible face. However, the first layer on a cuboid measuring 5x1x1 also requires twenty-two cubes; similarly the first layer on cuboids measuring 5x3x1, 7x2x1, and 11x1x1 all contain forty-six cubes. We shall define C( n ) to represent the number of cuboids that contain n cubes in one of its layers. So C(22) = 2, C(46) = 4, C(78) = 5, and C(118) = 8. It turns out that 154 is the least value of n for which C( n ) = 10. Find the least value of n for which C( n ) = 1000. \href{https://projecteuler.net/problem=126}{Zadanie 126}
\end{zadanie}
\begin{zadanie}
	The radical of n , rad( n ), is the product of distinct prime factors of n . For example, 504 = 2 3 $\times$ 3 2 $\times$ 7, so rad(504) = 2 $\times$ 3 $\times$ 7 = 42. We shall define the triplet of positive integers ( a , b , c ) to be an abc-hit if: For example, (5, 27, 32) is an abc-hit, because: It turns out that abc-hits are quite rare and there are only thirty-one abc-hits for c < 1000, with $\Sigma$ c = 12523. Find $\Sigma$ c for c < 120000. \href{https://projecteuler.net/problem=127}{Zadanie 127}
\end{zadanie}
\begin{zadanie}
	A hexagonal tile with number 1 is surrounded by a ring of six hexagonal tiles, starting at "12 o'clock" and numbering the tiles 2 to 7 in an anti-clockwise direction. New rings are added in the same fashion, with the next rings being numbered 8 to 19, 20 to 37, 38 to 61, and so on. The diagram below shows the first three rings. By finding the difference between tile n and each of its six neighbours we shall define PD( n ) to be the number of those differences which are prime. For example, working clockwise around tile 8 the differences are 12, 29, 11, 6, 1, and 13. So PD(8) = 3. In the same way, the differences around tile 17 are 1, 17, 16, 1, 11, and 10, hence PD(17) = 2. It can be shown that the maximum value of PD( n ) is 3. If all of the tiles for which PD( n ) = 3 are listed in ascending order to form a sequence, the 10th tile would be 271. Find the 2000th tile in this sequence. \href{https://projecteuler.net/problem=128}{Zadanie 128}
\end{zadanie}
\begin{zadanie}
	A number consisting entirely of ones is called a repunit. We shall define R( k ) to be a repunit of length k ; for example, R(6) = 111111. Given that n is a positive integer and GCD( n , 10) = 1, it can be shown that there always exists a value, k , for which R( k ) is divisible by n , and let A( n ) be the least such value of k ; for example, A(7) = 6 and A(41) = 5. The least value of n for which A( n ) first exceeds ten is 17. Find the least value of n for which A( n ) first exceeds one-million. \href{https://projecteuler.net/problem=129}{Zadanie 129}
\end{zadanie}
\begin{zadanie}
	A number consisting entirely of ones is called a repunit. We shall define R( k ) to be a repunit of length k ; for example, R(6) = 111111. Given that n is a positive integer and GCD( n , 10) = 1, it can be shown that there always exists a value, k , for which R( k ) is divisible by n , and let A( n ) be the least such value of k ; for example, A(7) = 6 and A(41) = 5. You are given that for all primes, p > 5, that p - 1 is divisible by A( p ). For example, when p = 41, A(41) = 5, and 40 is divisible by 5. However, there are rare composite values for which this is also true; the first five examples being 91, 259, 451, 481, and 703. Find the sum of the first twenty-five composite values of n for which GCD( n , 10) = 1 and n - 1 is divisible by A( n ). \href{https://projecteuler.net/problem=130}{Zadanie 130}
\end{zadanie}
\begin{zadanie}
	There are some prime values, p , for which there exists a positive integer, n , such that the expression n 3 + n 2 p is a perfect cube. For example, when p = 19, 8 3 + 8 2 $\times$19 = 12 3 . What is perhaps most surprising is that for each prime with this property the value of n is unique, and there are only four such primes below one-hundred. How many primes below one million have this remarkable property? \href{https://projecteuler.net/problem=131}{Zadanie 131}
\end{zadanie}
\begin{zadanie}
	A number consisting entirely of ones is called a repunit. We shall define R( k ) to be a repunit of length k . For example, R(10) = 1111111111 = 11$\times$41$\times$271$\times$9091, and the sum of these prime factors is 9414. Find the sum of the first forty prime factors of R(10 9 ). \href{https://projecteuler.net/problem=132}{Zadanie 132}
\end{zadanie}
\begin{zadanie}
	A number consisting entirely of ones is called a repunit. We shall define R( k ) to be a repunit of length k ; for example, R(6) = 111111. Let us consider repunits of the form R(10 n ). Although R(10), R(100), or R(1000) are not divisible by 17, R(10000) is divisible by 17. Yet there is no value of n for which R(10 n ) will divide by 19. In fact, it is remarkable that 11, 17, 41, and 73 are the only four primes below one-hundred that can be a factor of R(10 n ). Find the sum of all the primes below one-hundred thousand that will never be a factor of R(10 n ). \href{https://projecteuler.net/problem=133}{Zadanie 133}
\end{zadanie}
\begin{zadanie}
	Consider the consecutive primes p 1 = 19 and p 2 = 23. It can be verified that 1219 is the smallest number such that the last digits are formed by p 1 whilst also being divisible by p 2 . In fact, with the exception of p 1 = 3 and p 2 = 5, for every pair of consecutive primes, p 2 > p 1 , there exist values of n for which the last digits are formed by p 1 and n is divisible by p 2 . Let S be the smallest of these values of n . Find $\Sigma$ S for every pair of consecutive primes with 5 $\leq$ p 1 $\leq$ 1000000. \href{https://projecteuler.net/problem=134}{Zadanie 134}
\end{zadanie}
\begin{zadanie}
	Given the positive integers, x , y , and z , are consecutive terms of an arithmetic progression, the least value of the positive integer, n , for which the equation, x 2 - y 2 - z 2 = n , has exactly two solutions is n = 27: 34 2 - 27 2 - 20 2 = 12 2 - 9 2 - 6 2 = 27 It turns out that n = 1155 is the least value which has exactly ten solutions. How many values of n less than one million have exactly ten distinct solutions? \href{https://projecteuler.net/problem=135}{Zadanie 135}
\end{zadanie}
\begin{zadanie}
	The positive integers, x , y , and z , are consecutive terms of an arithmetic progression. Given that n is a positive integer, the equation, x 2 - y 2 - z 2 = n , has exactly one solution when n = 20: 13 2 - 10 2 - 7 2 = 20 In fact there are twenty-five values of n below one hundred for which the equation has a unique solution. How many values of n less than fifty million have exactly one solution? \href{https://projecteuler.net/problem=136}{Zadanie 136}
\end{zadanie}
\begin{zadanie}
	Consider the infinite polynomial series $A\_F(x) = x F\_1 + x^2 F\_2 + x^3 F\_3 + \dots$, where $F\_k$ is the $k$th term in the Fibonacci sequence: $1, 1, 2, 3, 5, 8, \dots$; that is, $F\_k = F\_{k-1} + F\_{k-2}$, $F\_1 = 1$ and $F\_2 = 1$. For this problem we shall be interested in values of $x$ for which $A\_F(x)$ is a positive integer. The corresponding values of x for the first five natural numbers are shown below. We shall call $A\_F(x)$ a golden nugget if $x$ is rational, because they become increasingly rarer; for example, the 10th golden nugget is 74049690. Find the 15th golden nugget. \href{https://projecteuler.net/problem=137}{Zadanie 137}
\end{zadanie}
\begin{zadanie}
	Consider the isosceles triangle with base length, $b = 16$, and legs, $L = 17$. By using the Pythagorean theorem it can be seen that the height of the triangle, $h = \sqrt{17^2 - 8^2} = 15$, which is one less than the base length. With $b = 272$ and $L = 305$, we get $h = 273$, which is one more than the base length, and this is the second smallest isosceles triangle with the property that $h = b \pm 1$. Find $\sum L$ for the twelve smallest isosceles triangles for which $h = b \pm 1$ and $b$, $L$ are positive integers. \href{https://projecteuler.net/problem=138}{Zadanie 138}
\end{zadanie}
\begin{zadanie}
	Let ( a , b , c ) represent the three sides of a right angle triangle with integral length sides. It is possible to place four such triangles together to form a square with length c . For example, (3, 4, 5) triangles can be placed together to form a 5 by 5 square with a 1 by 1 hole in the middle and it can be seen that the 5 by 5 square can be tiled with twenty-five 1 by 1 squares. However, if (5, 12, 13) triangles were used then the hole would measure 7 by 7 and these could not be used to tile the 13 by 13 square. Given that the perimeter of the right triangle is less than one-hundred million, how many Pythagorean triangles would allow such a tiling to take place? \href{https://projecteuler.net/problem=139}{Zadanie 139}
\end{zadanie}
\begin{zadanie}
	Consider the infinite polynomial series $A\_G(x) = x G\_1 + x^2 G\_2 + x^3 G\_3 + \cdots$, where $G\_k$ is the $k$th term of the second order recurrence relation $G\_k = G\_{k-1} + G\_{k-2}$, $G\_1 = 1$ and $G\_2 = 4$; that is, $1, 4, 5, 9, 14, 23, \dots$ . For this problem we shall be concerned with values of $x$ for which $A\_G(x)$ is a positive integer. The corresponding values of $x$ for the first five natural numbers are shown below. We shall call $A\_G(x)$ a golden nugget if $x$ is rational, because they become increasingly rarer; for example, the 20th golden nugget is 211345365. Find the sum of the first thirty golden nuggets. \href{https://projecteuler.net/problem=140}{Zadanie 140}
\end{zadanie}
\begin{zadanie}
	A positive integer, n , is divided by d and the quotient and remainder are q and r respectively. In addition d , q , and r are consecutive positive integer terms in a geometric sequence, but not necessarily in that order. For example, 58 divided by 6 has quotient 9 and remainder 4. It can also be seen that 4, 6, 9 are consecutive terms in a geometric sequence (common ratio 3/2). We will call such numbers, n , progressive. Some progressive numbers, such as 9 and 10404 = 102 2 , happen to also be perfect squares. The sum of all progressive perfect squares below one hundred thousand is 124657. Find the sum of all progressive perfect squares below one trillion (10 12 ). \href{https://projecteuler.net/problem=141}{Zadanie 141}
\end{zadanie}
\begin{zadanie}
	Find the smallest x + y + z with integers x > y > z > 0 such that x + y, x - y, x + z, x - z, y + z, y - z are all perfect squares. \href{https://projecteuler.net/problem=142}{Zadanie 142}
\end{zadanie}
\begin{zadanie}
	Let ABC be a triangle with all interior angles being less than 120 degrees. Let X be any point inside the triangle and let XA = p, XC = q, and XB = r. Fermat challenged Torricelli to find the position of X such that p + q + r was minimised. Torricelli was able to prove that if equilateral triangles AOB, BNC and AMC are constructed on each side of triangle ABC, the circumscribed circles of AOB, BNC, and AMC will intersect at a single point, T, inside the triangle. Moreover he proved that T, called the Torricelli/Fermat point, minimises p + q + r. Even more remarkable, it can be shown that when the sum is minimised, AN = BM = CO = p + q + r and that AN, BM and CO also intersect at T. If the sum is minimised and a, b, c, p, q and r are all positive integers we shall call triangle ABC a Torricelli triangle. For example, a = 399, b = 455, c = 511 is an example of a Torricelli triangle, with p + q + r = 784. Find the sum of all distinct values of p + q + r $\leq$ 120000 for Torricelli triangles. \href{https://projecteuler.net/problem=143}{Zadanie 143}
\end{zadanie}
\begin{zadanie}
	In laser physics, a "white cell" is a mirror system that acts as a delay line for the laser beam. The beam enters the cell, bounces around on the mirrors, and eventually works its way back out. The specific white cell we will be considering is an ellipse with the equation 4 x 2 + y 2 = 100 The section corresponding to -0.01 $\leq$ x $\leq$ +0.01 at the top is missing, allowing the light to enter and exit through the hole. The light beam in this problem starts at the point (0.0,10.1) just outside the white cell, and the beam first impacts the mirror at (1.4,-9.6). Each time the laser beam hits the surface of the ellipse, it follows the usual law of reflection "angle of incidence equals angle of reflection." That is, both the incident and reflected beams make the same angle with the normal line at the point of incidence. In the figure on the left, the red line shows the first two points of contact between the laser beam and the wall of the white cell; the blue line shows the line tangent to the ellipse at the point of incidence of the first bounce. The slope m of the tangent line at any point ( x , y ) of the given ellipse is: m = -4 x / y The normal line is perpendicular to this tangent line at the point of incidence. The animation on the right shows the first 10 reflections of the beam. How many times does the beam hit the internal surface of the white cell before exiting? \href{https://projecteuler.net/problem=144}{Zadanie 144}
\end{zadanie}
\begin{zadanie}
	Some positive integers n have the property that the sum [ n + reverse( n ) ] consists entirely of odd (decimal) digits. For instance, 36 + 63 = 99 and 409 + 904 = 1313. We will call such numbers reversible ; so 36, 63, 409, and 904 are reversible. Leading zeroes are not allowed in either n or reverse( n ). There are 120 reversible numbers below one-thousand. How many reversible numbers are there below one-billion (10 9 )? \href{https://projecteuler.net/problem=145}{Zadanie 145}
\end{zadanie}
\begin{zadanie}
	The smallest positive integer n for which the numbers n 2 +1, n 2 +3, n 2 +7, n 2 +9, n 2 +13, and n 2 +27 are consecutive primes is 10. The sum of all such integers n below one-million is 1242490. What is the sum of all such integers n below 150 million? \href{https://projecteuler.net/problem=146}{Zadanie 146}
\end{zadanie}
\begin{zadanie}
	In a 3x2 cross-hatched grid, a total of 37 different rectangles could be situated within that grid as indicated in the sketch. There are 5 grids smaller than 3x2, vertical and horizontal dimensions being important, i.e. 1x1, 2x1, 3x1, 1x2 and 2x2. If each of them is cross-hatched, the following number of different rectangles could be situated within those smaller grids: Adding those to the 37 of the 3x2 grid, a total of 72 different rectangles could be situated within 3x2 and smaller grids. How many different rectangles could be situated within 47x43 and smaller grids? \href{https://projecteuler.net/problem=147}{Zadanie 147}
\end{zadanie}
\begin{zadanie}
	We can easily verify that none of the entries in the first seven rows of Pascal's triangle are divisible by 7: However, if we check the first one hundred rows, we will find that only 2361 of the 5050 entries are not divisible by 7. Find the number of entries which are not divisible by 7 in the first one billion (10 9 ) rows of Pascal's triangle. \href{https://projecteuler.net/problem=148}{Zadanie 148}
\end{zadanie}
\begin{zadanie}
	Looking at the table below, it is easy to verify that the maximum possible sum of adjacent numbers in any direction (horizontal, vertical, diagonal or anti-diagonal) is 16 (= 8 + 7 + 1). Now, let us repeat the search, but on a much larger scale: First, generate four million pseudo-random numbers using a specific form of what is known as a "Lagged Fibonacci Generator": For 1 $\leq$ k $\leq$ 55, s k = [100003 - 200003 k + 300007 k 3 ] (modulo 1000000) - 500000. For 56 $\leq$ k $\leq$ 4000000, s k = [ s k-24 + s k-55 + 1000000] (modulo 1000000) - 500000. Thus, s 10 = -393027 and s 100 = 86613. The terms of s are then arranged in a 2000$\times$2000 table, using the first 2000 numbers to fill the first row (sequentially), the next 2000 numbers to fill the second row, and so on. Finally, find the greatest sum of (any number of) adjacent entries in any direction (horizontal, vertical, diagonal or anti-diagonal). \href{https://projecteuler.net/problem=149}{Zadanie 149}
\end{zadanie}
\begin{zadanie}
	In a triangular array of positive and negative integers, we wish to find a sub-triangle such that the sum of the numbers it contains is the smallest possible. In the example below, it can be easily verified that the marked triangle satisfies this condition having a sum of -42. We wish to make such a triangular array with one thousand rows, so we generate 500500 pseudo-random numbers s k in the range $\pm$2 19 , using a type of random number generator (known as a Linear Congruential Generator) as follows: t := 0 for k = 1 up to k = 500500: t := (615949* t + 797807) modulo 2 20 s k := t -2 19 Thus: s 1 = 273519, s 2 = -153582, s 3 = 450905 etc Our triangular array is then formed using the pseudo-random numbers thus: Sub-triangles can start at any element of the array and extend down as far as we like (taking-in the two elements directly below it from the next row, the three elements directly below from the row after that, and so on). The "sum of a sub-triangle" is defined as the sum of all the elements it contains. Find the smallest possible sub-triangle sum. \href{https://projecteuler.net/problem=150}{Zadanie 150}
\end{zadanie}
\begin{zadanie}
	A printing shop runs 16 batches (jobs) every week and each batch requires a sheet of special colour-proofing paper of size A5. Every Monday morning, the foreman opens a new envelope, containing a large sheet of the special paper with size A1. He proceeds to cut it in half, thus getting two sheets of size A2. Then he cuts one of them in half to get two sheets of size A3 and so on until he obtains the A5-size sheet needed for the first batch of the week. All the unused sheets are placed back in the envelope. At the beginning of each subsequent batch, he takes from the envelope one sheet of paper at random. If it is of size A5, he uses it. If it is larger, he repeats the 'cut-in-half' procedure until he has what he needs and any remaining sheets are always placed back in the envelope. Excluding the first and last batch of the week, find the expected number of times (during each week) that the foreman finds a single sheet of paper in the envelope. Give your answer rounded to six decimal places using the format x.xxxxxx. \href{https://projecteuler.net/problem=151}{Zadanie 151}
\end{zadanie}
\begin{zadanie}
	There are several ways to write the number 1/2 as a sum of inverse squares using distinct integers. For instance, the numbers {2,3,4,5,7,12,15,20,28,35} can be used: $$\dfrac{1}{2} = \dfrac{1}{2^2} + \dfrac{1}{3^2} + \dfrac{1}{4^2} + \dfrac{1}{5^2} +\\
		\quad \dfrac{1}{7^2} + \dfrac{1}{12^2} + \dfrac{1}{15^2} + \dfrac{1}{20^2} +\\
		\quad \dfrac{1}{28^2} + \dfrac{1}{35^2}$$ In fact, only using integers between 2 and 45 inclusive, there are exactly three ways to do it, the remaining two being: {2,3,4,6,7,9,10,20,28,35,36,45} and {2,3,4,6,7,9,12,15,28,30,35,36,45}. How many ways are there to write the number 1/2 as a sum of inverse squares using distinct integers between 2 and 80 inclusive? \href{https://projecteuler.net/problem=152}{Zadanie 152}
\end{zadanie}
\begin{zadanie}
	As we all know the equation x 2 =-1 has no solutions for real x . If we however introduce the imaginary number i this equation has two solutions: x=i and x=-i . If we go a step further the equation ( x -3) 2 =-4 has two complex solutions: x =3+2 i and x =3-2 i . x =3+2 i and x =3-2 i are called each others' complex conjugate. Numbers of the form a + bi are called complex numbers. In general a + bi and a - bi are each other's complex conjugate. A Gaussian Integer is a complex number a + bi such that both a and b are integers. The regular integers are also Gaussian integers (with b =0). To distinguish them from Gaussian integers with b $\neq$ 0 we call such integers "rational integers." A Gaussian integer is called a divisor of a rational integer n if the result is also a Gaussian integer. If for example we divide 5 by 1+2 i we can simplify $\dfrac{5}{1 + 2i}$ in the following manner: Multiply numerator and denominator by the complex conjugate of 1+2 i : 1-2 i . The result is $\dfrac{5}{1 + 2i} = \dfrac{5}{1 + 2i}\dfrac{1 - 2i}{1 - 2i} = \dfrac{5(1 - 2i)}{1 - (2i)^2} = \dfrac{5(1 - 2i)}{1 - (-4)} = \dfrac{5(1 - 2i)}{5} = 1 - 2i$. So 1+2 i is a divisor of 5. Note that 1+ i is not a divisor of 5 because $\dfrac{5}{1 + i} = \dfrac{5}{2} - \dfrac{5}{2}i$. Note also that if the Gaussian Integer ( a + bi ) is a divisor of a rational integer n , then its complex conjugate ( a - bi ) is also a divisor of n . In fact, 5 has six divisors such that the real part is positive: {1, 1 + 2 i , 1 - 2 i , 2 + i , 2 - i , 5}. The following is a table of all of the divisors for the first five positive rational integers: For divisors with positive real parts, then, we have: $\sum \limits\_{n = 1}^{5} {s(n)} = 35$. For $\sum \limits\_{n = 1}^{10^5} {s(n)} = 17924657155$. What is $\sum \limits\_{n = 1}^{10^8} {s(n)}$? \href{https://projecteuler.net/problem=153}{Zadanie 153}
\end{zadanie}
\begin{zadanie}
	A triangular pyramid is constructed using spherical balls so that each ball rests on exactly three balls of the next lower level. Then, we calculate the number of paths leading from the apex to each position: A path starts at the apex and progresses downwards to any of the three spheres directly below the current position. Consequently, the number of paths to reach a certain position is the sum of the numbers immediately above it (depending on the position, there are up to three numbers above it). The result is Pascal's pyramid and the numbers at each level n are the coefficients of the trinomial expansion ( x + y + z ) n . How many coefficients in the expansion of ( x + y + z ) 200000 are multiples of 10 12 ? \href{https://projecteuler.net/problem=154}{Zadanie 154}
\end{zadanie}
\begin{zadanie}
	An electric circuit uses exclusively identical capacitors of the same value C. The capacitors can be connected in series or in parallel to form sub-units, which can then be connected in series or in parallel with other capacitors or other sub-units to form larger sub-units, and so on up to a final circuit. Using this simple procedure and up to n identical capacitors, we can make circuits having a range of different total capacitances. For example, using up to n =3 capacitors of 60 F each, we can obtain the following 7 distinct total capacitance values: If we denote by D ( n ) the number of distinct total capacitance values we can obtain when using up to n equal-valued capacitors and the simple procedure described above, we have: D (1)=1, D (2)=3, D (3)=7 ... Find D (18). Reminder : When connecting capacitors C 1 , C 2 etc in parallel, the total capacitance is C T =C 1 +C 2 +..., whereas when connecting them in series, the overall capacitance is given by: $\dfrac{1}{C\_T} = \dfrac{1}{C\_1} + \dfrac{1}{C\_2} + ...$ \href{https://projecteuler.net/problem=155}{Zadanie 155}
\end{zadanie}
\begin{zadanie}
	Starting from zero the natural numbers are written down in base 10 like this: 0 1 2 3 4 5 6 7 8 9 10 11 12.... Consider the digit d =1. After we write down each number n , we will update the number of ones that have occurred and call this number f ( n ,1). The first values for f ( n ,1), then, are as follows: Note that f ( n ,1) never equals 3. So the first two solutions of the equation f ( n ,1)= n are n =0 and n =1. The next solution is n =199981. In the same manner the function f ( n,d ) gives the total number of digits d that have been written down after the number n has been written. In fact, for every digit d $\neq$ 0, 0 is the first solution of the equation f ( n,d )= n . Let s ( d ) be the sum of all the solutions for which f ( n,d )= n . You are given that s (1)=22786974071. Find $\Sigma$ s ( d ) for 1 $\leq$ d $\leq$ 9. Note: if, for some n , f ( n,d )= n for more than one value of d this value of n is counted again for every value of d for which f ( n,d )= n . \href{https://projecteuler.net/problem=156}{Zadanie 156}
\end{zadanie}
\begin{zadanie}
	Consider the diophantine equation 1 / a + 1 / b = p / 10 n with a, b, p, n positive integers and a $\leq$ b . For n =1 this equation has 20 solutions that are listed below: How many solutions has this equation for 1 $\leq$ n $\leq$ 9? \href{https://projecteuler.net/problem=157}{Zadanie 157}
\end{zadanie}
\begin{zadanie}
	Taking three different letters from the 26 letters of the alphabet, character strings of length three can be formed. Examples are 'abc', 'hat' and 'zyx'. When we study these three examples we see that for 'abc' two characters come lexicographically after its neighbour to the left. For 'hat' there is exactly one character that comes lexicographically after its neighbour to the left. For 'zyx' there are zero characters that come lexicographically after its neighbour to the left. In all there are 10400 strings of length 3 for which exactly one character comes lexicographically after its neighbour to the left. We now consider strings of n $\leq$ 26 different characters from the alphabet. For every n , p( n ) is the number of strings of length n for which exactly one character comes lexicographically after its neighbour to the left. What is the maximum value of p( n )? \href{https://projecteuler.net/problem=158}{Zadanie 158}
\end{zadanie}
\begin{zadanie}
	A composite number can be factored many different ways. For instance, not including multiplication by one, 24 can be factored in 7 distinct ways: Recall that the digital root of a number, in base 10, is found by adding together the digits of that number, and repeating that process until a number is arrived at that is less than 10. Thus the digital root of 467 is 8. We shall call a Digital Root Sum (DRS) the sum of the digital roots of the individual factors of our number. The chart below demonstrates all of the DRS values for 24. The maximum Digital Root Sum of 24 is 11. The function mdrs( n ) gives the maximum Digital Root Sum of n . So mdrs(24)=11. Find $\Sigma$ mdrs( n ) for 1 < n < 1,000,000. \href{https://projecteuler.net/problem=159}{Zadanie 159}
\end{zadanie}
\begin{zadanie}
	For any N, let f(N) be the last five digits before the trailing zeroes in N!. For example, 9! = 362880 so f(9)=36288 10! = 3628800 so f(10)=36288 20! = 2432902008176640000 so f(20)=17664 Find f(1,000,000,000,000) \href{https://projecteuler.net/problem=160}{Zadanie 160}
\end{zadanie}
\begin{zadanie}
	A triomino is a shape consisting of three squares joined via the edges.
	There are two basic forms: If all possible orientations are taken into account there are six: Any n by m grid for which nxm is divisible by 3 can be tiled with triominoes. If we consider tilings that can be obtained by reflection or rotation from another tiling as different there are 41 ways a 2 by 9 grid can be tiled with triominoes: In how many ways can a 9 by 12 grid be tiled in this way by triominoes? \href{https://projecteuler.net/problem=161}{Zadanie 161}
\end{zadanie}
\begin{zadanie}
	In the hexadecimal number system numbers are represented using 16 different digits: The hexadecimal number AF when written in the decimal number system equals 10x16+15=175. In the 3-digit hexadecimal numbers 10A, 1A0, A10, and A01 the digits 0,1 and A are all present. Like numbers written in base ten we write hexadecimal numbers without leading zeroes. How many hexadecimal numbers containing at most sixteen hexadecimal digits exist with all of the digits 0,1, and A present at least once? Give your answer as a hexadecimal number. (A,B,C,D,E and F in upper case, without any leading or trailing code that marks the number as hexadecimal and without leading zeroes , e.g. 1A3F and not: 1a3f and not 0x1a3f and not \$1A3F and not \#1A3F and not 0000001A3F) \href{https://projecteuler.net/problem=162}{Zadanie 162}
\end{zadanie}
\begin{zadanie}
	Consider an equilateral triangle in which straight lines are drawn from each vertex to the middle of the opposite side, such as in the size 1 triangle in the sketch below. Sixteen triangles of either different shape or size or orientation or location can now be observed in that triangle. Using size 1 triangles as building blocks, larger triangles can be formed, such as the size 2 triangle in the above sketch. One-hundred and four triangles of either different shape or size or orientation or location can now be observed in that size 2 triangle. It can be observed that the size 2 triangle contains 4 size 1 triangle building blocks. A size 3 triangle would contain 9 size 1 triangle building blocks and a size n triangle would thus contain n 2 size 1 triangle building blocks. If we denote T( n ) as the number of triangles present in a triangle of size n , then T(1) = 16 T(2) = 104 Find T(36). \href{https://projecteuler.net/problem=163}{Zadanie 163}
\end{zadanie}
\begin{zadanie}
	How many 20 digit numbers n (without any leading zero) exist such that no three consecutive digits of n have a sum greater than 9? \href{https://projecteuler.net/problem=164}{Zadanie 164}
\end{zadanie}
\begin{zadanie}
	A segment is uniquely defined by its two endpoints. By considering two line segments in plane geometry there are three possibilities: the segments have zero points, one point, or infinitely many points in common. Moreover when two segments have exactly one point in common it might be the case that that common point is an endpoint of either one of the segments or of both. If a common point of two segments is not an endpoint of either of the segments it is an interior point of both segments. We will call a common point T of two segments L 1 and L 2 a true intersection point of L 1 and L 2 if T is the only common point of L 1 and L 2 and T is an interior point of both segments. Consider the three segments L 1 , L 2 , and L 3 : L 1 : (27, 44) to (12, 32) L 2 : (46, 53) to (17, 62) L 3 : (46, 70) to (22, 40) It can be verified that line segments L 2 and L 3 have a true intersection point. We note that as the one of the end points of L 3 : (22,40) lies on L 1 this is not considered to be a true point of intersection. L 1 and L 2 have no common point. So among the three line segments, we find one true intersection point. Now let us do the same for 5000 line segments. To this end, we generate 20000 numbers using the so-called "Blum Blum Shub" pseudo-random number generator. s 0 = 290797 s n+1 = s n $\times$s n (modulo 50515093) t n = s n (modulo 500) To create each line segment, we use four consecutive numbers t n . That is, the first line segment is given by: (t 1 , t 2 ) to (t 3 , t 4 ) The first four numbers computed according to the above generator should be: 27, 144, 12 and 232. The first segment would thus be (27,144) to (12,232). How many distinct true intersection points are found among the 5000 line segments? \href{https://projecteuler.net/problem=165}{Zadanie 165}
\end{zadanie}
\begin{zadanie}
	A 4x4 grid is filled with digits d, 0 $\leq$ d $\leq$ 9. It can be seen that in the grid 6 3 3 0 5 0 4 3 0 7 1 4 1 2 4 5 the sum of each row and each column has the value 12. Moreover the sum of each diagonal is also 12. In how many ways can you fill a 4x4 grid with the digits d, 0 $\leq$ d $\leq$ 9 so that each row, each column, and both diagonals have the same sum? \href{https://projecteuler.net/problem=166}{Zadanie 166}
\end{zadanie}
\begin{zadanie}
	For two positive integers a and b, the Ulam sequence U(a,b) is defined by U(a,b) 1 = a, U(a,b) 2 = b and for k > 2,
	U(a,b) k is the smallest integer greater than U(a,b) (k-1) which can be written in exactly one way as the sum of two distinct previous members of U(a,b). For example, the sequence U(1,2) begins with 1, 2, 3 = 1 + 2, 4 = 1 + 3, 6 = 2 + 4, 8 = 2 + 6, 11 = 3 + 8; 5 does not belong to it because 5 = 1 + 4 = 2 + 3 has two representations as the sum of two previous members, likewise 7 = 1 + 6 = 3 + 4. Find $\Sigma$ U(2,2 n +1) k for 2 $\leq$ n $\leq$10, where k = 10 11 . \href{https://projecteuler.net/problem=167}{Zadanie 167}
\end{zadanie}
\begin{zadanie}
	Consider the number 142857. We can right-rotate this number by moving the last digit (7) to the front of it, giving us 714285. It can be verified that 714285=5$\times$142857. This demonstrates an unusual property of 142857: it is a divisor of its right-rotation. Find the last 5 digits of the sum of all integers n , 10 < n < 10 100 , that have this property. \href{https://projecteuler.net/problem=168}{Zadanie 168}
\end{zadanie}
\begin{zadanie}
	Define f(0)=1 and f( n ) to be the number of different ways n can be expressed as a sum of integer powers of 2 using each power no more than twice. For example, f(10)=5 since there are five different ways to express 10: 1 + 1 + 8 1 + 1 + 4 + 4 1 + 1 + 2 + 2 + 4 2 + 4 + 4 2 + 8 What is f(10 25 )? \href{https://projecteuler.net/problem=169}{Zadanie 169}
\end{zadanie}
\begin{zadanie}
	Take the number 6 and multiply it by each of 1273 and 9854: 6 $\times$ 1273 = 7638 6 $\times$ 9854 = 59124 By concatenating these products we get the 1 to 9 pandigital 763859124. We will call 763859124 the "concatenated product of 6 and (1273,9854)". Notice too, that the concatenation of the input numbers, 612739854, is also 1 to 9 pandigital. The same can be done for 0 to 9 pandigital numbers. What is the largest 0 to 9 pandigital 10-digit concatenated product of an integer with two or more other integers, such that the concatenation of the input numbers is also a 0 to 9 pandigital 10-digit number? \href{https://projecteuler.net/problem=170}{Zadanie 170}
\end{zadanie}
\begin{zadanie}
	For a positive integer n , let f( n ) be the sum of the squares of the digits (in base 10) of n , e.g. f(3) = 3 2 = 9, f(25) = 2 2 + 5 2 = 4 + 25 = 29, f(442) = 4 2 + 4 2 + 2 2 = 16 + 16 + 4 = 36 Find the last nine digits of the sum of all n , 0 < n < 10 20 , such that f( n ) is a perfect square. \href{https://projecteuler.net/problem=171}{Zadanie 171}
\end{zadanie}
\begin{zadanie}
	How many 18-digit numbers n (without leading zeros) are there such that no digit occurs more than three times in n ? \href{https://projecteuler.net/problem=172}{Zadanie 172}
\end{zadanie}
\begin{zadanie}
	We shall define a square lamina to be a square outline with a square "hole" so that the shape possesses vertical and horizontal symmetry. For example, using exactly thirty-two square tiles we can form two different square laminae: With one-hundred tiles, and not necessarily using all of the tiles at one time, it is possible to form forty-one different square laminae. Using up to one million tiles how many different square laminae can be formed? \href{https://projecteuler.net/problem=173}{Zadanie 173}
\end{zadanie}
\begin{zadanie}
	We shall define a square lamina to be a square outline with a square "hole" so that the shape possesses vertical and horizontal symmetry. Given eight tiles it is possible to form a lamina in only one way: 3x3 square with a 1x1 hole in the middle. However, using thirty-two tiles it is possible to form two distinct laminae. If t represents the number of tiles used, we shall say that t = 8 is type L(1) and t = 32 is type L(2). Let N( n ) be the number of t $\leq$ 1000000 such that t is type L( n ); for example, N(15) = 832. What is $\Sigma$ N( n ) for 1 $\leq$ n $\leq$ 10? \href{https://projecteuler.net/problem=174}{Zadanie 174}
\end{zadanie}
\begin{zadanie}
	\href{https://projecteuler.net/problem=175}{Zadanie 175}
\end{zadanie}
\begin{zadanie}
	The four right-angled triangles with sides (9,12,15), (12,16,20), (5,12,13) and (12,35,37) all have one of the shorter sides (catheti) equal to 12. It can be shown that no other integer sided right-angled triangle exists with one of the catheti equal to 12. Find the smallest integer that can be the length of a cathetus of exactly 47547 different integer sided right-angled triangles. \href{https://projecteuler.net/problem=176}{Zadanie 176}
\end{zadanie}
\begin{zadanie}
	Let ABCD be a convex quadrilateral, with diagonals AC and BD. At each vertex the diagonal makes an angle with each of the two sides, creating eight corner angles. For example, at vertex A, the two angles are CAD, CAB. We call such a quadrilateral for which all eight corner angles have integer values when measured in degrees an "integer angled quadrilateral". An example of an integer angled quadrilateral is a square, where all eight corner angles are 45. Another example is given by DAC = 20, BAC = 60, ABD = 50, CBD = 30, BCA = 40, DCA = 30, CDB = 80, ADB = 50. What is the total number of non-similar integer angled quadrilaterals? Note: In your calculations you may assume that a calculated angle is integral if it is within a tolerance of 10 -9 of an integer value. \href{https://projecteuler.net/problem=177}{Zadanie 177}
\end{zadanie}
\begin{zadanie}
	\href{https://projecteuler.net/problem=178}{Zadanie 178}
\end{zadanie}
\begin{zadanie}
	Find the number of integers 1 < n < 10 7 , for which n and n + 1 have the same number of positive divisors. For example, 14 has the positive divisors 1, 2, 7, 14 while 15 has 1, 3, 5, 15. \href{https://projecteuler.net/problem=179}{Zadanie 179}
\end{zadanie}
\begin{zadanie}
	For any integer n , consider the three functions f 1, n ( x , y , z ) = x n +1 + y n +1 - z n +1 f 2, n ( x , y , z ) = ( xy + yz + zx )*( x n -1 + y n -1 - z n -1 ) f 3, n ( x , y , z ) = xyz *( x n -2 + y n -2 - z n -2 ) and their combination f n ( x , y , z ) = f 1, n ( x , y , z ) + f 2, n ( x , y , z ) - f 3, n ( x , y , z ) We call ( x , y , z ) a golden triple of order k if x , y , and z are all rational numbers of the form a / b with 0 < a < b $\leq$ k and there is (at least) one integer n , so that f n ( x , y , z ) = 0. Let s ( x , y , z ) = x + y + z . Let t = u / v be the sum of all distinct s ( x , y , z ) for all golden triples ( x , y , z ) of order 35. All the s ( x , y , z ) and t must be in reduced form. Find u + v . \href{https://projecteuler.net/problem=180}{Zadanie 180}
\end{zadanie}
\begin{zadanie}
	Having three black objects B and one white object W they can be grouped in 7 ways like this: In how many ways can sixty black objects B and forty white objects W be thus grouped? \href{https://projecteuler.net/problem=181}{Zadanie 181}
\end{zadanie}
\begin{zadanie}
	The RSA encryption is based on the following procedure: Generate two distinct primes p and q . Compute n=pq and $\phi$=( p -1)( q -1). Find an integer e , 1< e <$\phi$, such that gcd( e ,$\phi$)=1. A message in this system is a number in the interval [0, n -1]. A text to be encrypted is then somehow converted to messages (numbers in the interval [0, n -1]). To encrypt the text, for each message, m , c = m e mod n is calculated. To decrypt the text, the following procedure is needed: calculate d such that ed =1 mod $\phi$, then for each encrypted message, c , calculate m=c d mod n . There exist values of e and m such that m e mod n=m . We call messages m for which m e mod n=m unconcealed messages. An issue when choosing e is that there should not be too many unconcealed messages. For instance, let p =19 and q =37. Then n =19*37=703 and $\phi$=18*36=648. If we choose e =181, then, although gcd(181,648)=1 it turns out that all possible messages m (0$\leq$ m $\leq$ n -1) are unconcealed when calculating m e mod n . For any valid choice of e there exist some unconcealed messages. It's important that the number of unconcealed messages is at a minimum. Choose p =1009 and q =3643. Find the sum of all values of e , 1< e <$\phi$(1009,3643) and gcd( e ,$\phi$)=1, so that the number of unconcealed messages for this value of e is at a minimum. \href{https://projecteuler.net/problem=182}{Zadanie 182}
\end{zadanie}
\begin{zadanie}
	Let N be a positive integer and let N be split into k equal parts, r = N/ k , so that N = r + r + ... + r . Let P be the product of these parts, P = r $\times$ r $\times$ ... $\times$ r = r k . For example, if 11 is split into five equal parts, 11 = 2.2 + 2.2 + 2.2 + 2.2 + 2.2, then P = 2.2 5 = 51.53632. Let M(N) = P max for a given value of N. It turns out that the maximum for N = 11 is found by splitting eleven into four equal parts which leads to P max = (11/4) 4 ; that is, M(11) = 14641/256 = 57.19140625, which is a terminating decimal. However, for N = 8 the maximum is achieved by splitting it into three equal parts, so M(8) = 512/27, which is a non-terminating decimal. Let D(N) = N if M(N) is a non-terminating decimal and D(N) = -N if M(N) is a terminating decimal. For example, $\Sigma$ D(N) for 5 $\leq$ N $\leq$ 100 is 2438. Find $\Sigma$ D(N) for 5 $\leq$ N $\leq$ 10000. \href{https://projecteuler.net/problem=183}{Zadanie 183}
\end{zadanie}
\begin{zadanie}
	Consider the set I r of points ( x , y ) with integer co-ordinates in the interior of the circle with radius r , centered at the origin, i.e. x 2 + y 2 < r 2 . For a radius of 2, I 2 contains the nine points (0,0), (1,0), (1,1), (0,1), (-1,1), (-1,0), (-1,-1), (0,-1) and (1,-1). There are eight triangles having all three vertices in I 2 which contain the origin in the interior. Two of them are shown below, the others are obtained from these by rotation. For a radius of 3, there are 360 triangles containing the origin in the interior and having all vertices in I 3 and for I 5 the number is 10600. How many triangles are there containing the origin in the interior and having all three vertices in I 105 ? \href{https://projecteuler.net/problem=184}{Zadanie 184}
\end{zadanie}
\begin{zadanie}
	The game Number Mind is a variant of the well known game Master Mind. Instead of coloured pegs, you have to guess a secret sequence of digits. After each guess you're only told in how many places you've guessed the correct digit. So, if the sequence was 1234 and you guessed 2036, you'd be told that you have one correct digit; however, you would NOT be told that you also have another digit in the wrong place. For instance, given the following guesses for a 5-digit secret sequence, 90342 ;2 correct 70794 ;0 correct 39458 ;2 correct 34109 ;1 correct 51545 ;2 correct 12531 ;1 correct The correct sequence 39542 is unique. Based on the following guesses, 5616185650518293 ;2 correct 3847439647293047 ;1 correct 5855462940810587 ;3 correct 9742855507068353 ;3 correct 4296849643607543 ;3 correct 3174248439465858 ;1 correct 4513559094146117 ;2 correct 7890971548908067 ;3 correct 8157356344118483 ;1 correct 2615250744386899 ;2 correct 8690095851526254 ;3 correct 6375711915077050 ;1 correct 6913859173121360 ;1 correct 6442889055042768 ;2 correct 2321386104303845 ;0 correct 2326509471271448 ;2 correct 5251583379644322 ;2 correct 1748270476758276 ;3 correct 4895722652190306 ;1 correct 3041631117224635 ;3 correct 1841236454324589 ;3 correct 2659862637316867 ;2 correct Find the unique 16-digit secret sequence. \href{https://projecteuler.net/problem=185}{Zadanie 185}
\end{zadanie}
\begin{zadanie}
	Here are the records from a busy telephone system with one million users: The telephone number of the caller and the called number in record n are Caller(n) = S 2n-1 and Called(n) = S 2n where S 1,2,3,... come from the "Lagged Fibonacci Generator": For 1 $\leq$ k $\leq$ 55, S k = [100003 - 200003k + 300007k 3 ] (modulo 1000000) For 56 $\leq$ k, S k = [S k-24 + S k-55 ] (modulo 1000000) If Caller(n) = Called(n) then the user is assumed to have misdialled and the call fails; otherwise the call is successful. From the start of the records, we say that any pair of users X and Y are friends if X calls Y or vice-versa. Similarly, X is a friend of a friend of Z if X is a friend of Y and Y is a friend of Z; and so on for longer chains. The Prime Minister's phone number is 524287. After how many successful calls, not counting misdials, will 99% of the users (including the PM) be a friend, or a friend of a friend etc., of the Prime Minister? \href{https://projecteuler.net/problem=186}{Zadanie 186}
\end{zadanie}
\begin{zadanie}
	A composite is a number containing at least two prime factors. For example, 15 = 3 $\times$ 5; 9 = 3 $\times$ 3; 12 = 2 $\times$ 2 $\times$ 3. There are ten composites below thirty containing precisely two, not necessarily distinct, prime factors:
	4, 6, 9, 10, 14, 15, 21, 22, 25, 26. How many composite integers, n < 10 8 , have precisely two, not necessarily distinct, prime factors? \href{https://projecteuler.net/problem=187}{Zadanie 187}
\end{zadanie}
\begin{zadanie}
	The hyperexponentiation or tetration of a number a by a positive integer b, denoted by ab or b a, is recursively defined by: a1 = a, a(k+1) = a (ak) . Thus we have e.g. 32 = 3 3 = 27, hence 33 = 3 27 = 7625597484987 and 34 is roughly $10 3.6383346400240996*10^12$ . Find the last 8 digits of 17771855. \href{https://projecteuler.net/problem=188}{Zadanie 188}
\end{zadanie}
\begin{zadanie}
	Consider the following configuration of 64 triangles: We wish to colour the interior of each triangle with one of three colours: red, green or blue, so that no two neighbouring triangles have the same colour. Such a colouring shall be called valid. Here, two triangles are said to be neighbouring if they share an edge. Note: if they only share a vertex, then they are not neighbours. For example, here is a valid colouring of the above grid: A colouring C' which is obtained from a colouring C by rotation or reflection is considered distinct from C unless the two are identical. How many distinct valid colourings are there for the above configuration? \href{https://projecteuler.net/problem=189}{Zadanie 189}
\end{zadanie}
\begin{zadanie}
	Let S m = (x 1 , x 2 , ... , x m ) be the m-tuple of positive real numbers with x 1 + x 2 + ... + x m = m for which P m = x 1 * x 2 2 * ... * x m m is maximised. For example, it can be verified that [P 10 ] = 4112 ([ ] is the integer part function). Find [P m ] for 2 $\leq$ m $\leq$ 15. \href{https://projecteuler.net/problem=190}{Zadanie 190}
\end{zadanie}
\begin{zadanie}
	A particular school offers cash rewards to children with good attendance and punctuality. If they are absent for three consecutive days or late on more than one occasion then they forfeit their prize. During an n-day period a trinary string is formed for each child consisting of L's (late), O's (on time), and A's (absent). Although there are eighty-one trinary strings for a 4-day period that can be formed, exactly forty-three strings would lead to a prize: How many "prize" strings exist over a 30-day period? \href{https://projecteuler.net/problem=191}{Zadanie 191}
\end{zadanie}
\begin{zadanie}
	Let $x$ be a real number. A best approximation to $x$ for the denominator bound $d$ is a rational number $\frac r s $ in reduced form , with $s \le d$, such that any rational number which is closer to $x$ than $\frac r s$ has a denominator larger than $d$: For example, the best approximation to $\sqrt {13}$ for the denominator bound 20 is $\frac {18} 5$ and the best approximation to $\sqrt {13}$ for the denominator bound 30 is $\frac {101}{28}$. Find the sum of all denominators of the best approximations to $\sqrt n$ for the denominator bound $10^{12}$, where $n$ is not a perfect square and $ 1 < n \le 100000$. \href{https://projecteuler.net/problem=192}{Zadanie 192}
\end{zadanie}
\begin{zadanie}
	A positive integer n is called squarefree, if no square of a prime divides n , thus 1, 2, 3, 5, 6, 7, 10, 11 are squarefree, but not 4, 8, 9, 12. How many squarefree numbers are there below 2 50 ? \href{https://projecteuler.net/problem=193}{Zadanie 193}
\end{zadanie}
\begin{zadanie}
	Consider graphs built with the units A: and B: , where the units are glued along
	the vertical edges as in the graph . A configuration of type ( a , b , c ) is a graph thus built of a units A and b units B, where the graph's vertices are coloured using up to c colours, so that no two adjacent vertices have the same colour. The compound graph above is an example of a configuration of type (2,2,6), in fact of type (2,2, c ) for all c $\geq$ 4. Let N( a , b , c ) be the number of configurations of type ( a , b , c ). For example, N(1,0,3) = 24, N(0,2,4) = 92928 and N(2,2,3) = 20736. Find the last 8 digits of N(25,75,1984). \href{https://projecteuler.net/problem=194}{Zadanie 194}
\end{zadanie}
\begin{zadanie}
	Let's call an integer sided triangle with exactly one angle of 60 degrees a 60-degree triangle. Let r be the radius of the inscribed circle of such a 60-degree triangle. There are 1234 60-degree triangles for which r $\leq$ 100. Let T( n ) be the number of 60-degree triangles for which r $\leq$ n , so T(100) = 1234, T(1000) = 22767, and T(10000) = 359912. Find T(1053779). \href{https://projecteuler.net/problem=195}{Zadanie 195}
\end{zadanie}
\begin{zadanie}
	Build a triangle from all positive integers in the following way: Each positive integer has up to eight neighbours in the triangle. A set of three primes is called a prime triplet if one of the three primes has the other two as neighbours in the triangle. For example, in the second row, the prime numbers 2 and 3 are elements of some prime triplet. If row 8 is considered, it contains two primes which are elements of some prime triplet, i.e. 29 and 31. If row 9 is considered, it contains only one prime which is an element of some prime triplet: 37. Define S( n ) as the sum of the primes in row n which are elements of any prime triplet. Then S(8)=60 and S(9)=37. You are given that S(10000)=950007619. Find S(5678027) + S(7208785). \href{https://projecteuler.net/problem=196}{Zadanie 196}
\end{zadanie}
\begin{zadanie}
	Given is the function f ( x ) = 2 30.403243784- x 2  $\times$ 10 -9 (   is the floor-function), the sequence u n is defined by u 0 = -1 and u n +1 = f ( u n ). Find u n + u n +1 for n = 10 12 . Give your answer with 9 digits after the decimal point. \href{https://projecteuler.net/problem=197}{Zadanie 197}
\end{zadanie}
\begin{zadanie}
	A best approximation to a real number $x$ for the denominator bound $d$ is a rational number $\frac r s$ (in reduced form) with $s \le d$, so that any rational number $\frac p q$ which is closer to $x$ than $\frac r s$ has $q > d$. Usually the best approximation to a real number is uniquely determined for all denominator bounds. However, there are some exceptions, e.g. $\frac 9 {40}$ has the two best approximations $\frac 1 4$ and $\frac 1 5$ for the denominator bound $6$.
	We shall call a real number $x$ ambiguous , if there is at least one denominator bound for which $x$ possesses two best approximations. Clearly, an ambiguous number is necessarily rational. How many ambiguous numbers $x=\frac p q, 0 < x < \frac 1 {100}$, are there whose denominator $q$ does not exceed $10^8$? \href{https://projecteuler.net/problem=198}{Zadanie 198}
\end{zadanie}
\begin{zadanie}
	Three circles of equal radius are placed inside a larger circle such that each pair of circles is tangent to one another and the inner circles do not overlap. There are four uncovered "gaps" which are to be filled iteratively with more tangent circles. At each iteration, a maximally sized circle is placed in each gap, which creates more gaps for the next iteration. After 3 iterations (pictured), there are 108 gaps and the fraction of the area which is not covered by circles is 0.06790342, rounded to eight decimal places. What fraction of the area is not covered by circles after 10 iterations? Give your answer rounded to eight decimal places using the format x.xxxxxxxx . \href{https://projecteuler.net/problem=199}{Zadanie 199}
\end{zadanie}
\begin{zadanie}
	We shall define a sqube to be a number of the form, p 2 q 3 , where p and q are distinct primes. For example, 200 = 5 2 2 3 or 120072949 = 23 2 61 3 . The first five squbes are 72, 108, 200, 392, and 500. Interestingly, 200 is also the first number for which you cannot change any single digit to make a prime; we shall call such numbers, prime-proof. The next prime-proof sqube which contains the contiguous sub-string "200" is 1992008. Find the 200th prime-proof sqube containing the contiguous sub-string "200". \href{https://projecteuler.net/problem=200}{Zadanie 200}
\end{zadanie}
\begin{zadanie}
	For any set A of numbers, let sum(A) be the sum of the elements of A. Consider the set B = {1,3,6,8,10,11}. There are 20 subsets of B containing three elements, and their sums are: sum({1,3,6}) = 10, sum({1,3,8}) = 12, sum({1,3,10}) = 14, sum({1,3,11}) = 15, sum({1,6,8}) = 15, sum({1,6,10}) = 17, sum({1,6,11}) = 18, sum({1,8,10}) = 19, sum({1,8,11}) = 20, sum({1,10,11}) = 22, sum({3,6,8}) = 17, sum({3,6,10}) = 19, sum({3,6,11}) = 20, sum({3,8,10}) = 21, sum({3,8,11}) = 22, sum({3,10,11}) = 24, sum({6,8,10}) = 24, sum({6,8,11}) = 25, sum({6,10,11}) = 27, sum({8,10,11}) = 29. Some of these sums occur more than once, others are unique. For a set A, let U(A,k) be the set of unique sums of k-element subsets of A, in our example we find U(B,3) = {10,12,14,18,21,25,27,29} and sum(U(B,3)) = 156. Now consider the 100-element set S = {1 2 , 2 2 , ... , 100 2 }. S has 100891344545564193334812497256 50-element subsets. Determine the sum of all integers which are the sum of exactly one of the 50-element subsets of S, i.e. find sum(U(S,50)). \href{https://projecteuler.net/problem=201}{Zadanie 201}
\end{zadanie}
\begin{zadanie}
	Three mirrors are arranged in the shape of an equilateral triangle, with their reflective surfaces pointing inwards. There is an infinitesimal gap at each vertex of the triangle through which a laser beam may pass. Label the vertices A, B and C. There are 2 ways in which a laser beam may enter vertex C, bounce off 11 surfaces, then exit through the same vertex: one way is shown below; the other is the reverse of that. There are 80840 ways in which a laser beam may enter vertex C, bounce off 1000001 surfaces, then exit through the same vertex. In how many ways can a laser beam enter at vertex C, bounce off 12017639147 surfaces, then exit through the same vertex? \href{https://projecteuler.net/problem=202}{Zadanie 202}
\end{zadanie}
\begin{zadanie}
	The binomial coefficients $\displaystyle \binom n k$ can be arranged in triangular form, Pascal's triangle, like this: It can be seen that the first eight rows of Pascal's triangle contain twelve distinct numbers: 1,2,3,4,5,6,7,10,15,20,21and35. A positive integer n is called squarefree if no square of a prime divides n .
	Of the twelve distinct numbers in the first eight rows of Pascal's triangle, all except 4 and 20 are squarefree.
	The sum of the distinct squarefree numbers in the first eight rows is 105. Find the sum of the distinct squarefree numbers in the first 51 rows of Pascal's triangle. \href{https://projecteuler.net/problem=203}{Zadanie 203}
\end{zadanie}
\begin{zadanie}
	A Hamming number is a positive number which has no prime factor larger than 5. So the first few Hamming numbers are 1, 2, 3, 4, 5, 6, 8, 9, 10, 12, 15. There are 1105 Hamming numbers not exceeding 10 8 . We will call a positive number a generalised Hamming number of type n , if it has no prime factor larger than n . Hence the Hamming numbers are the generalised Hamming numbers of type 5. How many generalised Hamming numbers of type 100 are there which don't exceed 10 9 ? \href{https://projecteuler.net/problem=204}{Zadanie 204}
\end{zadanie}
\begin{zadanie}
	Peter has nine four-sided (pyramidal) dice, each with faces numbered 1, 2, 3, 4. Colin has six six-sided (cubic) dice, each with faces numbered 1, 2, 3, 4, 5, 6. Peter and Colin roll their dice and compare totals: the highest total wins. The result is a draw if the totals are equal. What is the probability that Pyramidal Pete beats Cubic Colin? Give your answer rounded to seven decimal places in the form 0.abcdefg \href{https://projecteuler.net/problem=205}{Zadanie 205}
\end{zadanie}
\begin{zadanie}
	Find the unique positive integer whose square has the form 1\_2\_3\_4\_5\_6\_7\_8\_9\_0, where each \_ is a single digit. \href{https://projecteuler.net/problem=206}{Zadanie 206}
\end{zadanie}
\begin{zadanie}
	For some positive integers k , there exists an integer partition of the form 4 t = 2 t + k , where 4 t , 2 t , and k are all positive integers and t is a real number. The first two such partitions are 4 1 = 2 1 + 2 and 4 1.5849625... = 2 1.5849625... + 6. Partitions where t is also an integer are called perfect . For any m $\geq$ 1 let P( m ) be the proportion of such partitions that are perfect with k $\leq$ m . Thus P(6) = 1/2. In the following table are listed some values of P( m ) P(5) = 1/1 P(10) = 1/2 P(15) = 2/3 P(20) = 1/2 P(25) = 1/2 P(30) = 2/5 ... P(180) = 1/4 P(185) = 3/13 Find the smallest m for which P( m ) < 1/12345 \href{https://projecteuler.net/problem=207}{Zadanie 207}
\end{zadanie}
\begin{zadanie}
	A robot moves in a series of one-fifth circular arcs (72), with a free choice of a clockwise or an anticlockwise arc for each step, but no turning on the spot. One of 70932 possible closed paths of 25 arcs starting northward is Given that the robot starts facing North, how many journeys of 70 arcs in length can it take that return it, after the final arc, to its starting position? (Any arc may be traversed multiple times.) \href{https://projecteuler.net/problem=208}{Zadanie 208}
\end{zadanie}
\begin{zadanie}
	A k -input binary truth table is a map from k input bits
	(binary digits, 0 [false] or 1 [true]) to 1 output bit. For example, the 2-input binary truth tables for the logical AND and XOR functions are: How many 6-input binary truth tables, , satisfy the formula for all 6-bit inputs ( a , b , c , d , e , f )? \href{https://projecteuler.net/problem=209}{Zadanie 209}
\end{zadanie}
\begin{zadanie}
	What is N(1,000,000,000)? \href{https://projecteuler.net/problem=210}{Zadanie 210}
\end{zadanie}
\begin{zadanie}
	For a positive integer n , let  2 ( n ) be the sum of the squares of its divisors. For example, Find the sum of all n , 0 < n < 64,000,000 such that  2 ( n ) is a perfect square. \href{https://projecteuler.net/problem=211}{Zadanie 211}
\end{zadanie}
\begin{zadanie}
	An axis-aligned cuboid , specified by parameters { (x 0 ,y 0 ,z 0 ), (dx,dy,dz) }, consists of all points (X,Y,Z) such that x 0 $\leq$ X $\leq$ x 0 +dx, y 0 $\leq$ Y $\leq$ y 0 +dy and z 0 $\leq$ Z $\leq$ z 0 +dz. The volume of the cuboid is the product, dx $\times$ dy $\times$ dz. The combined volume of a collection of cuboids is the volume of their union and will be less than the sum of the individual volumes if any cuboids overlap. Let C 1 ,...,C 50000 be a collection of 50000 axis-aligned cuboids such that C n has parameters x 0 = S 6 n -5 modulo 10000 y 0 = S 6 n -4 modulo 10000 z 0 = S 6 n -3 modulo 10000 dx = 1 + (S 6 n -2 modulo 399) dy = 1 + (S 6 n -1 modulo 399) dz = 1 + (S 6 n modulo 399) where S 1 ,...,S 300000 come from the "Lagged Fibonacci Generator": For 1 $\leq$ k $\leq$ 55, S k = [100003 - 200003 k + 300007 k 3 ] (modulo 1000000) For 56 $\leq$ k , S k = [S k -24 + S k -55 ] (modulo 1000000) Thus, C 1 has parameters {(7,53,183),(94,369,56)}, C 2 has parameters {(2383,3563,5079),(42,212,344)}, and so on. The combined volume of the first 100 cuboids, C 1 ,...,C 100 , is 723581599. What is the combined volume of all 50000 cuboids, C 1 ,...,C 50000 ? \href{https://projecteuler.net/problem=212}{Zadanie 212}
\end{zadanie}
\begin{zadanie}
	A 30$\times$30 grid of squares contains 900 fleas, initially one flea per square. When a bell is rung, each flea jumps to an adjacent square at random (usually 4 possibilities, except for fleas on the edge of the grid or at the corners). What is the expected number of unoccupied squares after 50 rings of the bell? Give your answer rounded to six decimal places. \href{https://projecteuler.net/problem=213}{Zadanie 213}
\end{zadanie}
\begin{zadanie}
	Let $\phi$ be Euler's totient function, i.e. for a natural number n ,
	$\phi$( n ) is the number of k , 1 $\leq$ k $\leq$ n , for which gcd( k , n ) = 1. By iterating $\phi$, each positive integer generates a decreasing chain of numbers ending in 1. E.g. if we start with 5 the sequence 5,4,2,1 is generated. Here is a listing of all chains with length 4: Only two of these chains start with a prime, their sum is 12. What is the sum of all primes less than 40000000 which generate a chain of length 25? \href{https://projecteuler.net/problem=214}{Zadanie 214}
\end{zadanie}
\begin{zadanie}
	Consider the problem of building a wall out of 2$\times$1 and 3$\times$1 bricks (horizontal$\times$vertical dimensions) such that, for extra strength, the gaps between horizontally-adjacent bricks never line up in consecutive layers, i.e. never form a "running crack". For example, the following 9$\times$3 wall is not acceptable due to the running crack shown in red: There are eight ways of forming a crack-free 9$\times$3 wall, written W(9,3) = 8. Calculate W(32,10). \href{https://projecteuler.net/problem=215}{Zadanie 215}
\end{zadanie}
\begin{zadanie}
	Consider numbers t ( n ) of the form t ( n ) = 2 n 2 -1 with n > 1. The first such numbers are 7, 17, 31, 49, 71, 97, 127 and 161. It turns out that only 49 = 7*7 and 161 = 7*23 are not prime. For n $\leq$ 10000 there are 2202 numbers t ( n ) that are prime. How many numbers t ( n ) are prime for n $\leq$ 50,000,000 ? \href{https://projecteuler.net/problem=216}{Zadanie 216}
\end{zadanie}
\begin{zadanie}
	A positive integer with k (decimal) digits is called balanced if its first  k / 2  digits sum to the same value as its last  k / 2  digits, where  x , pronounced ceiling of x , is the smallest integer $\geq$ x , thus  = 4 and 5 = 5. So, for example, all palindromes are balanced, as is 13722. Let T( n ) be the sum of all balanced numbers less than 10 n . Thus: T(1) = 45, T(2) = 540 and T(5) = 334795890. Find T(47) mod 3 15 \href{https://projecteuler.net/problem=217}{Zadanie 217}
\end{zadanie}
\begin{zadanie}
	Consider the right angled triangle with sides a=7, b=24 and c=25.
	The area of this triangle is 84, which is divisible by the perfect numbers 6 and 28. Moreover it is a primitive right angled triangle as gcd(a,b)=1 and gcd(b,c)=1. Also c is a perfect square. We will call a right angled triangle perfect if -it is a primitive right angled triangle -its hypotenuse is a perfect square We will call a right angled triangle super-perfect if -it is a perfect right angled triangle and -its area is a multiple of the perfect numbers 6 and 28. How many perfect right-angled triangles with c$\leq$10 16 exist that are not super-perfect? \href{https://projecteuler.net/problem=218}{Zadanie 218}
\end{zadanie}
\begin{zadanie}
	Let A and B be bit strings (sequences of 0's and 1's). If A is equal to the left most length( A ) bits of B , then A is said to be a prefix of B . For example, 00110 is a prefix of 00110 1001, but not of 00111 or 100110. A prefix-free code of size n is a collection of n distinct bit strings such that no string is a prefix of any other. For example, this is a prefix-free code of size 6: 0000, 0001, 001, 01, 10, 11 Now suppose that it costs one penny to transmit a '0' bit, but four pence to transmit a '1'. Then the total cost of the prefix-free code shown above is 35 pence, which happens to be the cheapest possible for the skewed pricing scheme in question. In short, we write Cost(6) = 35. What is Cost(10 9 ) ? \href{https://projecteuler.net/problem=219}{Zadanie 219}
\end{zadanie}
\begin{zadanie}
	Let D 0 be the two-letter string "Fa". For n$\geq$1, derive D n from D n-1 by the string-rewriting rules: "a" $\rightarrow$ "aRbFR" "b" $\rightarrow$ "LFaLb" Thus, D 0 = "Fa", D 1 = "FaRbFR", D 2 = "FaRbFRRLFaLbFR", and so on. These strings can be interpreted as instructions to a computer graphics program, with "F" meaning "draw forward one unit", "L" meaning "turn left 90 degrees", "R" meaning "turn right 90 degrees", and "a" and "b" being ignored. The initial position of the computer cursor is (0,0), pointing up towards (0,1). Then D n is an exotic drawing known as the Heighway Dragon of order n . For example, D 10 is shown below; counting each "F" as one step, the highlighted spot at (18,16) is the position reached after 500 steps. What is the position of the cursor after 10 12 steps in D 50 ? Give your answer in the form x , y with no spaces. \href{https://projecteuler.net/problem=220}{Zadanie 220}
\end{zadanie}
\begin{zadanie}
	We shall call a positive integer A an "Alexandrian integer", if there exist integers p , q , r such that: For example, 630 is an Alexandrian integer ( p =5, q =-7, r =-18).
	In fact, 630 is the 6 th Alexandrian integer, the first 6 Alexandrian integers being: 6, 42, 120, 156, 420 and 630. Find the 150000 th Alexandrian integer. \href{https://projecteuler.net/problem=221}{Zadanie 221}
\end{zadanie}
\begin{zadanie}
	What is the length of the shortest pipe, of internal radius 50mm, that can fully contain 21 balls of radii 30mm, 31mm, ..., 50mm? Give your answer in micrometres (10 -6 m) rounded to the nearest integer. \href{https://projecteuler.net/problem=222}{Zadanie 222}
\end{zadanie}
\begin{zadanie}
	Let us call an integer sided triangle with sides a $\leq$ b $\leq$ c barely acute if the sides satisfy a 2 + b 2 = c 2 + 1. How many barely acute triangles are there with perimeter $\leq$ 25,000,000? \href{https://projecteuler.net/problem=223}{Zadanie 223}
\end{zadanie}
\begin{zadanie}
	Let us call an integer sided triangle with sides a $\leq$ b $\leq$ c barely obtuse if the sides satisfy a 2 + b 2 = c 2 - 1. How many barely obtuse triangles are there with perimeter $\leq$ 75,000,000? \href{https://projecteuler.net/problem=224}{Zadanie 224}
\end{zadanie}
\begin{zadanie}
	The sequence 1, 1, 1, 3, 5, 9, 17, 31, 57, 105, 193, 355, 653, 1201 ... is defined by T 1 = T 2 = T 3 = 1 and T n = T n -1 + T n -2 + T n -3 . It can be shown that 27 does not divide any terms of this sequence. In fact, 27 is the first odd number with this property. Find the 124 th odd number that does not divide any terms of the above sequence. \href{https://projecteuler.net/problem=225}{Zadanie 225}
\end{zadanie}
\begin{zadanie}
	The blancmange curve is the set of points $(x, y)$ such that $0 \le x \le 1$ and $y = \sum \limits\_{n = 0}^{\infty} {\dfrac{s(2^n x)}{2^n}}$, where $s(x)$ is the distance from $x$ to the nearest integer. The area under the blancmange curve is equal to $\frac12$, shown in pink in the diagram below. Let C be the circle with centre $\left ( \frac{1}{4}, \frac{1}{2} \right )$ and radius $\frac{1}{4}$, shown in black in the diagram. What area under the blancmange curve is enclosed by C ? Give your answer rounded to eight decimal places in the form 0.abcdefgh \href{https://projecteuler.net/problem=226}{Zadanie 226}
\end{zadanie}
\begin{zadanie}
	" The Chase " is a game played with two dice and an even number of players. The players sit around a table; the game begins with two opposite players having one die each. On each turn, the two players with a die roll it. If a player rolls a 1, he passes the die to his neighbour on the left; if he rolls a 6, he passes the die to his neighbour on the right; otherwise, he keeps the die for the next turn. The game ends when one player has both dice after they have been rolled and passed; that player has then lost. In a game with 100 players, what is the expected number of turns the game lasts? Give your answer rounded to ten significant digits. \href{https://projecteuler.net/problem=227}{Zadanie 227}
\end{zadanie}
\begin{zadanie}
	Let S n be the regular n -sided polygon  or shape  whose vertices v k ( k =1,2,, n ) have coordinates: Each S n is to be interpreted as a filled shape consisting of all points on the perimeter and in the interior. The Minkowski sum , S + T , of two shapes S and T is the result of adding every point in S to every point in T , where point addition is performed coordinate-wise: ( u , v ) + ( x , y ) = ( u + x , v + y ). For example, the sum of S 3 and S 4 is the six-sided shape shown in pink below: How many sides does S 1864 + S 1865 ++ S 1909 have? \href{https://projecteuler.net/problem=228}{Zadanie 228}
\end{zadanie}
\begin{zadanie}
	Consider the number 3600. It is very special, because Similarly, we find that 88201 = 99 2 + 280 2 = 287 2 + 2$\times$54 2 = 283 2 + 3$\times$52 2 = 197 2 + 7$\times$84 2 . In 1747, Euler proved which numbers are representable as a sum of two squares.
	We are interested in the numbers n which admit representations of all of the following four types: where the a k and b k are positive integers. There are 75373 such numbers that do not exceed 10 7 . How many such numbers are there that do not exceed 2$\times$10 9 ? \href{https://projecteuler.net/problem=229}{Zadanie 229}
\end{zadanie}
\begin{zadanie}
	For any two strings of digits, A and B, we define F A,B to be the sequence (A,B,AB,BAB,ABBAB,...) in which each term is the concatenation of the previous two. Further, we define D A,B ( n ) to be the n th digit in the first term of F A,B that contains at least n digits. Example: Let A=1415926535, B=8979323846. We wish to find D A,B (35), say. The first few terms of F A,B are: 1415926535 8979323846 14159265358979323846 897932384614159265358979323846 1415926535897932384689793238461415 9 265358979323846 Then D A,B (35) is the 35 th digit in the fifth term, which is 9. Now we use for A the first 100 digits of  behind the decimal point: 14159265358979323846264338327950288419716939937510 58209749445923078164062862089986280348253421170679 and for B the next hundred digits: 82148086513282306647093844609550582231725359408128 48111745028410270193852110555964462294895493038196 . Find $\Sigma$ n = 0,1,...,17 10 n $\times$ D A,B ((127+19 n )$\times$7 n ) . \href{https://projecteuler.net/problem=230}{Zadanie 230}
\end{zadanie}
\begin{zadanie}
	The binomial coefficient $\displaystyle \binom {10} 3 = 120$. $120 = 2^3 \times 3 \times 5 = 2 \times 2 \times 2 \times 3 \times 5$, and $2 + 2 + 2 + 3 + 5 = 14$. So the sum of the terms in the prime factorisation of $\displaystyle \binom {10} 3$ is $14$. Find the sum of the terms in the prime factorisation of $\displaystyle \binom {20\,000\,000} {15\,000\,000}$. \href{https://projecteuler.net/problem=231}{Zadanie 231}
\end{zadanie}
\begin{zadanie}
	Two players share an unbiased coin and take it in turns to play "The Race". On Player 1's turn, he tosses the coin once: if it comes up Heads, he scores one point; if it comes up Tails, he scores nothing. On Player 2's turn, she chooses a positive integer T and tosses the coin T times: if it comes up all Heads, she scores 2 T -1 points; otherwise, she scores nothing. Player 1 goes first. The winner is the first to 100 or more points. On each turn Player 2 selects the number, T , of coin tosses that maximises the probability of her winning. What is the probability that Player 2 wins? Give your answer rounded to eight decimal places in the form 0. abcdefgh . \href{https://projecteuler.net/problem=232}{Zadanie 232}
\end{zadanie}
\begin{zadanie}
	Let f ( N ) be the number of points with integer coordinates that are on a circle passing through (0,0), ( N ,0),(0, N ), and ( N , N ). It can be shown that f (10000)=36. What is the sum of all positive integers N $\leq$10 11 such that f ( N )=420? \href{https://projecteuler.net/problem=233}{Zadanie 233}
\end{zadanie}
\begin{zadanie}
	For an integer n $\geq$ 4, we define the lower prime square root of n , denoted by lps( n ), as the largest prime $\leq$  n and the upper prime square root of n , ups( n ), as the smallest prime $\geq$  n . So, for example, lps(4) = 2 = ups(4), lps(1000) = 31, ups(1000) = 37. Let us call an integer n $\geq$ 4 semidivisible , if one of lps( n ) and ups( n ) divides n , but not both. The sum of the semidivisible numbers not exceeding 15 is 30, the numbers are 8, 10 and 12. 15 is not semidivisible because it is a multiple of both lps(15) = 3 and ups(15) = 5. As a further example, the sum of the 92 semidivisible numbers up to 1000 is 34825. What is the sum of all semidivisible numbers not exceeding 999966663333 ? \href{https://projecteuler.net/problem=234}{Zadanie 234}
\end{zadanie}
\begin{zadanie}
	Given is the arithmetic-geometric sequence u( k ) = (900-3 k ) r k -1 . Let s( n ) =  k =1... n u( k ). Find the value of r for which s(5000) = -600,000,000,000. Give your answer rounded to 12 places behind the decimal point. \href{https://projecteuler.net/problem=235}{Zadanie 235}
\end{zadanie}
\begin{zadanie}
	Suppliers 'A' and 'B' provided the following numbers of products for the luxury hamper market: Although the suppliers try very hard to ship their goods in perfect condition, there is inevitably some spoilage - i.e. products gone bad. The suppliers compare their performance using two types of statistic: To their surprise, the suppliers found that each of the five per-product spoilage rates was worse (higher) for 'B' than for 'A' by the same factor (ratio of spoilage rates), m >1; and yet, paradoxically, the overall spoilage rate was worse for 'A' than for 'B', also by a factor of m . There are thirty-five m >1 for which this surprising result could have occurred, the smallest of which is 1476/1475. What's the largest possible value of m ? Give your answer as a fraction reduced to its lowest terms, in the form u / v . \href{https://projecteuler.net/problem=236}{Zadanie 236}
\end{zadanie}
\begin{zadanie}
	Let T( n ) be the number of tours over a 4 $\times$ n playing board such that: The diagram shows one tour over a 4 $\times$ 10 board: T(10) is 2329. What is T(10 12 ) modulo 10 8 ? \href{https://projecteuler.net/problem=237}{Zadanie 237}
\end{zadanie}
\begin{zadanie}
	Create a sequence of numbers using the "Blum Blum Shub" pseudo-random number generator: Concatenate these numbers s 0 s 1 s 2  to create a string w of infinite length. Then, w = an 14025256741014958470038053646 an 14025256741014958470038053646 For a positive integer k , if no substring of w exists with a sum of digits equal to k , p ( k ) is defined to be zero. If at least one substring of w exists with a sum of digits equal to k , we define p ( k )= z , where z is the starting position of the earliest such substring. For instance: The substrings 1 \href{https://projecteuler.net/problem=238}{Zadanie 238}
\end{zadanie}
\begin{zadanie}
	A set of disks numbered 1 through 100 are placed in a line in random order. What is the probability that we have a partial derangement such that exactly 22 prime number discs are found away from their natural positions? (Any number of non-prime disks may also be found in or out of their natural positions.) Give your answer rounded to 12 places behind the decimal point in the form 0.abcdefghijkl. \href{https://projecteuler.net/problem=239}{Zadanie 239}
\end{zadanie}
\begin{zadanie}
	There are 1111 ways in which five 6-sided dice (sides numbered 1 to 6) can be rolled so that the top three sum to 15. Some examples are: D 1 ,D 2 ,D 3 ,D 4 ,D 5 = 4,3,6,3,5 D 1 ,D 2 ,D 3 ,D 4 ,D 5 = 4,3,3,5,6 D 1 ,D 2 ,D 3 ,D 4 ,D 5 = 3,3,3,6,6 D 1 ,D 2 ,D 3 ,D 4 ,D 5 = 6,6,3,3,3 In how many ways can twenty 12-sided dice (sides numbered 1 to 12) be rolled so that the top ten sum to 70? \href{https://projecteuler.net/problem=240}{Zadanie 240}
\end{zadanie}
\begin{zadanie}
	For a positive integer n , let ( n ) be the sum of all divisors of n , so e.g. (6) = 1 + 2 + 3 + 6 = 12. A perfect number, as you probably know, is a number with ( n ) = 2 n . Find the sum of all positive integers n $\leq$ 10 18 for which p ( n ) has the form k + 1  2 , where k is an integer. \href{https://projecteuler.net/problem=241}{Zadanie 241}
\end{zadanie}
\begin{zadanie}
	Given the set {1,2,..., n }, we define f ( n , k ) as the number of its k -element subsets with an odd sum of elements. For example, f (5,3)=4, since the set {1,2,3,4,5} has four 3-element subsets having an odd sum of elements, i.e.: {1,2,4}, {1,3,5}, {2,3,4} and {2,4,5}. When all three values n , k and f ( n , k ) are odd, we say that they make an odd-triplet [ n , k , f ( n , k )]. There are exactly five odd-triplets with n $\leq$10, namely: [1,1, f (1,1)=1], [5,1, f (5,1)=3], [5,5, f (5,5)=1], [9,1, f (9,1)=5] and [9,9, f (9,9)=1]. How many odd-triplets are there with n $\leq$10 12 ? \href{https://projecteuler.net/problem=242}{Zadanie 242}
\end{zadanie}
\begin{zadanie}
	A positive fraction whose numerator is less than its denominator is called a proper fraction. For any denominator, d , there will be d -1 proper fractions; for example, with d =12: 1 / 12 , 2 / 12 , 3 / 12 , 4 / 12 , 5 / 12 , 6 / 12 , 7 / 12 , 8 / 12 , 9 / 12 , 10 / 12 , 11 / 12 . We shall call a fraction that cannot be cancelled down a resilient fraction . Furthermore we shall define the resilience of a denominator, R ( d ), to be the ratio of its proper fractions that are resilient; for example, R (12) = 4 / 11 . In fact, d =12 is the smallest denominator having a resilience R ( d ) < 4 / 10 . Find the smallest denominator d , having a resilience R ( d ) < 15499 / 94744 . \href{https://projecteuler.net/problem=243}{Zadanie 243}
\end{zadanie}
\begin{zadanie}
	You probably know the game Fifteen Puzzle . Here, instead of numbered tiles, we have seven red tiles and eight blue tiles. A move is denoted by the uppercase initial of the direction (Left, Right, Up, Down) in which the tile is slid, e.g. starting from configuration ( S ), by the sequence LULUR we reach the configuration ( E ): For each path, its checksum is calculated by (pseudocode): For the sequence LULUR given above, the checksum would be 19761398. Now, starting from configuration ( S ),
	find all shortest ways to reach configuration ( T ). What is the sum of all checksums for the paths having the minimal length? \href{https://projecteuler.net/problem=244}{Zadanie 244}
\end{zadanie}
\begin{zadanie}
	We shall call a fraction that cannot be cancelled down a resilient fraction. Furthermore we shall define the resilience of a denominator, R ( d ), to be the ratio of its proper fractions that are resilient; for example, R (12) = 4  11 . Find the sum of all composite integers 1 < n $\leq$ 2$\times$10 11 , for which C ( n ) is a unit fraction . \href{https://projecteuler.net/problem=245}{Zadanie 245}
\end{zadanie}
\begin{zadanie}
	A definition for an ellipse is: Given a circle c with centre M and radius r and a point G such that d(G,M)<r, the locus of the points that are equidistant from c and G form an ellipse. Given are the points M(-2000,1500) and G(8000,1500). Given is also the circle c with centre M and radius 15000. The locus of the points that are equidistant from G and c form an ellipse e. From a point P outside e the two tangents t 1 and t 2 to the ellipse are drawn. Let the points where t 1 and t 2 touch the ellipse be R and S. For how many lattice points P is angle RPS greater than 45 degrees? \href{https://projecteuler.net/problem=246}{Zadanie 246}
\end{zadanie}
\begin{zadanie}
	Consider the region constrained by 1 $\leq$ x and 0 $\leq$ y $\leq$ 1 / x . Let S 1 be the largest square that can fit under the curve. Let S 2 be the largest square that fits in the remaining area, and so on. Let the index of S n be the pair (left, below) indicating the number of squares to the left of S n and the number of squares below S n . The diagram shows some such squares labelled by number. S 2 has one square to its left and none below, so the index of S 2 is (1,0). It can be seen that the index of S 32 is (1,1) as is the index of S 50 . 50 is the largest n for which the index of S n is (1,1). What is the largest n for which the index of S n is (3,3)? \href{https://projecteuler.net/problem=247}{Zadanie 247}
\end{zadanie}
\begin{zadanie}
	The first number n for which $\phi$(n)=13! is 6227180929. Find the 150,000 th such number. \href{https://projecteuler.net/problem=248}{Zadanie 248}
\end{zadanie}
\begin{zadanie}
	Let S = {2, 3, 5, ..., 4999} be the set of prime numbers less than 5000. Find the number of subsets of S , the sum of whose elements is a prime number. Enter the rightmost 16 digits as your answer. \href{https://projecteuler.net/problem=249}{Zadanie 249}
\end{zadanie}
\begin{zadanie}
	Find the number of non-empty subsets of {1 1 , 2 2 , 3 3 ,..., 250250 250250 }, the sum of whose elements is divisible by 250. Enter the rightmost 16 digits as your answer. \href{https://projecteuler.net/problem=250}{Zadanie 250}
\end{zadanie}
\begin{zadanie}
	A triplet of positive integers ( a , b , c ) is called a Cardano Triplet if it satisfies the condition: For example, (2,1,5) is a Cardano Triplet. There exist 149 Cardano Triplets for which a + b + c $\leq$ 1000. Find how many Cardano Triplets exist such that a + b + c $\leq$ 110,000,000. \href{https://projecteuler.net/problem=251}{Zadanie 251}
\end{zadanie}
\begin{zadanie}
	Given a set of points on a plane, we define a convex hole to be a convex polygon having as vertices any of the given points and not containing any of the given points in its interior (in addition to the vertices, other given points may lie on the perimeter of the polygon). As an example, the image below shows a set of twenty points and a few such convex holes. The convex hole shown as a red heptagon has an area equal to 1049694.5 square units, which is the highest possible area for a convex hole on the given set of points. For our example, we used the first 20 points ( T 2 k -1 , T 2 k ), for k =1,2,,20, produced with the pseudo-random number generator: i.e. (527,144), (-488,732), (-454,-947),  What is the maximum area for a convex hole on the set containing the first 500 points in the pseudo-random sequence? Specify your answer including one digit after the decimal point. \href{https://projecteuler.net/problem=252}{Zadanie 252}
\end{zadanie}
\begin{zadanie}
	A small child has a number caterpillar consisting of forty jigsaw pieces, each with one number on it, which, when connected together in a line, reveal the numbers 1 to 40 in order. Every night, the child's father has to pick up the pieces of the caterpillar that have been scattered across the play room. He picks up the pieces at random and places them in the correct order. As the caterpillar is built up in this way, it forms distinct segments that gradually merge together. The number of segments starts at zero (no pieces placed), generally increases up to about eleven or twelve, then tends to drop again before finishing at a single segment (all pieces placed). For example: Let M be the maximum number of segments encountered during a random tidy-up of the caterpillar. For a caterpillar of ten pieces, the number of possibilities for each M is so the most likely value of M is 3 and the average value is 385643  113400 = 3.400732, rounded to six decimal places. The most likely value of M for a forty-piece caterpillar is 11; but what is the average value of M ? Give your answer rounded to six decimal places. \href{https://projecteuler.net/problem=253}{Zadanie 253}
\end{zadanie}
\begin{zadanie}
	Define f( n ) as the sum of the factorials of the digits of n . For example, f(342) = 3! + 4! + 2! = 32. Define sf( n ) as the sum of the digits of f( n ). So sf(342) = 3 + 2 = 5. Define g( i ) to be the smallest positive integer n such that sf( n ) = i . Though sf(342) is 5, sf(25) is also 5, and it can be verified that g(5) is 25. Define sg( i ) as the sum of the digits of g( i ). So sg(5) = 2 + 5 = 7. Further, it can be verified that g(20) is 267 and $\Sigma$ sg( i ) for 1 $\leq$ i $\leq$ 20 is 156. What is $\Sigma$ sg( i ) for 1 $\leq$ i $\leq$ 150? \href{https://projecteuler.net/problem=254}{Zadanie 254}
\end{zadanie}
\begin{zadanie}
	We define the rounded-square-root of a positive integer n as the square root of n rounded to the nearest integer. The following procedure (essentially Heron's method adapted to integer arithmetic) finds the rounded-square-root of n : Let d be the number of digits of the number n . If d is odd, set $x\_0 = 2 \times 10^{(d-1)/2}$. If d is even, set $x\_0 = 7 \times 10^{(d-2)/2}$. Repeat: until $x\_{k+1} = x\_k$. As an example, let us find the rounded-square-root of n = 4321. n has 4 digits, so $x\_0 = 7 \times 10^{(4-2)/2} = 70$. $$x\_1 = \Biggl\lfloor{\dfrac{70 + \lceil{4321 / 70}\rceil}{2}}\Biggr\rfloor = 66\\
		x\_2 = \Biggl\lfloor{\dfrac{66 + \lceil{4321 / 66}\rceil}{2}}\Biggr\rfloor = 66$$ Since $x\_2 = x\_1$, we stop here. So, after just two iterations, we have found that the rounded-square-root of 4321 is 66 (the actual square root is 65.7343137). The number of iterations required when using this method is surprisingly low. For example, we can find the rounded-square-root of a 5-digit integer (10,000 $\leq$ n $\leq$ 99,999) with an average of 3.2102888889 iterations (the average value was rounded to 10 decimal places). Using the procedure described above, what is the average number of iterations required to find the rounded-square-root of a 14-digit number (10 13 $\leq$ n < 10 14 )? Give your answer rounded to 10 decimal places. Note: The symbols $\lfloor x \rfloor$ and $\lceil x \rceil$ represent the floor function and ceiling function respectively. \href{https://projecteuler.net/problem=255}{Zadanie 255}
\end{zadanie}
\begin{zadanie}
	Tatami are rectangular mats, used to completely cover the floor of a room, without overlap. Assuming that the only type of available tatami has dimensions 1$\times$2, there are obviously some limitations for the shape and size of the rooms that can be covered. For this problem, we consider only rectangular rooms with integer dimensions a , b and even size s = a  b . We use the term 'size' to denote the floor surface area of the room, and  without loss of generality  we add the condition a $\leq$ b . There is one rule to follow when laying out tatami: there must be no points where corners of four different mats meet. For example, consider the two arrangements below for a 4$\times$4 room: The arrangement on the left is acceptable, whereas the one on the right is not: a red " X " in the middle, marks the point where four tatami meet. Because of this rule, certain even-sized rooms cannot be covered with tatami: we call them tatami-free rooms. Further, we define T ( s ) as the number of tatami-free rooms of size s . The smallest tatami-free room has size s = 70 and dimensions 7$\times$10. All the other rooms of size s = 70 can be covered with tatami; they are: 1$\times$70, 2$\times$35 and 5$\times$14. Hence, T (70) = 1. Similarly, we can verify that T (1320) = 5 because there are exactly 5 tatami-free rooms of size s = 1320: 20$\times$66, 22$\times$60, 24$\times$55, 30$\times$44 and 33$\times$40. In fact, s = 1320 is the smallest room-size s for which T ( s ) = 5. Find the smallest room-size s for which T ( s ) = 200. \href{https://projecteuler.net/problem=256}{Zadanie 256}
\end{zadanie}
\begin{zadanie}
	Given is an integer sided triangle ABC with sides a $\leq$ b $\leq$ c. (AB = c, BC = a and AC = b). The angular bisectors of the triangle intersect the sides at points E, F and G (see picture below). The segments EF, EG and FG partition the triangle ABC into four smaller triangles: AEG, BFE, CGF and EFG. It can be proven that for each of these four triangles the ratio area(ABC)/area(subtriangle) is rational. However, there exist triangles for which some or all of these ratios are integral. How many triangles ABC with perimeter$\leq$100,000,000 exist so that the ratio area(ABC)/area(AEG) is integral? \href{https://projecteuler.net/problem=257}{Zadanie 257}
\end{zadanie}
\begin{zadanie}
	A sequence is defined as: Find g k mod 20092010 for k = 10 18 . \href{https://projecteuler.net/problem=258}{Zadanie 258}
\end{zadanie}
\begin{zadanie}
	A positive integer will be called reachable if it can result from an arithmetic expression obeying the following rules: For example, 42 is reachable, since (1/23) * ((4*5)-6) * (78-9) = 42. What is the sum of all positive reachable integers? \href{https://projecteuler.net/problem=259}{Zadanie 259}
\end{zadanie}
\begin{zadanie}
	A game is played with three piles of stones and two players. At her turn, a player removes one or more stones from the piles. However, if she takes stones from more than one pile, she must remove the same number of stones from each of the selected piles. In other words, the player chooses some N>0 and removes: A winning configuration is one where the first player can force a win. For example, (0,0,13), (0,11,11) and (5,5,5) are winning configurations because the first player can immediately remove all stones. A losing configuration is one where the second player can force a win, no matter what the first player does. For example, (0,1,2) and (1,3,3) are losing configurations: any legal move leaves a winning configuration for the second player. Consider all losing configurations (x i ,y i ,z i ) where x i $\leq$ y i $\leq$ z i $\leq$ 100. We can verify that $\Sigma$ (x i +y i +z i ) = 173895 for these. Find $\Sigma$ (x i +y i +z i ) where (x i ,y i ,z i ) ranges over the losing configurations with x i $\leq$ y i $\leq$ z i $\leq$ 1000. \href{https://projecteuler.net/problem=260}{Zadanie 260}
\end{zadanie}
\begin{zadanie}
	Let us call a positive integer k a square-pivot , if there is a pair of integers m > 0 and n $\geq$ k , such that the sum of the ( m +1) consecutive squares up to k equals the sum of the m consecutive squares from ( n +1) on: Some small square-pivots are Find the sum of all distinct square-pivots $\leq$ 10 10 . \href{https://projecteuler.net/problem=261}{Zadanie 261}
\end{zadanie}
\begin{zadanie}
	The following equation represents the continuous topography of a mountainous region, giving the elevation h at any point ( x , y ): A mosquito intends to fly from A(200,200) to B(1400,1400), without leaving the area given by 0$\leq$ x , y $\leq$1600. Because of the intervening mountains, it first rises straight up to a point A', having elevation f . Then, while remaining at the same elevation f , it flies around any obstacles until it arrives at a point B' directly above B. First, determine f min which is the minimum constant elevation allowing such a trip from A to B, while remaining in the specified area. Then, find the length of the shortest path between A' and B', while flying at that constant elevation f min . Give that length as your answer, rounded to three decimal places. Note : For convenience, the elevation function shown above is repeated below, in a form suitable for most programming languages: h=( 5000-0.005*(x*x+y*y+x*y)+12.5*(x+y) ) * exp( -abs(0.000001*(x*x+y*y)-0.0015*(x+y)+0.7) ) \href{https://projecteuler.net/problem=262}{Zadanie 262}
\end{zadanie}
\begin{zadanie}
	Consider the number 6. The divisors of 6 are: 1,2,3 and 6. Every number from 1 up to and including 6 can be written as a sum of distinct divisors of 6: 1=1, 2=2, 3=1+2, 4=1+3, 5=2+3, 6=6. A number n is called a practical number if every number from 1 up to and including n can be expressed as a sum of distinct divisors of n . A pair of consecutive prime numbers with a difference of six is called a sexy pair (since "sex" is the Latin word for "six"). The first sexy pair is (23, 29). We may occasionally find a triple-pair, which means three consecutive sexy prime pairs, such that the second member of each pair is the first member of the next pair. We shall call a number n such that : Find the sum of the first four engineers paradises. \href{https://projecteuler.net/problem=263}{Zadanie 263}
\end{zadanie}
\begin{zadanie}
	Consider all the triangles having: There are nine such triangles having a perimeter $\leq$ 50. Listed and shown in ascending order of their perimeter, they are: The sum of their perimeters, rounded to four decimal places, is 291.0089. Find all such triangles with a perimeter $\leq$ 10 5 . Enter as your answer the sum of their perimeters rounded to four decimal places. \href{https://projecteuler.net/problem=264}{Zadanie 264}
\end{zadanie}
\begin{zadanie}
	2 N binary digits can be placed in a circle so that all the N-digit clockwise subsequences are distinct. For N=3, two such circular arrangements are possible, ignoring rotations: For the first arrangement, the 3-digit subsequences, in clockwise order, are: 000, 001, 010, 101, 011, 111, 110 and 100. Each circular arrangement can be encoded as a number by concatenating the binary digits starting with the subsequence of all zeros as the most significant bits and proceeding clockwise. The two arrangements for N=3 are thus represented as 23 and 29: Calling S(N) the sum of the unique numeric representations, we can see that S(3) = 23 + 29 = 52. Find S(5). \href{https://projecteuler.net/problem=265}{Zadanie 265}
\end{zadanie}
\begin{zadanie}
	The divisors of 12 are: 1,2,3,4,6 and 12. The largest divisor of 12 that does not exceed the square root of 12 is 3. We shall call the largest divisor of an integer n that does not exceed the square root of n the pseudo square root (PSR) of n . It can be seen that PSR(3102)=47. Let p be the product of the primes below 190. Find PSR( p ) mod 10 16 . \href{https://projecteuler.net/problem=266}{Zadanie 266}
\end{zadanie}
\begin{zadanie}
	You are given a unique investment opportunity. Starting with 1 of capital, you can choose a fixed proportion, f , of your capital to bet on a fair coin toss repeatedly for 1000 tosses. Your return is double your bet for heads and you lose your bet for tails. For example, if f =1/4, for the first toss you bet 0.25, and if heads comes up you win 0.5 and so then have 1.5. You then bet 0.375 and if the second toss is tails, you have 1.125. Choosing f to maximize your chances of having at least 1,000,000,000 after 1,000 flips, what is the chance that you become a billionaire? All computations are assumed to be exact (no rounding), but give your answer rounded to 12 digits behind the decimal point in the form 0.abcdefghijkl. \href{https://projecteuler.net/problem=267}{Zadanie 267}
\end{zadanie}
\begin{zadanie}
	It can be verified that there are 23 positive integers less than 1000 that are divisible by at least four distinct primes less than 100. Find how many positive integers less than 10 16 are divisible by at least four distinct primes less than 100. \href{https://projecteuler.net/problem=268}{Zadanie 268}
\end{zadanie}
\begin{zadanie}
	A root or zero of a polynomial P( x ) is a solution to the equation P( x ) = 0. Define P n as the polynomial whose coefficients are the digits of n . For example, P 5703 ( x ) = 5 x 3 + 7 x 2 + 3. We can see that: Define Z( k ) as the number of positive integers, n , not exceeding k for which the polynomial P n has at least one integer root. It can be verified that Z(100 000) is 14696. What is Z(10 16 )? \href{https://projecteuler.net/problem=269}{Zadanie 269}
\end{zadanie}
\begin{zadanie}
	A square piece of paper with integer dimensions N$\times$N is placed with a corner at the origin and two of its sides along the x - and y -axes. Then, we cut it up respecting the following rules: Counting any reflections or rotations as distinct, we call C(N) the number of ways to cut an N$\times$N square. For example, C(1)=2 and C(2)=30 (shown below). What is C(30) mod 10 8 ? \href{https://projecteuler.net/problem=270}{Zadanie 270}
\end{zadanie}
\begin{zadanie}
	For a positive number n , define S( n ) as the sum of the integers x, for which 1< x < n and x 3 $\equiv$1 mod n . When n =91, there are 8 possible values for x , namely : 9, 16, 22, 29, 53, 74, 79, 81. Thus, S(91)=9+16+22+29+53+74+79+81=363. Find S(13082761331670030). \href{https://projecteuler.net/problem=271}{Zadanie 271}
\end{zadanie}
\begin{zadanie}
	For a positive number n , define C( n ) as the number of the integers x, for which 1< x < n and x 3 $\equiv$1 mod n . When n =91, there are 8 possible values for x , namely : 9, 16, 22, 29, 53, 74, 79, 81. Thus, C(91)=8. Find the sum of the positive numbers n $\leq$10 11 for which C( n )=242. \href{https://projecteuler.net/problem=272}{Zadanie 272}
\end{zadanie}
\begin{zadanie}
	Consider equations of the form: a 2 + b 2 = N , 0 $\leq$ a $\leq$ b , a , b and N integer. For N =65 there are two solutions: a =1, b =8 and a =4, b =7. We call S( N ) the sum of the values of a of all solutions of a 2 + b 2 = N , 0 $\leq$ a $\leq$ b , a , b and N integer. Thus S(65) = 1 + 4 = 5. Find $\Sigma$ S( N ), for all squarefree N only divisible by primes of the form 4 k +1 with 4 k +1 < 150. \href{https://projecteuler.net/problem=273}{Zadanie 273}
\end{zadanie}
\begin{zadanie}
	For each integer p > 1 coprime to 10 there is a positive divisibility multiplier m < p which preserves divisibility by p for the following function on any positive integer, n : f ( n ) = (all but the last digit of n ) + (the last digit of n ) * m That is, if m is the divisibility multiplier for p , then f ( n ) is divisible by p if and only if n is divisible by p . (When n is much larger than p , f ( n ) will be less than n and repeated application of f provides a multiplicative divisibility test for p .) For example, the divisibility multiplier for 113 is 34. f (76275) = 7627 + 5 * 34 = 7797 : 76275 and 7797 are both divisible by 113 f (12345) = 1234 + 5 * 34 = 1404 : 12345 and 1404 are both not divisible by 113 The sum of the divisibility multipliers for the primes that are coprime to 10 and less than 1000 is 39517. What is the sum of the divisibility multipliers for the primes that are coprime to 10 and less than 10 7 ? \href{https://projecteuler.net/problem=274}{Zadanie 274}
\end{zadanie}
\begin{zadanie}
	Let us define a balanced sculpture of order n as follows: When counting the sculptures, any arrangements which are simply reflections about the y -axis, are not counted as distinct. For example, the 18 balanced sculptures of order 6 are shown below; note that each pair of mirror images (about the y -axis) is counted as one sculpture: There are 964 balanced sculptures of order 10 and 360505 of order 15. How many balanced sculptures are there of order 18? \href{https://projecteuler.net/problem=275}{Zadanie 275}
\end{zadanie}
\begin{zadanie}
	Consider the triangles with integer sides a, b and c with a $\leq$ b $\leq$ c. An integer sided triangle (a,b,c) is called primitive if gcd(a,b,c) =1. How many primitive integer sided triangles exist with a perimeter not exceeding 10 000 000? \href{https://projecteuler.net/problem=276}{Zadanie 276}
\end{zadanie}
\begin{zadanie}
	A modified Collatz sequence of integers is obtained from a starting value $a\_1$ in the following way: $a\_{n+1} = \, \,\, \frac {a\_n} 3 \quad$ if $a\_n$ is divisible by $3$. We shall denote this as a large downward step, "D". $a\_{n+1} = \frac {4 a\_n+2} 3 \, \,$ if $a\_n$ divided by $3$ gives a remainder of $1$. We shall denote this as an upward step, "U". $a\_{n+1} = \frac {2 a\_n -1} 3 \, \,$ if $a\_n$ divided by $3$ gives a remainder of $2$. We shall denote this as a small downward step, "d". The sequence terminates when some $a\_n = 1$. Given any integer, we can list out the sequence of steps. For instance if $a\_1=231$, then the sequence $\{a\_n\}=\{231,77,51,17,11,7,10,14,9,3,1\}$ corresponds to the steps "DdDddUUdDD". Of course, there are other sequences that begin with that same sequence "DdDddUUdDD....". For instance, if $a\_1=1004064$, then the sequence is DdDddUUdDDDdUDUUUdDdUUDDDUdDD. In fact, $1004064$ is the smallest possible $a\_1 > 10^6$ that begins with the sequence DdDddUUdDD. What is the smallest $a\_1 > 10^{15}$ that begins with the sequence "UDDDUdddDDUDDddDdDddDDUDDdUUDd"? \href{https://projecteuler.net/problem=277}{Zadanie 277}
\end{zadanie}
\begin{zadanie}
	Given the values of integers $1 < a\_1 < a\_2 < \dots < a\_n$, consider the linear combination $q\_1 a\_1+q\_2 a\_2 + \dots + q\_n a\_n=b$, using only integer values $q\_k \ge 0$. Note that for a given set of $a\_k$, it may be that not all values of $b$ are possible. For instance, if $a\_1=5$ and $a\_2=7$, there are no $q\_1 \ge 0$ and $q\_2 \ge 0$ such that $b$ could be $1, 2, 3, 4, 6, 8, 9, 11, 13, 16, 18$ or $23$. In fact, $23$ is the largest impossible value of $b$ for $a\_1=5$ and $a\_2=7$. We therefore call $f(5, 7) = 23$. Similarly, it can be shown that $f(6, 10, 15)=29$ and $f(14, 22, 77) = 195$. Find $\displaystyle \sum f( p\, q,p \, r, q \, r)$, where $p$, $q$ and $r$ are prime numbers and $p < q < r < 5000$. \href{https://projecteuler.net/problem=278}{Zadanie 278}
\end{zadanie}
\begin{zadanie}
	How many triangles are there with integral sides, at least one integral angle (measured in degrees), and a perimeter that does not exceed 10 8 ? \href{https://projecteuler.net/problem=279}{Zadanie 279}
\end{zadanie}
\begin{zadanie}
	A laborious ant walks randomly on a 5x5 grid. The walk starts from the central square. At each step, the ant moves to an adjacent square at random, without leaving the grid; thus there are 2, 3 or 4 possible moves at each step depending on the ant's position. At the start of the walk, a seed is placed on each square of the lower row. When the ant isn't carrying a seed and reaches a square of the lower row containing a seed, it will start to carry the seed. The ant will drop the seed on the first empty square of the upper row it eventually reaches. What's the expected number of steps until all seeds have been dropped in the top row? Give your answer rounded to 6 decimal places. \href{https://projecteuler.net/problem=280}{Zadanie 280}
\end{zadanie}
\begin{zadanie}
	You are given a pizza (perfect circle) that has been cut into m  n equal pieces and you want to have exactly one topping on each slice. Let f ( m , n ) denote the number of ways you can have toppings on the pizza with m different toppings ( m $\geq$2), using each topping on exactly n slices ( n $\geq$1). Reflections are considered distinct, rotations are not. Thus, for instance, f (2,1)=1, f (2,2)= f (3,1)=2 and f (3,2)=16. f (3,2) is shown below: Find the sum of all f ( m , n ) such that f ( m , n )$\leq$10 15 . \href{https://projecteuler.net/problem=281}{Zadanie 281}
\end{zadanie}
\begin{zadanie}
	For non-negative integers $m$, $n$, the Ackermann function $A(m,n)$ is defined as follows:  For example $A(1,0) = 2$, $A(2,2) = 7$ and $A(3,4) = 125$. Find $\displaystyle\sum\_{n=0}^6 A(n,n)$ and give your answer mod $14^8$. \href{https://projecteuler.net/problem=282}{Zadanie 282}
\end{zadanie}
\begin{zadanie}
	Consider the triangle with sides 6, 8 and 10. It can be seen that the perimeter and the area are both equal to 24. So the area/perimeter ratio is equal to 1. Consider also the triangle with sides 13, 14 and 15. The perimeter equals 42 while the area is equal to 84. So for this triangle the area/perimeter ratio is equal to 2. Find the sum of the perimeters of all integer sided triangles for which the area/perimeter ratios are equal to positive integers not exceeding 1000. \href{https://projecteuler.net/problem=283}{Zadanie 283}
\end{zadanie}
\begin{zadanie}
	The 3-digit number 376 in the decimal numbering system is an example of numbers with the special property that its square ends with the same digits: 376 2 = 141376. Let's call a number with this property a steady square. Steady squares can also be observed in other numbering systems. In the base 14 numbering system, the 3-digit number c37 is also a steady square: c37 2 = aa0c37, and the sum of its digits is c+3+7=18 in the same numbering system. The letters a, b, c and d are used for the 10, 11, 12 and 13 digits respectively, in a manner similar to the hexadecimal numbering system. For 1 $\leq$ n $\leq$ 9, the sum of the digits of all the n-digit steady squares in the base 14 numbering system is 2d8 (582 decimal). Steady squares with leading 0's are not allowed. Find the sum of the digits of all the n-digit steady squares in the base 14 numbering system for 1 $\leq$ n $\leq$ 10000 (decimal) and give your answer in the base 14 system using lower case letters where necessary. \href{https://projecteuler.net/problem=284}{Zadanie 284}
\end{zadanie}
\begin{zadanie}
	Albert chooses a positive integer k , then two real numbers a , b are randomly chosen in the interval [0,1] with uniform distribution. The square root of the sum ( k  a +1) 2 +( k  b +1) 2 is then computed and rounded to the nearest integer. If the result is equal to k , he scores k points; otherwise he scores nothing. For example, if k =6, a =0.2 and b =0.85, then ( k  a +1) 2 +( k  b +1) 2 =42.05. The square root of 42.05 is 6.484... and when rounded to the nearest integer, it becomes 6. This is equal to k , so he scores 6 points. It can be shown that if he plays 10 turns with k =1, k =2, ..., k =10, the expected value of his total score, rounded to five decimal places, is 10.20914. If he plays 10 5 turns with k =1, k =2, k =3, ..., k =10 5 , what is the expected value of his total score, rounded to five decimal places? \href{https://projecteuler.net/problem=285}{Zadanie 285}
\end{zadanie}
\begin{zadanie}
	Barbara is a mathematician and a basketball player. She has found that the probability of scoring a point when shooting from a distance x is exactly (1- x / q ), where q is a real constant greater than 50. During each practice run, she takes shots from distances x =1, x =2, ..., x =50 and, according to her records, she has precisely a 2% chance to score a total of exactly 20 points. Find q and give your answer rounded to 10 decimal places. \href{https://projecteuler.net/problem=286}{Zadanie 286}
\end{zadanie}
\begin{zadanie}
	The quadtree encoding allows us to describe a 2 N $\times$2 N black and white image as a sequence of bits (0 and 1). Those sequences are to be read from left to right like this: Consider the following 4$\times$4 image (colored marks denote places where a split can occur): This image can be described by several sequences, for example :
	" 0 0 10101010 0 1011111011 0 10101010", of length 30, or " 0 10 0 101111101110", of length 16, which is the minimal sequence for this image. For a positive integer N , define D N as the 2 N $\times$2 N image with the following coloring scheme: What is the length of the minimal sequence describing D 24 ? \href{https://projecteuler.net/problem=287}{Zadanie 287}
\end{zadanie}
\begin{zadanie}
	For any prime p the number N( p , q ) is defined by
	N( p , q ) = $\Sigma$ n =0 to q T n * p n with T n generated by the following random number generator: S 0 = 290797 S n +1 = S n 2 mod 50515093 T n = S n mod p Let Nfac( p , q ) be the factorial of N( p , q ). Let NF( p , q ) be the number of factors p in Nfac( p , q ). You are given that NF(3,10000) mod 3 20 =624955285. Find NF(61,10 7 ) mod 61 10 \href{https://projecteuler.net/problem=288}{Zadanie 288}
\end{zadanie}
\begin{zadanie}
	Let C( x , y ) be a circle passing through the points ( x , y ), ( x , y +1), ( x +1, y ) and ( x +1, y +1). For positive integers m and n, let E( m , n ) be a configuration which consists of the m  n circles: { C( x , y ): 0$\leq$ x < m , 0$\leq$ y < n , x and y are integers } An Eulerian cycle on E( m , n ) is a closed path that passes through each arc exactly once. Many such paths are possible on E( m , n ), but we are only interested in those which are not self-crossing: A non-crossing path just touches itself at lattice points, but it never crosses itself. The image below shows E(3,3) and an example of an Eulerian non-crossing path. Let L( m , n ) be the number of Eulerian non-crossing paths on E( m , n ). For example, L(1,2)=2, L(2,2)=37 and L(3,3)=104290. Find L(6,10) mod 10 10 . \href{https://projecteuler.net/problem=289}{Zadanie 289}
\end{zadanie}
\begin{zadanie}
	How many integers 0 $\leq$ n < 10 18 have the property that the sum of the digits of n equals the sum of digits of 137 n ? \href{https://projecteuler.net/problem=290}{Zadanie 290}
\end{zadanie}
\begin{zadanie}
	A prime number $p$ is called a Panaitopol prime if $p = \dfrac{x^4 - y^4}{x^3 + y^3}$ for some positive integers $x$ and $y$. Find how many Panaitopol primes are less than 5$\times$10 15 . \href{https://projecteuler.net/problem=291}{Zadanie 291}
\end{zadanie}
\begin{zadanie}
	We shall define a pythagorean polygon to be a convex polygon with the following properties: For a given integer n , define P( n ) as the number of distinct pythagorean polygons for which the perimeter is $\leq$ n . Pythagorean polygons should be considered distinct as long as none is a translation of another. You are given that P(4)=1, P(30)=3655 and P(60)=891045. Find P(120). \href{https://projecteuler.net/problem=292}{Zadanie 292}
\end{zadanie}
\begin{zadanie}
	An even positive integer N will be called admissible, if it is a power of 2 or its distinct prime factors are consecutive primes. The first twelve admissible numbers are 2,4,6,8,12,16,18,24,30,32,36,48. If N is admissible, the smallest integer M > 1 such that N+M is prime, will be called the pseudo-Fortunate number for N. For example, N=630 is admissible since it is even and its distinct prime factors are the consecutive primes 2,3,5 and 7. The next prime number after 631 is 641; hence, the pseudo-Fortunate number for 630 is M=11. It can also be seen that the pseudo-Fortunate number for 16 is 3. Find the sum of all distinct pseudo-Fortunate numbers for admissible numbers N less than 10 9 . \href{https://projecteuler.net/problem=293}{Zadanie 293}
\end{zadanie}
\begin{zadanie}
	For a positive integer k, define d(k) as the sum of the digits of k in its usual decimal representation.
	Thus d(42) = 4+2 = 6. For a positive integer n, define S(n) as the number of positive integers k < 10 n with the following properties : Find S(11 12 ) and give your answer mod 10 9 . \href{https://projecteuler.net/problem=294}{Zadanie 294}
\end{zadanie}
\begin{zadanie}
	We call the convex area enclosed by two circles a lenticular hole if: Consider the circles: C 0 : x 2 + y 2 =25 C 1 : ( x +4) 2 +( y -4) 2 =1 C 2 : ( x -12) 2 +( y -4) 2 =65 The circles C 0 , C 1 and C 2 are drawn in the picture below. C 0 and C 1 form a lenticular hole, as well as C 0 and C 2 . We call an ordered pair of positive real numbers (r 1 , r 2 ) a lenticular pair if there exist two circles with radii r 1 and r 2 that form a lenticular hole.
	We can verify that (1, 5) and (5, 65) are the lenticular pairs of the example above. Let L(N) be the number of distinct lenticular pairs (r 1 , r 2 ) for which 0 < r 1 $\leq$ r 2 $\leq$ N. We can verify that L(10) = 30 and L(100) = 3442. Find L(100 000). \href{https://projecteuler.net/problem=295}{Zadanie 295}
\end{zadanie}
\begin{zadanie}
	Given is an integer sided triangle ABC with BC $\leq$ AC $\leq$ AB . k is the angular bisector of angle ACB . m is the tangent at C to the circumscribed circle of ABC . n is a line parallel to m through B . The intersection of n and k is called E . How many triangles ABC with a perimeter not exceeding 100 000 exist such that BE has integral length? \href{https://projecteuler.net/problem=296}{Zadanie 296}
\end{zadanie}
\begin{zadanie}
	Each new term in the Fibonacci sequence is generated by adding the previous two terms. Starting with 1 and 2, the first 10 terms will be: 1, 2, 3, 5, 8, 13, 21, 34, 55, 89. Every positive integer can be uniquely written as a sum of nonconsecutive terms of the Fibonacci sequence. For example, 100 = 3 + 8 + 89. Such a sum is called the Zeckendorf representation of the number. For any integer n >0, let z ( n ) be the number of terms in the Zeckendorf representation of n . Thus, z (5)=1, z (14)=2, z (100)=3 etc. Also, for 0< n <10 6 , $\Sigma$ z ( n )=7894453. Find $\Sigma$ z ( n ) for 0< n <10 17 . \href{https://projecteuler.net/problem=297}{Zadanie 297}
\end{zadanie}
\begin{zadanie}
	Larry and Robin play a memory game involving of a sequence of random numbers between 1 and 10, inclusive, that are called out one at a time. Each player can remember up to 5 previous numbers. When the called number is in a player's memory, that player is awarded a point. If it's not, the player adds the called number to his memory, removing another number if his memory is full. Both players start with empty memories. Both players always add new missed numbers to their memory but use a different strategy in deciding which number to remove: Larry's strategy is to remove the number that hasn't been called in the longest time. Robin's strategy is to remove the number that's been in the memory the longest time. Example game: Denoting Larry's score by L and Robin's score by R , what is the expected value of | L - R | after 50 turns? Give your answer rounded to eight decimal places using the format x.xxxxxxxx . \href{https://projecteuler.net/problem=298}{Zadanie 298}
\end{zadanie}
\begin{zadanie}
	Four points with integer coordinates are selected: A( a ,0), B( b ,0), C(0, c ) and D(0, d ), with 0< a < b and 0< c < d . Point P, also with integer coordinates, is chosen on the line AC so that the three triangles ABP, CDP and BDP are all similar . It is easy to prove that the three triangles can be similar, only if a = c . So, given that a = c , we are looking for triplets ( a , b , d ) such that at least one point P (with integer coordinates) exists on AC, making the three triangles ABP, CDP and BDP all similar. For example, if ( a , b , d )=(2,3,4), it can be easily verified that point P(1,1) satisfies the above condition. Note that the triplets (2,3,4) and (2,4,3) are considered as distinct, although point P(1,1) is common for both. If b + d <100, there are 92 distinct triplets ( a , b , d ) such that point P exists. If b + d <100 000, there are 320471 distinct triplets ( a , b , d ) such that point P exists. If b + d <100 000 000, how many distinct triplets ( a , b , d ) are there such that point P exists? \href{https://projecteuler.net/problem=299}{Zadanie 299}
\end{zadanie}
\begin{zadanie}
	In a very simplified form, we can consider proteins as strings consisting of hydrophobic (H) and polar (P) elements, e.g. HHPPHHHPHHPH. For this problem, the orientation of a protein is important; e.g. HPP is considered distinct from PPH. Thus, there are 2 n distinct proteins consisting of n elements. When one encounters these strings in nature, they are always folded in such a way that the number of H-H contact points is as large as possible, since this is energetically advantageous. As a result, the H-elements tend to accumulate in the inner part, with the P-elements on the outside. Natural proteins are folded in three dimensions of course, but we will only consider protein folding in two dimensions . The figure below shows two possible ways that our example protein could be folded (H-H contact points are shown with red dots). The folding on the left has only six H-H contact points, thus it would never occur naturally. On the other hand, the folding on the right has nine H-H contact points, which is optimal for this string. Assuming that H and P elements are equally likely to occur in any position along the string, the average number of H-H contact points in an optimal folding of a random protein string of length 8 turns out to be 850/2 8 =3.3203125. What is the average number of H-H contact points in an optimal folding of a random protein string of length 15? Give your answer using as many decimal places as necessary for an exact result. \href{https://projecteuler.net/problem=300}{Zadanie 300}
\end{zadanie}
\begin{zadanie}
	Nim is a game played with heaps of stones, where two players take it in turn to remove any number of stones from any heap until no stones remain. We'll consider the three-heap normal-play version of Nim, which works as follows: - At the start of the game there are three heaps of stones. - On his turn the player removes any positive number of stones from any single heap. - The first player unable to move (because no stones remain) loses. If ( n 1 , n 2 , n 3 ) indicates a Nim position consisting of heaps of size n 1 , n 2 and n 3 then there is a simple function X ( n 1 , n 2 , n 3 )  that you may look up or attempt to deduce for yourself  that returns: For example X (1,2,3) = 0 because, no matter what the current player does, his opponent can respond with a move that leaves two heaps of equal size, at which point every move by the current player can be mirrored by his opponent until no stones remain; so the current player loses. To illustrate: - current player moves to (1,2,1) - opponent moves to (1,0,1) - current player moves to (0,0,1) - opponent moves to (0,0,0), and so wins. For how many positive integers n $\leq$2 30 does X ( n ,2 n ,3 n ) = 0 ? \href{https://projecteuler.net/problem=301}{Zadanie 301}
\end{zadanie}
\begin{zadanie}
	A positive integer n is powerful if p 2 is a divisor of n for every prime factor p in n . A positive integer n is a perfect power if n can be expressed as a power of another positive integer. A positive integer n is an Achilles number if n is powerful but not a perfect power. For example, 864 and 1800 are Achilles numbers: 864 = 2 5 3 3 and 1800 = 2 3 3 2 5 2 . We shall call a positive integer S a Strong Achilles number if both S and $\phi$( S ) are Achilles numbers. 1 For example, 864 is a Strong Achilles number: $\phi$(864) = 288 = 2 5 3 2 . However, 1800 isn't a Strong Achilles number because: $\phi$(1800) = 480 = 2 5 3 1 5 1 . There are 7 Strong Achilles numbers below 10 4 and 656 below 10 8 . How many Strong Achilles numbers are there below 10 18 ? 1 $\phi$ denotes Euler's totient function . \href{https://projecteuler.net/problem=302}{Zadanie 302}
\end{zadanie}
\begin{zadanie}
	For a positive integer n , define f ( n ) as the least positive multiple of n that, written in base 10, uses only digits $\leq$ 2. Thus f (2)=2, f (3)=12, f (7)=21, f (42)=210, f (89)=1121222. Also, $\sum \limits\_{n = 1}^{100} {\dfrac{f(n)}{n}} = 11363107$. Find $\sum \limits\_{n=1}^{10000} {\dfrac{f(n)}{n}}$. \href{https://projecteuler.net/problem=303}{Zadanie 303}
\end{zadanie}
\begin{zadanie}
	For any positive integer n the function next\_prime( n ) returns the smallest prime p such that p> n . The sequence a( n ) is defined by: a(1)=next\_prime(10 14 ) and a( n )=next\_prime(a( n -1)) for n>1. The fibonacci sequence f( n ) is defined by:
	f(0)=0, f(1)=1 and f( n )=f( n -1)+f( n -2) for n >1. The sequence b( n ) is defined as f(a( n )). Find $\Sigma$ b( n ) for 1$\leq$ n $\leq$100 000. Give your answer mod 1234567891011. \href{https://projecteuler.net/problem=304}{Zadanie 304}
\end{zadanie}
\begin{zadanie}
	Let's call S the (infinite) string that is made by concatenating the consecutive positive integers (starting from 1) written down in base 10. Thus, S = 1234567891011121314151617181920212223242... It's easy to see that any number will show up an infinite number of times in S. Let's call f(n) the starting position of the n th occurrence of n in S. For example, f(1)=1, f(5)=81, f(12)=271 and f(7780)=111111365. Find $\Sigma$ f(3 k ) for 1$\leq$k$\leq$13. \href{https://projecteuler.net/problem=305}{Zadanie 305}
\end{zadanie}
\begin{zadanie}
	The following game is a classic example of Combinatorial Game Theory: Two players start with a strip of n white squares and they take alternate turns. On each turn, a player picks two contiguous white squares and paints them black. The first player who cannot make a move loses. So, for 1 $\leq$ n $\leq$ 5, there are 3 values of n for which the first player can force a win. Similarly, for 1 $\leq$ n $\leq$ 50, there are 40 values of n for which the first player can force a win. For 1 $\leq$ n $\leq$ 1 000 000, how many values of n are there for which the first player can force a win? \href{https://projecteuler.net/problem=306}{Zadanie 306}
\end{zadanie}
\begin{zadanie}
	k defects are randomly distributed amongst n integrated-circuit chips produced by a factory (any number of defects may be found on a chip and each defect is independent of the other defects). Let p( k,n ) represent the probability that there is a chip with at least 3 defects. For instance p(3,7) $\approx$ 0.0204081633. Find p(20 000, 1 000 000) and give your answer rounded to 10 decimal places in the form 0.abcdefghij \href{https://projecteuler.net/problem=307}{Zadanie 307}
\end{zadanie}
\begin{zadanie}
	A program written in the programming language Fractran consists of a list of fractions. The internal state of the Fractran Virtual Machine is a positive integer, which is initially set to a seed value. Each iteration of a Fractran program multiplies the state integer by the first fraction in the list which will leave it an integer. For example, one of the Fractran programs that John Horton Conway wrote for prime-generation consists of the following 14 fractions: Starting with the seed integer 2, successive iterations of the program produce the sequence: 15, 825, 725, 1925, 2275, 425, ..., 68, 4 , 30, ..., 136, 8 , 60, ..., 544, 32 , 240, ... The powers of 2 that appear in this sequence are 2 2 , 2 3 , 2 5 , ... It can be shown that all the powers of 2 in this sequence have prime exponents and that all the primes appear as exponents of powers of 2, in proper order! If someone uses the above Fractran program to solve Project Euler Problem 7 (find the 10001 st prime), how many iterations would be needed until the program produces 2 10001st prime ? \href{https://projecteuler.net/problem=308}{Zadanie 308}
\end{zadanie}
\begin{zadanie}
	In the classic "Crossing Ladders" problem, we are given the lengths x and y of two ladders resting on the opposite walls of a narrow, level street. We are also given the height h above the street where the two ladders cross and we are asked to find the width of the street ( w ). Here, we are only concerned with instances where all four variables are positive integers. For example, if x = 70, y = 119 and h = 30, we can calculate that w = 56. In fact, for integer values x , y , h and 0 < x < y < 200, there are only five triplets ( x , y , h ) producing integer solutions for w : (70, 119, 30), (74, 182, 21), (87, 105, 35), (100, 116, 35) and (119, 175, 40). For integer values x , y , h and 0 < x < y < 1 000 000, how many triplets ( x , y , h ) produce integer solutions for w ? \href{https://projecteuler.net/problem=309}{Zadanie 309}
\end{zadanie}
\begin{zadanie}
	Alice and Bob play the game Nim Square. Nim Square is just like ordinary three-heap normal play Nim, but the players may only remove a square number of stones from a heap. The number of stones in the three heaps is represented by the ordered triple (a,b,c). If 0$\leq$a$\leq$b$\leq$c$\leq$29 then the number of losing positions for the next player is 1160. Find the number of losing positions for the next player if 0$\leq$a$\leq$b$\leq$c$\leq$100 000. \href{https://projecteuler.net/problem=310}{Zadanie 310}
\end{zadanie}
\begin{zadanie}
	ABCD is a convex, integer sided quadrilateral with 1 $\leq$ AB < BC < CD < AD. BD has integer length. O is the midpoint of BD. AO has integer length. We'll call ABCD a biclinic integral quadrilateral if AO = CO $\leq$ BO = DO. For example, the following quadrilateral is a biclinic integral quadrilateral: AB = 19, BC = 29, CD = 37, AD = 43, BD = 48 and AO = CO = 23. Let B( N ) be the number of distinct biclinic integral quadrilaterals ABCD that satisfy AB 2 +BC 2 +CD 2 +AD 2 $\leq$ N . We can verify that B(10 000) = 49 and B(1 000 000) = 38239. Find B(10 000 000 000). \href{https://projecteuler.net/problem=311}{Zadanie 311}
\end{zadanie}
\begin{zadanie}
	- A Sierpiski graph of order-1 ( S 1 ) is an equilateral triangle. - S n +1 is obtained from S n by positioning three copies of S n so that every pair of copies has one common corner. Let C( n ) be the number of cycles that pass exactly once through all the vertices of S n . For example, C(3) = 8 because eight such cycles can be drawn on S 3 , as shown below: It can also be verified that : C(1) = C(2) = 1 C(5) = 71328803586048 C(10 000) mod 10 8 = 37652224 C(10 000) mod 13 8 = 617720485 Find C(C(C(10 000))) mod 13 8 . \href{https://projecteuler.net/problem=312}{Zadanie 312}
\end{zadanie}
\begin{zadanie}
	In a sliding game a counter may slide horizontally or vertically into an empty space. The objective of the game is to move the red counter from the top left corner of a grid to the bottom right corner; the space always starts in the bottom right corner. For example, the following sequence of pictures show how the game can be completed in five moves on a 2 by 2 grid. Let S(m,n) represent the minimum number of moves to complete the game on an m by n grid. For example, it can be verified that S(5,4) = 25. There are exactly 5482 grids for which S(m,n) = p 2 , where p < 100 is prime. How many grids does S(m,n) = p 2 , where p < 10 6 is prime? \href{https://projecteuler.net/problem=313}{Zadanie 313}
\end{zadanie}
\begin{zadanie}
	The moon has been opened up, and land can be obtained for free, but there is a catch. You have to build a wall around the land that you stake out, and building a wall on the moon is expensive. Every country has been allotted a 500 m by 500 m square area, but they will possess only that area which they wall in. 251001 posts have been placed in a rectangular grid with 1 meter spacing. The wall must be a closed series of straight lines, each line running from post to post. The bigger countries of course have built a 2000 m wall enclosing the entire 250 000 m 2 area. The Duchy of Grand Fenwick , has a tighter budget, and has asked you (their Royal Programmer) to compute what shape would get best maximum enclosed-area/wall-length ratio. You have done some preliminary calculations on a sheet of paper.
	For a 2000 meter wall enclosing the 250 000 m 2 area the
	enclosed-area/wall-length ratio is 125. Although not allowed , but to get an idea if this is anything better: if you place a circle inside the square area touching the four sides the area will be equal to *250 2 m 2 and the perimeter will be *500 m, so the enclosed-area/wall-length ratio will also be 125. However, if you cut off from the square four triangles with sides 75 m, 75 m and 752 m the total area becomes 238750 m 2 and the perimeter becomes 1400+3002 m. So this gives an enclosed-area/wall-length ratio of 130.87, which is significantly better. Find the maximum enclosed-area/wall-length ratio. Give your answer rounded to 8 places behind the decimal point in the form abc.defghijk. \href{https://projecteuler.net/problem=314}{Zadanie 314}
\end{zadanie}
\begin{zadanie}
	Sam and Max are asked to transform two digital clocks into two "digital root" clocks. A digital root clock is a digital clock that calculates digital roots step by step. When a clock is fed a number, it will show it and then it will start the calculation, showing all the intermediate values until it gets to the result. For example, if the clock is fed the number 137, it will show: " 137 " $\rightarrow$ " 11 " $\rightarrow$ " 2 " and then it will go black, waiting for the next number. Every digital number consists of some light segments: three horizontal (top, middle, bottom) and four vertical (top-left, top-right, bottom-left, bottom-right). Number " 1 " is made of vertical top-right and bottom-right, number " 4 " is made by middle horizontal and vertical top-left, top-right and bottom-right. Number " 8 " lights them all. The clocks consume energy only when segments are turned on/off. To turn on a " 2 " will cost 5 transitions, while a " 7 " will cost only 4 transitions. Sam and Max built two different clocks. Sam's clock is fed e.g. number 137: the clock shows " 137 ", then the panel is turned off, then the next number (" 11 ") is turned on, then the panel is turned off again and finally the last number (" 2 ") is turned on and, after some time, off. For the example, with number 137, Sam's clock requires: Max's clock works differently. Instead of turning off the whole panel, it is smart enough to turn off only those segments that won't be needed for the next number. For number 137, Max's clock requires: Of course, Max's clock consumes less power than Sam's one. The two clocks are fed all the prime numbers between A = 10 7 and B = 2$\times$10 7 . Find the difference between the total number of transitions needed by Sam's clock and that needed by Max's one. \href{https://projecteuler.net/problem=315}{Zadanie 315}
\end{zadanie}
\begin{zadanie}
	Let p = p 1 p 2 p 3 ... be an infinite sequence of random digits, selected from {0,1,2,3,4,5,6,7,8,9} with equal probability. It can be seen that p corresponds to the real number 0. p 1 p 2 p 3 .... It can also be seen that choosing a random real number from the interval [0,1) is equivalent to choosing an infinite sequence of random digits selected from {0,1,2,3,4,5,6,7,8,9} with equal probability. For any positive integer n with d decimal digits, let k be the smallest index such that p k , p k+1 , ... p k+d-1 are the decimal digits of n , in the same order. Also, let g ( n ) be the expected value of k ; it can be proven that g ( n ) is always finite and, interestingly, always an integer number. For example, if n = 535, then for p = 31415926 535 897...., we get k = 9 for p = 35528714365004956000049084876408468 535 4..., we get k = 36 etc and we find that g (535) = 1008. Given that $\sum \limits\_{n = 2}^{999} {g \left ( \left \lfloor \dfrac{10^6}{n} \right \rfloor \right )} = 27280188$, find $\sum \limits\_{n = 2}^{999999} {g \left ( \left \lfloor \dfrac{10^{16}}{n} \right \rfloor \right )}$. \href{https://projecteuler.net/problem=316}{Zadanie 316}
\end{zadanie}
\begin{zadanie}
	A firecracker explodes at a height of 100 m above level ground. It breaks into a large number of very small fragments, which move in every direction; all of them have the same initial velocity of 20 m/s. We assume that the fragments move without air resistance, in a uniform gravitational field with g=9.81 m/s 2 . Find the volume (in m 3 ) of the region through which the fragments move before reaching the ground. Give your answer rounded to four decimal places. \href{https://projecteuler.net/problem=317}{Zadanie 317}
\end{zadanie}
\begin{zadanie}
	Consider the real number $\sqrt 2 + \sqrt 3$. When we calculate the even powers of $\sqrt 2 + \sqrt 3$
	we get: $(\sqrt 2 + \sqrt 3)^2 = 9.898979485566356 \dots $ $(\sqrt 2 + \sqrt 3)^4 = 97.98979485566356 \dots $ $(\sqrt 2 + \sqrt 3)^6 = 969.998969071069263 \dots $ $(\sqrt 2 + \sqrt 3)^8 = 9601.99989585502907 \dots $ $(\sqrt 2 + \sqrt 3)^{10} = 95049.999989479221 \dots $ $(\sqrt 2 + \sqrt 3)^{12} = 940897.9999989371855 \dots $ $(\sqrt 2 + \sqrt 3)^{14} = 9313929.99999989263 \dots $ $(\sqrt 2 + \sqrt 3)^{16} = 92198401.99999998915 \dots $ It looks as if the number of consecutive nines at the beginning of the fractional part of these powers is non-decreasing. In fact it can be proven that the fractional part of $(\sqrt 2 + \sqrt 3)^{2 n}$ approaches $1$ for large $n$. Consider all real numbers of the form $\sqrt p + \sqrt q$ with $p$ and $q$ positive integers and $p < q$, such that the fractional part of $(\sqrt p + \sqrt q)^{ 2 n}$ approaches $1$ for large $n$. Let $C(p,q,n)$ be the number of consecutive nines at the beginning of the fractional part of $(\sqrt p + \sqrt q)^{ 2 n}$. Let $N(p,q)$ be the minimal value of $n$ such that $C(p,q,n) \ge 2011$. Find $\displaystyle \sum N(p,q) \,\, \text{ for } p+q \le 2011$. \href{https://projecteuler.net/problem=318}{Zadanie 318}
\end{zadanie}
\begin{zadanie}
	Let x 1 , x 2 ,..., x n be a sequence of length n such that: There are only five such sequences of length 2, namely:
	{2,4}, {2,5}, {2,6}, {2,7} and {2,8}. There are 293 such sequences of length 5; three examples are given below: {2,5,11,25,55}, {2,6,14,36,88}, {2,8,22,64,181}. Let t ( n ) denote the number of such sequences of length n . You are given that t (10) = 86195 and t (20) = 5227991891. Find t (10 10 ) and give your answer modulo 10 9 . \href{https://projecteuler.net/problem=319}{Zadanie 319}
\end{zadanie}
\begin{zadanie}
	Let N( i ) be the smallest integer n such that n ! is divisible by ( i !) 1234567890 Let S( u )= $\Sigma$ N( i ) for 10 $\leq$ i $\leq$ u . S(1000)=614538266565663. Find S(1 000 000) mod 10 18 . \href{https://projecteuler.net/problem=320}{Zadanie 320}
\end{zadanie}
\begin{zadanie}
	A horizontal row comprising of 2 n + 1 squares has n red counters placed at one end and n blue counters at the other end, being separated by a single empty square in the centre. For example, when n = 3. A counter can move from one square to the next (slide) or can jump over another counter (hop) as long as the square next to that counter is unoccupied. Let M( n ) represent the minimum number of moves/actions to completely reverse the positions of the coloured counters; that is, move all the red counters to the right and all the blue counters to the left. It can be verified M(3) = 15, which also happens to be a triangle number. If we create a sequence based on the values of n for which M( n ) is a triangle number then the first five terms would be: 1, 3, 10, 22, and 63, and their sum would be 99. Find the sum of the first forty terms of this sequence. \href{https://projecteuler.net/problem=321}{Zadanie 321}
\end{zadanie}
\begin{zadanie}
	Let T( m , n ) be the number of the binomial coefficients i C n that are divisible by 10 for n $\leq$ i < m ( i , m and n are positive integers). You are given that T(10 9 , 10 7 -10) = 989697000. Find T(10 18 , 10 12 -10). \href{https://projecteuler.net/problem=322}{Zadanie 322}
\end{zadanie}
\begin{zadanie}
	Let y 0 , y 1 , y 2 ,... be a sequence of random unsigned 32 bit integers (i.e. 0 $\leq$ y i < 2 32 , every value equally likely). For the sequence x i the following recursion is given: It can be seen that eventually there will be an index N such that x i = 2 32 -1 (a bit-pattern of all ones) for all i $\geq$ N. Find the expected value of N. Give your answer rounded to 10 digits after the decimal point. \href{https://projecteuler.net/problem=323}{Zadanie 323}
\end{zadanie}
\begin{zadanie}
	Let f ( n ) represent the number of ways one can fill a 3$\times$3$\times$ n tower with blocks of 2$\times$1$\times$1. You're allowed to rotate the blocks in any way you like; however, rotations, reflections etc of the tower itself are counted as distinct. For example (with q = 100000007) : f (2) = 229, f (4) = 117805, f (10) mod q = 96149360, f (10 3 ) mod q = 24806056, f (10 6 ) mod q = 30808124. Find f (10 10000 ) mod 100000007. \href{https://projecteuler.net/problem=324}{Zadanie 324}
\end{zadanie}
\begin{zadanie}
	A game is played with two piles of stones and two players. At her turn, a player removes a number of stones from the larger pile. The number of stones she removes must be a positive multiple of the number of stones in the smaller pile. E.g., let the ordered pair(6,14) describe a configuration with 6 stones in the smaller pile and 14 stones in the larger pile, then the first player can remove 6 or 12 stones from the larger pile. The player taking all the stones from a pile wins the game. A winning configuration is one where the first player can force a win. For example, (1,5), (2,6) and (3,12) are winning configurations because the first player can immediately remove all stones in the second pile. A losing configuration is one where the second player can force a win, no matter what the first player does. For example, (2,3) and (3,4) are losing configurations: any legal move leaves a winning configuration for the second player. Define S( N ) as the sum of ( x i + y i ) for all losing configurations ( x i , y i ), 0 < x i < y i $\leq$ N . We can verify that S(10) = 211 and S(10 4 ) = 230312207313. Find S(10 16 ) mod 7 10 . \href{https://projecteuler.net/problem=325}{Zadanie 325}
\end{zadanie}
\begin{zadanie}
	Let $a\_n$ be a sequence recursively defined by:$\quad a\_1=1,\quad\displaystyle a\_n=\biggl(\sum\_{k=1}^{n-1}k\cdot a\_k\biggr)\bmod n$. So the first 10 elements of $a\_n$ are: 1,1,0,3,0,3,5,4,1,9. Let f ( N,M ) represent the number of pairs ( p,q ) such that:  It can be seen that f (10,10)=4 with the pairs (3,3), (5,5), (7,9) and (9,10). You are also given that f (10 4 ,10 3 )=97158. Find f (10 12 ,10 6 ). \href{https://projecteuler.net/problem=326}{Zadanie 326}
\end{zadanie}
\begin{zadanie}
	A series of three rooms are connected to each other by automatic doors. Each door is operated by a security card. Once you enter a room the door automatically closes and that security card cannot be used again. A machine at the start will dispense an unlimited number of cards, but each room (including the starting room) contains scanners and if they detect that you are holding more than three security cards or if they detect an unattended security card on the floor, then all the doors will become permanently locked. However, each room contains a box where you may safely store any number of security cards for use at a later stage. If you simply tried to travel through the rooms one at a time then as you entered room 3 you would have used all three cards and would be trapped in that room forever! However, if you make use of the storage boxes, then escape is possible. For example, you could enter room 1 using your first card, place one card in the storage box, and use your third card to exit the room back to the start. Then after collecting three more cards from the dispensing machine you could use one to enter room 1 and collect the card you placed in the box a moment ago. You now have three cards again and will be able to travel through the remaining three doors. This method allows you to travel through all three rooms using six security cards in total. It is possible to travel through six rooms using a total of 123 security cards while carrying a maximum of 3 cards. Let C be the maximum number of cards which can be carried at any time. Let R be the number of rooms to travel through. Let M( C , R ) be the minimum number of cards required from the dispensing machine to travel through R rooms carrying up to a maximum of C cards at any time. For example, M(3,6)=123 and M(4,6)=23. And, $\Sigma$ M( C ,6)=146 for 3 $\leq$ C $\leq$ 4. You are given that $\Sigma$ M( C ,10)=10382 for 3 $\leq$ C $\leq$ 10. Find $\Sigma$ M( C ,30) for 3 $\leq$ C $\leq$ 40. \href{https://projecteuler.net/problem=327}{Zadanie 327}
\end{zadanie}
\begin{zadanie}
	We are trying to find a hidden number selected from the set of integers {1, 2, ..., n } by asking questions. Each number (question) we ask, has a cost equal to the number asked and we get one of three possible answers: Given the value of n , an optimal strategy minimizes the total cost (i.e. the sum of all the questions asked) for the worst possible case . E.g. If n =3, the best we can do is obviously to ask the number " 2 ". The answer will immediately lead us to find the hidden number (at a total cost = 2). If n =8, we might decide to use a "binary search" type of strategy: Our first question would be " 4 " and if the hidden number is higher than 4 we will need one or two additional questions. Let our second question be " 6 ". If the hidden number is still higher than 6, we will need a third question in order to discriminate between 7 and 8. Thus, our third question will be " 7 " and the total cost for this worst-case scenario will be 4+6+7= 17 . We can improve considerably the worst-case cost for n =8, by asking " 5 " as our first question. If we are told that the hidden number is higher than 5, our second question will be " 7 ", then we'll know for certain what the hidden number is (for a total cost of 5+7= 12 ). If we are told that the hidden number is lower than 5, our second question will be " 3 " and if the hidden number is lower than 3 our third question will be " 1 ", giving a total cost of 5+3+1= 9 . Since 12 > 9 , the worst-case cost for this strategy is 12 . That's better than what we achieved previously with the "binary search" strategy; it is also better than or equal to any other strategy. So, in fact, we have just described an optimal strategy for n =8. Let C( n ) be the worst-case cost achieved by an optimal strategy for n , as described above. Thus C(1) = 0, C(2) = 1, C(3) = 2 and C(8) = 12. Similarly, C(100) = 400 and $\sum \limits\_{n = 1}^{100} {C(n)} = 17575$. Find $\sum \limits\_{n = 1}^{200000} {C(n)}$ . \href{https://projecteuler.net/problem=328}{Zadanie 328}
\end{zadanie}
\begin{zadanie}
	Susan has a prime frog. Her frog is jumping around over 500 squares numbered 1 to 500.
	He can only jump one square to the left or to the right, with equal probability, and he cannot jump outside the range [1;500]. (if it lands at either end, it automatically jumps to the only available square on the next move.) When he is on a square with a prime number on it, he croaks 'P' (PRIME) with probability 2/3 or 'N' (NOT PRIME) with probability 1/3 just before jumping to the next square. When he is on a square with a number on it that is not a prime he croaks 'P' with probability 1/3 or 'N' with probability 2/3 just before jumping to the next square. Given that the frog's starting position is random with the same probability for every square, and given that she listens to his first 15 croaks, what is the probability that she hears the sequence PPPPNNPPPNPPNPN? \href{https://projecteuler.net/problem=329}{Zadanie 329}
\end{zadanie}
\begin{zadanie}
	For example, Find A(10 9 ) + B(10 9 ) and give your answer mod 77 777 777. \href{https://projecteuler.net/problem=330}{Zadanie 330}
\end{zadanie}
\begin{zadanie}
	N $\times$ N disks are placed on a square game board. Each disk has a black side and white side. At each turn, you may choose a disk and flip all the disks in the same row and the same column as this disk: thus 2$\times$ N -1 disks are flipped. The game ends when all disks show their white side. The following example shows a game on a 5$\times$5 board. It can be proven that 3 is the minimal number of turns to finish this game. The bottom left disk on the N $\times$ N board has coordinates (0,0); the bottom right disk has coordinates ( N -1,0) and the top left disk has coordinates (0, N -1). Let C N be the following configuration of a board with N $\times$ N disks: A disk at ( x , y ) satisfying $N - 1 \le \sqrt{x^2 + y^2} \lt N$, shows its black side; otherwise, it shows its white side. C 5 is shown above. Let T( N ) be the minimal number of turns to finish a game starting from configuration C N or 0 if configuration C N is unsolvable. We have shown that T(5)=3. You are also given that T(10)=29 and T(1 000)=395253. Find $\sum \limits\_{i = 3}^{31} {T(2^i - i)}$. \href{https://projecteuler.net/problem=331}{Zadanie 331}
\end{zadanie}
\begin{zadanie}
	A spherical triangle is a figure formed on the surface of a sphere by three great circular arcs intersecting pairwise in three vertices. Let C( r ) be the sphere with the centre (0,0,0) and radius r . Let Z( r ) be the set of points on the surface of C( r ) with integer coordinates. Let T( r ) be the set of spherical triangles with vertices in Z( r ).
	Degenerate spherical triangles, formed by three points on the same great arc, are not included in T( r ). Let A( r ) be the area of the smallest spherical triangle in T( r ). For example A(14) is 3.294040 rounded to six decimal places. Find $\sum \limits\_{r = 1}^{50} {A(r)}$. Give your answer rounded to six decimal places. \href{https://projecteuler.net/problem=332}{Zadanie 332}
\end{zadanie}
\begin{zadanie}
	All positive integers can be partitioned in such a way that each and every term of the partition can be expressed as 2 i x3 j , where i,j $\geq$ 0. Let's consider only those such partitions where none of the terms can divide any of the other terms. For example, the partition of 17 = 2 + 6 + 9 = (2 1 x3 0 + 2 1 x3 1 + 2 0 x3 2 ) would not be valid since 2 can divide 6. Neither would the partition 17 = 16 + 1 = (2 4 x3 0 + 2 0 x3 0 ) since 1 can divide 16. The only valid partition of 17 would be 8 + 9 = (2 3 x3 0 + 2 0 x3 2 ). Many integers have more than one valid partition, the first being 11 having the following two partitions. 11 = 2 + 9 = (2 1 x3 0 + 2 0 x3 2 ) 11 = 8 + 3 = (2 3 x3 0 + 2 0 x3 1 ) Let's define P( n ) as the number of valid partitions of n . For example, P(11) = 2. Let's consider only the prime integers q which would have a single valid partition such as P(17). The sum of the primes q <100 such that P( q )=1 equals 233. Find the sum of the primes q <1000000 such that P( q )=1. \href{https://projecteuler.net/problem=333}{Zadanie 333}
\end{zadanie}
\begin{zadanie}
	In Plato's heaven, there exist an infinite number of bowls in a straight line. Each bowl either contains some or none of a finite number of beans. A child plays a game, which allows only one kind of move: removing two beans from any bowl, and putting one in each of the two adjacent bowls. The game ends when each bowl contains either one or no beans. For example, consider two adjacent bowls containing 2 and 3 beans respectively, all other bowls being empty. The following eight moves will finish the game: You are given the following sequences: $\def\htmltext#1{\style{font-family:inherit;}{\text{#1}}}$
	$

		\qquad t\_0 = 123456,\cr
		\qquad t\_i = \cases{
			\;\;\frac{t\_{i-1}}{2},$\htmltext{if }t\_{i-1}\htmltext{ is even}$\cr
			\left\lfloor\frac{t\_{i-1}}{2}\right\rfloor\oplus 926252,$\htmltext{if }t\_{i-1}\htmltext{ is odd}$\cr}\cr
		\qquad\htmltext{where }\lfloor x\rfloor\htmltext{ is the floor function }\cr
		\qquad\!\htmltext{and }\oplus\htmltext{is the bitwise XOR operator.}\cr
		\qquad b\_i = (t\_i\bmod2^{11}) + 1.\cr

	$ The first two terms of the last sequence are $b\_1 = 289$ and $b\_2 = 145$. If we start with $b\_1$ and $b\_2$ beans in two adjacent bowls, $3419100$ moves would be required to finish the game. Consider now $1500$ adjacent bowls containing $b\_1, b\_2, \ldots, b\_{1500}$ beans respectively, all other bowls being empty. Find how many moves it takes before the game ends. \href{https://projecteuler.net/problem=334}{Zadanie 334}
\end{zadanie}
\begin{zadanie}
	Whenever Peter feels bored, he places some bowls, containing one bean each, in a circle. After this, he takes all the beans out of a certain bowl and drops them one by one in the bowls going clockwise. He repeats this, starting from the bowl he dropped the last bean in, until the initial situation appears again. For example with 5 bowls he acts as follows: So with 5 bowls it takes Peter 15 moves to return to the initial situation. Let M ( x ) represent the number of moves required to return to the initial situation, starting with x bowls. Thus, M (5) = 15. It can also be verified that M (100) = 10920. Find M (2 k +1). Give your answer modulo 7 9 . \href{https://projecteuler.net/problem=335}{Zadanie 335}
\end{zadanie}
\begin{zadanie}
	A train is used to transport four carriages in the order: ABCD. However, sometimes when the train arrives to collect the carriages they are not in the correct order. To rearrange the carriages they are all shunted on to a large rotating turntable. After the carriages are uncoupled at a specific point the train moves off the turntable pulling the carriages still attached with it. The remaining carriages are rotated 180 degrees. All of the carriages are then rejoined and this process is repeated as often as necessary in order to obtain the least number of uses of the turntable. Some arrangements, such as ADCB, can be solved easily: the carriages are separated between A and D, and after DCB are rotated the correct order has been achieved. However, Simple Simon, the train driver, is not known for his efficiency, so he always solves the problem by initially getting carriage A in the correct place, then carriage B, and so on. Using four carriages, the worst possible arrangements for Simon, which we shall call maximix arrangements , are DACB and DBAC; each requiring him five rotations (although, using the most efficient approach, they could be solved using just three rotations). The process he uses for DACB is shown below. It can be verified that there are 24 maximix arrangements for six carriages, of which the tenth lexicographic maximix arrangement is DFAECB. Find the 2011 th lexicographic maximix arrangement for eleven carriages. \href{https://projecteuler.net/problem=336}{Zadanie 336}
\end{zadanie}
\begin{zadanie}
	Let {a 1 , a 2 ,..., a n } be an integer sequence of length n such that: Let S( N ) be the number of such sequences with a n $\leq$ N . For example, S(10) = 4: {6}, {6, 8}, {6, 8, 9} and {6, 10}. We can verify that S(100) = 482073668 and S(10 000) mod 10 8 = 73808307. Find S(20 000 000) mod 10 8 . 1 $\phi$ denotes Euler's totient function . \href{https://projecteuler.net/problem=337}{Zadanie 337}
\end{zadanie}
\begin{zadanie}
	A rectangular sheet of grid paper with integer dimensions w $\times$ h is given. Its grid spacing is 1. When we cut the sheet along the grid lines into two pieces and rearrange those pieces without overlap, we can make new rectangles with different dimensions. For example, from a sheet with dimensions 9 $\times$ 4 , we can make rectangles with dimensions 18 $\times$ 2, 12 $\times$ 3 and 6 $\times$ 6 by cutting and rearranging as below: Similarly, from a sheet with dimensions 9 $\times$ 8 , we can make rectangles with dimensions 18 $\times$ 4 and 12 $\times$ 6 . For a pair w and h , let F( w , h ) be the number of distinct rectangles that can be made from a sheet with dimensions w $\times$ h . For example, F(2,1) = 0, F(2,2) = 1, F(9,4) = 3 and F(9,8) = 2. Note that rectangles congruent to the initial one are not counted in F( w , h ). Note also that rectangles with dimensions w $\times$ h and dimensions h $\times$ w are not considered distinct. For an integer N , let G( N ) be the sum of F( w , h ) for all pairs w and h which satisfy 0 < h $\leq$ w $\leq$ N . We can verify that G(10) = 55, G(10 3 ) = 971745 and G(10 5 ) = 9992617687. Find G(10 12 ). Give your answer modulo 10 8 . \href{https://projecteuler.net/problem=338}{Zadanie 338}
\end{zadanie}
\begin{zadanie}
	"And he came towards a valley, through which ran a river; and the borders of the valley were wooded, and on each side of the river were level meadows. And on one side of the river he saw a flock of white sheep, and on the other a flock of black sheep. And whenever one of the white sheep bleated, one of the black sheep would cross over and become white; and when one of the black sheep bleated, one of the white sheep would cross over and become black." en.wikisource.org Initially each flock consists of n sheep. Each sheep (regardless of colour) is equally likely to be the next sheep to bleat. After a sheep has bleated and a sheep from the other flock has crossed over, Peredur may remove a number of white sheep in order to maximize the expected final number of black sheep. Let E( n ) be the expected final number of black sheep if Peredur uses an optimal strategy. You are given that E(5) = 6.871346 rounded to 6 places behind the decimal point. Find E(10 000) and give your answer rounded to 6 places behind the decimal point. \href{https://projecteuler.net/problem=339}{Zadanie 339}
\end{zadanie}
\begin{zadanie}
	For fixed integers a, b, c, define the crazy function F( n ) as follows: F( n ) = n - c for all n > b F( n ) = F(a + F(a + F(a + F(a + n )))) for all n $\leq$ b. Also, define $S(a, b, c) = \sum \limits\_{n = 0}^{b} {F(n)}$. For example, if a = 50, b = 2000 and c = 40, then F(0) = 3240 and F(2000) = 2040. Also, S(50, 2000, 40) = 5204240. Find the last 9 digits of S(21 7 , 7 21 , 12 7 ). \href{https://projecteuler.net/problem=340}{Zadanie 340}
\end{zadanie}
\begin{zadanie}
	The Golomb's self-describing sequence {G( n )} is the only nondecreasing sequence of natural numbers such that n appears exactly G( n ) times in the sequence. The values of G( n ) for the first few n are You are given that G(10 3 ) = 86, G(10 6 ) = 6137. You are also given that $\Sigma$ G( n 3 ) = 153506976 for 1 $\leq$ n < 10 3 . Find $\Sigma$ G( n 3 ) for 1 $\leq$ n < 10 6 . \href{https://projecteuler.net/problem=341}{Zadanie 341}
\end{zadanie}
\begin{zadanie}
	Consider the number 50. 50 2 = 2500 = 2 2 $\times$ 5 4 , so $\phi$(2500) = 2 $\times$ 4 $\times$ 5 3 = 8 $\times$ 5 3 = 2 3 $\times$ 5 3 . 1 So 2500 is a square and $\phi$(2500) is a cube. Find the sum of all numbers n, 1 < n < 10 10 such that $\phi$(n 2 ) is a cube. 1 $\phi$ denotes Euler's totient function . \href{https://projecteuler.net/problem=342}{Zadanie 342}
\end{zadanie}
\begin{zadanie}
	For any positive integer k , a finite sequence a i of fractions x i /y i is defined by: a 1 = 1/ k and a i = (x i -1 +1)/(y i -1 -1) reduced to lowest terms for i >1. When a i reaches some integer n , the sequence stops. (That is, when y i =1.) Define f( k ) = n . For example, for k = 20: 1/20 $\rightarrow$ 2/19 $\rightarrow$ 3/18 = 1/6 $\rightarrow$ 2/5 $\rightarrow$ 3/4 $\rightarrow$ 4/3 $\rightarrow$ 5/2 $\rightarrow$ 6/1 = 6 So f(20) = 6. Also f(1) = 1, f(2) = 2, f(3) = 1 and $\Sigma$ f( k 3 ) = 118937 for 1 $\leq$ k $\leq$ 100. Find $\Sigma$ f( k 3 ) for 1 $\leq$ k $\leq$ 2$\times$10 6 . \href{https://projecteuler.net/problem=343}{Zadanie 343}
\end{zadanie}
\begin{zadanie}
	One variant of N.G. de Bruijn's silver dollar game can be described as follows: On a strip of squares a number of coins are placed, at most one coin per square. Only one coin, called the silver dollar , has any value. Two players take turns making moves. At each turn a player must make either a regular or a special move. A regular move consists of selecting one coin and moving it one or more squares to the left. The coin cannot move out of the strip or jump on or over another coin. Alternatively, the player can choose to make the special move of pocketing the leftmost coin rather than making a regular move. If no regular moves are possible, the player is forced to pocket the leftmost coin. The winner is the player who pockets the silver dollar. A winning configuration is an arrangement of coins on the strip where the first player can force a win no matter what the second player does. Let W( n , c ) be the number of winning configurations for a strip of n squares, c worthless coins and one silver dollar. You are given that W(10,2) = 324 and W(100,10) = 1514704946113500. Find W(1 000 000, 100) modulo the semiprime 1000 036 000 099 (= 1 000 003  1 000 033). \href{https://projecteuler.net/problem=344}{Zadanie 344}
\end{zadanie}
\begin{zadanie}
	We define the Matrix Sum of a matrix as the maximum sum of matrix elements with each element being the only one in his row and column. For example, the Matrix Sum of the matrix below equals 3315 ( = 863 + 383 + 343 + 959 + 767): Find the Matrix Sum of: \href{https://projecteuler.net/problem=345}{Zadanie 345}
\end{zadanie}
\begin{zadanie}
	The number 7 is special, because 7 is 111 written in base 2, and 11 written in base 6 (i.e. 7 10 = 11 6 = 111 2 ). In other words, 7 is a repunit in at least two bases b > 1. We shall call a positive integer with this property a strong repunit. It can be verified that there are 8 strong repunits below 50: {1,7,13,15,21,31,40,43}. Furthermore, the sum of all strong repunits below 1000 equals 15864. \href{https://projecteuler.net/problem=346}{Zadanie 346}
\end{zadanie}
\begin{zadanie}
	The largest integer $\leq$ 100 that is only divisible by both the primes 2 and 3 is 96, as 96=32*3=2 5 *3.
	For two distinct primes p and q let M(p,q,N) be the largest positive integer $\leq$N only divisible
	by both p and q and M(p,q,N)=0 if such a positive integer does not exist. E.g. M(2,3,100)=96. M(3,5,100)=75 and not 90 because 90 is divisible by 2 ,3 and 5. Also M(2,73,100)=0 because there does not exist a positive integer $\leq$ 100 that is divisible by both 2 and 73. Let S(N) be the sum of all distinct M(p,q,N).
	S(100)=2262. Find S(10 000 000). \href{https://projecteuler.net/problem=347}{Zadanie 347}
\end{zadanie}
\begin{zadanie}
	Many numbers can be expressed as the sum of a square and a cube. Some of them in more than one way. Consider the palindromic numbers that can be expressed as the sum of a square and a cube, both greater than 1, in exactly 4 different ways. For example, 5229225 is a palindromic number and it can be expressed in exactly 4 different ways: 2285 2 + 20 3 2223 2 + 66 3 1810 2 + 125 3 1197 2 + 156 3 Find the sum of the five smallest such palindromic numbers. \href{https://projecteuler.net/problem=348}{Zadanie 348}
\end{zadanie}
\begin{zadanie}
	An ant moves on a regular grid of squares that are coloured either black or white. The ant is always oriented in one of the cardinal directions (left, right, up or down) and moves from square to adjacent square according to the following rules: - if it is on a black square, it flips the colour of the square to white, rotates 90 degrees counterclockwise and moves forward one square. - if it is on a white square, it flips the colour of the square to black, rotates 90 degrees clockwise and moves forward one square. Starting with a grid that is entirely white, how many squares are black after 10 18 moves of the ant? \href{https://projecteuler.net/problem=349}{Zadanie 349}
\end{zadanie}
\begin{zadanie}
	A list of size n is a sequence of n natural numbers. Examples are (2,4,6), (2,6,4), (10,6,15,6), and (11). The greatest common divisor , or gcd, of a list is the largest natural number that divides all entries of the list. Examples: gcd(2,6,4) = 2, gcd(10,6,15,6) = 1 and gcd(11) = 11. The least common multiple , or lcm, of a list is the smallest natural number divisible by each entry of the list. Examples: lcm(2,6,4) = 12, lcm(10,6,15,6) = 30 and lcm(11) = 11. Let f( G , L , N ) be the number of lists of size N with gcd $\geq$ G and lcm $\leq$ L . For example: f(10, 100, 1) = 91. f(10, 100, 2) = 327. f(10, 100, 3) = 1135. f(10, 100, 1000) mod 101 4 = 3286053. Find f(10 6 , 10 12 , 10 18 ) mod 101 4 . \href{https://projecteuler.net/problem=350}{Zadanie 350}
\end{zadanie}
\begin{zadanie}
	A hexagonal orchard of order n is a triangular lattice made up of points within a regular hexagon with side n . The following is an example of a hexagonal orchard of order 5: Highlighted in green are the points which are hidden from the center by a point closer to it. It can be seen that for a hexagonal orchard of order 5, 30 points are hidden from the center. Let H( n ) be the number of points hidden from the center in a hexagonal orchard of order n . H(5) = 30. H(10) = 138. H(1 000) = 1177848. Find H(100 000 000). \href{https://projecteuler.net/problem=351}{Zadanie 351}
\end{zadanie}
\begin{zadanie}
	Each one of the 25 sheep in a flock must be tested for a rare virus, known to affect 2% of the sheep population.
	An accurate and extremely sensitive PCR test exists for blood samples, producing a clear positive / negative result, but it is very time-consuming and expensive. Because of the high cost, the vet-in-charge suggests that instead of performing 25 separate tests, the following procedure can be used instead: The sheep are split into 5 groups of 5 sheep in each group. For each group, the 5 samples are mixed together and a single test is performed. Then, Since the probability of infection for any specific animal is only 0.02, the first test (on the pooled samples) for each group will be: Thus, the expected number of tests for each group is 1 + 0.0960792032 $\times$ 5 = 1.480396016. Consequently, all 5 groups can be screened using an average of only 1.480396016 $\times$ 5 = 7.40198008 tests, which represents a huge saving of more than 70% ! Although the scheme we have just described seems to be very efficient, it can still be improved considerably (always assuming that the test is sufficiently sensitive and that there are no adverse effects caused by mixing different samples). E.g.: To simplify the very wide range of possibilities, there is one restriction we place when devising the most cost-efficient testing scheme: whenever we start with a mixed sample, all the sheep contributing to that sample must be fully screened (i.e. a verdict of infected / virus-free must be reached for all of them) before we start examining any other animals. Using the optimal strategy, let T( s , p ) represent the average number of tests needed to screen a flock of s sheep for a virus having probability p to be present in any individual. Thus, rounded to six decimal places, T(25, 0.02) = 4.155452 and T(25, 0.10) = 12.702124. Find $\Sigma$ T(10000, p) for p=0.01, 0.02, 0.03, ... 0.50. Give your answer rounded to six decimal places. \href{https://projecteuler.net/problem=352}{Zadanie 352}
\end{zadanie}
\begin{zadanie}
	A moon could be described by the sphere C( r ) with centre (0,0,0) and radius r . There are stations on the moon at the points on the surface of C( r ) with integer coordinates. The station at (0,0, r ) is called North Pole station, the station at (0,0,- r ) is called South Pole station. All stations are connected with each other via the shortest road on the great arc through the stations. A journey between two stations is risky. If d is the length of the road between two stations, ( d /( r )) 2 is a measure for the risk of the journey (let us call it the risk of the road). If the journey includes more than two stations, the risk of the journey is the sum of risks of the used roads. A direct journey from the North Pole station to the South Pole station has the length  r and risk 1. The journey from the North Pole station to the South Pole station via (0, r ,0) has the same length, but a smaller risk: ($\frac12$ r /( r )) 2 +($\frac12$ r /( r )) 2 =0.5. The minimal risk of a journey from the North Pole station to the South Pole station on C( r ) is M( r ). You are given that M(7)=0.1784943998 rounded to 10 digits behind the decimal point. Find $\Sigma$ M(2 n -1) for 1$\leq$n$\leq$15. Give your answer rounded to 10 digits behind the decimal point in the form a.bcdefghijk. \href{https://projecteuler.net/problem=353}{Zadanie 353}
\end{zadanie}
\begin{zadanie}
	Consider a honey bee's honeycomb where each cell is a perfect regular hexagon with side length $1$. One particular cell is occupied by the queen bee. For a positive real number $L$, let $\text{B}(L)$ count the cells with distance $L$ from the queen bee cell (all distances are measured from centre to centre); you may assume that the honeycomb is large enough to accommodate for any distance we wish to consider. For example, $\text{B}(\sqrt 3)=6$, $\text{B}(\sqrt {21}) = 12$ and $\text{B}(111\,111\,111) = 54$. Find the number of $L \le 5 \times 10^{11}$ such that $\text{B}(L) = 450$. \href{https://projecteuler.net/problem=354}{Zadanie 354}
\end{zadanie}
\begin{zadanie}
	Define Co( n ) to be the maximal possible sum of a set of mutually co-prime elements from {1,2,..., n }. For example Co(10) is 30 and hits that maximum on the subset {1,5,7,8,9}. You are given that Co(30) = 193 and Co(100) = 1356. Find Co(200000). \href{https://projecteuler.net/problem=355}{Zadanie 355}
\end{zadanie}
\begin{zadanie}
	Let a n be the largest real root of a polynomial g (x) = x 3 - 2 n x 2 + n . For example, a 2 = 3.86619826... Find the last eight digits of $\sum \limits\_{i = 1}^{30} {\left \lfloor a\_i^{987654321} \right \rfloor}$. Note : $\lfloor a \rfloor$ represents the floor function. \href{https://projecteuler.net/problem=356}{Zadanie 356}
\end{zadanie}
\begin{zadanie}
	Consider the divisors of 30: 1,2,3,5,6,10,15,30. It can be seen that for every divisor d of 30, d +30/ d is prime. Find the sum of all positive integers n not exceeding 100 000 000 such that
	for every divisor d of n , d + n / d is prime. \href{https://projecteuler.net/problem=357}{Zadanie 357}
\end{zadanie}
\begin{zadanie}
	A cyclic number with n digits has a very interesting property: When it is multiplied by 1, 2, 3, 4, ... n , all the products have exactly the same digits, in the same order, but rotated in a circular fashion! The smallest cyclic number is the 6-digit number 142857 : 142857 $\times$ 1 = 142857 142857 $\times$ 2 = 285714 142857 $\times$ 3 = 428571 142857 $\times$ 4 = 571428 142857 $\times$ 5 = 714285 142857 $\times$ 6 = 857142 The next cyclic number is 0588235294117647 with 16 digits : 0588235294117647 $\times$ 1 = 0588235294117647 0588235294117647 $\times$ 2 = 1176470588235294 0588235294117647 $\times$ 3 = 1764705882352941 ... 0588235294117647 $\times$ 16 = 9411764705882352 Note that for cyclic numbers, leading zeros are important. There is only one cyclic number for which, the eleven leftmost digits are 00000000137 and the five rightmost digits are 56789 (i.e., it has the form 00000000137...56789 with an unknown number of digits in the middle). Find the sum of all its digits. \href{https://projecteuler.net/problem=358}{Zadanie 358}
\end{zadanie}
\begin{zadanie}
	An infinite number of people (numbered 1, 2, 3, etc.) are lined up to get a room at Hilbert's newest infinite hotel. The hotel contains an infinite number of floors (numbered 1, 2, 3, etc.), and each floor contains an infinite number of rooms (numbered 1, 2, 3, etc.). Initially the hotel is empty. Hilbert declares a rule on how the n th person is assigned a room: person n gets the first vacant room in the lowest numbered floor satisfying either of the following: Person 1 gets room 1 in floor 1 since floor 1 is empty. Person 2 does not get room 2 in floor 1 since 1 + 2 = 3 is not a perfect square. Person 2 instead gets room 1 in floor 2 since floor 2 is empty. Person 3 gets room 2 in floor 1 since 1 + 3 = 4 is a perfect square. Eventually, every person in the line gets a room in the hotel. Define P( f , r ) to be n if person n occupies room r in floor f , and 0 if no person occupies the room. Here are a few examples: P(1, 1) = 1 P(1, 2) = 3 P(2, 1) = 2 P(10, 20) = 440 P(25, 75) = 4863 P(99, 100) = 19454 Find the sum of all P( f , r ) for all positive f and r such that f $\times$ r = 71328803586048 and give the last 8 digits as your answer. \href{https://projecteuler.net/problem=359}{Zadanie 359}
\end{zadanie}
\begin{zadanie}
	Given two points (x 1 ,y 1 ,z 1 ) and (x 2 ,y 2 ,z 2 ) in three dimensional space, the Manhattan distance between those points is defined as |x 1 -x 2 |+|y 1 -y 2 |+|z 1 -z 2 |. Let C( r ) be a sphere with radius r and center in the origin O(0,0,0). Let I( r ) be the set of all points with integer coordinates on the surface of C( r ). Let S( r ) be the sum of the Manhattan distances of all elements of I( r ) to the origin O. E.g. S(45)=34518. Find S(10 10 ). \href{https://projecteuler.net/problem=360}{Zadanie 360}
\end{zadanie}
\begin{zadanie}
	The Thue-Morse sequence {T n } is a binary sequence satisfying: The first several terms of {T n } are given as follows: 01101001 10010 1101001011001101001.... We define {A n } as the sorted sequence of integers such that the binary expression of each element appears as a subsequence in {T n }. For example, the decimal number 18 is expressed as 10010 in binary. 10010 appears in {T n } (T 8 to T 12 ), so 18 is an element of {A n }. The decimal number 14 is expressed as 1110 in binary. 1110 never appears in {T n }, so 14 is not an element of {A n }. The first several terms of A n are given as follows: We can also verify that A 100 = 3251 and A 1000 = 80852364498. Find the last 9 digits of $\sum \limits\_{k = 1}^{18} {A\_{10^k}}$. \href{https://projecteuler.net/problem=361}{Zadanie 361}
\end{zadanie}
\begin{zadanie}
	Consider the number 54. 54 can be factored in 7 distinct ways into one or more factors larger than 1: 54, 2$\times$27, 3$\times$18, 6$\times$9, 3$\times$3$\times$6, 2$\times$3$\times$9 and 2$\times$3$\times$3$\times$3. If we require that the factors are all squarefree only two ways remain: 3$\times$3$\times$6 and 2$\times$3$\times$3$\times$3. Let's call Fsf( n ) the number of ways n can be factored into one or more squarefree factors larger than 1, so
	Fsf(54)=2. Let S( n ) be $\Sigma$ Fsf( k ) for k =2 to n . S(100)=193. Find S(10 000 000 000). \href{https://projecteuler.net/problem=362}{Zadanie 362}
\end{zadanie}
\begin{zadanie}
	A cubic Bzier curve is defined by four points: P 0 , P 1 , P 2 and P 3 . The curve is constructed as follows: On the segments P 0 P 1 , P 1 P 2 and P 2 P 3 the points Q 0 ,Q 1 and Q 2 are drawn such that P 0 Q 0 / P 0 P 1 = P 1 Q 1 / P 1 P 2 = P 2 Q 2 / P 2 P 3 = t (t in [0,1]). On the segments Q 0 Q 1 and Q 1 Q 2 the points R 0 and R 1 are drawn such that Q 0 R 0 / Q 0 Q 1 = Q 1 R 1 / Q 1 Q 2 = t for the same value of t. On the segment R 0 R 1 the point B is drawn such that R 0 B / R 0 R 1 = t for the same value of t. The Bzier curve defined by the points P 0 , P 1 , P 2 , P 3 is the locus of B as Q 0 takes all possible positions on the segment P 0 P 1 . (Please note that for all points the value of t is the same.) At this (external) web address you will find an applet that allows you to drag the points P 0 , P 1 , P 2 and P 3 to see what the Bzier curve (green curve) defined by those points looks like. You can also drag the point Q 0 along the segment P 0 P 1 . From the construction it is clear that the Bzier curve will be tangent to the segments P 0 P 1 in P 0 and P 2 P 3 in P 3 . A cubic Bzier curve with P 0 =(1,0), P 1 =(1, v ), P 2 =( v ,1) and P 3 =(0,1) is used to approximate a quarter circle. The value v > 0 is chosen such that the area enclosed by the lines OP 0 , OP 3 and the curve is equal to  / 4 (the area of the quarter circle). \href{https://projecteuler.net/problem=363}{Zadanie 363}
\end{zadanie}
\begin{zadanie}
	There are N seats in a row. N people come after each other to fill the seats according to the following rules: We can verify that T(10) = 61632 and T(1 000) mod 100 000 007 = 47255094. Find T(1 000 000) mod 100 000 007. \href{https://projecteuler.net/problem=364}{Zadanie 364}
\end{zadanie}
\begin{zadanie}
	The binomial coefficient C(10 18 ,10 9 ) is a number with more than 9 billion (9$\times$10 9 ) digits. Let M(n,k,m) denote the binomial coefficient C(n,k) modulo m. Calculate $\Sigma$ M(10 18 ,10 9 ,pqr) for 1000<p<q<r<5000 and p,q,r prime. \href{https://projecteuler.net/problem=365}{Zadanie 365}
\end{zadanie}
\begin{zadanie}
	Two players, Anton and Bernhard, are playing the following game. There is one pile of n stones. The first player may remove any positive number of stones, but not the whole pile. Thereafter, each player may remove at most twice the number of stones his opponent took on the previous move. The player who removes the last stone wins. E.g. n=5 If the first player takes anything more than one stone the next player will be able to take all remaining stones. If the first player takes one stone, leaving four, his opponent will take also one stone, leaving three stones. The first player cannot take all three because he may take at most 2x1=2 stones. So let's say he takes also one stone, leaving 2. The second player can take the two remaining stones and wins. So 5 is a losing position for the first player. For some winning positions there is more than one possible move for the first player. E.g. when n=17 the first player can remove one or four stones. Let M(n) be the maximum number of stones the first player can take from a winning position at his first turn and M(n)=0 for any other position. $\Sigma$ M(n) for n$\leq$100 is 728. Find $\Sigma$ M(n) for n$\leq$10 18 .
	Give your answer modulo 10 8 . \href{https://projecteuler.net/problem=366}{Zadanie 366}
\end{zadanie}
\begin{zadanie}
	Bozo sort , not to be confused with the slightly less efficient bogo sort , consists out of checking if the input sequence is sorted and if not swapping randomly two elements. This is repeated until eventually the sequence is sorted. If we consider all permutations of the first 4 natural numbers as input the expectation value of the number of swaps, averaged over all 4! input sequences is 24.75. The already sorted sequence takes 0 steps. In this problem we consider the following variant on bozo sort. If the sequence is not in order we pick three elements at random and shuffle these three elements randomly. All 3!=6 permutations of those three elements are equally likely. The already sorted sequence will take 0 steps. If we consider all permutations of the first 4 natural numbers as input the expectation value of the number of shuffles, averaged over all 4! input sequences is 27.5. Consider as input sequences the permutations of the first 11 natural numbers. Averaged over all 11! input sequences, what is the expected number of shuffles this sorting algorithm will perform? Give your answer rounded to the nearest integer. \href{https://projecteuler.net/problem=367}{Zadanie 367}
\end{zadanie}
\begin{zadanie}
	The harmonic series $1 + \dfrac{1}{2} + \dfrac{1}{3} + \dfrac{1}{4} + ...$ is well known to be divergent. If we however omit from this series every term where the denominator has a 9 in it, the series remarkably enough converges to approximately 22.9206766193. This modified harmonic series is called the Kempner series. Let us now consider another modified harmonic series by omitting from the harmonic series every term where the denominator has 3 or more equal consecutive digits.
	One can verify that out of the first 1200 terms of the harmonic series, only 20 terms will be omitted. These 20 omitted terms are: $$\dfrac{1}{111}, \dfrac{1}{222}, \dfrac{1}{333}, \dfrac{1}{444}, \dfrac{1}{555}, \dfrac{1}{666}, \dfrac{1}{777}, \dfrac{1}{888}, \dfrac{1}{999}, \dfrac{1}{1000}, \dfrac{1}{1110}, \\\ \dfrac{1}{1111}, \dfrac{1}{1112}, \dfrac{1}{1113}, \dfrac{1}{1114}, \dfrac{1}{1115}, \dfrac{1}{1116}, \dfrac{1}{1117}, \dfrac{1}{1118}, \dfrac{1}{1119}$$ This series converges as well. Find the value the series converges to. Give your answer rounded to 10 digits behind the decimal point. \href{https://projecteuler.net/problem=368}{Zadanie 368}
\end{zadanie}
\begin{zadanie}
	In a standard 52 card deck of playing cards, a set of 4 cards is a Badugi if it contains 4 cards with no pairs and no two cards of the same suit. Let f( n ) be the number of ways to choose n cards with a 4 card subset that is a Badugi. For example, there are 2598960 ways to choose five cards from a standard 52 card deck, of which 514800 contain a 4 card subset that is a Badugi, so f(5) = 514800. Find $\Sigma$ f( n ) for 4 $\leq$ n $\leq$ 13. \href{https://projecteuler.net/problem=369}{Zadanie 369}
\end{zadanie}
\begin{zadanie}
	Let us define a geometric triangle as an integer sided triangle with sides a $\leq$ b $\leq$ c so that its sides form a geometric progression , i.e. b 2 = a  c . An example of such a geometric triangle is the triangle with sides a = 144, b = 156 and c = 169. There are 861805 geometric triangles with perimeter $\leq$ 10 6 . How many geometric triangles exist with perimeter $\leq$ 2.510 13 ? \href{https://projecteuler.net/problem=370}{Zadanie 370}
\end{zadanie}
\begin{zadanie}
	Oregon licence plates consist of three letters followed by a three digit number (each digit can be from [0..9]). While driving to work Seth plays the following game: Whenever the numbers of two licence plates seen on his trip add to 1000 that's a win. E.g. MIC-012 and HAN-988 is a win and RYU-500 and SET-500 too. (as long as he sees them in the same trip). Find the expected number of plates he needs to see for a win. Give your answer rounded to 8 decimal places behind the decimal point. Note: We assume that each licence plate seen is equally likely to have any three digit number on it. \href{https://projecteuler.net/problem=371}{Zadanie 371}
\end{zadanie}
\begin{zadanie}
	Let $R(M, N)$ be the number of lattice points $(x, y)$ which satisfy $M\!\lt\!x\!\le\!N$, $M\!\lt\!y\!\le\!N$ and $\large\left\lfloor\!\frac{y^2}{x^2}\!\right\rfloor$ is odd. We can verify that $R(0, 100) = 3019$ and $R(100, 10000) = 29750422$. Find $R(2\cdot10^6, 10^9)$. Note : $\lfloor x\rfloor$ represents the floor function. \href{https://projecteuler.net/problem=372}{Zadanie 372}
\end{zadanie}
\begin{zadanie}
	Every triangle has a circumscribed circle that goes through the three vertices.
	Consider all integer sided triangles for which the radius of the circumscribed circle is integral as well. Let S( n ) be the sum of the radii of the circumscribed circles of all such triangles for which the radius does not exceed n . S(100)=4950 and S(1200)=1653605. Find S(10 7 ). \href{https://projecteuler.net/problem=373}{Zadanie 373}
\end{zadanie}
\begin{zadanie}
	An integer partition of a number n is a way of writing n as a sum of positive integers. Partitions that differ only in the order of their summands are considered the same.
	A partition of n into distinct parts is a partition of n in which every part occurs at most once. The partitions of 5 into distinct parts are: 5, 4+1 and 3+2. Let f( n ) be the maximum product of the parts of any such partition of n into distinct parts and let m( n ) be the number of elements of any such partition of n with that product. So f(5)=6 and m(5)=2. For n =10 the partition with the largest product is 10=2+3+5, which gives f(10)=30 and m(10)=3. And their product, f(10)m(10) = 303 = 90 It can be verified that $\Sigma$ f( n )m( n ) for 1 $\leq$ n $\leq$ 100 = 1683550844462. Find $\Sigma$ f( n )m( n ) for 1 $\leq$ n $\leq$ 10 14 . Give your answer modulo 982451653, the 50 millionth prime. \href{https://projecteuler.net/problem=374}{Zadanie 374}
\end{zadanie}
\begin{zadanie}
	Let S n be an integer sequence produced with the following pseudo-random number generator: Let A( i , j ) be the minimum of the numbers S i , S i +1 , ... , S j for i $\leq$ j . Let M( N ) = $\Sigma$ A( i , j ) for 1 $\leq$ i $\leq$ j $\leq$ N . We can verify that M(10) = 432256955 and M(10 000) = 3264567774119. Find M(2 000 000 000). \href{https://projecteuler.net/problem=375}{Zadanie 375}
\end{zadanie}
\begin{zadanie}
	Consider the following set of dice with nonstandard pips: Die A: 1 4 4 4 4 4 Die B: 2 2 2 5 5 5 Die C: 3 3 3 3 3 6 A game is played by two players picking a die in turn and rolling it. The player who rolls the highest value wins. If the first player picks die A and the second player picks die B we get P(second player wins) = 7 / 12 > 1 / 2 If the first player picks die B and the second player picks die C we get P(second player wins) = 7 / 12 > 1 / 2 If the first player picks die C and the second player picks die A we get P(second player wins) = 25 / 36 > 1 / 2 So whatever die the first player picks, the second player can pick another die and have a larger than 50% chance of winning. A set of dice having this property is called a nontransitive set of dice . We wish to investigate how many sets of nontransitive dice exist. We will assume the following conditions: For N = 7 we find there are 9780 such sets. How many are there for N = 30 ? \href{https://projecteuler.net/problem=376}{Zadanie 376}
\end{zadanie}
\begin{zadanie}
	There are 16 positive integers that do not have a zero in their digits and that have a digital sum equal to 5, namely: 5, 14, 23, 32, 41, 113, 122, 131, 212, 221, 311, 1112, 1121, 1211, 2111 and 11111. Their sum is 17891. Let f ( n ) be the sum of all positive integers that do not have a zero in their digits and have a digital sum equal to n . Find $\displaystyle \sum\_{i=1}^{17} f(13^i)$. Give the last 9 digits as your answer. \href{https://projecteuler.net/problem=377}{Zadanie 377}
\end{zadanie}
\begin{zadanie}
	Let dT( n ) be the number of divisors of T( n ). E.g.:
	T(7) = 28 and dT(7) = 6. Let Tr( n ) be the number of triples ( i , j , k ) such that 1 $\leq$ i < j < k $\leq$ n and dT( i ) > dT( j ) > dT( k ). Tr(20) = 14, Tr(100) = 5772 and Tr(1000) = 11174776. Find Tr(60 000 000). Give the last 18 digits of your answer. \href{https://projecteuler.net/problem=378}{Zadanie 378}
\end{zadanie}
\begin{zadanie}
	Let f ( n ) be the number of couples ( x , y ) with x and y positive integers, x $\leq$ y and the least common multiple of x and y equal to n . Let g be the summatory function of f , i.e.: g ( n ) = $\Sigma$ f ( i ) for 1 $\leq$ i $\leq$ n . You are given that g (10 6 ) = 37429395. Find g (10 12 ). \href{https://projecteuler.net/problem=379}{Zadanie 379}
\end{zadanie}
\begin{zadanie}
	An m$\times$n maze is an m$\times$n rectangular grid with walls placed between grid cells such that there is exactly one path from the top-left square to any other square. The following are examples of a 9$\times$12 maze and a 15$\times$20 maze: Let C(m,n) be the number of distinct m$\times$n mazes. Mazes which can be formed by rotation and reflection from another maze are considered distinct. It can be verified that C(1,1) = 1, C(2,2) = 4, C(3,4) = 2415, and C(9,12) = 2.5720e46 (in scientific notation rounded to 5 significant digits). Find C(100,500) and write your answer in scientific notation rounded to 5 significant digits. When giving your answer, use a lowercase e to separate mantissa and exponent.
	E.g. if the answer is 1234567891011 then the answer format would be 1.2346e12. \href{https://projecteuler.net/problem=380}{Zadanie 380}
\end{zadanie}
\begin{zadanie}
	For a prime p let S(p) = ( $\Sigma$ (p-k)!) mod(p) for 1 $\leq$ k $\leq$ 5. For example, if p=7, (7-1)! + (7-2)! + (7-3)! + (7-4)! + (7-5)! = 6! + 5! + 4! + 3! + 2! = 720+120+24+6+2 = 872. As 872 mod(7) = 4, S(7) = 4. It can be verified that $\Sigma$ S(p) = 480 for 5 $\leq$ p < 100. Find $\Sigma$ S(p) for 5 $\leq$ p < 10 8 . \href{https://projecteuler.net/problem=381}{Zadanie 381}
\end{zadanie}
\begin{zadanie}
	A polygon is a flat shape consisting of straight line segments that are joined to form a closed chain or circuit. A polygon consists of at least three sides and does not self-intersect. A set S of positive numbers is said to generate a polygon P if: For example: The set {3, 4, 5} generates a polygon with sides 3, 4, and 5 (a triangle). The set {6, 9, 11, 24} generates a polygon with sides 6, 9, 11, and 24 (a quadrilateral). The sets {1, 2, 3} and {2, 3, 4, 9} do not generate any polygon at all. Consider the sequence s, defined as follows: Let U n be the set {s 1 , s 2 , ..., s n }. For example, U 10 = {1, 2, 3, 4, 6, 9, 13, 19, 28, 41}. Let f( n ) be the number of subsets of U n which generate at least one polygon. For example, f(5) = 7, f(10) = 501 and f(25) = 18635853. Find the last 9 digits of f(10 18 ). \href{https://projecteuler.net/problem=382}{Zadanie 382}
\end{zadanie}
\begin{zadanie}
	Let f 5 ( n ) be the largest integer x for which 5 x divides n . For example, f 5 (625000) = 7. Let T 5 ( n ) be the number of integers i which satisfy f 5 ((2 i -1)!) < 2f 5 ( i !) and 1 $\leq$ i $\leq$ n . It can be verified that T 5 (10 3 ) = 68 and T 5 (10 9 ) = 2408210. Find T 5 (10 18 ). \href{https://projecteuler.net/problem=383}{Zadanie 383}
\end{zadanie}
\begin{zadanie}
	Define the sequence a(n) as the number of adjacent pairs of ones in the binary expansion of n (possibly overlapping). E.g.: a(5) = a(101 2 ) = 0, a(6) = a(110 2 ) = 1, a(7) = a(111 2 ) = 2 Define the sequence b(n) = (-1) a(n) . This sequence is called the Rudin-Shapiro sequence. Also consider the summatory sequence of b(n): $s(n) = \sum \limits\_{i = 0}^{n} {b(i)}$. The first couple of values of these sequences are: n 0 1 2 3 4 5 6 7 a(n) 0 0 0 1 0 0 1 2 b(n) 1 1 1 -1 1 1 -1 1 s(n) 1 2 3 2 3 4 3 4 The sequence s(n) has the remarkable property that all elements are positive and every positive integer k occurs exactly k times. Define g(t,c), with 1 $\leq$ c $\leq$ t, as the index in s(n) for which t occurs for the c'th time in s(n). E.g.: g(3,3) = 6, g(4,2) = 7 and g(54321,12345) = 1220847710. Let F(n) be the fibonacci sequence defined by: F(0)=F(1)=1 and F(n)=F(n-1)+F(n-2) for n>1. Define GF(t)=g(F(t),F(t-1)). Find $\sum$GF(t) for 2$\leq$t$\leq$45. \href{https://projecteuler.net/problem=384}{Zadanie 384}
\end{zadanie}
\begin{zadanie}
	For any triangle T in the plane, it can be shown that there is a unique ellipse with largest area that is completely inside T . For a given n , consider triangles T such that: - the vertices of T have integer coordinates with absolute value $\leq$ n, and - the foci 1 of the largest-area ellipse inside T are (13,0) and (-13,0). Let A( n ) be the sum of the areas of all such triangles. For example, if n = 8, there are two such triangles. Their vertices are (-4,-3),(-4,3),(8,0) and (4,3),(4,-3),(-8,0), and the area of each triangle is 36. Thus A(8) = 36 + 36 = 72. It can be verified that A(10) = 252, A(100) = 34632 and A(1000) = 3529008. Find A(1 000 000 000). 1 The foci (plural of focus ) of an ellipse are two points A and B such that for every point P on the boundary of the ellipse, AP + PB is constant. \href{https://projecteuler.net/problem=385}{Zadanie 385}
\end{zadanie}
\begin{zadanie}
	Let n be an integer and S ( n ) be the set of factors of n . A subset A of S ( n ) is called an antichain of S ( n ) if A contains only one element or if none of the elements of A divides any of the other elements of A . For example: S (30) = {1, 2, 3, 5, 6, 10, 15, 30} {2, 5, 6} is not an antichain of S (30). {2, 3, 5} is an antichain of S (30). Let N ( n ) be the maximum length of an antichain of S ( n ). Find $\Sigma$ N ( n ) for 1 $\leq$ n $\leq$ 10 8 \href{https://projecteuler.net/problem=386}{Zadanie 386}
\end{zadanie}
\begin{zadanie}
	A Harshad or Niven number is a number that is divisible by the sum of its digits. 201 is a Harshad number because it is divisible by 3 (the sum of its digits.) When we truncate the last digit from 201, we get 20, which is a Harshad number. When we truncate the last digit from 20, we get 2, which is also a Harshad number. Let's call a Harshad number that, while recursively truncating the last digit, always results in a Harshad number a right truncatable Harshad number. Also: 201/3=67 which is prime. Let's call a Harshad number that, when divided by the sum of its digits, results in a prime a strong Harshad number . Now take the number 2011 which is prime. When we truncate the last digit from it we get 201, a strong Harshad number that is also right truncatable. Let's call such primes strong, right truncatable Harshad primes . You are given that the sum of the strong, right truncatable Harshad primes less than 10000 is 90619. Find the sum of the strong, right truncatable Harshad primes less than 10 14 . \href{https://projecteuler.net/problem=387}{Zadanie 387}
\end{zadanie}
\begin{zadanie}
	Consider all lattice points (a,b,c) with 0 $\leq$ a,b,c $\leq$ N. From the origin O(0,0,0) all lines are drawn to the other lattice points. Let D(N) be the number of distinct such lines. You are given that D(1 000 000) = 831909254469114121. Find D(10 10 ). Give as your answer the first nine digits followed by the last nine digits. \href{https://projecteuler.net/problem=388}{Zadanie 388}
\end{zadanie}
\begin{zadanie}
	An unbiased single 4-sided die is thrown and its value, T , is noted. T unbiased 6-sided dice are thrown and their scores are added together. The sum, C , is noted. C unbiased 8-sided dice are thrown and their scores are added together. The sum, O , is noted. O unbiased 12-sided dice are thrown and their scores are added together. The sum, D , is noted. D unbiased 20-sided dice are thrown and their scores are added together. The sum, I , is noted. Find the variance of I , and give your answer rounded to 4 decimal places. \href{https://projecteuler.net/problem=389}{Zadanie 389}
\end{zadanie}
\begin{zadanie}
	Consider the triangle with sides $\sqrt 5$, $\sqrt {65}$ and $\sqrt {68}$.
	It can be shown that this triangle has area $9$. $S(n)$ is the sum of the areas of all triangles with sides $\sqrt{1+b^2}$, $\sqrt {1+c^2}$ and $\sqrt{b^2+c^2}\,$ (for positive integers $b$ and $c$) that have an integral area not exceeding $n$. The example triangle has $b=2$ and $c=8$. $S(10^6)=18018206$. Find $S(10^{10})$. \href{https://projecteuler.net/problem=390}{Zadanie 390}
\end{zadanie}
\begin{zadanie}
	Let s k be the number of 1s when writing the numbers from 0 to k in binary. For example, writing 0 to 5 in binary, we have 0, 1, 10, 11, 100, 101. There are seven 1s, so s 5 = 7. The sequence S = { s k : k $\geq$ 0} starts {0, 1, 2, 4, 5, 7, 9, 12, ...}. A game is played by two players. Before the game starts, a number n is chosen. A counter c starts at 0. At each turn, the player chooses a number from 1 to n (inclusive) and increases c by that number. The resulting value of c must be a member of S. If there are no more valid moves, the player loses. For example: Let n = 5. c starts at 0. Player 1 chooses 4, so c becomes 0 + 4 = 4. Player 2 chooses 5, so c becomes 4 + 5 = 9. Player 1 chooses 3, so c becomes 9 + 3 = 12. etc. Note that c must always belong to S, and each player can increase c by at most n . Let M( n ) be the highest number the first player can choose at her first turn to force a win, and M( n ) = 0 if there is no such move. For example, M(2) = 2, M(7) = 1 and M(20) = 4. Given $\Sigma$ (M( n )) 3 = 8150 for 1 $\leq$ n $\leq$ 20. Find $\Sigma$ (M( n )) 3 for 1 $\leq$ n $\leq$ 1000. \href{https://projecteuler.net/problem=391}{Zadanie 391}
\end{zadanie}
\begin{zadanie}
	A rectilinear grid is an orthogonal grid where the spacing between the gridlines does not have to be equidistant. An example of such grid is logarithmic graph paper. Consider rectilinear grids in the Cartesian coordinate system with the following properties: E.g. here is a picture of the solution for N = 10: Find the positions for N = 400. Give as your answer the area occupied by the red cells rounded to 10 digits behind the decimal point. \href{https://projecteuler.net/problem=392}{Zadanie 392}
\end{zadanie}
\begin{zadanie}
	An n $\times$ n grid of squares contains n 2 ants, one ant per square. All ants decide to move simultaneously to an adjacent square (usually 4 possibilities, except for ants on the edge of the grid or at the corners). We define f ( n ) to be the number of ways this can happen without any ants ending on the same square and without any two ants crossing the same edge between two squares. You are given that f (4) = 88. Find f (10). \href{https://projecteuler.net/problem=393}{Zadanie 393}
\end{zadanie}
\begin{zadanie}
	Jeff eats a pie in an unusual way. The pie is circular. He starts with slicing an initial cut in the pie along a radius. While there is at least a given fraction F of pie left, he performs the following procedure: - He makes two slices from the pie centre to any point of what is remaining of the pie border, any point on the remaining pie border equally likely. This will divide the remaining pie into three pieces. - Going counterclockwise from the initial cut, he takes the first two pie pieces and eats them. When less than a fraction F of pie remains, he does not repeat this procedure. Instead, he eats all of the remaining pie. For x $\geq$ 1, let E( x ) be the expected number of times Jeff repeats the procedure above with F = 1 / x . It can be verified that E(1) = 1, E(2) $\approx$ 1.2676536759, and E(7.5) $\approx$ 2.1215732071. Find E(40) rounded to 10 decimal places behind the decimal point. \href{https://projecteuler.net/problem=394}{Zadanie 394}
\end{zadanie}
\begin{zadanie}
	The Pythagorean tree is a fractal generated by the following procedure: Start with a unit square. Then, calling one of the sides its base (in the animation, the bottom side is the base): It can be shown that there exists at least one rectangle, whose sides are parallel to the largest square of the Pythagorean tree, which encloses the Pythagorean tree completely. Find the smallest area possible for such a bounding rectangle, and give your answer rounded to 10 decimal places. \href{https://projecteuler.net/problem=395}{Zadanie 395}
\end{zadanie}
\begin{zadanie}
	For any positive integer n, the nth weak Goodstein sequence {g 1 , g 2 , g 3 , ...} is defined as: For example, the 6th weak Goodstein sequence is {6, 11, 17, 25, ...}: It can be shown that every weak Goodstein sequence terminates. Let G( n ) be the number of nonzero elements in the n th weak Goodstein sequence. It can be verified that G(2) = 3, G(4) = 21 and G(6) = 381. It can also be verified that $\Sigma$ G( n ) = 2517 for 1 $\leq$ n < 8. Find the last 9 digits of $\Sigma$ G( n ) for 1 $\leq$ n < 16. \href{https://projecteuler.net/problem=396}{Zadanie 396}
\end{zadanie}
\begin{zadanie}
	On the parabola y = x 2 / k , three points A( a , a 2 / k ), B( b , b 2 / k ) and C( c , c 2 / k ) are chosen. Let F( K , X ) be the number of the integer quadruplets ( k , a , b , c ) such that at least one angle of the triangle ABC is 45-degree, with 1 $\leq$ k $\leq$ K and - X $\leq$ a < b < c $\leq$ X . For example, F(1, 10) = 41 and F(10, 100) = 12492. Find F(10 6 , 10 9 ). \href{https://projecteuler.net/problem=397}{Zadanie 397}
\end{zadanie}
\begin{zadanie}
	Inside a rope of length n , n -1 points are placed with distance 1 from each other and from the endpoints. Among these points, we choose m -1 points at random and cut the rope at these points to create m segments. Let E( n , m ) be the expected length of the second-shortest segment.
	For example, E(3, 2) = 2 and E(8, 3) = 16/7.
	Note that if multiple segments have the same shortest length the length of the second-shortest segment is defined as the same as the shortest length. Find E(10 7 , 100).
	Give your answer rounded to 5 decimal places behind the decimal point. \href{https://projecteuler.net/problem=398}{Zadanie 398}
\end{zadanie}
\begin{zadanie}
	The first 15 fibonacci numbers are: 1,1,2,3,5,8,13,21,34,55,89,144,233,377,610. It can be seen that 8 and 144 are not squarefree: 8 is divisible by 4 and 144 is divisible by 4 and by 9. So the first 13 squarefree fibonacci numbers are: 1,1,2,3,5,13,21,34,55,89,233,377 and 610. The 200th squarefree fibonacci number is:
	971183874599339129547649988289594072811608739584170445. The last sixteen digits of this number are: 1608739584170445 and in scientific notation this number can be written as 9.7e53. Find the 100 000 000th squarefree fibonacci number. Give as your answer its last sixteen digits followed by a comma followed by the number in scientific notation (rounded to one digit after the decimal point). For the 200th squarefree number the answer would have been: 1608739584170445,9.7e53 Note: For this problem, assume that for every prime p, the first fibonacci number divisible by p is not divisible by p 2 (this is part of Wall's conjecture ). This has been verified for primes $\leq$ 310 15 , but has not been proven in general. If it happens that the conjecture is false, then the accepted answer to this problem isn't guaranteed to be the 100 000 000th squarefree fibonacci number, rather it represents only a lower bound for that number. \href{https://projecteuler.net/problem=399}{Zadanie 399}
\end{zadanie}
\begin{zadanie}
	A Fibonacci tree is a binary tree recursively defined as: On such a tree two players play a take-away game. On each turn a player selects a node and removes that node along with the subtree rooted at that node. The player who is forced to take the root node of the entire tree loses. Here are the winning moves of the first player on the first turn for T( k ) from k =1 to k =6. For example, f (5) = 1 and f (10) = 17. Find f (10000). Give the last 18 digits of your answer. \href{https://projecteuler.net/problem=400}{Zadanie 400}
\end{zadanie}
\begin{zadanie}
	The divisors of 6 are 1,2,3 and 6. The sum of the squares of these numbers is 1+4+9+36=50. Let sigma2(n) represent the sum of the squares of the divisors of n.
	Thus sigma2(6)=50. Find SIGMA2(10 15 ) modulo 10 9 . \href{https://projecteuler.net/problem=401}{Zadanie 401}
\end{zadanie}
\begin{zadanie}
	It can be shown that the polynomial n 4 + 4 n 3 + 2 n 2 + 5 n is a multiple of 6 for every integer n . It can also be shown that 6 is the largest integer satisfying this property. Define M( a , b , c ) as the maximum m such that n 4 + a n 3 + b n 2 + c n is a multiple of m for all integers n . For example, M(4, 2, 5) = 6. Also, define S( N ) as the sum of M( a , b , c ) for all 0 < a , b , c $\leq$ N . We can verify that S(10) = 1972 and S(10000) = 2024258331114. Let F k be the Fibonacci sequence: F 0 = 0, F 1 = 1 and F k = F k -1 + F k -2 for k $\geq$ 2. Find the last 9 digits of $\Sigma$ S(F k ) for 2 $\leq$ k $\leq$ 1234567890123. \href{https://projecteuler.net/problem=402}{Zadanie 402}
\end{zadanie}
\begin{zadanie}
	For integers a and b , we define D ( a , b ) as the domain enclosed by the parabola y = x 2 and the line y = a  x + b : D ( a , b ) = { ( x , y ) | x 2 $\leq$ y $\leq$ a  x + b }. L( a , b ) is defined as the number of lattice points contained in D ( a , b ). For example, L(1, 2) = 8 and L(2, -1) = 1. We also define S( N ) as the sum of L( a , b ) for all the pairs ( a , b ) such that the area of D ( a , b ) is a rational number and | a |,| b | $\leq$ N . We can verify that S(5) = 344 and S(100) = 26709528. Find S(10 12 ). Give your answer mod 10 8 . \href{https://projecteuler.net/problem=403}{Zadanie 403}
\end{zadanie}
\begin{zadanie}
	E a is an ellipse with an equation of the form x 2 + 4y 2 = 4 a 2 . E a ' is the rotated image of E a by  degrees counterclockwise around the origin O(0, 0) for 0 <  < 90. b is the distance to the origin of the two intersection points closest to the origin and c is the distance of the two other intersection points. We call an ordered triplet ( a , b , c ) a canonical ellipsoidal triplet if a , b and c are positive integers. For example, (209, 247, 286) is a canonical ellipsoidal triplet. Let C( N ) be the number of distinct canonical ellipsoidal triplets ( a , b , c ) for a $\leq$ N . It can be verified that C(10 3 ) = 7, C(10 4 ) = 106 and C(10 6 ) = 11845. Find C(10 17 ). \href{https://projecteuler.net/problem=404}{Zadanie 404}
\end{zadanie}
\begin{zadanie}
	We wish to tile a rectangle whose length is twice its width. Let T (0) be the tiling consisting of a single rectangle. For n > 0, let T ( n ) be obtained from T ( n -1) by replacing all tiles in the following manner: The following animation demonstrates the tilings T ( n ) for n from 0 to 5: Let f ( n ) be the number of points where four tiles meet in T ( n ). For example, f (1) = 0, f (4) = 82 and f (10 9 ) mod 17 7 = 126897180. Find f (10 k ) for k = 10 18 , give your answer modulo 17 7 . \href{https://projecteuler.net/problem=405}{Zadanie 405}
\end{zadanie}
\begin{zadanie}
	We are trying to find a hidden number selected from the set of integers {1, 2, ..., n } by asking questions. Each number (question) we ask, we get one of three possible answers: Given the value of n , a , and b , an optimal strategy minimizes the total cost for the worst possible case . For example, if n = 5, a = 2, and b = 3, then we may begin by asking " 2 " as our first question. If we are told that 2 is higher than the hidden number (for a cost of b =3), then we are sure that " 1 " is the hidden number (for a total cost of 3 ). If we are told that 2 is lower than the hidden number (for a cost of a =2), then our next question will be " 4 ". If we are told that 4 is higher than the hidden number (for a cost of b =3), then we are sure that " 3 " is the hidden number (for a total cost of 2+3= 5 ). If we are told that 4 is lower than the hidden number (for a cost of a =2), then we are sure that " 5 " is the hidden number (for a total cost of 2+2= 4 ). Thus, the worst-case cost achieved by this strategy is 5 . It can also be shown that this is the lowest worst-case cost that can be achieved. So, in fact, we have just described an optimal strategy for the given values of n , a , and b . Let $C(n, a, b)$ be the worst-case cost achieved by an optimal strategy for the given values of n , a and b . Here are a few examples: $C(5, 2, 3) = 5$ $C(500, \sqrt 2, \sqrt 3) = 13.22073197\dots$ $C(20000, 5, 7) = 82$ $C(2000000, \sqrt 5, \sqrt 7) = 49.63755955\dots$ Let $F\_k$ be the Fibonacci numbers: $F\_k=F\_{k-1}+F\_{k-2}$ with base cases $F\_1=F\_2= 1$. Find $\displaystyle \sum \limits\_{k = 1}^{30} {C \left (10^{12}, \sqrt{k}, \sqrt{F\_k} \right )}$, and give your answer rounded to 8 decimal places behind the decimal point. \href{https://projecteuler.net/problem=406}{Zadanie 406}
\end{zadanie}
\begin{zadanie}
	If we calculate a 2 mod 6 for 0 $\leq$ a $\leq$ 5 we get: 0,1,4,3,4,1. The largest value of a such that a 2 $\equiv$ a mod 6 is 4. Let's call M( n ) the largest value of a < n such that a 2 $\equiv$ a (mod n ). So M(6) = 4. Find $\Sigma$ M( n ) for 1 $\leq$ n $\leq$ 10 7 . \href{https://projecteuler.net/problem=407}{Zadanie 407}
\end{zadanie}
\begin{zadanie}
	Let's call a lattice point ( x , y ) inadmissible if x , y and x + y are all positive perfect squares. For example, (9, 16) is inadmissible, while (0, 4), (3, 1) and (9, 4) are not. Consider a path from point ( x 1 , y 1 ) to point ( x 2 , y 2 ) using only unit steps north or east. Let's call such a path admissible if none of its intermediate points are inadmissible. Let P( n ) be the number of admissible paths from (0, 0) to ( n , n ). It can be verified that P(5) = 252, P(16) = 596994440 and P(1000) mod 1000000007 = 341920854. Find P(10000000) mod 1000000007. \href{https://projecteuler.net/problem=408}{Zadanie 408}
\end{zadanie}
\begin{zadanie}
	Let n be a positive integer. Consider nim positions where: Let W( n ) be the number of winning nim positions satisfying the above
	conditions (a position is winning if the first player has a winning strategy). For example, W(1) = 1, W(2) = 6, W(3) = 168, W(5) = 19764360 and W(100) mod 1000000007 = 384777056. Find W(10000000) mod 1000000007. \href{https://projecteuler.net/problem=409}{Zadanie 409}
\end{zadanie}
\begin{zadanie}
	Let C be the circle with radius r , x 2 + y 2 = r 2 . We choose two points P( a , b ) and Q(- a , c ) so that the line passing through P and Q is tangent to C. For example, the quadruplet ( r , a , b , c ) = (2, 6, 2, -7) satisfies this property. Let F( R , X ) be the number of the integer quadruplets ( r , a , b , c ) with this property, and with 0 < r $\leq$ R and 0 < a $\leq$ X . We can verify that F(1, 5) = 10, F(2, 10) = 52 and F(10, 100) = 3384. Find F(10 8 , 10 9 ) + F(10 9 , 10 8 ). \href{https://projecteuler.net/problem=410}{Zadanie 410}
\end{zadanie}
\begin{zadanie}
	Let n be a positive integer. Suppose there are stations at the coordinates ( x , y ) = (2 i mod n , 3 i mod n ) for 0 $\leq$ i $\leq$ 2 n . We will consider stations with the same coordinates as the same station. We wish to form a path from (0, 0) to ( n , n ) such that the x and y coordinates never decrease. Let S( n ) be the maximum number of stations such a path can pass through. For example, if n = 22, there are 11 distinct stations, and a valid path can pass through at most 5 stations. Therefore, S(22) = 5.
	The case is illustrated below, with an example of an optimal path: It can also be verified that S(123) = 14 and S(10000) = 48. Find $\Sigma$ S( k 5 ) for 1 $\leq$ k $\leq$ 30. \href{https://projecteuler.net/problem=411}{Zadanie 411}
\end{zadanie}
\begin{zadanie}
	For integers m , n (0$\leq$ n < m ), let L( m , n ) be an m $\times$ m grid with the top-right n $\times$ n grid removed. For example, L(5, 3) looks like this: We want to number each cell of L( m , n ) with consecutive integers 1, 2, 3, ... such that the number in every cell is smaller than the number below it and to the left of it. For example, here are two valid numberings of L(5,3): Let LC( m , n ) be the number of valid numberings of L( m , n ). It can be verified that LC(3,0) = 42, LC(5,3) = 250250, LC(6,3) = 406029023400 and LC(10,5) mod 76543217 = 61251715. Find LC(10000,5000) mod 76543217. \href{https://projecteuler.net/problem=412}{Zadanie 412}
\end{zadanie}
\begin{zadanie}
	We say that a d -digit positive number (no leading zeros) is a one-child number if exactly one of its sub-strings is divisible by d . For example, 5671 is a 4-digit one-child number. Among all its sub-strings 5, 6, 7, 1, 56, 67, 71, 567, 671 and 5671, only 56 is divisible by 4. Similarly, 104 is a 3-digit one-child number because only 0 is divisible by 3. 1132451 is a 7-digit one-child number because only 245 is divisible by 7. Let F( N ) be the number of the one-child numbers less than N . We can verify that F(10) = 9, F(10 3 ) = 389 and F(10 7 ) = 277674. Find F(10 19 ). \href{https://projecteuler.net/problem=413}{Zadanie 413}
\end{zadanie}
\begin{zadanie}
	6174 is a remarkable number; if we sort its digits in increasing order and subtract that number from the number you get when you sort the digits in decreasing order, we get 7641-1467=6174. Even more remarkable is that if we start from any 4 digit number and repeat this process of sorting and subtracting, we'll eventually end up with 6174 or immediately with 0 if all digits are equal. This also works with numbers that have less than 4 digits if we pad the number with leading zeroes until we have 4 digits. E.g. let's start with the number 0837: 8730-0378=8352 8532-2358=6174 6174 is called the Kaprekar constant . The process of sorting and subtracting and repeating this until either 0 or the Kaprekar constant is reached is called the Kaprekar routine . We can consider the Kaprekar routine for other bases and number of digits. Unfortunately, it is not guaranteed a Kaprekar constant exists in all cases; either the routine can end up in a cycle for some input numbers or the constant the routine arrives at can be different for different input numbers. However, it can be shown that for 5 digits and a base b = 6t+3$\neq$9, a Kaprekar constant exists. E.g. base 15: (10,4,14,9,5) 15 base 21: (14,6,20,13,7) 21 Define C b to be the Kaprekar constant in base b for 5 digits.
	Define the function sb(i) to be Define S(b) as the sum of sb(i) for 0 < i < b 5 . E.g. S(15) = 5274369 S(111) = 400668930299 Find the sum of S(6k+3) for 2 $\leq$ k $\leq$ 300. Give the last 18 digits as your answer. \href{https://projecteuler.net/problem=414}{Zadanie 414}
\end{zadanie}
\begin{zadanie}
	A set of lattice points S is called a titanic set if there exists a line passing through exactly two points in S. An example of a titanic set is S = {(0, 0), (0, 1), (0, 2), (1, 1), (2, 0), (1, 0)}, where the line passing through (0, 1) and (2, 0) does not pass through any other point in S. On the other hand, the set {(0, 0), (1, 1), (2, 2), (4, 4)} is not a titanic set since the line passing through any two points in the set also passes through the other two. For any positive integer N , let T ( N ) be the number of titanic sets S whose every point ( x , y ) satisfies 0 $\leq$ x , y $\leq$ N .
	It can be verified that T (1) = 11, T (2) = 494, T (4) = 33554178, T (111) mod 10 8 = 13500401 and T (10 5 )mod10 8 = 63259062. Find T (10 11 )mod10 8 . \href{https://projecteuler.net/problem=415}{Zadanie 415}
\end{zadanie}
\begin{zadanie}
	A row of n squares contains a frog in the leftmost square. By successive jumps the frog goes to the rightmost square and then back to the leftmost square. On the outward trip he jumps one, two or three squares to the right, and on the homeward trip he jumps to the left in a similar manner. He cannot jump outside the squares. He repeats the round-trip travel m times. Let F( m , n ) be the number of the ways the frog can travel so that at most one square remains unvisited. For example, F(1, 3) = 4, F(1, 4) = 15, F(1, 5) = 46, F(2, 3) = 16 and F(2, 100) mod 10 9 = 429619151. Find the last 9 digits of F(10, 10 12 ). \href{https://projecteuler.net/problem=416}{Zadanie 416}
\end{zadanie}
\begin{zadanie}
	A unit fraction contains 1 in the numerator. The decimal representation of the unit fractions with denominators 2 to 10 are given: Where 0.1(6) means 0.166666..., and has a 1-digit recurring cycle. It can be seen that 1 / 7 has a 6-digit recurring cycle. Unit fractions whose denominator has no other prime factors than 2 and/or 5 are not considered to have a recurring cycle. We define the length of the recurring cycle of those unit fractions as 0. Let L(n) denote the length of the recurring cycle of 1/n.
	You are given that $\Sigma$ L(n) for 3 $\leq$ n $\leq$ 1 000 000 equals 55535191115. Find $\Sigma$ L(n) for 3 $\leq$ n $\leq$ 100 000 000. \href{https://projecteuler.net/problem=417}{Zadanie 417}
\end{zadanie}
\begin{zadanie}
	Let n be a positive integer. An integer triple ( a , b , c ) is called a factorisation triple of n if: Define f ( n ) to be a + b + c for the factorisation triple ( a , b , c ) of n which minimises c / a . One can show that this triple is unique. For example, f (165) = 19, f (100100) = 142 and f (20!) = 4034872. Find f (43!). \href{https://projecteuler.net/problem=418}{Zadanie 418}
\end{zadanie}
\begin{zadanie}
	The look and say sequence goes 1, 11, 21, 1211, 111221, 312211, 13112221, 1113213211, ... The sequence starts with 1 and all other members are obtained by describing the previous member in terms of consecutive digits. It helps to do this out loud: 1 is 'one one' $\rightarrow$ 11 11 is 'two ones' $\rightarrow$ 21 21 is 'one two and one one' $\rightarrow$ 1211 1211 is 'one one, one two and two ones' $\rightarrow$ 111221 111221 is 'three ones, two twos and one one' $\rightarrow$ 312211 ... Define A(n), B(n) and C(n) as the number of ones, twos and threes in the n'th element of the sequence respectively. One can verify that A(40) = 31254, B(40) = 20259 and C(40) = 11625. Find A(n), B(n) and C(n) for n = 10 12 . Give your answer modulo 2 30 and separate your values for A, B and C by a comma. E.g. for n = 40 the answer would be 31254,20259,11625 \href{https://projecteuler.net/problem=419}{Zadanie 419}
\end{zadanie}
\begin{zadanie}
	A positive integer matrix is a matrix whose elements are all positive integers. Some positive integer matrices can be expressed as a square of a positive integer matrix in two different ways. Here is an example: We define F( N ) as the number of the 2x2 positive integer matrices which have a trace less than N and which can be expressed as a square of a positive integer matrix in two different ways. We can verify that F(50) = 7 and F(1000) = 1019. Find F(10 7 ). \href{https://projecteuler.net/problem=420}{Zadanie 420}
\end{zadanie}
\begin{zadanie}
	Numbers of the form n 15 +1 are composite for every integer n > 1. For positive integers n and m let s ( n,m ) be defined as the sum of the distinct prime factors of n 15 +1 not exceeding m . Find $\Sigma$ s ( n ,10 8 ) for 1 $\leq$ n $\leq$ 10 11 . \href{https://projecteuler.net/problem=421}{Zadanie 421}
\end{zadanie}
\begin{zadanie}
	Let H be the hyperbola defined by the equation 12 x 2 + 7 x y - 12 y 2 = 625. Next, define X as the point (7, 1). It can be seen that X is in H. Now we define a sequence of points in H, {P i : i $\geq$ 1}, as: You are given that P 3 = (-19/2, -229/24), P 4 = (1267/144, -37/12) and P 7 = (17194218091/143327232, 274748766781/1719926784). Find P n for n = 11 14 in the following format: If P n = ( a / b , c / d ) where the fractions are in lowest terms and the denominators are positive, then the answer is ( a + b + c + d ) mod 1000000007. For n = 7, the answer would have been: 806236837. \href{https://projecteuler.net/problem=422}{Zadanie 422}
\end{zadanie}
\begin{zadanie}
	Let n be a positive integer. A 6-sided die is thrown n times. Let c be the number of pairs of consecutive throws that give the same value. For example, if n = 7 and the values of the die throws are (1,1,5,6,6,6,3), then the following pairs of consecutive throws give the same value: ( 1,1 ,5,6,6,6,3) (1,1,5, 6,6 ,6,3) (1,1,5,6, 6,6 ,3) Therefore, c = 3 for (1,1,5,6,6,6,3). Define C( n ) as the number of outcomes of throwing a 6-sided die n times such that c does not exceed ( n ). 1 For example, C(3) = 216, C(4) = 1290, C(11) = 361912500 and C(24) = 4727547363281250000. Define S( L ) as $\Sigma$ C( n ) for 1 $\leq$ n $\leq$ L . For example, S(50) mod 1000000007 = 832833871. Find S(50000000) mod 1000000007. 1  denotes the prime-counting function , i.e. ( n ) is the number of primes $\leq$ n . \href{https://projecteuler.net/problem=423}{Zadanie 423}
\end{zadanie}
\begin{zadanie}
	The above is an example of a cryptic kakuro (also known as cross sums, or even sums cross) puzzle, with its final solution on the right. (The common rules of kakuro puzzles can be found easily on numerous internet sites. Other related information can also be currently found at krazydad.com whose author has provided the puzzle data for this challenge.) The downloadable text file ( kakuro200.txt ) contains the description of 200 such puzzles, a mix of 5x5 and 6x6 types. The first puzzle in the file is the above example which is coded as follows: 6,X,X,(vCC),(vI),X,X,X,(hH),B,O,(vCA),(vJE),X,(hFE,vD),O,O,O,O,(hA),O,I,(hJC,vB),O,O,(hJC),H,O,O,O,X,X,X,(hJE),O,O,X The first character is a numerical digit indicating the size of the information grid. It would be either a 6 (for a 5x5 kakuro puzzle) or a 7 (for a 6x6 puzzle) followed by a comma (,). The extra top line and left column are needed to insert information. The content of each cell is then described and followed by a comma, going left to right and starting with the top line. X = Gray cell, not required to be filled by a digit. O (upper case letter)= White empty cell to be filled by a digit. A = Or any one of the upper case letters from A to J to be replaced by its equivalent digit in the solved puzzle. ( ) = Location of the encrypted sums. Horizontal sums are preceded by a lower case "h" and vertical sums are preceded by a lower case "v". Those are followed by one or two upper case letters depending if the sum is a single digit or double digit one. For double digit sums, the first letter would be for the "tens" and the second one for the "units". When the cell must contain information for both a horizontal and a vertical sum, the first one is always for the horizontal sum and the two are separated by a comma within the same set of brackets, ex.: (hFE,vD). Each set of brackets is also immediately followed by a comma. The description of the last cell is followed by a Carriage Return/Line Feed (CRLF) instead of a comma. The required answer to each puzzle is based on the value of each letter necessary to arrive at the solution and according to the alphabetical order. As indicated under the example puzzle, its answer would be 8426039571. At least 9 out of the 10 encrypting letters are always part of the problem description. When only 9 are given, the missing one must be assigned the remaining digit. You are given that the sum of the answers for the first 10 puzzles in the file is 64414157580. Find the sum of the answers for the 200 puzzles. \href{https://projecteuler.net/problem=424}{Zadanie 424}
\end{zadanie}
\begin{zadanie}
	Two positive numbers A and B are said to be connected (denoted by "A  B") if one of these conditions holds: (1) A and B have the same length and differ in exactly one digit; for example, 123  173. (2) Adding one digit to the left of A (or B) makes B (or A); for example, 23  223 and 123  23. We call a prime P a 2's relative if there exists a chain of connected primes between 2 and P and no prime in the chain exceeds P. For example, 127 is a 2's relative. One of the possible chains is shown below: 2  3  13  113  103  107  127 However, 11 and 103 are not 2's relatives. Let F(N) be the sum of the primes $\leq$ N which are not 2's relatives. We can verify that F(10 3 ) = 431 and F(10 4 ) = 78728. Find F(10 7 ). \href{https://projecteuler.net/problem=425}{Zadanie 425}
\end{zadanie}
\begin{zadanie}
	Consider an infinite row of boxes. Some of the boxes contain a ball. For example, an initial configuration of 2 consecutive occupied boxes followed by 2 empty boxes, 2 occupied boxes, 1 empty box, and 2 occupied boxes can be denoted by the sequence (2, 2, 2, 1, 2), in which the number of consecutive occupied and empty boxes appear alternately. A turn consists of moving each ball exactly once according to the following rule: Transfer the leftmost ball which has not been moved to the nearest empty box to its right. After one turn the sequence (2, 2, 2, 1, 2) becomes (2, 2, 1, 2, 3) as can be seen below; note that we begin the new sequence starting at the first occupied box. A system like this is called a Box-Ball System or BBS for short. It can be shown that after a sufficient number of turns, the system evolves to a state where the consecutive numbers of occupied boxes is invariant. In the example below, the consecutive numbers of occupied boxes evolves to [1, 2, 3]; we shall call this the final state. We define the sequence { t i }: Starting from the initial configuration ( t 0 , t 1 , , t 10 ), the final state becomes [1, 3, 10, 24, 51, 75]. Starting from the initial configuration ( t 0 , t 1 , , t 10 000 000 ), find the final state. Give as your answer the sum of the squares of the elements of the final state. For example, if the final state is [1, 2, 3] then 14 ( = 1 2 + 2 2 + 3 2 ) is your answer. \href{https://projecteuler.net/problem=426}{Zadanie 426}
\end{zadanie}
\begin{zadanie}
	A sequence of integers S = {s i } is called an n-sequence if it has n elements and each element s i satisfies 1 $\leq$ s i $\leq$ n . Thus there are n n distinct n -sequences in total.
	For example, the sequence S = {1, 5, 5, 10, 7, 7, 7, 2, 3, 7} is a 10-sequence. For any sequence S, let L(S) be the length of the longest contiguous subsequence of S with the same value.
	For example, for the given sequence S above, L(S) = 3, because of the three consecutive 7's. Let f ( n ) = $\Sigma$ L(S) for all n -sequences S. For example, f (3) = 45, f (7) = 1403689 and f (11) = 481496895121. Find f (7500000) mod 1000000009. \href{https://projecteuler.net/problem=427}{Zadanie 427}
\end{zadanie}
\begin{zadanie}
	Let a , b and c be positive numbers. Let W, X, Y, Z be four collinear points where |WX| = a , |XY| = b , |YZ| = c and |WZ| = a + b + c . Let C in be the circle having the diameter XY. Let C out be the circle having the diameter WZ. The triplet ( a , b , c ) is called a necklace triplet if you can place k $\geq$ 3 distinct circles C 1 , C 2 , ..., C k such that: For example, (5, 5, 5) and (4, 3, 21) are necklace triplets, while it can be shown that (2, 2, 5) is not. Let T( n ) be the number of necklace triplets ( a , b , c ) such that a , b and c are positive integers, and b $\leq$ n .
	For example, T(1)=9, T(20)=732 and T(3000)=438106. Find T(1000000000). \href{https://projecteuler.net/problem=428}{Zadanie 428}
\end{zadanie}
\begin{zadanie}
	A unitary divisor d of a number n is a divisor of n that has the property gcd( d, n/d ) = 1. The unitary divisors of 4! = 24 are 1, 3, 8 and 24. The sum of their squares is 1 2 + 3 2 + 8 2 + 24 2 = 650. Let S( n ) represent the sum of the squares of the unitary divisors of n . Thus S(4!)=650. Find S(100 000 000!) modulo 1 000 000 009. \href{https://projecteuler.net/problem=429}{Zadanie 429}
\end{zadanie}
\begin{zadanie}
	N disks are placed in a row, indexed 1 to N from left to right. Each disk has a black side and white side. Initially all disks show their white side. At each turn, two, not necessarily distinct, integers A and B between 1 and N (inclusive) are chosen uniformly at random. All disks with an index from A to B (inclusive) are flipped. The following example shows the case N = 8. At the first turn A = 5 and B = 2, and at the second turn A = 4 and B = 6. Let E( N , M ) be the expected number of disks that show their white side after M turns. We can verify that E(3, 1) = 10/9, E(3, 2) = 5/3, E(10, 4) $\approx$ 5.157 and E(100, 10) $\approx$ 51.893. Find E(10 10 , 4000). Give your answer rounded to 2 decimal places behind the decimal point. \href{https://projecteuler.net/problem=430}{Zadanie 430}
\end{zadanie}
\begin{zadanie}
	Fred the farmer arranges to have a new storage silo installed on his farm and having an obsession for all things square he is absolutely devastated when he discovers that it is circular. Quentin, the representative from the company that installed the silo, explains that they only manufacture cylindrical silos, but he points out that it is resting on a square base. Fred is not amused and insists that it is removed from his property. Quick thinking Quentin explains that when granular materials are delivered from above a conical slope is formed and the natural angle made with the horizontal is called the angle of repose. For example if the angle of repose, $\alpha = 30$ degrees, and grain is delivered at the centre of the silo then a perfect cone will form towards the top of the cylinder. In the case of this silo, which has a diameter of 6m, the amount of space wasted would be approximately 32.648388556 m 3 . However, if grain is delivered at a point on the top which has a horizontal distance of $x$ metres from the centre then a cone with a strangely curved and sloping base is formed. He shows Fred a picture. We shall let the amount of space wasted in cubic metres be given by $V(x)$. If $x = 1.114785284$, which happens to have three squared decimal places, then the amount of space wasted, $V(1.114785284) \approx 36$. Given the range of possible solutions to this problem there is exactly one other option: $V(2.511167869) \approx 49$. It would be like knowing that the square is king of the silo, sitting in splendid glory on top of your grain. Fred's eyes light up with delight at this elegant resolution, but on closer inspection of Quentin's drawings and calculations his happiness turns to despondency once more. Fred points out to Quentin that it's the radius of the silo that is 6 metres, not the diameter, and the angle of repose for his grain is 40 degrees. However, if Quentin can find a set of solutions for this particular silo then he will be more than happy to keep it. If Quick thinking Quentin is to satisfy frustratingly fussy Fred the farmer's appetite for all things square then determine the values of $x$ for all possible square space wastage options and calculate $\sum x$ correct to 9 decimal places. \href{https://projecteuler.net/problem=431}{Zadanie 431}
\end{zadanie}
\begin{zadanie}
	Let S( n,m ) = $\Sigma$$\phi$( n $\times$ i ) for 1 $\leq$ i $\leq$ m . ($\phi$ is Euler's totient function) You are given that S(510510,10 6 )= 45480596821125120. Find S(510510,10 11 ). Give the last 9 digits of your answer. \href{https://projecteuler.net/problem=432}{Zadanie 432}
\end{zadanie}
\begin{zadanie}
	Let E( x 0 , y 0 ) be the number of steps it takes to determine the greatest common divisor of x 0 and y 0 with Euclid's algorithm . More formally: x 1 = y 0 , y 1 = x 0 mod y 0 x n = y n -1 , y n = x n -1 mod y n -1 E( x 0 , y 0 ) is the smallest n such that y n = 0. We have E(1,1) = 1, E(10,6) = 3 and E(6,10) = 4. Define S(N) as the sum of E( x,y ) for 1 $\leq$ x,y $\leq$ N. We have S(1) = 1, S(10) = 221 and S(100) = 39826. Find S(510 6 ). \href{https://projecteuler.net/problem=433}{Zadanie 433}
\end{zadanie}
\begin{zadanie}
	Recall that a graph is a collection of vertices and edges connecting the vertices, and that two vertices connected by an edge are called adjacent. Graphs can be embedded in Euclidean space by associating each vertex with a point in the Euclidean space. A flexible graph is an embedding of a graph where it is possible to move one or more vertices continuously so that the distance between at least two nonadjacent vertices is altered while the distances between each pair of adjacent vertices is kept constant. A rigid graph is an embedding of a graph which is not flexible. Informally, a graph is rigid if by replacing the vertices with fully rotating hinges and the edges with rods that are unbending and inelastic, no parts of the graph can be moved independently from the rest of the graph. The grid graphs embedded in the Euclidean plane are not rigid, as the following animation demonstrates: However, one can make them rigid by adding diagonal edges to the cells. For example, for the 2x3 grid graph, there are 19 ways to make the graph rigid: Note that for the purposes of this problem, we do not consider changing the orientation of a diagonal edge or adding both diagonal edges to a cell as a different way of making a grid graph rigid. Let R ( m , n ) be the number of ways to make the m $\times$ n grid graph rigid. E.g. R (2,3) = 19 and R (5,5) = 23679901 Define S ( N ) as $\Sigma$ R ( i , j ) for 1 $\leq$ i , j $\leq$ N . E.g. S (5) = 25021721. Find S (100), give your answer modulo 1000000033. \href{https://projecteuler.net/problem=434}{Zadanie 434}
\end{zadanie}
\begin{zadanie}
	The Fibonacci numbers {f n , n $\geq$ 0} are defined recursively as f n = f n-1 + f n-2 with base cases f 0 = 0 and f 1 = 1. Define the polynomials {F n , n $\geq$ 0} as F n (x) = $\Sigma$f i x i for 0 $\leq$ i $\leq$ n. For example, F 7 (x) = x + x 2 + 2x 3 + 3x 4 + 5x 5 + 8x 6 + 13x 7 , and F 7 (11) = 268357683. Let n = 10 15 . Find the sum [$\Sigma$ 0$\leq$x$\leq$100 F n (x)] mod 1307674368000 (= 15!). \href{https://projecteuler.net/problem=435}{Zadanie 435}
\end{zadanie}
\begin{zadanie}
	Julie proposes the following wager to her sister Louise. She suggests they play a game of chance to determine who will wash the dishes. For this game, they shall use a generator of independent random numbers uniformly distributed between 0 and 1. The game starts with S = 0. The first player, Louise, adds to S different random numbers from the generator until S > 1 and records her last random number ' x '. The second player, Julie, continues adding to S different random numbers from the generator until S > 2 and records her last random number ' y '. The player with the highest number wins and the loser washes the dishes, i.e. if y > x the second player wins. For e x ample, if the first player draws 0.62 and 0.44, the first player turn ends since 0.62+0.44 > 1 and x = 0.44. If the second players draws 0.1, 0.27 and 0.91, the second player turn ends since 0.62+0.44+0.1+0.27+0.91 > 2 and y = 0.91.
	Since y > x , the second player wins. Louise thinks about it for a second, and objects: "That's not fair". What is the probability that the second player wins? Give your answer rounded to 10 places behind the decimal point in the form 0.abcdefghij \href{https://projecteuler.net/problem=436}{Zadanie 436}
\end{zadanie}
\begin{zadanie}
	When we calculate 8 n modulo 11 for n=0 to 9 we get: 1, 8, 9, 6, 4, 10, 3, 2, 5, 7. As we see all possible values from 1 to 10 occur. So 8 is a primitive root of 11. But there is more: If we take a closer look we see: 1+8=9 8+9=17$\equiv$6 mod 11 9+6=15$\equiv$4 mod 11 6+4=10 4+10=14$\equiv$3 mod 11 10+3=13$\equiv$2 mod 11 3+2=5 2+5=7 5+7=12$\equiv$1 mod 11. \href{https://projecteuler.net/problem=437}{Zadanie 437}
\end{zadanie}
\begin{zadanie}
	For an n -tuple of integers t = ( a 1 , ..., a n ), let ( x 1 , ..., x n ) be the solutions of the polynomial equation x n + a 1 x n -1 + a 2 x n -2 + ... + a n -1 x + a n = 0. Consider the following two conditions: In the case of n = 4, there are 12 n -tuples of integers which satisfy both conditions. We define S( t ) as the sum of the absolute values of the integers in t . For n = 4 we can verify that $\Sigma$ S( t ) = 2087 for all n -tuples t which satisfy both conditions. Find $\Sigma$ S( t ) for n = 7. \href{https://projecteuler.net/problem=438}{Zadanie 438}
\end{zadanie}
\begin{zadanie}
	Let d ( k ) be the sum of all divisors of k . We define the function S( N ) = $\sum\_{i=1}^N \sum\_{j=1}^Nd(i \cdot j)$. For example, S(3) = d (1) + d (2) + d (3) + d (2) + d (4) + d (6) + d (3) + d (6) + d (9) = 59. You are given that S(10 3 ) = 563576517282 and S(10 5 ) mod 10 9 = 215766508. Find S(10 11 ) mod 10 9 . \href{https://projecteuler.net/problem=439}{Zadanie 439}
\end{zadanie}
\begin{zadanie}
	We want to tile a board of length n and height 1 completely, with either 1 $\times$ 2 blocks or 1 $\times$ 1 blocks with a single decimal digit on top: For example, here are some of the ways to tile a board of length n = 8: Let T( n ) be the number of ways to tile a board of length n as described above. For example, T(1) = 10 and T(2) = 101. Let S( L ) be the triple sum $\Sigma$ a , b , c gcd(T( c a ), T( c b )) for 1 $\leq$ a , b , c $\leq$ L . For example: S(2) = 10444 S(3) = 1292115238446807016106539989 S(4) mod 987898789 = 670616280. Find S(2000) mod 987898789. \href{https://projecteuler.net/problem=440}{Zadanie 440}
\end{zadanie}
\begin{zadanie}
	For an integer M , we define R( M ) as the sum of 1/( p  q ) for all the integer pairs p and q which satisfy all of these conditions: We also define S( N ) as the sum of R( i ) for 2 $\leq$ i $\leq$ N . We can verify that S(2) = R(2) = 1/2, S(10) $\approx$ 6.9147 and S(100) $\approx$ 58.2962. Find S(10 7 ). Give your answer rounded to four decimal places. \href{https://projecteuler.net/problem=441}{Zadanie 441}
\end{zadanie}
\begin{zadanie}
	An integer is called eleven-free if its decimal expansion does not contain any substring representing a power of 11 except 1. For example, 2404 and 13431 are eleven-free, while 911 and 4121331 are not. Let E( n ) be the n th positive eleven-free integer. For example, E(3) = 3, E(200) = 213 and E(500000) = 531563. Find E(10 18 ). \href{https://projecteuler.net/problem=442}{Zadanie 442}
\end{zadanie}
\begin{zadanie}
	Let g( n ) be a sequence defined as follows: g(4) = 13, g( n ) = g( n -1) + gcd( n , g( n -1)) for n > 4. The first few values are: You are given that g(1000) = 2524 and g(1000000) = 2624152. Find g(10 15 ). \href{https://projecteuler.net/problem=443}{Zadanie 443}
\end{zadanie}
\begin{zadanie}
	A group of p people decide to sit down at a round table and play a lottery-ticket trading game. Each person starts off with a randomly-assigned, unscratched lottery ticket. Each ticket, when scratched, reveals a whole-pound prize ranging anywhere from 1 to  p , with no two tickets alike. The goal of the game is for each person to maximize his ticket winnings upon leaving the game. An arbitrary person is chosen to be the first player. Going around the table, each player has only one of two options: 1. The player can scratch his ticket and reveal its worth to everyone at the table. 2. The player can trade his unscratched ticket for a previous player's scratched ticket, and then leave the game with that ticket. The previous player then scratches his newly-acquired ticket and reveals its worth to everyone at the table. The game ends once all tickets have been scratched. All players still remaining at the table must leave with their currently-held tickets. Assume that each player uses the optimal strategy for maximizing the expected value of his ticket winnings. Let E( p ) represent the expected number of players left at the table when the game ends in a game consisting of p players (e.g. E(111)=5.2912 when rounded to 5 significant digits). Let $S\_1(N) = \sum \limits\_{p = 1}^{N} {E(p)}$. Let $S\_k(N) = \sum \limits\_{p = 1}^{N} {S\_{k-1}(p)}$ for $k \gt 1$. Find S 20 (10 14 ) and write the answer in scientific notation rounded to 10 significant digits. Use a lowercase e to separate mantissa and exponent (e.g. S 3 (100) = 5.983679014e5). \href{https://projecteuler.net/problem=444}{Zadanie 444}
\end{zadanie}
\begin{zadanie}
	For every integer n>1, the family of functions f n,a,b is defined by f n,a,b ( x )$\equiv$a x +b mod n for a,b, x integer and 0<a<n, 0$\leq$b<n, 0$\leq$ x <n. We will call f n,a,b a retraction if f n,a,b (f n,a,b ( x ))$\equiv$f n,a,b ( x ) mod n for every 0$\leq$ x <n. Let R(n) be the number of retractions for n. You are given that $\Sigma$ R(c) for c=C(100 000,k), and 1 $\leq$ k $\leq$99 999 $\equiv$628701600 (mod 1 000 000 007). (C(n,k) is the binomial coefficient). Find $\Sigma$ R(c) for c=C(10 000 000,k), and 1 $\leq$k$\leq$ 9 999 999. Give your answer modulo 1 000 000 007. \href{https://projecteuler.net/problem=445}{Zadanie 445}
\end{zadanie}
\begin{zadanie}
	For every integer n>1, the family of functions f n,a,b is defined by f n,a,b ( x )$\equiv$a x +b mod n for a,b, x integer and 0<a<n, 0$\leq$b<n, 0$\leq$ x <n. We will call f n,a,b a retraction if f n,a,b (f n,a,b ( x ))$\equiv$f n,a,b ( x ) mod n for every 0$\leq$ x <n. Let R(n) be the number of retractions for n. F(N)=$\Sigma$R(n 4 +4) for 1$\leq$n$\leq$N. F(1024)=77532377300600. Find F(10 7 ) (mod 1 000 000 007) \href{https://projecteuler.net/problem=446}{Zadanie 446}
\end{zadanie}
\begin{zadanie}
	For every integer n>1, the family of functions f n,a,b is defined by f n,a,b ( x )$\equiv$a x +b mod n for a,b, x integer and 0<a<n, 0$\leq$b<n, 0$\leq$ x <n. We will call f n,a,b a retraction if f n,a,b (f n,a,b ( x ))$\equiv$f n,a,b ( x ) mod n for every 0$\leq$ x <n. Let R(n) be the number of retractions for n. F(N)=$\Sigma$R(n) for 2$\leq$n$\leq$N. F(10 7 )$\equiv$638042271 (mod 1 000 000 007). Find F(10 14 ) (mod 1 000 000 007). \href{https://projecteuler.net/problem=447}{Zadanie 447}
\end{zadanie}
\begin{zadanie}
	The function lcm (a,b) denotes the least common multiple of a and b. Let A(n) be the average of the values of lcm(n,i) for 1$\leq$i$\leq$n. E.g: A(2)=(2+2)/2=2 and A(10)=(10+10+30+20+10+30+70+40+90+10)/10=32. Find S(99999999019) mod 999999017. \href{https://projecteuler.net/problem=448}{Zadanie 448}
\end{zadanie}
\begin{zadanie}
	Phil the confectioner is making a new batch of chocolate covered candy. Each candy centre is shaped like an ellipsoid of revolution defined by the equation: b 2 x 2 + b 2 y 2 + a 2 z 2 = a 2 b 2 . Phil wants to know how much chocolate is needed to cover one candy centre with a uniform coat of chocolate one millimeter thick. Find the amount of chocolate in mm 3 required if a=3 mm and b=1 mm. Give your answer as the number rounded to 8 decimal places behind the decimal point. \href{https://projecteuler.net/problem=449}{Zadanie 449}
\end{zadanie}
\begin{zadanie}
	A hypocycloid is the curve drawn by a point on a small circle rolling inside a larger circle. The parametric equations of a hypocycloid centered at the origin, and starting at the right most point is given by: $x(t) = (R - r) \cos(t) + r \cos(\frac {R - r} r t)$ $y(t) = (R - r) \sin(t) - r \sin(\frac {R - r} r t)$ Where R is the radius of the large circle and r the radius of the small circle. Let $C(R, r)$ be the set of distinct points with integer coordinates on the hypocycloid with radius R and r and for which there is a corresponding value of t such that $\sin(t)$ and $\cos(t)$ are rational numbers. Let $S(R, r) = \sum\_{(x,y) \in C(R, r)} |x| + |y|$ be the sum of the absolute values of the x and y coordinates of the points in $C(R, r)$. Let $T(N) = \sum\_{R = 3}^N \sum\_{r=1}^{\lfloor \frac {R - 1} 2 \rfloor} S(R, r)$ be the sum of $S(R, r)$ for R and r positive integers, $R\leq N$ and $2r < R$. Note: (-625, 0) is not an element of C(2500, 1000) because $\sin(t)$ is not a rational number for the corresponding values of t . S (3, 1) = (|3| + |0|) + (|-1| + |2|) + (|-1| + |0|) + (|-1| + |-2|) = 10 T (3) = 10; T (10) = 524 ; T (100) = 580442; T (10 3 ) = 583108600. Find T (10 6 ). \href{https://projecteuler.net/problem=450}{Zadanie 450}
\end{zadanie}
\begin{zadanie}
	Consider the number 15. There are eight positive numbers less than 15 which are coprime to 15: 1, 2, 4, 7, 8, 11, 13, 14. The modular inverses of these numbers modulo 15 are: 1, 8, 4, 13, 2, 11, 7, 14 because 11 mod 15=1 28=16 mod 15=1 44=16 mod 15=1 713=91 mod 15=1 1111=121 mod 15=1 1414=196 mod 15=1 Let I(n) be the largest positive number m smaller than n-1 such that the modular inverse of m modulo n equals m itself. So I(15)=11. Also I(100)=51 and I(7)=1. Find $\Sigma$ I(n) for 3$\leq$n$\leq$2$\times$10 7 \href{https://projecteuler.net/problem=451}{Zadanie 451}
\end{zadanie}
\begin{zadanie}
	Define F( m , n ) as the number of n -tuples of positive integers for which the product of the elements doesn't exceed m . F(10, 10) = 571. F(10 6 , 10 6 ) mod 1234567891 = 252903833. Find F(10 9 , 10 9 ) mod 1234567891. \href{https://projecteuler.net/problem=452}{Zadanie 452}
\end{zadanie}
\begin{zadanie}
	A simple quadrilateral is a polygon that has four distinct vertices, has no straight angles and does not self-intersect. Let Q(m, n) be the number of simple quadrilaterals whose vertices are lattice points with coordinates (x,y) satisfying 0 $\leq$ x $\leq$ m and 0 $\leq$ y $\leq$ n. For example, Q(2, 2) = 94 as can be seen below: It can also be verified that Q(3, 7) = 39590, Q(12, 3) = 309000 and Q(123, 45) = 70542215894646. Find Q(12345, 6789) mod 135707531. \href{https://projecteuler.net/problem=453}{Zadanie 453}
\end{zadanie}
\begin{zadanie}
	In the following equation x , y , and n are positive integers. For a limit L we define F( L ) as the number of solutions which satisfy x < y $\leq$ L . We can verify that F(15) = 4 and F(1000) = 1069. Find F(10 12 ). \href{https://projecteuler.net/problem=454}{Zadanie 454}
\end{zadanie}
\begin{zadanie}
	Let f(n) be the largest positive integer x less than 10 9 such that the last 9 digits of n x form the number x (including leading zeros), or zero if no such integer exists. For example: Find $\Sigma$ f(n), 2 $\leq$ n $\leq$ 10 6 . \href{https://projecteuler.net/problem=455}{Zadanie 455}
\end{zadanie}
\begin{zadanie}
	Define: x n = (1248 n mod 32323) - 16161 y n = (8421 n mod 30103) - 15051 P n = {( x 1 , y 1 ), ( x 2 , y 2 ), ..., ( x n , y n )} For example, P 8 = {(-14913, -6630), (-10161, 5625), (5226, 11896), (8340, -10778), (15852, -5203), (-15165, 11295), (-1427, -14495), (12407, 1060)}. Let C( n ) be the number of triangles whose vertices are in P n which contain the origin in the interior. Examples: C(8) = 20 C(600) = 8950634 C(40000) = 2666610948988 Find C(2000000). \href{https://projecteuler.net/problem=456}{Zadanie 456}
\end{zadanie}
\begin{zadanie}
	Let f ( n ) = n 2 - 3 n - 1. Let p be a prime. Let R( p ) be the smallest positive integer n such that f ( n ) mod p 2 = 0 if such an integer n exists, otherwise R( p ) = 0. Let SR( L ) be $\Sigma$ R( p ) for all primes not exceeding L . Find SR(10 7 ). \href{https://projecteuler.net/problem=457}{Zadanie 457}
\end{zadanie}
\begin{zadanie}
	Consider the alphabet A made out of the letters of the word "project": A={c,e,j,o,p,r,t}. Let T(n) be the number of strings of length n consisting of letters from A that do not have a substring that is one of the 5040 permutations of "project". Find T(10 12 ). Give the last 9 digits of your answer. \href{https://projecteuler.net/problem=458}{Zadanie 458}
\end{zadanie}
\begin{zadanie}
	The flipping game is a two player game played on a N by N square board. Each square contains a disk with one side white and one side black. The game starts with all disks showing their white side. A turn consists of flipping all disks in a rectangle with the following properties: Players alternate turns. A player wins by turning the grid all black. Let W(N) be the number of winning moves for the first player on a N by N board with all disks white, assuming perfect play. W(1) = 1, W(2) = 0, W(5) = 8 and W(10 2 ) = 31395. For N=5, the first player's eight winning first moves are: Find W(10 6 ). \href{https://projecteuler.net/problem=459}{Zadanie 459}
\end{zadanie}
\begin{zadanie}
	On the Euclidean plane, an ant travels from point A(0, 1) to point B( d , 1) for an integer d . In each step, the ant at point ( x 0 , y 0 ) chooses one of the lattice points ( x 1 , y 1 ) which satisfy x 1 $\geq$ 0 and y 1 $\geq$ 1 and goes straight to ( x 1 , y 1 ) at a constant velocity v . The value of v depends on y 0 and y 1 as follows: The left image is one of the possible paths for d = 4. First the ant goes from A(0, 1) to P 1 (1, 3) at velocity (3 - 1) / (ln(3) - ln(1)) $\approx$ 1.8205. Then the required time is sqrt(5) / 1.8205 $\approx$ 1.2283. From P 1 (1, 3) to P 2 (3, 3) the ant travels at velocity 3 so the required time is 2 / 3 $\approx$ 0.6667. From P 2 (3, 3) to B(4, 1) the ant travels at velocity (1 - 3) / (ln(1) - ln(3)) $\approx$ 1.8205 so the required time is sqrt(5) / 1.8205 $\approx$ 1.2283. Thus the total required time is 1.2283 + 0.6667 + 1.2283 = 3.1233. The right image is another path. The total required time is calculated as 0.98026 + 1 + 0.98026 = 2.96052. It can be shown that this is the quickest path for d = 4. Let F( d ) be the total required time if the ant chooses the quickest path. For example, F(4) $\approx$ 2.960516287. We can verify that F(10) $\approx$ 4.668187834 and F(100) $\approx$ 9.217221972. Find F(10000). Give your answer rounded to nine decimal places. \href{https://projecteuler.net/problem=460}{Zadanie 460}
\end{zadanie}
\begin{zadanie}
	Let f n ( k )= e k / n -1, for all non-negative integers k . Remarkably, f 200 (6)+ f 200 (75)+ f 200 (89)+ f 200 (226)= 3.1415926 44529$\approx$  . In fact, it is the best approximation of  of the form f n ( a )+ f n ( b )+ f n ( c )+ f n ( d ) for n =200. Let g ( n )= a 2 + b 2 + c 2 + d 2 for a , b , c , d that minimize the error: | f n ( a )+ f n ( b )+ f n ( c )+ f n ( d )-  | (where | x | denotes the absolute value of x ). You are given g (200)=6 2 +75 2 +89 2 +226 2 =64658. Find g (10000). \href{https://projecteuler.net/problem=461}{Zadanie 461}
\end{zadanie}
\begin{zadanie}
	A 3-smooth number is an integer which has no prime factor larger than 3. For an integer N , we define S( N ) as the set of 3-smooth numbers less than or equal to N . For example, S(20) = { 1, 2, 3, 4, 6, 8, 9, 12, 16, 18 }. We define F( N ) as the number of permutations of S( N ) in which each element comes after all of its proper divisors. This is one of the possible permutations for N = 20. - 1, 2, 4, 3, 9, 8, 16, 6, 18, 12. This is not a valid permutation because 12 comes before its divisor 6. - 1, 2, 4, 3, 9, 8, 12 , 16, 6 , 18. We can verify that F(6) = 5, F(8) = 9, F(20) = 450 and F(1000) $\approx$ 8.8521816557e21. Find F(10 18 ). Give as your answer its scientific notation rounded to ten digits after the decimal point. When giving your answer, use a lowercase e to separate mantissa and exponent. E.g. if the answer is 112,233,445,566,778,899 then the answer format would be 1.1223344557e17. \href{https://projecteuler.net/problem=462}{Zadanie 462}
\end{zadanie}
\begin{zadanie}
	The function $f$ is defined for all positive integers as follows: The function $S(n)$ is defined as $\sum\_{i=1}^{n}f(i)$. $S(8)=22$ and $S(100)=3604$. Find $S(3^{37})$. Give the last 9 digits of your answer. \href{https://projecteuler.net/problem=463}{Zadanie 463}
\end{zadanie}
\begin{zadanie}
	The Mbius function , denoted  ( n ), is defined as: Let P( a , b ) be the number of integers n in the interval [ a , b ] such that  ( n ) = 1. Let N( a , b ) be the number of integers n in the interval [ a , b ] such that  ( n ) = -1. For example, P(2,10) = 2 and N(2,10) = 4. Let C( n ) be the number of integer pairs ( a , b ) such that: For example, C(10)=13, C(500)=16676 and C(10000)=20155319. Find C(20000000). \href{https://projecteuler.net/problem=464}{Zadanie 464}
\end{zadanie}
\begin{zadanie}
	The kernel of a polygon is defined by the set of points from which the entire polygon's boundary is visible. We define a polar polygon as a polygon for which the origin is strictly contained inside its kernel. For this problem, a polygon can have collinear consecutive vertices. However, a polygon still cannot have self-intersection and cannot have zero area. For example, only the first of the following is a polar polygon (the kernels of the second, third, and fourth do not strictly contain the origin, and the fifth does not have a kernel at all): Notice that the first polygon has three consecutive collinear vertices. Let P( n ) be the number of polar polygons such that the vertices ( x , y ) have integer coordinates whose absolute values are not greater than n . Note that polygons should be counted as different if they have different set of edges, even if they enclose the same area. For example, the polygon with vertices [(0,0),(0,3),(1,1),(3,0)] is distinct from the polygon with vertices [(0,0),(0,3),(1,1),(3,0),(1,0)]. For example, P(1) = 131, P(2) = 1648531, P(3) = 1099461296175 and P(343) mod 1000000007 = 937293740. Find P(7 13 ) mod 1000000007. \href{https://projecteuler.net/problem=465}{Zadanie 465}
\end{zadanie}
\begin{zadanie}
	Let P( m , n ) be the number of distinct terms in an m $\times$ n multiplication table. For example, a 3$\times$4 multiplication table looks like this: There are 8 distinct terms {1,2,3,4,6,8,9,12}, therefore P(3,4) = 8. You are given that: P(64,64) = 1263, P(12,345) = 1998, and P(32,10 15 ) = 13826382602124302. Find P(64,10 16 ). \href{https://projecteuler.net/problem=466}{Zadanie 466}
\end{zadanie}
\begin{zadanie}
	An integer s is called a superinteger of another integer n if the digits of n form a subsequence of the digits of s . For example, 2718281828 is a superinteger of 18828, while 314159 is not a superinteger of 151. Let p ( n ) be the n th prime number, and let c ( n ) be the n th composite number. For example, p (1) = 2, p (10) = 29, c (1) = 4 and c (10) = 18. { p ( i ) : i $\geq$ 1} = {2, 3, 5, 7, 11, 13, 17, 19, 23, 29, ...} { c ( i ) : i $\geq$ 1} = {4, 6, 8, 9, 10, 12, 14, 15, 16, 18, ...} Let P D the sequence of the digital roots of { p ( i )} (C D is defined similarly for { c ( i )}): P D = {2, 3, 5, 7, 2, 4, 8, 1, 5, 2, ...} C D = {4, 6, 8, 9, 1, 3, 5, 6, 7, 9, ...} Let P n be the integer formed by concatenating the first n elements of P D (C n is defined similarly for C D ). P 10 = 2357248152 C 10 = 4689135679 Let f ( n ) be the smallest positive integer that is a common superinteger of P n and C n . For example, f (10) = 2357246891352679, and f (100) mod 1000000007 = 771661825. Find f (10000) mod 1000000007. \href{https://projecteuler.net/problem=467}{Zadanie 467}
\end{zadanie}
\begin{zadanie}
	An integer is called B -smooth if none of its prime factors is greater than $B$. Let $S\_B(n)$ be the largest $B$-smooth divisor of $n$. Examples: $S\_1(10)=1$ $S\_4(2100) = 12$ $S\_{17}(2496144) = 5712$ Define $\displaystyle F(n)=\sum\_{B=1}^n \sum\_{r=0}^n S\_B(\binom n r)$. Here, $\displaystyle \binom n r$ denotes the binomial coefficient. Examples: $F(11) = 3132$ $F(1111) \mod 1\,000\,000\,993 = 706036312$ $F(111\,111) \mod 1\,000\,000\,993 = 22156169$ Find $F(11\,111\,111) \mod 1\,000\,000\,993$. \href{https://projecteuler.net/problem=468}{Zadanie 468}
\end{zadanie}
\begin{zadanie}
	In a room N chairs are placed around a round table. Knights enter the room one by one and choose at random an available empty chair. To have enough elbow room the knights always leave at least one empty chair between each other. When there aren't any suitable chairs left, the fraction C of empty chairs is determined. We also define E( N ) as the expected value of C . We can verify that E(4) = 1/2 and E(6) = 5/9. Find E(10 18 ). Give your answer rounded to fourteen decimal places in the form 0.abcdefghijklmn. \href{https://projecteuler.net/problem=469}{Zadanie 469}
\end{zadanie}
\begin{zadanie}
	Consider a single game of Ramvok: Let t represent the maximum number of turns the game lasts. If t = 0, then the game ends immediately. Otherwise, on each turn i , the player rolls a die. After rolling, if i < t the player can either stop the game and receive a prize equal to the value of the current roll, or discard the roll and try again next turn. If i = t , then the roll cannot be discarded and the prize must be accepted. Before the game begins, t is chosen by the player, who must then pay an up-front cost ct for some constant c . For c = 0, t can be chosen to be infinite (with an up-front cost of 0). Let R( d , c ) be the expected profit (i.e. net gain) that the player receives from a single game of optimally-played Ramvok, given a fair d -sided die and cost constant c . For example, R(4, 0.2) = 2.65. Assume that the player has sufficient funds for paying any/all up-front costs. Now consider a game of Super Ramvok: In Super Ramvok, the game of Ramvok is played repeatedly, but with a slight modification. After each game, the die is altered. The alteration process is as follows: The die is rolled once, and if the resulting face has its pips visible, then that face is altered to be blank instead. If the face is already blank, then it is changed back to its original value. After the alteration is made, another game of Ramvok can begin (and during such a game, at each turn, the die is rolled until a face with a value on it appears). The player knows which faces are blank and which are not at all times. The game of Super Ramvok ends once all faces of the die are blank. Let S( d , c ) be the expected profit that the player receives from an optimally-played game of Super Ramvok, given a fair d -sided die to start (with all sides visible), and cost constant c . For example, S(6, 1) = 208.3. Let F( n ) = $\Sigma$ 4$\leq$ d $\leq$ n $\Sigma$ 0$\leq$ c $\leq$ n S( d , c ). Calculate F(20), rounded to the nearest integer. \href{https://projecteuler.net/problem=470}{Zadanie 470}
\end{zadanie}
\begin{zadanie}
	The triangle $\Delta$ABC is inscribed in an ellipse with equation $\frac {x^2} {a^2} + \frac {y^2} {b^2} = 1$, 0<2 b < a , a and b integers. Let r ( a , b ) be the radius of the incircle of $\Delta$ABC when the incircle has center (2 b , 0) and A has coordinates $\left( \frac a 2, \frac {\sqrt 3} 2 b\right)$. For example, r (3,1)=$\frac12$, r (6,2)=1, r (12,3)=2. Let $G(n) = \sum\_{a=3}^n \sum\_{b=1}^{\lfloor \frac {a - 1} 2 \rfloor} r(a, b)$ You are given G (10) = 20.59722222, G (100) = 19223.60980 (rounded to 10 significant digits). Find G (10 11 ). Give your answer in scientific notation rounded to 10 significant digits. Use a lowercase e to separate mantissa and exponent. For G (10) the answer would have been 2.059722222e1. \href{https://projecteuler.net/problem=471}{Zadanie 471}
\end{zadanie}
\begin{zadanie}
	There are N seats in a row. N people come one after another to fill the seats according to the following rules: Note that due to rule 1, some seats will surely be left unoccupied, and the maximum number of people that can be seated is less than N (for N > 1). Here are the possible seating arrangements for N = 15: We see that if the first person chooses correctly, the 15 seats can seat up to 7 people. We can also see that the first person has 9 choices to maximize the number of people that may be seated. Let f(N) be the number of choices the first person has to maximize the number of occupants for N seats in a row. Thus, f(1)=1, f(15)=9, f(20)=6, and f(500)=16. Also, $\Sigma$ f(N) = 83 for 1$\leq$ N $\leq$20 and $\Sigma$ f(N) = 13343 for 1$\leq$ N $\leq$500. Find $\Sigma$ f(N) for 1$\leq$ N $\leq$10 12 . Give the last 8 digits of your answer. \href{https://projecteuler.net/problem=472}{Zadanie 472}
\end{zadanie}
\begin{zadanie}
	Let $\varphi$ be the golden ratio: $\varphi=\frac{1+\sqrt{5}}{2}.$ Remarkably it is possible to write every positive integer as a sum of powers of $\varphi$ even if we require that every power of $\varphi$ is used at most once in this sum. Even then this representation is not unique. We can make it unique by requiring that no powers with consecutive exponents are used and that the representation is finite. E.g: $2=\varphi+\varphi^{-2}$ and $3=\varphi^{2}+\varphi^{-2}$ To represent this sum of powers of $\varphi$ we use a string of 0's and 1's with a point to indicate where the negative exponents start. We call this the representation in the phigital numberbase . So $1=1\_{\varphi}$, $2=10.01\_{\varphi}$, $3=100.01\_{\varphi}$ and $14=100100.001001\_{\varphi}$. The strings representing 1, 2 and 14 in the phigital number base are palindromic, while the string representing 3 is not. (the phigital point is not the middle character). The sum of the positive integers not exceeding 1000 whose phigital representation is palindromic is 4345. Find the sum of the positive integers not exceeding $10^{10}$ whose phigital representation is palindromic. \href{https://projecteuler.net/problem=473}{Zadanie 473}
\end{zadanie}
\begin{zadanie}
	For a positive integer n and digits d , we define F( n , d ) as the number of the divisors of n whose last digits equal d . For example, F(84, 4) = 3. Among the divisors of 84 (1, 2, 3, 4, 6, 7, 12, 14, 21, 28, 42, 84), three of them (4, 14, 84) have the last digit 4. We can also verify that F(12!, 12) = 11 and F(50!, 123) = 17888. Find F(10 6 !, 65432) modulo (10 16 + 61). \href{https://projecteuler.net/problem=474}{Zadanie 474}
\end{zadanie}
\begin{zadanie}
	12 n musicians participate at a music festival. On the first day, they form 3 n quartets and practice all day. It is a disaster. At the end of the day, all musicians decide they will never again agree to play with any member of their quartet. On the second day, they form 4 n trios, each musician avoiding his previous quartet partners. Let f (12 n ) be the number of ways to organize the trios amongst the 12 n musicians. You are given f (12) = 576 and f (24) mod 1 000 000 007 = 509089824. Find f (600) mod 1 000 000 007. \href{https://projecteuler.net/problem=475}{Zadanie 475}
\end{zadanie}
\begin{zadanie}
	Let R ( a , b , c ) be the maximum area covered by three non-overlapping circles inside a triangle with edge lengths a , b and c . Let S ( n ) be the average value of R ( a , b , c ) over all integer triplets ( a , b , c ) such that 1$\leq$ a $\leq$ b $\leq$ c < a + b $\leq$ n You are given S (2) = R (1, 1, 1) $\approx$ 0.31998, S (5) $\approx$ 1.25899. Find S (1803) rounded to 5 decimal places behind the decimal point. \href{https://projecteuler.net/problem=476}{Zadanie 476}
\end{zadanie}
\begin{zadanie}
	The number sequence game starts with a sequence S of N numbers written on a line. Two players alternate turns. At his turn, a player must select and remove either the first or the last number remaining in the sequence. The player score is the sum of all the numbers he has taken. Each player attempts to maximize his own sum. Player 1 score is 1 + 10 = 11. Let F ( N ) be the score of player 1 if both players follow the optimal strategy for the sequence S = { s 1 , s 2 ,..., s N } defined as: The sequence begins with S ={0,45,2070,4284945,753524550,478107844,894218625,...}. You are given F (2)=45, F (4)=4284990, F (100)=26365463243, F (10 4 )=2495838522951. Find F (10 8 ). \href{https://projecteuler.net/problem=477}{Zadanie 477}
\end{zadanie}
\begin{zadanie}
	Let us consider mixtures of three substances: A , B and C . A mixture can be described by a ratio of the amounts of A , B , and C in it, i.e., ( a : b : c ). For example, a mixture described by the ratio (2:3:5) contains 20\% A , 30\% B and 50\% C . For the purposes of this problem, we cannot separate the individual components from a mixture. However, we can combine different amounts of different mixtures to form mixtures with new ratios. For example, say we have three mixtures with ratios (3:0:2), (3:6:11) and (3:3:4). By mixing 10 units of the first, 20 units of the second and 30 units of the third, we get a new mixture with ratio (6:5:9), since: (10 3 / 5 +20 3 / 20 +30 3 / 10 :10 0 / 5 +20 6 / 20 +30 3 / 10 :10 2 / 5 +20 11 / 20 +30 4 / 10 )
	= (18:15:27) = (6:5:9) However, with the same three mixtures, it is impossible to form the ratio (3:2:1), since the amount of B is always less than the amount of C . Let n be a positive integer. Suppose that for every triple of integers ( a , b , c ) with 0 $\leq$ a , b , c $\leq$ n and gcd( a , b , c ) = 1, we have a mixture with ratio ( a : b : c ). Let M( n ) be the set of all such mixtures. For example, M(2) contains the 19 mixtures with the following ratios: {(0:0:1), (0:1:0), (0:1:1), (0:1:2), (0:2:1), (1:0:0), (1:0:1), (1:0:2), (1:1:0), (1:1:1), (1:1:2), (1:2:0), (1:2:1), (1:2:2), (2:0:1), (2:1:0), (2:1:1), (2:1:2), (2:2:1)}. Let E( n ) be the number of subsets of M( n ) which can produce the mixture with ratio (1 : 1 : 1), i.e., the mixture with equal parts A , B and C . We can verify that E(1) = 103, E(2) = 520447, E(10)mod11 8 = 82608406 and E(500)mod11 8 = 13801403. Find E(10000000)mod11 8 . \href{https://projecteuler.net/problem=478}{Zadanie 478}
\end{zadanie}
\begin{zadanie}
	Let a k , b k , and c k represent the three solutions (real or complex numbers) to the expression 1/ x = ( k / x ) 2 ( k + x 2 ) - kx . For instance, for k = 5, we see that { a 5 , b 5 , c 5 } is approximately {5.727244, -0.363622+2.057397i, -0.363622-2.057397i}. Let S( n ) = $\Sigma$( a k + b k ) p ( b k + c k ) p ( c k + a k ) p for all integers p , k such that 1 $\leq$ p , k $\leq$ n . Interestingly, S( n ) is always an integer. For example, S(4) = 51160. Find S(10 6 ) modulo 1 000 000 007. \href{https://projecteuler.net/problem=479}{Zadanie 479}
\end{zadanie}
\begin{zadanie}
	Consider all the words which can be formed by selecting letters, in any order, from the phrase: Suppose those with 15 letters or less are listed in alphabetical order and numbered sequentially starting at 1. The list would include: Define P ( w ) as the position of the word w . Define W ( p ) as the word in position p . We can see that P ( w ) and W ( p ) are inverses: P ( W ( p )) = p and W ( P ( w )) = w . Examples: Find W ( P (legionary) + P (calorimeters) - P (annihilate) + P (orchestrated) - P (fluttering)). Give your answer using lowercase characters (no punctuation or space). \href{https://projecteuler.net/problem=480}{Zadanie 480}
\end{zadanie}
\begin{zadanie}
	A group of chefs (numbered #1, #2, etc) participate in a turn-based strategic cooking competition. On each chef's turn, he/she cooks up a dish to the best of his/her ability and gives it to a separate panel of judges for taste-testing. Let S( k ) represent chef # k 's skill level (which is publicly known). More specifically, S( k ) is the probability that chef # k 's dish will be assessed favorably by the judges (on any/all turns). If the dish receives a favorable rating, then the chef must choose one other chef to be eliminated from the competition. The last chef remaining in the competition is the winner. The game always begins with chef #1, with the turn order iterating sequentially over the rest of the chefs still in play. Then the cycle repeats from the lowest-numbered chef. All chefs aim to optimize their chances of winning within the rules as stated, assuming that the other chefs behave in the same manner. In the event that a chef has more than one equally-optimal elimination choice, assume that the chosen chef is always the one with the next-closest turn. Define W n ( k ) as the probability that chef # k wins in a competition with n chefs. If we have S(1) = 0.25, S(2) = 0.5, and S(3) = 1, then W 3 (1) = 0.29375. Going forward, we assign S( k ) = F k /F n+1 over all 1 $\leq$ k $\leq$ n , where F k is a Fibonacci number: F k = F k-1 + F k-2 with base cases F 1 = F 2 = 1. Then, for example, when considering a competition with n = 7 chefs, we have W 7 (1) = 0.08965042, W 7 (2) = 0.20775702, W 7 (3) = 0.15291406, W 7 (4) = 0.14554098, W 7 (5) = 0.15905291, W 7 (6) = 0.10261412, and W 7 (7) = 0.14247050, rounded to 8 decimal places each. Let E( n ) represent the expected number of dishes cooked in a competition with n chefs. For instance, E(7) = 42.28176050. Find E(14) rounded to 8 decimal places. \href{https://projecteuler.net/problem=481}{Zadanie 481}
\end{zadanie}
\begin{zadanie}
	ABC is an integer sided triangle with incenter I and perimeter p. The segments IA, IB and IC have integral length as well. Let L = p + |IA| + |IB| + |IC|. Let S(P) = $\Sigma$ L for all such triangles where p $\leq$ P. For example, S(10 3 ) = 3619. Find S(10 7 ). \href{https://projecteuler.net/problem=482}{Zadanie 482}
\end{zadanie}
\begin{zadanie}
	We define a permutation as an operation that rearranges the order of the elements {1, 2, 3, ..., n}.
	There are n! such permutations, one of which leaves the elements in their initial order.
	For n = 3 we have 3! = 6 permutations: - P 1 = keep the initial order - P 2 = exchange the 1 st and 2 nd elements - P 3 = exchange the 1 st and 3 rd elements - P 4 = exchange the 2 nd and 3 rd elements - P 5 = rotate the elements to the right - P 6 = rotate the elements to the left If we select one of these permutations, and we re-apply the same permutation repeatedly, we eventually restore the initial order. For a permutation P i , let f(P i ) be the number of steps required to restore the initial order by applying the permutation P i repeatedly. For n = 3, we obtain: - f(P 1 ) = 1 : (1,2,3) $\rightarrow$ (1,2,3) - f(P 2 ) = 2 : (1,2,3) $\rightarrow$ (2,1,3) $\rightarrow$ (1,2,3) - f(P 3 ) = 2 : (1,2,3) $\rightarrow$ (3,2,1) $\rightarrow$ (1,2,3) - f(P 4 ) = 2 : (1,2,3) $\rightarrow$ (1,3,2) $\rightarrow$ (1,2,3) - f(P 5 ) = 3 : (1,2,3) $\rightarrow$ (3,1,2) $\rightarrow$ (2,3,1) $\rightarrow$ (1,2,3) - f(P 6 ) = 3 : (1,2,3) $\rightarrow$ (2,3,1) $\rightarrow$ (3,1,2) $\rightarrow$ (1,2,3) Let g(n) be the average value of f 2 (P i ) over all permutations P i of length n. g(3) = (1 2 + 2 2 + 2 2 + 2 2 + 3 2 + 3 2 )/3! = 31/6 $\approx$ 5.166666667e0 g(5) = 2081/120 $\approx$ 1.734166667e1 g(20) = 12422728886023769167301/2432902008176640000 $\approx$ 5.106136147e3 Find g(350) and write the answer in scientific notation rounded to 10 significant digits, using a lowercase e to separate mantissa and exponent, as in the examples above. \href{https://projecteuler.net/problem=483}{Zadanie 483}
\end{zadanie}
\begin{zadanie}
	The arithmetic derivative is defined by For example, 20 ' = 24. Find $\Sigma$ gcd ( k , k' ) for 1 < k $\leq$ 5$\times$10 15 . Note: gcd ( x , y ) denotes the greatest common divisor of x and y . \href{https://projecteuler.net/problem=484}{Zadanie 484}
\end{zadanie}
\begin{zadanie}
	Let d(n) be the number of divisors of n. Let M(n,k) be the maximum value of d(j) for n $\leq$ j $\leq$ n+k-1. Let S(u,k) be the sum of M(n,k) for 1 $\leq$ n $\leq$ u-k+1. You are given that S(1000,10)=17176. Find S(100 000 000,100 000). \href{https://projecteuler.net/problem=485}{Zadanie 485}
\end{zadanie}
\begin{zadanie}
	Let F 5 ( n ) be the number of strings s such that: For example, F 5 (4)=0, F 5 (5)=8, F 5 (6)=42 and F 5 (11)=3844. Let D( L ) be the number of integers n such that 5$\leq$ n $\leq$ L and F 5 ( n ) is divisible by 87654321. For example, D(10 7 )=0 and D(510 9 )=51. Find D(10 18 ). \href{https://projecteuler.net/problem=486}{Zadanie 486}
\end{zadanie}
\begin{zadanie}
	Let f k ( n ) be the sum of the k th powers of the first n positive integers. For example, f 2 (10) = 1 2 + 2 2 + 3 2 + 4 2 + 5 2 + 6 2 + 7 2 + 8 2 + 9 2 + 10 2 = 385. Let S k ( n ) be the sum of f k ( i ) for 1 $\leq$ i $\leq$ n . For example, S 4 (100) = 35375333830. What is $\Sigma$ (S 10000 (10 12 ) mod p) over all primes p between 2  10 9 and 2  10 9 + 2000? \href{https://projecteuler.net/problem=487}{Zadanie 487}
\end{zadanie}
\begin{zadanie}
	Alice and Bob have enjoyed playing Nim every day. However, they finally got bored of playing ordinary three-heap Nim. So, they added an extra rule: - Must not make two heaps of the same size. The triple ( a , b , c ) indicates the size of three heaps. Under this extra rule, (2,4,5) is one of the losing positions for the next player. To illustrate: - Alice moves to (2,4,3) - Bob moves to (0,4,3) - Alice moves to (0,2,3) - Bob moves to (0,2,1) Unlike ordinary three-heap Nim, (0,1,2) and its permutations are the end states of this game. For an integer N , we define F( N ) as the sum of a + b + c for all the losing positions for the next player, with 0 < a < b < c < N . For example, F(8) = 42, because there are 4 losing positions for the next player, (1,3,5), (1,4,6), (2,3,6) and (2,4,5). We can also verify that F(128) = 496062. Find the last 9 digits of F(10 18 ). \href{https://projecteuler.net/problem=488}{Zadanie 488}
\end{zadanie}
\begin{zadanie}
	Let G ( a , b ) be the smallest non-negative integer n for which gcd ( n 3 + b , ( n + a ) 3 + b ) is maximized. For example, G (1, 1) = 5 because gcd( n 3 + 1, ( n + 1) 3 + 1) reaches its maximum value of 7 for n = 5, and is smaller for 0 $\leq$ n < 5. Let H ( m , n ) = $\Sigma$ G ( a , b ) for 1 $\leq$ a $\leq$ m , 1 $\leq$ b $\leq$ n . You are given H (5, 5) = 128878 and H (10, 10) = 32936544. Find H (18, 1900). \href{https://projecteuler.net/problem=489}{Zadanie 489}
\end{zadanie}
\begin{zadanie}
	There are n stones in a pond, numbered 1 to n . Consecutive stones are spaced one unit apart. A frog sits on stone 1. He wishes to visit each stone exactly once, stopping on stone n . However, he can only jump from one stone to another if they are at most 3 units apart. In other words, from stone i , he can reach a stone j if 1 $\leq$ j $\leq$ n and j is in the set { i -3, i -2, i -1, i +1, i +2, i +3}. Let f( n ) be the number of ways he can do this. For example, f(6) = 14, as shown below: 1 $\rightarrow$ 2 $\rightarrow$ 3 $\rightarrow$ 4 $\rightarrow$ 5 $\rightarrow$ 6 1 $\rightarrow$ 2 $\rightarrow$ 3 $\rightarrow$ 5 $\rightarrow$ 4 $\rightarrow$ 6 1 $\rightarrow$ 2 $\rightarrow$ 4 $\rightarrow$ 3 $\rightarrow$ 5 $\rightarrow$ 6 1 $\rightarrow$ 2 $\rightarrow$ 4 $\rightarrow$ 5 $\rightarrow$ 3 $\rightarrow$ 6 1 $\rightarrow$ 2 $\rightarrow$ 5 $\rightarrow$ 3 $\rightarrow$ 4 $\rightarrow$ 6 1 $\rightarrow$ 2 $\rightarrow$ 5 $\rightarrow$ 4 $\rightarrow$ 3 $\rightarrow$ 6 1 $\rightarrow$ 3 $\rightarrow$ 2 $\rightarrow$ 4 $\rightarrow$ 5 $\rightarrow$ 6 1 $\rightarrow$ 3 $\rightarrow$ 2 $\rightarrow$ 5 $\rightarrow$ 4 $\rightarrow$ 6 1 $\rightarrow$ 3 $\rightarrow$ 4 $\rightarrow$ 2 $\rightarrow$ 5 $\rightarrow$ 6 1 $\rightarrow$ 3 $\rightarrow$ 5 $\rightarrow$ 2 $\rightarrow$ 4 $\rightarrow$ 6 1 $\rightarrow$ 4 $\rightarrow$ 2 $\rightarrow$ 3 $\rightarrow$ 5 $\rightarrow$ 6 1 $\rightarrow$ 4 $\rightarrow$ 2 $\rightarrow$ 5 $\rightarrow$ 3 $\rightarrow$ 6 1 $\rightarrow$ 4 $\rightarrow$ 3 $\rightarrow$ 2 $\rightarrow$ 5 $\rightarrow$ 6 1 $\rightarrow$ 4 $\rightarrow$ 5 $\rightarrow$ 2 $\rightarrow$ 3 $\rightarrow$ 6 Other examples are f(10) = 254 and f(40) = 1439682432976. Let S( L ) = $\Sigma$ f( n ) 3 for 1 $\leq$ n $\leq$ L . Examples: S(10) = 18230635 S(20) = 104207881192114219 S(1 000) mod 10 9 = 225031475 S(1 000 000) mod 10 9 = 363486179 Find S(10 14 ) mod 10 9 . \href{https://projecteuler.net/problem=490}{Zadanie 490}
\end{zadanie}
\begin{zadanie}
	We call a positive integer double pandigital if it uses all the digits 0 to 9 exactly twice (with no leading zero). For example, 40561817703823564929 is one such number. How many double pandigital numbers are divisible by 11? \href{https://projecteuler.net/problem=491}{Zadanie 491}
\end{zadanie}
\begin{zadanie}
	Define the sequence a 1 , a 2 , a 3 , ... as: Examples: a 3 = 2359 a 6 = 269221280981320216750489044576319 a 6 mod 1000000007 = 203064689 a 100 mod 1000000007 = 456482974 Define B( x , y , n ) as $\Sigma$ (a n mod p ) for every prime p such that x $\leq$ p $\leq$ x + y . Examples: B(10 9 , 10 3 , 10 3 ) = 23674718882 B(10 9 , 10 3 , 10 15 ) = 20731563854 Find B(10 9 , 10 7 , 10 15 ). \href{https://projecteuler.net/problem=492}{Zadanie 492}
\end{zadanie}
\begin{zadanie}
	70 coloured balls are placed in an urn, 10 for each of the seven rainbow colours. What is the expected number of distinct colours in 20 randomly picked balls? Give your answer with nine digits after the decimal point (a.bcdefghij). \href{https://projecteuler.net/problem=493}{Zadanie 493}
\end{zadanie}
\begin{zadanie}
	The Collatz sequence is defined as:
	$a\_{i+1} = \left\{ \large{\frac {a\_i} 2 \atop 3 a\_i+1} {\text{if }a\_i\text{ is even} \atop \text{if }a\_i\text{ is odd}} \right.$. The Collatz conjecture states that starting from any positive integer, the sequence eventually reaches the cycle 1,4,2,1.... We shall define the sequence prefix p(n) for the Collatz sequence starting with a 1 = n as the sub-sequence of all numbers not a power of 2 (2 0 =1 is considered a power of 2 for this problem). For example: p (13) = {13, 40, 20, 10, 5} p (8) = {} Any number invalidating the conjecture would have an infinite length sequence prefix. Let S m be the set of all sequence prefixes of length m . Two sequences {a 1 , a 2 , ..., a m } and {b 1 , b 2 , ..., b m } in S m are said to belong to the same prefix family if a i < a j if and only if b i < b j for all 1 $\leq$ i,j $\leq$ m . For example, in S 4 , {6, 3, 10, 5} is in the same family as {454, 227, 682, 341}, but not {113, 340, 170, 85}. Let f(m) be the number of distinct prefix families in S m . You are given f (5) = 5, f (10) = 55, f (20) = 6771. Find f(90). \href{https://projecteuler.net/problem=494}{Zadanie 494}
\end{zadanie}
\begin{zadanie}
	Let W ( n , k ) be the number of ways in which n can be written as the product of k distinct positive integers. For example, W (144,4) = 7. There are 7 ways in which 144 can be written as a product of 4 distinct positive integers: Note that permutations of the integers themselves are not considered distinct. Furthermore, W (100!,10) modulo 1000000007 = 287549200. Find W (10000!,30) modulo 1000000007. \href{https://projecteuler.net/problem=495}{Zadanie 495}
\end{zadanie}
\begin{zadanie}
	Given an integer sided triangle ABC: Let I be the incenter of ABC. Let D be the intersection between the line AI and the circumcircle of ABC (A $\neq$ D). We define F( L ) as the sum of BC for the triangles ABC that satisfy AC = DI and BC $\leq$ L . For example, F(15) = 45 because the triangles ABC with (BC,AC,AB) = (6,4,5), (12,8,10), (12,9,7), (15,9,16) satisfy the conditions. Find F(10 9 ). \href{https://projecteuler.net/problem=496}{Zadanie 496}
\end{zadanie}
\begin{zadanie}
	Bob is very familiar with the famous mathematical puzzle/game, "Tower of Hanoi," which consists of three upright rods and disks of different sizes that can slide onto any of the rods. The game begins with a stack of n disks placed on the leftmost rod in descending order by size. The objective of the game is to move all of the disks from the leftmost rod to the rightmost rod, given the following restrictions: Moving on to a variant of this game, consider a long room k units (square tiles) wide, labeled from 1 to k in ascending order. Three rods are placed at squares a , b , and c , and a stack of n disks is placed on the rod at square a . Bob begins the game standing at square b . His objective is to play the Tower of Hanoi game by moving all of the disks to the rod at square c . However, Bob can only pick up or set down a disk if he is on the same square as the rod / stack in question. Unfortunately, Bob is also drunk. On a given move, Bob will either stumble one square to the left or one square to the right with equal probability, unless Bob is at either end of the room, in which case he can only move in one direction. Despite Bob's inebriated state, he is still capable of following the rules of the game itself, as well as choosing when to pick up or put down a disk. The following animation depicts a side-view of a sample game for n = 3, k = 7, a = 2, b = 4, and c = 6: Let E( n , k , a , b , c ) be the expected number of squares that Bob travels during a single optimally-played game. A game is played optimally if the number of disk-pickups is minimized. Interestingly enough, the result is always an integer. For example, E(2,5,1,3,5) = 60 and E(3,20,4,9,17) = 2358. Find the last nine digits of $\Sigma$ 1$\leq$ n $\leq$10000 E( n ,10 n ,3 n ,6 n ,9 n ). \href{https://projecteuler.net/problem=497}{Zadanie 497}
\end{zadanie}
\begin{zadanie}
	For positive integers n and m , we define two polynomials F n ( x ) = x n and G m ( x ) = ( x -1) m . We also define a polynomial R n , m ( x ) as the remainder of the division of F n ( x ) by G m ( x ). For example, R 6,3 ( x ) = 15 x 2 - 24 x + 10. Let C( n , m , d ) be the absolute value of the coefficient of the d -th degree term of R n , m ( x ). We can verify that C(6, 3, 1) = 24 and C(100, 10, 4) = 227197811615775. Find C(10 13 , 10 12 , 10 4 ) mod 999999937. \href{https://projecteuler.net/problem=498}{Zadanie 498}
\end{zadanie}
\begin{zadanie}
	A gambler decides to participate in a special lottery. In this lottery the gambler plays a series of one or more games. Each game costs m pounds to play and starts with an initial pot of 1 pound. The gambler flips an unbiased coin. Every time a head appears, the pot is doubled and the gambler continues. When a tail appears, the game ends and the gambler collects the current value of the pot. The gambler is certain to win at least 1 pound, the starting value of the pot, at the cost of m pounds, the initial fee. The gambler cannot continue to play if his fortune falls below m pounds.
	Let p m ( s ) denote the probability that the gambler will never run out of money in this lottery given his initial fortune s and the cost per game m . For example p 2 (2)$\approx$0.2522, p 2 (5)$\approx$0.6873 and p 6 (10000)$\approx$0.9952 (note: p m ( s )=0 for s < m ). Find p 15 (10 9 ) and give your answer rounded to 7 decimal places behind the decimal point in the form 0.abcdefg. \href{https://projecteuler.net/problem=499}{Zadanie 499}
\end{zadanie}
\begin{zadanie}
	The number of divisors of 120 is 16. In fact 120 is the smallest number having 16 divisors. Find the smallest number with 2 500500 divisors. Give your answer modulo 500500507. \href{https://projecteuler.net/problem=500}{Zadanie 500}
\end{zadanie}
\begin{zadanie}
	The eight divisors of 24 are 1, 2, 3, 4, 6, 8, 12 and 24.
	The ten numbers not exceeding 100 having exactly eight divisors are 24, 30, 40, 42, 54, 56, 66, 70, 78 and 88.
	Let f ( n ) be the count of numbers not exceeding n with exactly eight divisors. You are given f (100)=10, f (1000)=180 and f (10 6 )=224427. Find f (10 12 ). \href{https://projecteuler.net/problem=501}{Zadanie 501}
\end{zadanie}
\begin{zadanie}
	We define a block to be a rectangle with a height of 1 and an integer-valued length. Let a castle be a configuration of stacked blocks. Given a game grid that is w units wide and h units tall, a castle is generated according to the following rules: The following is a sample castle for w =8 and h =5: Let F( w , h ) represent the number of valid castles, given grid parameters w and h . For example, F(4,2) = 10, F(13,10) = 3729050610636, F(10,13) = 37959702514, and F(100,100) mod 1 000 000 007 = 841913936. Find (F(10 12 ,100) + F(10000,10000) + F(100,10 12 )) mod 1 000 000 007. \href{https://projecteuler.net/problem=502}{Zadanie 502}
\end{zadanie}
\begin{zadanie}
	Alice is playing a game with n cards numbered 1 to n . A game consists of iterations of the following steps. (1) Alice picks one of the cards at random. (2) Alice cannot see the number on it. Instead, Bob, one of her friends, sees the number and tells Alice how many previously-seen numbers are bigger than the number which he is seeing. (3) Alice can end or continue the game. If she decides to end, the number becomes her score. If she decides to continue, the card is removed from the game and she returns to (1). If there is no card left, she is forced to end the game. Let F( n ) be the Alice's expected score if she takes the optimized strategy to minimize her score. For example, F(3) = 5/3. At the first iteration, she should continue the game. At the second iteration, she should end the game if Bob says that one previously-seen number is bigger than the number which he is seeing, otherwise she should continue the game. We can also verify that F(4) = 15/8 and F(10) $\approx$ 2.5579365079. Find F(10 6 ). Give your answer rounded to 10 decimal places behind the decimal point. \href{https://projecteuler.net/problem=503}{Zadanie 503}
\end{zadanie}
\begin{zadanie}
	Let ABCD be a quadrilateral whose vertices are lattice points lying on the coordinate axes as follows: A ( a , 0), B (0, b ), C ( -c , 0), D (0, -d ), where 1 $\leq$ a , b , c , d $\leq$ m and a , b , c , d , m are integers. It can be shown that for m = 4 there are exactly 256 valid ways to construct ABCD . Of these 256 quadrilaterals, 42 of them strictly contain a square number of lattice points. How many quadrilaterals ABCD strictly contain a square number of lattice points for m = 100? \href{https://projecteuler.net/problem=504}{Zadanie 504}
\end{zadanie}
\begin{zadanie}
	Let: $\begin{array}{ll} x(0)=0 \\ x(1)=1 \\ x(2k)=(3x(k)+2x(\lfloor \frac k 2 \rfloor)) \text{ mod } 2^{60} \text{ for } k \ge 1 \text {, where } \lfloor \text { } \rfloor \text { is the floor function} \\ x(2k+1)=(2x(k)+3x(\lfloor \frac k 2 \rfloor)) \text{ mod } 2^{60} \text{ for } k \ge 1 \\ y\_n(k)=\left\{{\begin{array}{lc} x(k)  \text{if } k \ge n \\ 2^{60} - 1 - max(y\_n(2k),y\_n(2k+1))  \text{if } k < n \end{array}} \right. \\ A(n)=y\_n(1) \end{array}$ You are given: $\begin{array}{ll} x(2)=3 \\ x(3)=2 \\ x(4)=11 \\ y\_4(4)=11 \\ y\_4(3)=2^{60}-9\\ y\_4(2)=2^{60}-12 \\ y\_4(1)=A(4)=8 \\ A(10)=2^{60}-34\\ A(10^3)=101881 \end{array}$ \href{https://projecteuler.net/problem=505}{Zadanie 505}
\end{zadanie}
\begin{zadanie}
	Consider the infinite repeating sequence of digits: 1234321234321234321... Amazingly, you can break this sequence of digits into a sequence of integers such that the sum of the digits in the n 'th value is n . The sequence goes as follows: 1, 2, 3, 4, 32, 123, 43, 2123, 432, 1234, 32123, ... Let v n be the n 'th value in this sequence. For example, v 2 =2, v 5 =32 and v 11 =32123. Let S ( n ) be v 1 + v 2 +...+ v n . For example, S (11)=36120, and S (1000)mod123454321=18232686. Find S (10 14 )mod123454321. \href{https://projecteuler.net/problem=506}{Zadanie 506}
\end{zadanie}
\begin{zadanie}
	Let $t\_n$ be the tribonacci numbers defined as: $t\_0 = t\_1 = 0$; $t\_2 = 1$; $t\_n = t\_{n-1} + t\_{n-2} + t\_{n-3}$ for $n \ge 3$ and let $r\_n = t\_n \text{ mod } 10^7$. For each pair of Vectors $V\_n=(v\_1,v\_2,v\_3)$ and $W\_n=(w\_1,w\_2,w\_3)$ with $v\_1=r\_{12n-11}-r\_{12n-10}, v\_2=r\_{12n-9}+r\_{12n-8}, v\_3=r\_{12n-7} \cdot r\_{12n-6}$ and $w\_1=r\_{12n-5}-r\_{12n-4}, w\_2=r\_{12n-3}+r\_{12n-2}, w\_3=r\_{12n-1} \cdot r\_{12n}$ we define $S(n)$ as the minimal value of the manhattan length of the vector $D=k \cdot V\_n+l \cdot W\_n$ measured as $|k \cdot v\_1+l \cdot w\_1|+|k \cdot v\_2+l \cdot w\_2|+|k \cdot v\_3+l \cdot w\_3|$ for any integers $k$ and $l$ with $(k,l)\neq (0,0)$. The first vector pair is (-1, 3, 28), (-11, 125, 40826). You are given that $S(1)=32$ and $\sum\_{n=1}^{10} S(n)=130762273722$. Find $\sum\_{n=1}^{20000000} S(n)$. \href{https://projecteuler.net/problem=507}{Zadanie 507}
\end{zadanie}
\begin{zadanie}
	Consider the Gaussian integer i-1. A base i-1 representation of a Gaussian integer a + b i is a finite sequence of digits d n -1 d n -2 ... d 1 d 0 such that: Here are base i-1 representations of a few Gaussian integers: 11+24i $\rightarrow$ 111010110001101 24-11i $\rightarrow$ 110010110011 8+0i $\rightarrow$ 111000000 -5+0i $\rightarrow$ 11001101 0+0i $\rightarrow$ 0 \href{https://projecteuler.net/problem=508}{Zadanie 508}
\end{zadanie}
\begin{zadanie}
	Anton and Bertrand love to play three pile Nim. However, after a lot of games of Nim they got bored and changed the rules somewhat. They may only take a number of stones from a pile that is a proper divisor of the number of stones present in the pile. E.g. if a pile at a certain moment contains 24 stones they may take only 1,2,3,4,6,8 or 12 stones from that pile. So if a pile contains one stone they can't take the last stone from it as 1 isn't a proper divisor of 1. The first player that can't make a valid move loses the game. Of course both Anton and Bertrand play optimally. The triple ( a , b , c ) indicates the number of stones in the three piles. Let S ( n ) be the number of winning positions for the next player for 1 $\leq$ a , b , c $\leq$ n . S (10) = 692 and S (100) = 735494. Find S (123456787654321) modulo 1234567890. \href{https://projecteuler.net/problem=509}{Zadanie 509}
\end{zadanie}
\begin{zadanie}
	Circles A and B are tangent to each other and to line L at three distinct points. Circle C is inside the space between A, B and L, and tangent to all three. Let r A , r B and r C be the radii of A, B and C respectively. Let S ( n )= $\Sigma$ r A + r B + r C , for 0< r A $\leq$ r B $\leq$ n where r A , r B and r C are integers.
	The only solution for 0< r A $\leq$ r B $\leq$5 is r A =4, r B =4 and r C =1, so S (5)=4+4+1=9.
	You are also given S (100)=3072. Find S (10 9 ). \href{https://projecteuler.net/problem=510}{Zadanie 510}
\end{zadanie}
\begin{zadanie}
	Let Seq ( n , k ) be the number of positive-integer sequences { a i } 1$\leq$i$\leq$ n of length n such that: Examples: Seq (3,4) = 4, and the 4 sequences are: {1, 1, 3} {1, 3, 1} {3, 1, 1} {3, 3, 3} Seq (4,11) = 8, and the 8 sequences are: {1, 1, 1, 4} {1, 1, 4, 1} {1, 4, 1, 1} {4, 1, 1, 1} {2, 2, 2, 1} {2, 2, 1, 2} {2, 1, 2, 2} {1, 2, 2, 2} The last nine digits of Seq (1111,24) are 840643584. Find the last nine digits of Seq (1234567898765,4321). \href{https://projecteuler.net/problem=511}{Zadanie 511}
\end{zadanie}
\begin{zadanie}
	Let $\varphi(n)$ be Euler's totient function. Let $f(n)=(\sum\_{i=1}^{n}\varphi(n^i)) \text{ mod } (n+1)$. Let $g(n)=\sum\_{i=1}^{n} f(i)$. $g(100)=2007$. Find $g(5 \times 10^8)$. \href{https://projecteuler.net/problem=512}{Zadanie 512}
\end{zadanie}
\begin{zadanie}
	ABC is an integral sided triangle with sides a$\leq$b$\leq$c. m c is the median connecting C and the midpoint of AB. F(n) is the number of such triangles with c$\leq$n for which m c has integral length as well. F(10)=3 and F(50)=165. Find F(100000). \href{https://projecteuler.net/problem=513}{Zadanie 513}
\end{zadanie}
\begin{zadanie}
	A geoboard (of order N ) is a square board with equally-spaced pins protruding from the surface, representing an integer point lattice for coordinates 0 $\leq$ x , y $\leq$ N . John begins with a pinless geoboard. Each position on the board is a hole that can be filled with a pin. John decides to generate a random integer between 1 and N +1 (inclusive) for each hole in the geoboard. If the random integer is equal to 1 for a given hole, then a pin is placed in that hole. After John is finished generating numbers for all ( N +1) 2 holes and placing any/all corresponding pins, he wraps a tight rubberband around the entire group of pins protruding from the board. Let S represent the shape that is formed. S can also be defined as the smallest convex shape that contains all the pins. The above image depicts a sample layout for N = 4. The green markers indicate positions where pins have been placed, and the blue lines collectively represent the rubberband. For this particular arrangement, S has an area of 6. If there are fewer than three pins on the board (or if all pins are collinear), S can be assumed to have zero area. Let E( N ) be the expected area of S given a geoboard of order N . For example, E(1) = 0.18750, E(2) = 0.94335, and E(10) = 55.03013 when rounded to five decimal places each. Calculate E(100) rounded to five decimal places. \href{https://projecteuler.net/problem=514}{Zadanie 514}
\end{zadanie}
\begin{zadanie}
	Let d ( p , n ,0) be the multiplicative inverse of n modulo prime p , defined as n $\times$ d ( p , n ,0)=1mod p . Let d ( p , n , k ) = $\sum\_{i=1}^n$ d ( p , i , k -1) for k $\geq$1. Let D ( a , b , k ) = $\sum$( d ( p , p -1, k ) mod p ) for all primes a $\leq$ p < a + b . You are given: Find D (10 9 ,10 5 ,10 5 ). \href{https://projecteuler.net/problem=515}{Zadanie 515}
\end{zadanie}
\begin{zadanie}
	5-smooth numbers are numbers whose largest prime factor doesn't exceed 5. 5-smooth numbers are also called Hamming numbers. Let S(L) be the sum of the numbers n not exceeding L such that Euler's totient function $\phi$(n) is a Hamming number. S(100)=3728. Find S(10 12 ). Give your answer modulo 2 32 . \href{https://projecteuler.net/problem=516}{Zadanie 516}
\end{zadanie}
\begin{zadanie}
	For every real number $a \gt 1$ is given the sequence $g\_a$ by: $g\_{a}(x)=1$ for $x \lt a$ $g\_{a}(x)=g\_{a}(x-1)+g\_a(x-a)$ for $x \ge a$ $G(n)=g\_{\sqrt {n}}(n)$ $G(90)=7564511$. Find $\sum G(p)$ for $p$ prime and $10000000 \lt p \lt 10010000$ Give your answer modulo 1000000007. \href{https://projecteuler.net/problem=517}{Zadanie 517}
\end{zadanie}
\begin{zadanie}
	Let S( n ) =  a + b + c over all triples ( a , b , c ) such that: For example, S(100) = 1035 with the following triples: (2, 5, 11), (2, 11, 47), (5, 11, 23), (5, 17, 53), (7, 11, 17), (7, 23, 71), (11, 23, 47), (17, 23, 31), (17, 41, 97), (31, 47, 71), (71, 83, 97) Find S(10 8 ). \href{https://projecteuler.net/problem=518}{Zadanie 518}
\end{zadanie}
\begin{zadanie}
	An arrangement of coins in one or more rows with the bottom row being a block without gaps and every coin in a higher row touching exactly two coins in the row below is called a fountain of coins. Let f ( n ) be the number of possible fountains with n coins. For 4 coins there are three possible arrangements: Therefore f (4)=3 while f (10)=78. Let T ( n ) be the number of all possible colourings with three colours for all f ( n ) different fountains with n coins, given the condition that no two touching coins have the same colour. Below you see the possible colourings for one of the three valid fountains for 4 coins: You are given that T (4)=48 and T (10)=17760. Find the last 9 digits of T (20000). \href{https://projecteuler.net/problem=519}{Zadanie 519}
\end{zadanie}
\begin{zadanie}
	We define a simber to be a positive integer in which any odd digit, if present, occurs an odd number of times, and any even digit, if present, occurs an even number of times. For example, 141221242 is a 9-digit simber because it has three 1's, four 2's and two 4's. Let Q( n ) be the count of all simbers with at most n digits. You are given Q(7) = 287975 and Q(100) mod 1 000 000 123 = 123864868. Find ( $\Sigma$ 1$\leq$ u $\leq$39 Q(2 u )) mod 1 000 000 123. \href{https://projecteuler.net/problem=520}{Zadanie 520}
\end{zadanie}
\begin{zadanie}
	Let smpf( n ) be the smallest prime factor of n . smpf(91)=7 because 91=7$\times$13 and smpf(45)=3 because 45=3$\times$3$\times$5. Let S( n ) be the sum of smpf( i ) for 2 $\leq$ i $\leq$ n . E.g. S(100)=1257. Find S(10 12 ) mod 10 9 . \href{https://projecteuler.net/problem=521}{Zadanie 521}
\end{zadanie}
\begin{zadanie}
	Despite the popularity of Hilbert's infinite hotel, Hilbert decided to try managing extremely large finite hotels, instead. To cut costs, Hilbert wished to power the new hotel with his own special generator. Each floor would send power to the floor above it, with the top floor sending power back down to the bottom floor. That way, Hilbert could have the generator placed on any given floor (as he likes having the option) and have electricity flow freely throughout the entire hotel. Unfortunately, the contractors misinterpreted the schematics when they built the hotel. They informed Hilbert that each floor sends power to another floor at random, instead. This may compromise Hilbert's freedom to have the generator placed anywhere, since blackouts could occur on certain floors. For example, consider a sample flow diagram for a three-story hotel: If the generator were placed on the first floor, then every floor would receive power. But if it were placed on the second or third floors instead, then there would be a blackout on the first floor. Note that while a given floor can receive power from many other floors at once, it can only send power to one other floor. To resolve the blackout concerns, Hilbert decided to have a minimal number of floors rewired. To rewire a floor is to change the floor it sends power to. In the sample diagram above, all possible blackouts can be avoided by rewiring the second floor to send power to the first floor instead of the third floor. Let F( n ) be the sum of the minimum number of floor rewirings needed over all possible power-flow arrangements in a hotel of n floors. For example, F(3) = 6, F(8) = 16276736, and F(100) mod 135707531 = 84326147. Find F(12344321) mod 135707531. \href{https://projecteuler.net/problem=522}{Zadanie 522}
\end{zadanie}
\begin{zadanie}
	Consider the following algorithm for sorting a list: For example, the list { 4 1 3 2 } is sorted as follows: Let F ( L ) be the number of times step 2a is executed to sort list L . For example, F ({ 4 1 3 2 }) = 5. Let E ( n ) be the expected value of F ( P ) over all permutations P of the integers {1, 2, ..., n }. You are given E (4) = 3.25 and E (10) = 115.725. Find E (30). Give your answer rounded to two digits after the decimal point. \href{https://projecteuler.net/problem=523}{Zadanie 523}
\end{zadanie}
\begin{zadanie}
	Consider the following algorithm for sorting a list: For example, the list { 4 1 3 2 } is sorted as follows: Let F ( L ) be the number of times step 2a is executed to sort list L . For example, F ({ 4 1 3 2 }) = 5. We can list all permutations P of the integers {1, 2, ..., n } in lexicographical order , and assign to each permutation an index I n ( P ) from 1 to n ! corresponding to its position in the list. Let Q ( n , k ) = min( I n ( P )) for F ( P ) = k , the index of the first permutation requiring exactly k steps to sort with First Sort. If there is no permutation for which F ( P ) = k , then Q ( n , k ) is undefined. For n = 4 we have: Let R ( k ) = min( Q ( n , k )) over all n for which Q ( n , k ) is defined. Find R (12 12 ). \href{https://projecteuler.net/problem=524}{Zadanie 524}
\end{zadanie}
\begin{zadanie}
	An ellipse E ( a , b ) is given at its initial position by equation: $\frac {x^2} {a^2} + \frac {(y - b)^2} {b^2} = 1$ The ellipse rolls without slipping along the x axis for one complete turn. Interestingly, the length of the curve generated by a focus is independent from the size of the minor axis: $F(a,b) = 2 \pi \text{ } max(a,b)$ This is not true for the curve generated by the ellipse center. Let C ( a , b ) be the length of the curve generated by the center of the ellipse as it rolls without slipping for one turn. You are given C (2, 4) ~ 21.38816906. Find C (1, 4) + C (3, 4). Give your answer rounded to 8 digits behind the decimal point in the form ab.cdefghij . \href{https://projecteuler.net/problem=525}{Zadanie 525}
\end{zadanie}
\begin{zadanie}
	Let f ( n ) be the largest prime factor of n . Let g ( n ) = f ( n ) + f ( n +1) + f ( n +2) + f ( n +3) + f ( n +4) + f ( n +5) + f ( n +6) + f ( n +7) + f ( n +8), the sum of the largest prime factor of each of nine consecutive numbers starting with n . Let h ( n ) be the maximum value of g ( k ) for 2 $\leq$ k $\leq$ n . You are given: Find h (10 16 ). \href{https://projecteuler.net/problem=526}{Zadanie 526}
\end{zadanie}
\begin{zadanie}
	A secret integer t is selected at random within the range 1 $\leq$ t $\leq$ n . The goal is to guess the value of t by making repeated guesses, via integer g . After a guess is made, there are three possible outcomes, in which it will be revealed that either g < t , g = t , or g > t . Then the process can repeat as necessary. Normally, the number of guesses required on average can be minimized with a binary search: Given a lower bound L and upper bound H (initialized to L = 1 and H = n ), let g = ( L + H )/2 where  is the integer floor function. If g = t , the process ends. Otherwise, if g < t , set L = g +1, but if g > t instead, set H = g -1. After setting the new bounds, the search process repeats, and ultimately ends once t is found. Even if t can be deduced without searching, assume that a search will be required anyway to confirm the value. Your friend Bob believes that the standard binary search is not that much better than his randomized variant: Instead of setting g = ( L + H )/2, simply let g be a random integer between L and H , inclusive. The rest of the algorithm is the same as the standard binary search. This new search routine will be referred to as a random binary search . Given that 1 $\leq$ t $\leq$ n for random t , let B( n ) be the expected number of guesses needed to find t using the standard binary search, and let R( n ) be the expected number of guesses needed to find t using the random binary search. For example, B(6) = 2.33333333 and R(6) = 2.71666667 when rounded to 8 decimal places. Find R(10 10 ) - B(10 10 ) rounded to 8 decimal places. \href{https://projecteuler.net/problem=527}{Zadanie 527}
\end{zadanie}
\begin{zadanie}
	Let S( n , k , b ) represent the number of valid solutions to x 1 + x 2 + ... + x k $\leq$ n , where 0 $\leq$ x m $\leq$ b m for all 1 $\leq$ m $\leq$ k . For example, S(14,3,2) = 135, S(200,5,3) = 12949440, and S(1000,10,5) mod 1 000 000 007 = 624839075. Find ( $\Sigma$ 10 $\leq$ k $\leq$ 15 S(10 k , k , k )) mod 1 000 000 007. \href{https://projecteuler.net/problem=528}{Zadanie 528}
\end{zadanie}
\begin{zadanie}
	A 10-substring of a number is a substring of its digits that sum to 10. For example, the 10-substrings of the number 3523014 are: A number is called 10-substring-friendly if every one of its digits belongs to a 10-substring. For example, 3523014 is 10-substring-friendly, but 28546 is not. Let T ( n ) be the number of 10-substring-friendly numbers from 1 to 10 n (inclusive). For example T (2) = 9 and T (5) = 3492. Find T (10 18 ) mod 1 000 000 007. \href{https://projecteuler.net/problem=529}{Zadanie 529}
\end{zadanie}
\begin{zadanie}
	Every divisor d of a number n has a complementary divisor n / d . Let f ( n ) be the sum of the greatest common divisor of d and n / d over all positive divisors d of n , that is
	$f(n)=\displaystyle\sum\limits\_{d|n}\, \text{gcd}(d,\frac n d)$. Let F be the summatory function of f , that is
	$F(k)=\displaystyle\sum\limits\_{n=1}^k \, f(n)$. You are given that F (10)=32 and F (1000)=12776. Find F (10 15 ). \href{https://projecteuler.net/problem=530}{Zadanie 530}
\end{zadanie}
\begin{zadanie}
	Let g( a,n,b,m ) be the smallest non-negative solution x to the system: x = a mod n x = b mod m if such a solution exists, otherwise 0. E.g. g(2,4,4,6)=10, but g(3,4,4,6)=0. Let $\phi$( n ) be Euler's totient function. Let f(n,m)=g($\phi$(n),n,$\phi$(m),m) Find $\Sigma$ f(n,m) for 1000000 $\leq$ n < m < 1005000 \href{https://projecteuler.net/problem=531}{Zadanie 531}
\end{zadanie}
\begin{zadanie}
	Bob is a manufacturer of nanobots and wants to impress his customers by giving them a ball coloured by his new nanobots as a present. His nanobots can be programmed to select and locate exactly one other bot precisely and, after activation, move towards this bot along the shortest possible path and draw a coloured line onto the surface while moving. Placed on a plane, the bots will start to move towards their selected bots in a straight line. In contrast, being placed on a ball, they will start to move along a geodesic as the shortest possible path. However, in both cases, whenever their target moves they will adjust their direction instantaneously to the new shortest possible path. All bots will move at the same speed after their simultaneous activation until each bot reaches its goal. Now Bob places n bots on the ball (with radius 1) equidistantly on a small circle with radius 0.999 and programs each of them to move toward the next nanobot sitting counterclockwise on that small circle. After activation, the bots move in a sort of spiral until they finally meet at one point on the ball. Using three bots, Bob finds that every bot will draw a line of length 2.84, resulting in a total length of 8.52 for all three bots, each time rounded to two decimal places. The coloured ball looks like this: In order to show off a little with his presents, Bob decides to use just enough bots to make sure that the line each bot draws is longer than 1000. What is the total length of all lines drawn with this number of bots, rounded to two decimal places? \href{https://projecteuler.net/problem=532}{Zadanie 532}
\end{zadanie}
\begin{zadanie}
	The Carmichael function ( n ) is defined as the smallest positive integer m such that a m = 1 modulo n for all integers a coprime with n . For example (8) = 2 and (240) = 4. Define L( n ) as the smallest positive integer m such that ( k )$\geq$ n for all k $\geq$ m . For example, L(6)=241 and L(100)=20174525281. Find L(20000000). Give the last 9 digits of your answer. \href{https://projecteuler.net/problem=533}{Zadanie 533}
\end{zadanie}
\begin{zadanie}
	The classical eight queens puzzle is the well known problem of placing eight chess queens on a 8$\times$8 chessboard so that no two queens threaten each other. Allowing configurations to reappear in rotated or mirrored form, a total of 92 distinct configurations can be found for eight queens. The general case asks for the number of distinct ways of placing n queens on a n $\times$ n board, e.g. you can find 2 distinct configurations for n =4. Let's define a weak queen on a n $\times$ n board to be a piece which can move any number of squares if moved horizontally, but a maximum of n -1- w squares if moved vertically or diagonally, 0$\leq$w<n being the "weakness factor". For example, a weak queen on a n $\times$ n board with a weakness factor of w =1 located in the bottom row will not be able to threaten any square in the top row as the weak queen would need to move n -1 squares vertically or diagonally to get there, but may only move n -2 squares in these directions. In contrast, the weak queen is not handicapped horizontally, thus threatening every square in its own row, independently from its current position in that row. Let Q( n , w ) be the number of ways n weak queens with weakness factor w can be placed on a n $\times$ n board so that no two queens threaten each other. It can be shown, for example, that Q(4,0)=2, Q(4,2)=16 and Q(4,3)=256. Let $S(n)=\displaystyle\sum\_{w=0}^{n-1} Q(n,w)$. You are given that S(4)=276 and S(5)=3347. Find S(14). \href{https://projecteuler.net/problem=534}{Zadanie 534}
\end{zadanie}
\begin{zadanie}
	Consider the infinite integer sequence S starting with: S = 1, 1, 2, 1, 3, 2, 4, 1, 5, 3, 6, 2, 7, 8, 4, 9, 1, 10, 11, 5, ... Circle the first occurrence of each integer. S = , 1, , 1, , 2, , 1, , 3, , 2, , , 4, , 1, , , 5, ... The sequence is characterized by the following properties: Let T( n ) be the sum of the first n elements of the sequence. You are given T(1)=1, T(20)=86, T(10 3 )=364089 and T(10 9 )=498676527978348241. Find T(10 18 ). Give the last 9 digits of your answer. \href{https://projecteuler.net/problem=535}{Zadanie 535}
\end{zadanie}
\begin{zadanie}
	Let S( n ) be the sum of all positive integers m not exceeding n having the following property: a m +4 $\equiv$ a (mod m ) for all integers a . The values of m $\leq$ 100 that satisfy this property are 1, 2, 3, 5 and 21, thus S(100) = 1+2+3+5+21 = 32. You are given S(10 6 ) = 22868117. Find S(10 12 ). \href{https://projecteuler.net/problem=536}{Zadanie 536}
\end{zadanie}
\begin{zadanie}
	Let  ( x ) be the prime counting function, i.e. the number of prime numbers less than or equal to x . For example,  (1)=0,  (2)=1,  (100)=25. Let T ( n , k ) be the number of k -tuples ( x 1 ,, x k ) which satisfy: 1. every x i is a positive integer; 2. $\displaystyle \sum\_{i=1}^k \pi(x\_i)=n$ For example T (3,3)=19. The 19 tuples are (1,1,5), (1,5,1), (5,1,1), (1,1,6), (1,6,1), (6,1,1), (1,2,3), (1,3,2), (2,1,3), (2,3,1), (3,1,2), (3,2,1), (1,2,4), (1,4,2), (2,1,4), (2,4,1), (4,1,2), (4,2,1), (2,2,2). You are given T (10,10) = 869 985 and T (10 3 ,10 3 ) $\equiv$ 578 270 566 (mod 1 004 535 809). Find T (20 000, 20 000) mod 1 004 535 809. \href{https://projecteuler.net/problem=537}{Zadanie 537}
\end{zadanie}
\begin{zadanie}
	Consider a positive integer sequence S =( s 1 , s 2 ,..., s n ). Let f( S ) be the perimeter of the maximum-area quadrilateral whose side lengths are 4 elements ( s i , s j , s k , s l ) of S (all i , j , k , l distinct). If there are many quadrilaterals with the same maximum area, then choose the one with the largest perimeter. For example, if S =(8,9,14,9,27), then we can take the elements (9,14,9,27) and form an isosceles trapezium with parallel side lengths 14 and 27 and both leg lengths 9. The area of this quadrilateral is 127.611470879... It can be shown that this is the largest area for any quadrilateral that can be formed using side lengths from S . Therefore, f( S )=9+14+9+27=59. Let u n =2 B(3 n ) +3 B(2 n ) +B( n +1), where B( k ) is the number of 1 bits of k in base 2. For example, B(6)=2, B(10)=2 and B(15)=4, and u 5 =2 4 +3 2 +2=27. Also, let U n be the sequence ( u 1 , u 2 ,..., u n ). For example, U 10 =(8,9,14,9,27,16,36,9,27,28). It can be shown that f( U 5 )=59, f( U 10 )=118, f( U 150 )=3223. It can also be shown that $\Sigma$ f( U n )=234761 for 4$\leq$n$\leq$150. Find $\Sigma$ f( U n ) for 4$\leq$n$\leq$3000000. \href{https://projecteuler.net/problem=538}{Zadanie 538}
\end{zadanie}
\begin{zadanie}
	Start from an ordered list of all integers from 1 to n. Going from left to right, remove the first number and every other number afterward until the end of the list. Repeat the procedure from right to left, removing the right most number and every other number from the numbers left. Continue removing every other numbers, alternating left to right and right to left, until a single number remains. Starting with n = 9, we have: 1 2 3 4 5 6 7 8 9 2 4 6 8 2 6 6 Let P ( n ) be the last number left starting with a list of length n . Let $\displaystyle S(n) = \sum\_{k=1}^n P(k)$. You are given P(1)=1, P(9) = 6, P(1000)=510, S(1000)=268271. Find S(10 18 ) mod 987654321. \href{https://projecteuler.net/problem=539}{Zadanie 539}
\end{zadanie}
\begin{zadanie}
	A Pythagorean triple consists of three positive integers $a, b$ and $c$ satisfying $a^2+b^2=c^2$. The triple is called primitive if $a, b$ and $c$ are relatively prime. Let P($n$) be the number of primitive Pythagorean triples with $a < b < c \le n$. For example P(20) = 3, since there are three triples: (3,4,5), (5,12,13) and (8,15,17). You are given that P(10 6 ) = 159139. Find P(3141592653589793). \href{https://projecteuler.net/problem=540}{Zadanie 540}
\end{zadanie}
\begin{zadanie}
	The n th harmonic number H n is defined as the sum of the multiplicative inverses of the first n positive integers, and can be written as a reduced fraction a n /b n . $H\_n = \displaystyle \sum\_{k=1}^n \frac 1 k = \frac {a\_n} {b\_n}$, with $\text {gcd}(a\_n, b\_n)=1$. Let M ( p ) be the largest value of n such that b n is not divisible by p . For example, M (3) = 68 because $H\_{68} = \frac {a\_{68}} {b\_{68}} = \frac {14094018321907827923954201611} {2933773379069966367528193600}$, b 68 =2933773379069966367528193600 is not divisible by 3, but all larger harmonic numbers have denominators divisible by 3. You are given M (7) = 719102. Find M (137). \href{https://projecteuler.net/problem=541}{Zadanie 541}
\end{zadanie}
\begin{zadanie}
	Let S ( k ) be the sum of three or more distinct positive integers having the following properties: S (4)=4+2+1=7 S (10)=9+6+4=19 S (12)=12+6+3=21 S (1000)=1000+900+810+729=3439 Let $T(n) = \sum\_{k=4}^n (-1)^k S(k)$. T (1000) = 2268 Find T (10 17 ). \href{https://projecteuler.net/problem=542}{Zadanie 542}
\end{zadanie}
\begin{zadanie}
	Define function P( n , k ) = 1 if n can be written as the sum of k prime numbers (with repetitions allowed), and P( n , k ) = 0 otherwise. For example, P(10,2) = 1 because 10 can be written as either 3 + 7 or 5 + 5, but P(11,2) = 0 because no two primes can sum to 11. Let S( n ) be the sum of all P( i , k ) over 1 $\leq$ i , k $\leq$ n . For example, S(10) = 20, S(100) = 2402, and S(1000) = 248838. Let F( k ) be the k th Fibonacci number (with F(0) = 0 and F(1) = 1). Find the sum of all S(F( k )) over 3 $\leq$ k $\leq$ 44 \href{https://projecteuler.net/problem=543}{Zadanie 543}
\end{zadanie}
\begin{zadanie}
	Let F( r , c , n ) be the number of ways to colour a rectangular grid with r rows and c columns using at most n colours such that no two adjacent cells share the same colour. Cells that are diagonal to each other are not considered adjacent. For example, F(2,2,3) = 18, F(2,2,20) = 130340, and F(3,4,6) = 102923670. Let S( r , c , n ) = $\sum\_{k=1}^{n}$ F( r , c , k ). For example, S(4,4,15) mod 10 9 +7 = 325951319. Find S(9,10,1112131415) mod 10 9 +7. \href{https://projecteuler.net/problem=544}{Zadanie 544}
\end{zadanie}
\begin{zadanie}
	The sum of the k th powers of the first n positive integers can be expressed as a polynomial of degree k +1 with rational coefficients, the Faulhaber's Formulas : $1^k + 2^k + ... + n^k = \sum\_{i=1}^n i^k = \sum\_{i=1}^{k+1} a\_{i} n^i = a\_{1} n + a\_{2} n^2 + ... + a\_{k} n^k + a\_{k+1} n^{k + 1}$, where a i 's are rational coefficients that can be written as reduced fractions p i / q i (if a i =0, we shall consider q i =1). For example, $1^4 + 2^4 + ... + n^4 = -\frac 1 {30} n + \frac 1 3 n^3 + \frac 1 2 n^4 + \frac 1 5 n^5.$ Define D( k ) as the value of q 1 for the sum of k th powers (i.e. the denominator of the reduced fraction a 1 ). Define F( m ) as the m th value of k $\geq$1 for which D( k )=20010. You are given D(4) = 30 (since a 1 =-1/30), D(308)=20010, F(1)=308, F(10)=96404. Find F(10 5 ). \href{https://projecteuler.net/problem=545}{Zadanie 545}
\end{zadanie}
\begin{zadanie}
	Define f k ( n ) = $\sum\_{i=0}^{n}$ f k ($\lfloor\frac{i}{k}\rfloor$) where f k (0) = 1 and $\lfloor x \rfloor$ denotes the floor function. For example, f 5 (10) = 18, f 7 (100) = 1003, and f 2 (10 3 ) = 264830889564. Find $(\sum\_{k=2}^{10}$ f k (10 14 )$)$ mod (10 9 +7). \href{https://projecteuler.net/problem=546}{Zadanie 546}
\end{zadanie}
\begin{zadanie}
	Assuming that two points are chosen randomly (with uniform distribution ) within a rectangle, it is possible to determine the expected value of the distance between these two points. For example, the expected distance between two random points in a unit square is about 0.521405, while the expected distance between two random points in a rectangle with side lengths 2 and 3 is about 1.317067. Now we define a hollow square lamina of size n to be an integer sized square with side length n $\geq$ 3 consisting of n 2 unit squares from which a rectangle consisting of x $\times$ y unit squares (1 $\leq$ x , y $\leq$ n - 2) within the original square has been removed. For n = 3 there exists only one hollow square lamina: For n = 4 you can find 9 distinct hollow square laminae, allowing shapes to reappear in rotated or mirrored form: Let S( n ) be the sum of the expected distance between two points chosen randomly within each of the possible hollow square laminae of size n . The two points have to lie within the area left after removing the inner rectangle, i.e. the gray-colored areas in the illustrations above. For example, S(3) = 1.6514 and S(4) = 19.6564, rounded to four digits after the decimal point. Find S(40) rounded to four digits after the decimal point. \href{https://projecteuler.net/problem=547}{Zadanie 547}
\end{zadanie}
\begin{zadanie}
	A gozinta chain for n is a sequence {1,a,b,..., n } where each element properly divides the next. There are eight gozinta chains for 12: {1,12} ,{1,2,12}, {1,2,4,12}, {1,2,6,12}, {1,3,12}, {1,3,6,12}, {1,4,12} and {1,6,12}. Let g( n ) be the number of gozinta chains for n , so g(12)=8. g(48)=48 and g(120)=132. Find the sum of the numbers n not exceeding 10 16 for which g( n )= n . \href{https://projecteuler.net/problem=548}{Zadanie 548}
\end{zadanie}
\begin{zadanie}
	The smallest number m such that 10 divides m! is m=5. The smallest number m such that 25 divides m! is m=10. Let s( n ) be the smallest number m such that n divides m!. So s(10)=5 and s(25)=10. Let S( n ) be $\Sigma$s( i ) for 2 $\leq$ i $\leq$ n . S(100)=2012. Find S(10 8 ). \href{https://projecteuler.net/problem=549}{Zadanie 549}
\end{zadanie}
\begin{zadanie}
	Two players are playing a game. There are k piles of stones.
	When it is his turn a player has to choose a pile and replace it by two piles of stones under the following two conditions: The first player unable to make a valid move loses. Let f( n , k ) be the number of winning positions for the first player, assuming perfect play, when the game is played with k piles each having between 2 and n stones (inclusively). f(10,5)=40085. Find f(10 7 ,10 12 ). Give your answer modulo 987654321. \href{https://projecteuler.net/problem=550}{Zadanie 550}
\end{zadanie}
\begin{zadanie}
	Let a 0 , a 1 , a 2 , ... be an integer sequence defined by: The sequence starts with 1, 1, 2, 4, 8, 16, 23, 28, 38, 49, ... You are given a 10 6 =31054319. Find a 10 15 . \href{https://projecteuler.net/problem=551}{Zadanie 551}
\end{zadanie}
\begin{zadanie}
	Let A n be the smallest positive integer satisfying A n mod p i = i for all 1 $\leq$ i $\leq$ n , where p i is the i -th prime. For example A 2 = 5, since this is the smallest positive solution of the system of equations The system of equations for A 3 adds another constraint. That is, A 3 is the smallest positive solution of and hence A 3 = 23. Similarly, one gets A 4 = 53 and A 5 = 1523. Let S( n ) be the sum of all primes up to n that divide at least one element in the sequence A . For example, S(50) = 69 = 5 + 23 + 41, since 5 divides A 2 , 23 divides A 3 and 41 divides A 10 = 5765999453. No other prime number up to 50 divides an element in A . Find S(300000). \href{https://projecteuler.net/problem=552}{Zadanie 552}
\end{zadanie}
\begin{zadanie}
	Let P ( n ) be the set of the first n positive integers {1, 2, ..., n }. Let Q( n ) be the set of all the non-empty subsets of P ( n ). Let R( n ) be the set of all the non-empty subsets of Q ( n ). An element X  R ( n ) is a non-empty subset of Q ( n ), so it is itself a set. From X we can construct a graph as follows: For example, X ={{1},{1,2,3},{3},{5,6},{6,7}} results in the following graph: This graph has two connected components . Let C ( n , k ) be the number of elements of R ( n ) that have exactly k connected components in their graph. You are given C (2,1)=6, C (3,1)=111, C (4,2)=486, C (100,10)mod1000000007=728209718. Find C (10 4 ,10)mod1000000007. \href{https://projecteuler.net/problem=553}{Zadanie 553}
\end{zadanie}
\begin{zadanie}
	On a chess board, a centaur moves like a king or a knight. The diagram below shows the valid moves of a centaur (represented by an inverted king) on an 8x8 board. It can be shown that at most n 2 non-attacking centaurs can be placed on a board of size 2 n $\times$2 n . Let C ( n ) be the number of ways to place n 2 centaurs on a 2 n $\times$2 n board so that no centaur attacks another directly. For example C (1)=4, C (2)=25, C (10)=1477721. Let F i be the i th Fibonacci number defined as F 1 = F 2 =1 and F i = F i -1 + F i -2 for i >2. Find $\displaystyle \left( \sum\_{i=2}^{90} C(F\_i) \right) \text{mod } (10^8+7)$. \href{https://projecteuler.net/problem=554}{Zadanie 554}
\end{zadanie}
\begin{zadanie}
	The McCarthy 91 function is defined as follows:
	$$
		M\_{91}(n) = \begin{cases} n - 10  \text{if } n > 100 \\ M\_{91}(M\_{91}(n+11))  \text{if } 0 \leq n \leq 100 \end{cases}
	$$ We can generalize this definition by abstracting away the constants into new variables: $$
		M\_{m,k,s}(n) = \begin{cases} n - s  \text{if } n > m \\ M\_{m,k,s}(M\_{m,k,s}(n+k))  \text{if } 0 \leq n \leq m \end{cases}
	$$ This way, we have $M\_{91} = M\_{100,11,10}$. Let $F\_{m,k,s}$ be the set of fixed points of $M\_{m,k,s}$. That is, $$F\_{m,k,s}= \left\{ n \in \mathbb{N} \, | \, M\_{m,k,s}(n) = n \right\}$$ For example, the only fixed point of $M\_{91}$ is $n = 91$. In other words, $F\_{100,11,10}= \{91\}$. Now, define $SF(m,k,s)$ as the sum of the elements in $F\_{m,k,s}$ and let $S(p,m) = \displaystyle \sum\_{1 \leq s < k \leq p}{SF(m,k,s)}$. For example, $S(10, 10) = 225$ and $S(1000, 1000)=208724467$. Find $S(10^6, 10^6)$. \href{https://projecteuler.net/problem=555}{Zadanie 555}
\end{zadanie}
\begin{zadanie}
	A Gaussian integer is a number z = a + b i where a , b are integers and i 2 =-1. Gaussian integers are a subset of the complex numbers, and the integers are the subset of Gaussian integers for which b =0. A Gaussian integer unit is one for which a 2 + b 2 =1, i.e. one of 1, i, -1, -i. Let's define a proper Gaussian integer as one for which a >0 and b $\geq$0. A Gaussian integer z 1 =a 1 + b 1 i is said to be divisible by z 2 =a 2 + b 2 i if z 3 = a 3 + b 3 i= z 1 / z 2 is a Gaussian integer. $\frac {z\_1} {z\_2} = \frac {a\_1 + b\_1 i} {a\_2 + b\_2 i} = \frac {(a\_1 + b\_1 i)(a\_2 - b\_2 i)} {(a\_2 + b\_2 i)(a\_2 - b\_2 i)} = \frac {a\_1 a\_2 + b\_1 b\_2} {a\_2^2 + b\_2^2} + \frac {a\_2 b\_1 - a\_1 b\_2} {a\_2^2 + b\_2^2}i = a\_3 + b\_3 i$ So, z 1 is divisible by z 2 if $\frac {a\_1 a\_2 + b\_1 b\_2} {a\_2^2 + b\_2^2}$ and $\frac {a\_2 b\_1 - a\_1 b\_2} {a\_2^2 + b\_2^2}$ are integers. For example, 2 is divisible by 1+i because 2/(1+i)=1-i is a Gaussian integer. A Gaussian prime is a Gaussian integer that is divisible only by a unit, itself or itself times a unit. For example, 1+2i is a Gaussian prime, because it is only divisible by 1, i, -1, -i, 1+2i, i(1+2i)=i-2, -(1+2i)=-1-2i and -i(1+2i)=2-i. 2 is not a Gaussian prime as it is divisible by 1+i. A Gaussian integer can be uniquely factored as the product of a unit and proper Gaussian primes. For example 2=-i(1+i) 2 and 1+3i=(1+i)(2+i). A Gaussian integer is said to be squarefree if its prime factorization does not contain repeated proper Gaussian primes. So 2 is not squarefree over the Gaussian integers, but 1+3i is. Units and Gaussian primes are squarefree by definition. Let f ( n ) be the count of proper squarefree Gaussian integers with a 2 + b 2 $\leq$n. For example f (10)=7 because 1, 1+i, 1+2i, 1+3i=(1+i)(2+i), 2+i, 3 and 3+i=-i(1+i)(1+2i) are squarefree, while 2=-i(1+i) 2 and 2+2i=-i(1+i) 3 are not. You are given f (10 2 )=54, f (10 4 )=5218 and f (10 8 )=52126906. Find f (10 14 ). \href{https://projecteuler.net/problem=556}{Zadanie 556}
\end{zadanie}
\begin{zadanie}
	A triangle is cut into four pieces by two straight lines, each starting at one vertex and ending on the opposite edge. This results in forming three smaller triangular pieces, and one quadrilateral. If the original triangle has an integral area, it is often possible to choose cuts such that all of the four pieces also have integral area. For example, the diagram below shows a triangle of area 55 that has been cut in this way. Representing the areas as a, b, c and d , in the example above, the individual areas are a = 22, b = 8, c = 11 and d = 14. It is also possible to cut a triangle of area 55 such that a = 20, b = 2, c = 24, d = 9. Define a triangle cutting quadruple ( a, b, c, d ) as a valid integral division of a triangle, where a is the area of the triangle between the two cut vertices, d is the area of the quadrilateral and b and c are the areas of the two other triangles, with the restriction that b $\leq$ c . The two solutions described above are (22,8,11,14) and (20,2,24,9). These are the only two possible quadruples that have a total area of 55. Define S( n ) as the sum of the area of the uncut triangles represented by all valid quadruples with a+b+c+d $\leq$ n . For example, S(20) = 259. Find S(10000). \href{https://projecteuler.net/problem=557}{Zadanie 557}
\end{zadanie}
\begin{zadanie}
	Let r be the real root of the equation x 3 = x 2 +1. Every positive integer can be written as the sum of distinct increasing powers of r . If we require the number of terms to be finite and the difference between any two exponents to be three or more, then the representation is unique. For example, 3= r -10 + r -5 + r -1 + r 2 and 10= r -10 + r -7 + r 6 . Interestingly, the relation holds for the complex roots of the equation. Let w ( n ) be the number of terms in this unique representation of n . Thus w (3)=4 and w (10)=3. More formally, for all positive integers n , we have: n = $\displaystyle \sum\_{k=-\infty}^{\infty}$ b k r k under the conditions that: b k is 0 or 1 for all k ; b k + b k +1 + b k +2 $\leq$1 for all k ; w ( n ) = $\displaystyle \sum\_{k=-\infty}^{\infty}$ b k is finite. Let S( m )= $\displaystyle \sum\_{j=1}^{m}$ w ( j 2 ). You are given S(10) = 61 and S(1000) = 19403. Find S(5 000 000). \href{https://projecteuler.net/problem=558}{Zadanie 558}
\end{zadanie}
\begin{zadanie}
	An ascent of a column j in a matrix occurs if the value of column j is smaller than the value of column j +1 in all rows. Let P( k , r , n ) be the number of r x n matrices with the following properties: For example, P(1, 2, 3) = 19, P(2, 4, 6) = 65508751 and P(7, 5, 30) mod 1000000123 = 161858102. Find Q(50000) mod 1000000123. \href{https://projecteuler.net/problem=559}{Zadanie 559}
\end{zadanie}
\begin{zadanie}
	Coprime Nim is just like ordinary normal play Nim, but the players may only remove a number of stones from a pile that is coprime with the current size of the pile. Two players remove stones in turn. The player who removes the last stone wins. Let L( n , k ) be the number of losing starting positions for the first player, assuming perfect play, when the game is played with k piles, each having between 1 and n -1 stones inclusively. For example, L(5,2)=6 since the losing initial positions are (1,1), (2,2), (2,4), (3,3), (4,2) and (4,4). You are also given L(10,5)=9964, L(10,10)=472400303, L(10 3 ,10 3 )mod1000000007=954021836. Find L(10 7 ,10 7 )mod1000000007 \href{https://projecteuler.net/problem=560}{Zadanie 560}
\end{zadanie}
\begin{zadanie}
	Let $S(n)$ be the number of pairs $(a,b)$ of distinct divisors of $n$ such that $a$ divides $b$. For $n=6$ we get the following pairs: $(1,2), (1,3), (1,6),( 2,6)$ and $(3,6)$. So $S(6)=5$. Let $p\_m\#$ be the product of the first $m$ prime numbers, so $p\_2\# = 2*3 = 6$. Let $E(m, n)$ be the highest integer $k$ such that $2^k$ divides $S((p\_m\#)^n)$. $E(2,1) = 0$ since $2^0$ is the highest power of 2 that divides S(6)=5. Let $Q(n)=\sum\_{i=1}^{n} E(904961, i)$ $Q(8)=2714886$. Evaluate $Q(10^{12})$. \href{https://projecteuler.net/problem=561}{Zadanie 561}
\end{zadanie}
\begin{zadanie}
	Construct triangle ABC such that: Let R be the circumradius of triangle ABC and T( r ) = R / r . For r = 5, one possible triangle has vertices (-4,-3), (4,2) and (1,0) with perimeter $\sqrt{13}+\sqrt{34}+\sqrt{89}$ and circumradius R = $\sqrt {\frac {19669} 2 }$, so T(5) =$\sqrt {\frac {19669} {50} }$. You are given T(10) ~ 97.26729 and T(100) ~ 9157.64707. Find T(10 7 ). Give your answer rounded to the nearest integer. \href{https://projecteuler.net/problem=562}{Zadanie 562}
\end{zadanie}
\begin{zadanie}
	A company specialises in producing large rectangular metal sheets, starting from unit square metal plates. The welding is performed by a range of robots of increasing size. Unfortunately, the programming options of these robots are rather limited. Each one can only process up to 25 identical rectangles of metal, which they can weld along either edge to produce a larger rectangle. The only programmable variables are the number of rectangles to be processed (up to and including 25), and whether to weld the long or short edge. For example, the first robot could be programmed to weld together 11 raw unit square plates to make a 11$\times$1 strip. The next could take 10 of these 11$\times$1 strips, and weld them either to make a longer 110$\times$1 strip, or a 11$\times$10 rectangle. Many, but not all, possible dimensions of metal sheets can be constructed in this way. One regular customer has a particularly unusual order. He always demands that the finished product should have an exact area, and that the long side must not be more than 10% larger than the short side. If these requirements can be met in more than one way, in terms of the exact dimensions of the two sides, then he demands that all variants are produced. For example, if he were to ask for metal sheet of area 889200, then there are three final dimensions that can be produced: 900$\times$988, 912$\times$975 and 936$\times$950. The target area of 889200 is the smallest area which can be manufactured in three different variants, within the limitations of the robot welders. Let M(n) be the minimal area that can be manufactured in exactly n variants with the longer edge not greater than 10% bigger than the shorter edge. Hence M(3) = 889200. Find $ \sum\_{n=2}^{100} M(n)$. \href{https://projecteuler.net/problem=563}{Zadanie 563}
\end{zadanie}
\begin{zadanie}
	A line segment of length $2n-3$ is randomly split into $n$ segments of integer length ($n \ge 3$). In the sequence given by this split, the segments are then used as consecutive sides of a convex $n$-polygon, formed in such a way that its area is maximal. All of the $\binom{2n-4} {n-1}$ possibilities for splitting up the initial line segment occur with the same probability. Let $E(n)$ be the expected value of the area that is obtained by this procedure. For example, for $n=3$ the only possible split of the line segment of length $3$ results in three line segments with length $1$, that form an equilateral triangle with an area of $\frac 1 4 \sqrt{3}$. Therefore $E(3)=0.433013$, rounded to $6$ decimal places. For $n=4$ you can find $4$ different possible splits, each of which is composed of three line segments with length $1$ and one line segment with length $2$. All of these splits lead to the same maximal quadrilateral with an area of $\frac 3 4 \sqrt{3}$, thus $E(4)=1.299038$, rounded to $6$ decimal places. Let $S(k)=\displaystyle \sum\_{n=3}^k E(n)$. For example, $S(3)=0.433013$, $S(4)=1.732051$, $S(5)=4.604767$ and $S(10)=66.955511$, rounded to $6$ decimal places each. Find $S(50)$, rounded to $6$ decimal places. \href{https://projecteuler.net/problem=564}{Zadanie 564}
\end{zadanie}
\begin{zadanie}
	Let $\sigma(n)$ be the sum of the divisors of $n$. E.g. the divisors of 4 are 1, 2 and 4, so $\sigma(4)=7$. The numbers $n$ not exceeding 20 such that 7 divides $\sigma(n)$ are: 4,12,13 and 20, the sum of these numbers being 49. Let $S(n , d)$ be the sum of the numbers $i$ not exceeding $n$ such that $d$ divides $\sigma(i)$. So $S(20 , 7)=49$. You are given: $S(10^6,2017)=150850429$ and $S(10^9 , 2017)=249652238344557$. Find $S(10^{11} , 2017)$ \href{https://projecteuler.net/problem=565}{Zadanie 565}
\end{zadanie}
\begin{zadanie}
	Adam plays the following game with his birthday cake. He cuts a piece forming a circular sector of 60 degrees and flips the piece upside down, with the icing on the bottom. He then rotates the cake by 60 degrees counterclockwise, cuts an adjacent 60 degree piece and flips it upside down. He keeps repeating this, until after a total of twelve steps, all the icing is back on top. Amazingly, this works for any piece size, even if the cutting angle is an irrational number: all the icing will be back on top after a finite number of steps. Now, Adam tries something different: he alternates cutting pieces of size $x=\frac{360}{9}$ degrees, $y=\frac{360}{10}$ degrees and $z=\frac{360 }{\sqrt{11}}$ degrees. The first piece he cuts has size x and he flips it. The second has size y and he flips it. The third has size z and he flips it. He repeats this with pieces of size x , y and z in that order until all the icing is back on top, and discovers he needs 60 flips altogether. Let F ( a , b , c ) be the minimum number of piece flips needed to get all the icing back on top for pieces of size $x=\frac{360}{a}$ degrees, $y=\frac{360}{b}$ degrees and $z=\frac{360}{\sqrt{c}}$ degrees. Let $G(n) = \sum\_{9 \le a < b < c \le n} F(a,b,c)$, for integers a , b and c . You are given that F (9,10,11)=60, F (10,14,16)=506, F (15,16,17)=785232. You are also given G (11)=60, G (14)=58020 and G (17)=1269260. Find G (53). \href{https://projecteuler.net/problem=566}{Zadanie 566}
\end{zadanie}
\begin{zadanie}
	Tom has built a random generator that is connected to a row of $n$ light bulbs. Whenever the random generator is activated each of the $n$ lights is turned on with the probability of $\frac 1 2$, independently of its former state or the state of the other light bulbs. While discussing with his friend Jerry how to use his generator, they invent two different games, they call the reciprocal games : Both games consist of $n$ turns. Each turn is started by choosing a number $k$ randomly between (and including) $1$ and $n$, with equal probability of $\frac 1 n$ for each number, while the possible win for that turn is the reciprocal of $k$, that is $\frac 1 k$. In game A, Tom activates his random generator once in each turn. If the number of lights turned on is the same as the previously chosen number $k$, Jerry wins and gets $\frac 1 k$, otherwise he will receive nothing for that turn. Jerry's expected win after playing the total game A consisting of $n$ turns is called $J\_A(n)$. For example $J\_A(6)=0.39505208$, rounded to 8 decimal places. For each turn in game B, after $k$ has been randomly selected, Tom keeps reactivating his random generator until exactly $k$ lights are turned on. After that Jerry takes over and reactivates the random generator until he, too, has generated a pattern with exactly $k$ lights turned on. If this pattern is identical to Tom's last pattern, Jerry wins and gets $\frac 1 k$, otherwise he will receive nothing. Jerry's expected win after the total game B consisting of $n$ turns is called $J\_B(n)$. For example $J\_B(6)=0.43333333$, rounded to 8 decimal places. Let $\displaystyle S(m)=\sum\_{n=1}^m (J\_A(n)+J\_B(n))$. For example $S(6)=7.58932292$, rounded to 8 decimal places. Find S(123456789), rounded to 8 decimal places. \href{https://projecteuler.net/problem=567}{Zadanie 567}
\end{zadanie}
\begin{zadanie}
	Tom has built a random generator that is connected to a row of $n$ light bulbs. Whenever the random generator is activated each of the $n$ lights is turned on with the probability of $\frac 1 2$, independently of its former state or the state of the other light bulbs. While discussing with his friend Jerry how to use his generator, they invent two different games, they call the reciprocal games : Both games consist of $n$ turns. Each turn is started by choosing a number $k$ randomly between (and including) $1$ and $n$, with equal probability of $\frac 1 n$ for each number, while the possible win for that turn is the reciprocal of $k$, that is $\frac 1 k$. In game A, Tom activates his random generator once in each turn. If the number of lights turned on is the same as the previously chosen number $k$, Jerry wins and gets $\frac 1 k$, otherwise he will receive nothing for that turn. Jerry's expected win after playing the total game A consisting of $n$ turns is called $J\_A(n)$. For example $J\_A(6)=0.39505208$, rounded to 8 decimal places. For each turn in game B, after $k$ has been randomly selected, Tom keeps reactivating his random generator until exactly $k$ lights are turned on. After that Jerry takes over and reactivates the random generator until he, too, has generated a pattern with exactly $k$ lights turned on. If this pattern is identical to Tom's last pattern, Jerry wins and gets $\frac 1 k$, otherwise he will receive nothing. Jerry's expected win after the total game B consisting of $n$ turns is called $J\_B(n)$. For example $J\_B(6)=0.43333333$, rounded to 8 decimal places. Let $D(n)=J\_B(n)-J\_A(n)$. For example, $D(6) = 0.03828125$. Find the 7 most significant digits of $D(123456789)$ after removing all leading zeros. (If, for example, we had asked for the 7 most significant digits of $D(6)$, the answer would have been 3828125.) \href{https://projecteuler.net/problem=568}{Zadanie 568}
\end{zadanie}
\begin{zadanie}
	A mountain range consists of a line of mountains with slopes of exactly 45, and heights governed by the prime numbers, p n . The up-slope of the k th mountain is of height p 2 k -1 , and the downslope is p 2 k . The first few foot-hills of this range are illustrated below. Tenzing sets out to climb each one in turn, starting from the lowest. At the top of each peak, he looks back and counts how many of the previous peaks he can see. In the example above, the eye-line from the third mountain is drawn in red, showing that he can only see the peak of the second mountain from this viewpoint. Similarly, from the 9 th mountain, he can see three peaks, those of the 5 th , 7 th and 8 th mountain. Let P( k ) be the number of peaks that are visible looking back from the k th mountain. Hence P(3)=1 and P(9)=3. Also $\displaystyle \sum\_{k=1}^{100} P(k) = 227$. Find $\displaystyle \sum\_{k=1}^{2500000} P(k)$. \href{https://projecteuler.net/problem=569}{Zadanie 569}
\end{zadanie}
\begin{zadanie}
	A snowflake of order n is formed by overlaying an equilateral triangle (rotated by 180 degrees) onto each equilateral triangle of the same size in a snowflake of order n-1. A snowflake of order 1 is a single equilateral triangle. Some areas of the snowflake are overlaid repeatedly. In the above picture, blue represents the areas that are one layer thick, red two layers thick, yellow three layers thick, and so on. For an order n snowflake, let A(n) be the number of triangles that are one layer thick, and let B(n) be the number of triangles that are three layers thick. Define G(n) = gcd(A(n), B(n)). E.g. A(3) = 30, B(3) = 6, G(3)=6 A(11) = 3027630, B(11) = 19862070, G(11) = 30 Further, G(500) = 186 and $\sum\_{n=3}^{500}G(n)=5124$ Find $\displaystyle \sum\_{n=3}^{10^7}G(n)$. \href{https://projecteuler.net/problem=570}{Zadanie 570}
\end{zadanie}
\begin{zadanie}
	A positive number is pandigital in base b if it contains all digits from 0 to b -1 at least once when written in base b . A n-super-pandigital number is a number that is simultaneously pandigital in all bases from 2 to n inclusively. For example 978=1111010010 2 =1100020 3 =33102 4 =12403 5 is the smallest 5-super-pandigital number. Similarly, 1093265784 is the smallest 10-super-pandigital number. The sum of the 10 smallest 10-super-pandigital numbers is 20319792309. What is the sum of the 10 smallest 12-super-pandigital numbers? \href{https://projecteuler.net/problem=571}{Zadanie 571}
\end{zadanie}
\begin{zadanie}
	A matrix $M$ is called idempotent if $M^2 = M$. Let $M$ be a three by three matrix : $M=\begin{pmatrix} a  b  c \\ d  e  f\\ g h i\\
		\end{pmatrix}$. Let C( n ) be the number of idempotent three by three matrices $M$ with integer elements such that $ -n \le a,b,c,d,e,f,g,h,i \le n$. C(1)=164 and C(2)=848. Find C(200). \href{https://projecteuler.net/problem=572}{Zadanie 572}
\end{zadanie}
\begin{zadanie}
	$n$ runners in very different training states want to compete in a race. Each one of them is given a different starting number $k$ $(1\leq k \leq n)$ according to his (constant) individual racing speed being $v\_k=\frac{k}{n}$. In order to give the slower runners a chance to win the race, $n$ different starting positions are chosen randomly (with uniform distribution) and independently from each other within the racing track of length $1$. After this, the starting position nearest to the goal is assigned to runner $1$, the next nearest starting position to runner $2$ and so on, until finally the starting position furthest away from the goal is assigned to runner $n$. The winner of the race is the runner who reaches the goal first. Interestingly, the expected running time for the winner is $\frac{1}{2}$, independently of the number of runners. Moreover, while it can be shown that all runners will have the same expected running time of $\frac{n}{n+1}$, the race is still unfair, since the winning chances may differ significantly for different starting numbers: Let $P\_{n,k}$ be the probability for runner $k$ to win a race with $n$ runners and $E\_n = \sum\_{k=1}^n k P\_{n,k}$ be the expected starting number of the winner in that race. It can be shown that, for example,
	$P\_{3,1}=\frac{4}{9}$, $P\_{3,2}=\frac{2}{9}$, $P\_{3,3}=\frac{1}{3}$ and $E\_3=\frac{17}{9}$ for a race with $3$ runners. You are given that $E\_4=2.21875$, $E\_5=2.5104$ and $E\_{10}=3.66021568$. Find $E\_{1000000}$ rounded to 4 digits after the decimal point. \href{https://projecteuler.net/problem=573}{Zadanie 573}
\end{zadanie}
\begin{zadanie}
	Let $q$ be a prime and $A \ge B >0$ be two integers with the following properties: It can be shown that, given these conditions, any sum $A+B<q^2$ and any difference $1<A-B<q^2$ has to be a prime number. Thus you can verify that a number $p$ is prime by showing that either $p=A+B<q^2$ or $p=A-B<q^2$ for some $A,B,q$ fulfilling the conditions listed above. Let $V(p)$ be the smallest possible value of $A$ in any sum $p=A+B$ and any difference $p=A-B$, that verifies $p$ being prime. Examples: $V(2)=1$, since $2=1+1< 2^2$. $V(37)=22$, since $37=22+15=2 \cdot 11+3 \cdot 5< 7^2$ is the associated sum with the smallest possible $A$. $V(151)=165$ since $151=165-14=3 \cdot 5 \cdot 11 - 2 \cdot 7<13^2$ is the associated difference with the smallest possible $A$. Let $S(n)$ be the sum of $V(p)$ for all primes $p<n$. For example, $S(10)=10$ and $S(200)=7177$. Find $S(3800)$. \href{https://projecteuler.net/problem=574}{Zadanie 574}
\end{zadanie}
\begin{zadanie}
	It was quite an ordinary day when a mysterious alien vessel appeared as if from nowhere. After waiting several hours and receiving no response it is decided to send a team to investigate, of which you are included. Upon entering the vessel you are met by a friendly holographic figure, Katharina, who explains the purpose of the vessel, Eulertopia. She claims that Eulertopia is almost older than time itself. Its mission was to take advantage of a combination of incredible computational power and vast periods of time to discover the answer to life, the universe, and everything. Hence the resident cleaning robot, Leonhard, along with his housekeeping responsibilities, was built with a powerful computational matrix to ponder the meaning of life as he wanders through a massive 1000 by 1000 square grid of rooms. She goes on to explain that the rooms are numbered sequentially from left to right, row by row. So, for example, if Leonhard was wandering around a 5 by 5 grid then the rooms would be numbered in the following way. Many millenia ago Leonhard reported to Katharina to have found the answer and he is willing to share it with any life form who proves to be worthy of such knowledge. Katharina further explains that the designers of Leonhard were given instructions to program him with equal probability of remaining in the same room or travelling to an adjacent room. However, it was not clear to them if this meant (i) an equal probability being split equally between remaining in the room and the number of available routes, or, (ii) an equal probability (50%) of remaining in the same room and then the other 50% was to be split equally between the number of available routes. The records indicate that they decided to flip a coin. Heads would mean that the probability of remaining was dynamically related to the number of exits whereas tails would mean that they program Leonhard with a fixed 50% probability of remaining in a particular room. Unfortunately there is no record of the outcome of the coin, so without further information we would need to assume that there is equal probability of either of the choices being implemented. Katharina suggests it should not be too challenging to determine that the probability of finding him in a square numbered room in a 5 by 5 grid after unfathomable periods of time would be approximately 0.177976190476 [12 d.p.]. In order to prove yourself worthy of visiting the great oracle you must calculate the probability of finding him in a square numbered room in the 1000 by 1000 lair in which he has been wandering. (Give your answer rounded to 12 decimal places) \href{https://projecteuler.net/problem=575}{Zadanie 575}
\end{zadanie}
\begin{zadanie}
	A bouncing point moves counterclockwise along a circle with circumference $1$ with jumps of constant length $l<1$, until it hits a gap of length $g<1$, that is placed in a distance $d$ counterclockwise from the starting point. The gap does not include the starting point, that is $g+d<1$. Let $S(l,g,d)$ be the sum of the length of all jumps, until the point falls into the gap. It can be shown that $S(l,g,d)$ is finite for any irrational jump size $l$, regardless of the values of $g$ and $d$. Examples: $S(\sqrt{\frac 1 2}, 0.06, 0.7)=0.7071 \dots$, $S(\sqrt{\frac 1 2}, 0.06, 0.3543)=1.4142 \dots$ and $S(\sqrt{\frac 1 2}, 0.06, 0.2427)=16.2634 \dots$. Let $M(n, g)$ be the maximum of $ \sum S(\sqrt{\frac 1 p}, g, d)$ for all primes $p \le n$ and any valid value of $d$. Examples: $M(3, 0.06) =29.5425 \dots$, since $S(\sqrt{\frac 1 2}, 0.06, 0.2427)+S(\sqrt{\frac 1 3}, 0.06, 0.2427)=29.5425 \dots$ is the maximal reachable sum for $g=0.06$. $M(10, 0.01)=266.9010 \dots$ Find $M(100, 0.00002)$, rounded to 4 decimal places. \href{https://projecteuler.net/problem=576}{Zadanie 576}
\end{zadanie}
\begin{zadanie}
	An equilateral triangle with integer side length $n \ge 3$ is divided into $n^2$ equilateral triangles with side length 1 as shown in the diagram below. The vertices of these triangles constitute a triangular lattice with $\frac{(n+1)(n+2)} 2$ lattice points. Let $H(n)$ be the number of all regular hexagons that can be found by connecting 6 of these points. For example, $H(3)=1$, $H(6)=12$ and $H(20)=966$. Find $\displaystyle \sum\_{n=3}^{12345} H(n)$. \href{https://projecteuler.net/problem=577}{Zadanie 577}
\end{zadanie}
\begin{zadanie}
	Any positive integer can be written as a product of prime powers: p 1 a 1 $\times$ p 2 a 2 $\times$...$\times$ p k a k , where p i are distinct prime integers, a i >0 and p i < p j if i < j . A decreasing prime power positive integer is one for which a i $\geq$ a j if i < j . For example, 1, 2, 15=3$\times$5, 360=2 3 $\times$3 2 $\times$5 and 1000=2 3 $\times$5 3 are decreasing prime power integers. Let C( n ) be the count of decreasing prime power positive integers not exceeding n . C(100)=94 since all positive integers not exceeding 100 have decreasing prime powers except 18, 50, 54, 75, 90 and 98. You are given C(10 6 )=922052. Find C(10 13 ). \href{https://projecteuler.net/problem=578}{Zadanie 578}
\end{zadanie}
\begin{zadanie}
	A lattice cube is a cube in which all vertices have integer coordinates. Let C( n ) be the number of different lattice cubes in which the coordinates of all vertices range between (and including) 0 and n . Two cubes are hereby considered different if any of their vertices have different coordinates. For example, C(1)=1, C(2)=9, C(4)=100, C(5)=229, C(10)=4469 and C(50)=8154671. Different cubes may contain different numbers of lattice points. For example, the cube with the vertices (0, 0, 0), (3, 0, 0), (0, 3, 0), (0, 0, 3), (0, 3, 3), (3, 0, 3), (3, 3, 0), (3, 3, 3) contains 64 lattice points (56 lattice points on the surface including the 8 vertices and 8 points within the cube). In contrast, the cube with the vertices (0, 2, 2), (1, 4, 4), (2, 0, 3), (2, 3, 0), (3, 2, 5), (3, 5, 2), (4, 1, 1), (5, 3, 3) contains only 40 lattice points (20 points on the surface and 20 points within the cube), although both cubes have the same side length 3. Let S( n ) be the sum of the lattice points contained in the different lattice cubes in which the coordinates of all vertices range between (and including) 0 and n . For example, S(1)=8, S(2)=91, S(4)=1878, S(5)=5832, S(10)=387003 and S(50)=29948928129. Find S(5000) mod 10 9 . \href{https://projecteuler.net/problem=579}{Zadanie 579}
\end{zadanie}
\begin{zadanie}
	A Hilbert number is any positive integer of the form $4k+1$ for integer $k\geq 0$. We shall define a squarefree Hilbert number as a Hilbert number which is not divisible by the square of any Hilbert number other than one. For example, $117$ is a squarefree Hilbert number, equaling $9\times13$. However $6237$ is a Hilbert number that is not squarefree in this sense, as it is divisible by $9^2$. The number $3969$ is also not squarefree, as it is divisible by both $9^2$ and $21^2$. There are $2327192$ squarefree Hilbert numbers below $10^7$. How many squarefree Hilbert numbers are there below $10^{16}$? \href{https://projecteuler.net/problem=580}{Zadanie 580}
\end{zadanie}
\begin{zadanie}
	A number is p-smooth if it has no prime factors larger than p. Let T be the sequence of triangular numbers, ie T( n )= n ( n +1)/2. Find the sum of all indices n such that T( n ) is 47-smooth. \href{https://projecteuler.net/problem=581}{Zadanie 581}
\end{zadanie}
\begin{zadanie}
	Let a, b and c be the sides of an integer sided triangle with one angle of 120 degrees, a$\leq$b$\leq$c and b-a$\leq$100. Let T( n ) be the number of such triangles with c$\leq$ n . T(1000)=235 and T(10 8 )=1245. Find T(10 100 ). \href{https://projecteuler.net/problem=582}{Zadanie 582}
\end{zadanie}
\begin{zadanie}
	A standard envelope shape is a convex figure consisting of an isosceles triangle (the flap) placed on top of a rectangle. An example of an envelope with integral sides is shown below. Note that to form a sensible envelope, the perpendicular height of the flap (BCD) must be smaller than the height of the rectangle (ABDE). In the envelope illustrated, not only are all the sides integral, but also all the diagonals (AC, AD, BD, BE and CE) are integral too. Let us call an envelope with these properties a Heron envelope . Let S( p ) be the sum of the perimeters of all the Heron envelopes with a perimeter less than or equal to p . You are given that S(10 4 ) = 884680. Find S(10 7 ). \href{https://projecteuler.net/problem=583}{Zadanie 583}
\end{zadanie}
\begin{zadanie}
	A long long time ago in a galaxy far far away, the Wimwians, inhabitants of planet WimWi, discovered an unmanned drone that had landed on their planet. On examining the drone, they uncovered a device that sought the answer for the so called " Birthday Problem ". The description of the problem was as follows: If people on your planet were to enter a very large room one by one, what will be the expected number of people in the room when you first find 3 people with Birthdays within 1 day from each other . The description further instructed them to enter the answer into the device and send the drone into space again. Startled by this turn of events, the Wimwians consulted their best mathematicians. Each year on Wimwi has 10 days and the mathematicians assumed equally likely birthdays and ignored leap years (leap years in Wimwi have 11 days), and found 5.78688636 to be the required answer. As such, the Wimwians entered this answer and sent the drone back into space. After traveling light years away, the drone then landed on planet Joka. The same events ensued except this time, the numbers in the device had changed due to some unknown technical issues. The description read: If people on your planet were to enter a very large room one by one, what will be the expected number of people in the room when you first find 3 people with Birthdays within 7 days from each other . With a 100-day year on the planet, the Jokars (inhabitants of Joka) found the answer to be 8.48967364 (rounded to 8 decimal places because the device allowed only 8 places after the decimal point) assuming equally likely birthdays. They too entered the answer into the device and launched the drone into space again. This time the drone landed on planet Earth. As before the numbers in the problem description had changed. It read: If people on your planet were to enter a very large room one by one, what will be the expected number of people in the room when you first find 4 people with Birthdays within 7 days from each other . What would be the answer (rounded to eight places after the decimal point) the people of Earth have to enter into the device for a year with 365 days? Ignore leap years. Also assume that all birthdays are equally likely and independent of each other. \href{https://projecteuler.net/problem=584}{Zadanie 584}
\end{zadanie}
\begin{zadanie}
	Consider the term $\small \sqrt{x+\sqrt{y}+\sqrt{z}}$ that is representing a nested square root . $x$, $y$ and $z$ are positive integers and $y$ and $z$ are not allowed to be perfect squares, so the number below the outer square root is irrational. Still it can be shown that for some combinations of $x$, $y$ and $z$ the given term can be simplified into a sum and/or difference of simple square roots of integers, actually denesting the square roots in the initial expression. Here are some examples of this denesting: $\small \sqrt{3+\sqrt{2}+\sqrt{2}}=\sqrt{2}+\sqrt{1}=\sqrt{2}+1$ $\small \sqrt{8+\sqrt{15}+\sqrt{15}}=\sqrt{5}+\sqrt{3}$ $\small \sqrt{20+\sqrt{96}+\sqrt{12}}=\sqrt{9}+\sqrt{6}+\sqrt{3}-\sqrt{2}=3+\sqrt{6}+\sqrt{3}-\sqrt{2}$ $\small \sqrt{28+\sqrt{160}+\sqrt{108}}=\sqrt{15}+\sqrt{6}+\sqrt{5}-\sqrt{2}$ As you can see the integers used in the denested expression may also be perfect squares resulting in further simplification. Let F($n$) be the number of different terms $\small \sqrt{x+\sqrt{y}+\sqrt{z}}$, that can be denested into the sum and/or difference of a finite number of square roots, given the additional condition that $0<x \le n$. That is, $\small \displaystyle \sqrt{x+\sqrt{y}+\sqrt{z}}=\sum\_{i=1}^k s\_i\sqrt{a\_i}$ with $k$, $x$, $y$, $z$ and all $a\_i$ being positive integers, all $s\_i =\pm 1$ and $x\le n$. Furthermore $y$ and $z$ are not allowed to be perfect squares. Nested roots with the same value are not considered different, for example $\small \sqrt{7+\sqrt{3}+\sqrt{27}}$, $\small \sqrt{7+\sqrt{12}+\sqrt{12}}$ and $\small \sqrt{7+\sqrt{27}+\sqrt{3}}$, that can all three be denested into $\small 2+\sqrt{3}$, would only be counted once. You are given that F(10)=17, F(15)=46, F(20)=86, F(30)=213 and F(100)=2918 and F(5000)=11134074. Find F(5000000). \href{https://projecteuler.net/problem=585}{Zadanie 585}
\end{zadanie}
\begin{zadanie}
	The number 209 can be expressed as $a^2 + 3ab + b^2$ in two distinct ways: $ \qquad 209 = 8^2 + 3\cdot 8\cdot 5 + 5^2$ $ \qquad 209 = 13^2 + 3\cdot13\cdot 1 + 1^2$ Let $f(n,r)$ be the number of integers $k$ not exceeding $n$ that can be expressed as $k=a^2 + 3ab + b^2$, with $a\gt b>0$ integers, in exactly $r$ different ways. You are given that $f(10^5, 4) = 237$ and $f(10^8, 6) = 59517$. Find $f(10^{15}, 40)$. \href{https://projecteuler.net/problem=586}{Zadanie 586}
\end{zadanie}
\begin{zadanie}
	A square is drawn around a circle as shown in the diagram below on the left. We shall call the blue shaded region the L-section. A line is drawn from the bottom left of the square to the top right as shown in the diagram on the right. We shall call the orange shaded region a concave triangle. It should be clear that the concave triangle occupies exactly half of the L-section. Two circles are placed next to each other horizontally, a rectangle is drawn around both circles, and a line is drawn from the bottom left to the top right as shown in the diagram below. This time the concave triangle occupies approximately 36.46% of the L-section. If n circles are placed next to each other horizontally, a rectangle is drawn around the n circles, and a line is drawn from the bottom left to the top right, then it can be shown that the least value of n for which the concave triangle occupies less than 10% of the L-section is n =15. What is the least value of n for which the concave triangle occupies less than 0.1% of the L-section? \href{https://projecteuler.net/problem=587}{Zadanie 587}
\end{zadanie}
\begin{zadanie}
	The coefficients in the expansion of $(x+1)^k$ are called binomial coefficients . Analoguously the coefficients in the expansion of $(x^4+x^3+x^2+x+1)^k$ are called quintinomial coefficients . (quintus= Latin for fifth). Consider the expansion of $(x^4+x^3+x^2+x+1)^3$: $x^{12}+3x^{11}+6x^{10}+10x^9+15x^8+18x^7+19x^6+18x^5+15x^4+10x^3+6x^2+3x+1$ As we can see 7 out of the 13 quintinomial coefficients for $k=3$ are odd. Let $Q(k)$ be the number of odd coefficients in the expansion of $(x^4+x^3+x^2+x+1)^k$. So $Q(3)=7$. You are given $Q(10)=17$ and $Q(100)=35$. Find $\sum\_{k=1}^{18}Q(10^k) $. \href{https://projecteuler.net/problem=588}{Zadanie 588}
\end{zadanie}
\begin{zadanie}
	Christopher Robin and Pooh Bear love the game of Poohsticks so much that they invented a new version which allows them to play for longer before one of them wins and they have to go home for tea. The game starts as normal with both dropping a stick simultaneously on the upstream side of a bridge. But rather than the game ending when one of the sticks emerges on the downstream side, instead they fish their sticks out of the water, and drop them back in again on the upstream side. The game only ends when one of the sticks emerges from under the bridge ahead of the other one having also 'lapped' the other stick - that is, having made one additional journey under the bridge compared to the other stick. On a particular day when playing this game, the time taken for a stick to travel under the bridge varies between a minimum of 30 seconds, and a maximum of 60 seconds. The time taken to fish a stick out of the water and drop it back in again on the other side is 5 seconds. The current under the bridge has the unusual property that the sticks' journey time is always an integral number of seconds, and it is equally likely to emerge at any of the possible times between 30 and 60 seconds (inclusive). It turns out that under these circumstances, the expected time for playing a single game is 1036.15 seconds (rounded to 2 decimal places). This time is measured from the point of dropping the sticks for the first time, to the point where the winning stick emerges from under the bridge having lapped the other. The stream flows at different rates each day, but maintains the property that the journey time in seconds is equally distributed amongst the integers from a minimum, $n$, to a maximum, $m$, inclusive. Let the expected time of play in seconds be $E(m,n)$. Hence $E(60,30)=1036.15...$ Let $S(k)=\sum\_{m=2}^k\sum\_{n=1}^{m-1}E(m,n)$. For example $S(5)=7722.82$ rounded to 2 decimal places. Find $S(100)$ and give your answer rounded to 2 decimal places. \href{https://projecteuler.net/problem=589}{Zadanie 589}
\end{zadanie}
\begin{zadanie}
	Let H( n ) denote the number of sets of positive integers such that the least common multiple of the integers in the set equals n . E.g.: The integers in the following ten sets all have a least common multiple of 6: {2,3}, {1,2,3}, {6}, {1,6}, {2,6} ,{1,2,6}, {3,6}, {1,3,6}, {2,3,6} and {1,2,3,6}. Thus H(6)=10. Let L( n ) denote the least common multiple of the numbers 1 through n . E.g. L(6) is the least common multiple of the numbers 1,2,3,4,5,6 and L(6) equals 60. Let HL( n ) denote H(L( n )). You are given HL(4)=H(12)=44. Find HL(50000). Give your answer modulo 10 9 . \href{https://projecteuler.net/problem=590}{Zadanie 590}
\end{zadanie}
\begin{zadanie}
	Given a non-square integer $d$, any real $x$ can be approximated arbitrarily close by quadratic integers $a+b\sqrt{d}$, where $a,b$ are integers. For example, the following inequalities approximate $\pi$ with precision $10^{-13}$: $$4375636191520\sqrt{2}-6188084046055 < \pi < 721133315582\sqrt{2}-1019836515172 $$ We call $BQA\_d(x,n)$ the quadratic integer closest to $x$ with the absolute values of $a,b$ not exceeding $n$. We also define the integral part of a quadratic integer as $I\_d(a+b\sqrt{d}) = a$. You are given that: Find the sum of $|I\_d(BQA\_d(\pi,10^{13}))|$ for all non-square positive integers less than 100. \href{https://projecteuler.net/problem=591}{Zadanie 591}
\end{zadanie}
\begin{zadanie}
	For any N , let f ( N ) be the last twelve hexadecimal digits before the trailing zeroes in N !. For example, the hexadecimal representation of 20! is 21C3677C82B40000, so f (20) is the digit sequence 21C3677C82B4. Find f (20!). Give your answer as twelve hexadecimal digits, using uppercase for the digits A to F. \href{https://projecteuler.net/problem=592}{Zadanie 592}
\end{zadanie}
\begin{zadanie}
	We define two sequences $S = \{S(1), S(2), ..., S(n)\}$ and $S\_2 = \{S\_2(1), S\_2(2), ..., S\_2(n)\}$: $S(k) = (p\_k)^k$ mod $10007$ where $p\_k$ is the $k$th prime number. $S\_2(k) = S(k) + S(\lfloor\frac{k}{10000}\rfloor + 1)$ where $\lfloor \cdot \rfloor$ denotes the floor function. Then let $M(i, j)$ be the median of elements $S\_2(i)$ through $S\_2(j)$, inclusive. For example, $M(1, 10) = 2021.5$ and $M(10^2, 10^3) = 4715.0$. Let $F(n, k) = \sum\_{i=1}^{n-k+1} M(i, i + k - 1)$. For example, $F(100, 10) = 463628.5$ and $F(10^5, 10^4) = 675348207.5$. Find $F(10^7, 10^5)$. If the sum is not an integer, use $.5$ to denote a half. Otherwise, use $.0$ instead. \href{https://projecteuler.net/problem=593}{Zadanie 593}
\end{zadanie}
\begin{zadanie}
	For a polygon $P$, let $t(P)$ be the number of ways in which $P$ can be tiled using rhombi and squares with edge length 1. Distinct rotations and reflections are counted as separate tilings. For example, if $O$ is a regular octagon with edge length 1, then $t(O) = 8$. As it happens, all these 8 tilings are rotations of one another: Let $O\_{a,b}$ be the equal-angled convex octagon whose edges alternate in length between $a$ and $b$. For example, here is $O\_{2,1}$, with one of its tilings: You are given that $t(O\_{1,1})=8$, $t(O\_{2,1})=76$ and $t(O\_{3,2})=456572$. Find $t(O\_{4,2})$. \href{https://projecteuler.net/problem=594}{Zadanie 594}
\end{zadanie}
\begin{zadanie}
	A deck of cards numbered from 1 to n is shuffled randomly such that each permutation is equally likely. The cards are to be sorted into ascending order using the following technique: Let S(n) be the expected number of shuffles needed to sort the cards. Since the order is checked before the first shuffle, S(1) = 0. You are given that S(2) = 1, and S(5) = 4213/871. Find S(52), and give your answer rounded to 8 decimal places. \href{https://projecteuler.net/problem=595}{Zadanie 595}
\end{zadanie}
\begin{zadanie}
	Let T( r ) be the number of integer quadruplets x , y , z , t such that x 2 + y 2 + z 2 + t 2 $\leq$ r 2 . In other words, T( r ) is the number of lattice points in the four-dimensional hyperball of radius r . You are given that T(2) = 89, T(5) = 3121, T(100) = 493490641 and T(10 4 ) = 49348022079085897. Find T(10 8 ) mod 1000000007. \href{https://projecteuler.net/problem=596}{Zadanie 596}
\end{zadanie}
\begin{zadanie}
	Suppose that, in a particular race, each boat $B\_j$ rows at a steady speed $v\_j = -$log$X\_j$ metres per second, where the $X\_j$ are chosen randomly (with uniform distribution) between 0 and 1, independently from one another. These speeds are relative to the riverbank: you may disregard the flow of the river. Let $p(n,L)$ be the probability that the new order is an even permutation of the starting order, when there are $n$ boats in the division and $L$ is the course length. For example, with $n=3$ and $L=160$, labelling the boats as $A$,$B$,$C$ in starting order with $C$ highest, the different possible outcomes of the race are as follows: Therefore, $p(3,160) = 4/15 + 4/27 = 56/135$. You are also given that $p(4,400)=0.5107843137$, rounded to 10 digits after the decimal point. Find $p(13,1800)$ rounded to 10 digits after the decimal point. \href{https://projecteuler.net/problem=597}{Zadanie 597}
\end{zadanie}
\begin{zadanie}
	Consider the number 48. There are five pairs of integers $a$ and $b$ ($a \leq b$) such that $a \times b=48$: (1,48), (2,24), (3,16), (4,12) and (6,8). It can be seen that both 6 and 8 have 4 divisors. So of those five pairs one consists of two integers with the same number of divisors. In general: Let $C(n)$ be the number of pairs of positive integers $a \times b=n$, ($a \leq b$) such that $a$ and $b$ have the same number of divisors; so $C(48)=1$. You are given $C(10!)=3$: (1680, 2160), (1800, 2016) and (1890,1920). Find $C(100!)$ \href{https://projecteuler.net/problem=598}{Zadanie 598}
\end{zadanie}
\begin{zadanie}
	The well-known Rubik's Cube puzzle has many fascinating mathematical properties. The 2$\times$2$\times$2 variant has 8 cubelets with a total of 24 visible faces, each with a coloured sticker. Successively turning faces will rearrange the cubelets, although not all arrangements of cubelets are reachable without dismantling the puzzle. Suppose that we wish to apply new stickers to a 2$\times$2$\times$2 Rubik's cube in a non-standard colouring. Specifically, we have $n$ different colours available (with an unlimited supply of stickers of each colour), and we place one sticker on each of the 24 faces in any arrangement that we please. We are not required to use all the colours, and if desired the same colour may appear in more than one face of a single cubelet. We say that two such colourings $c\_1,c\_2$ are essentially distinct if a cube coloured according to $c\_1$ cannot be made to match a cube coloured according to $c\_2$ by performing mechanically possible Rubik's Cube moves. For example, with two colours available, there are 183 essentially distinct colourings. How many essentially distinct colourings are there with 10 different colours available? \href{https://projecteuler.net/problem=599}{Zadanie 599}
\end{zadanie}
\begin{zadanie}
	Let H ( n ) be the number of distinct integer sided equiangular convex hexagons with perimeter not exceeding n . Hexagons are distinct if and only if they are not congruent . You are given H (6)=1, H (12)=10, H (100)=31248. Find H (55106). Equiangular hexagons with perimeter not exceeding 12 \href{https://projecteuler.net/problem=600}{Zadanie 600}
\end{zadanie}
\begin{zadanie}
	For every positive number $n$ we define the function $streak(n)=k$ as the smallest positive integer $k$ such that $n+k$ is not divisible by $k+1$. E.g: 13 is divisible by 1 14 is divisible by 2 15 is divisible by 3 16 is divisible by 4 17 is NOT divisible by 5 So $streak(13) = 4$. Similarly: 120 is divisible by 1 121 is NOT divisible by 2 So $streak(120) = 1$. Define $P(s, N)$ to be the number of integers $n$, $1 < n < N$, for which $streak(n) = s$. So $P(3, 14) = 1$ and $P(6, 10^6) = 14286$. Find the sum, as $i$ ranges from 1 to 31, of $P(i, 4^i)$. \href{https://projecteuler.net/problem=601}{Zadanie 601}
\end{zadanie}
\begin{zadanie}
	Alice enlists the help of some friends to generate a random number, using a single unfair coin. She and her friends sit around a table and, starting with Alice, they take it in turns to toss the coin. Everyone keeps a count of how many heads they obtain individually. The process ends as soon as Alice obtains a Head. At this point, Alice multiplies all her friends' Head counts together to obtain her random number. As an illustration, suppose Alice is assisted by Bob, Charlie, and Dawn, who are seated round the table in that order, and that they obtain the sequence of Head/Tail outcomes THHHTTTTTHHTH beginning and ending with Alice. Then Bob and Charlie each obtain 2 heads, and Dawn obtains 1 head. Alice's random number is therefore $2\times 2\times 1 = 4$. Define $e(n, p)$ to be the expected value of Alice's random number, where $n$ is the number of friends helping (excluding Alice herself), and $p$ is the probability of the coin coming up Tails. It turns out that, for any fixed $n$, $e(n, p)$ is always a polynomial in $p$. For example, $e(3, p) = p^3 + 4p^2 + p$. Define $c(n, k)$ to be the coefficient of $p^k$ in the polynomial $e(n, p)$. So $c(3, 1) = 1$, $c(3, 2) = 4$, and $c(3, 3) = 1$. You are given that $c(100, 40) \equiv 986699437 \text{ } (\text{mod } 10^9+7)$. Find $c(10000000, 4000000) \mod 10^9+7$. \href{https://projecteuler.net/problem=602}{Zadanie 602}
\end{zadanie}
\begin{zadanie}
	Let $S(n)$ be the sum of all contiguous integer-substrings that can be formed from the integer $n$. The substrings need not be distinct. For example, $S(2024) = 2 + 0 + 2 + 4 + 20 + 02 + 24 + 202 + 024 + 2024 = 2304$. Let $P(n)$ be the integer formed by concatenating the first $n$ primes together. For example, $P(7) = 2357111317$. Let $C(n, k)$ be the integer formed by concatenating $k$ copies of $P(n)$ together. For example, $C(7, 3) = 235711131723571113172357111317$. Evaluate $S(C(10^6, 10^{12}))$ mod $(10^9 + 7)$. \href{https://projecteuler.net/problem=603}{Zadanie 603}
\end{zadanie}
\begin{zadanie}
	Let $F(N)$ be the maximum number of lattice points in an axis-aligned $N\times N$ square that the graph of a single strictly convex increasing function can pass through. You are given that $F(1) = 2$, $F(3) = 3$, $F(9) = 6$, $F(11) = 7$, $F(100) = 30$ and $F(50000) = 1898$. Below is the graph of a function reaching the maximum 3 for $N=3$: Find $F(10^{18})$. \href{https://projecteuler.net/problem=604}{Zadanie 604}
\end{zadanie}
\begin{zadanie}
	Consider an $n$-player game played in consecutive pairs: Round $1$ takes place between players $1$ and $2$, round $2$ takes place between players $2$ and $3$, and so on and so forth, all the way up to round $n$, which takes place between players $n$ and $1$. Then round $n+1$ takes place between players $1$ and $2$ as the entire cycle starts again. In other words, during round $r$, player $((r-1) \bmod n) + 1$ faces off against player $(r \bmod n) + 1$. During each round, a fair coin is tossed to decide which of the two players wins that round. If any given player wins both rounds $r$ and $r+1$, then that player wins the entire game. Let $P\_n(k)$ be the probability that player $k$ wins in an $n$-player game, in the form of a reduced fraction. For example, $P\_3(1) = 12/49$ and $P\_6(2) = 368/1323$. Let $M\_n(k)$ be the product of the reduced numerator and denominator of $P\_n(k)$. For example, $M\_3(1) = 588$ and $M\_6(2) = 486864$. Find the last $8$ digits of $M\_{10^8+7}(10^4+7)$. \href{https://projecteuler.net/problem=605}{Zadanie 605}
\end{zadanie}
\begin{zadanie}
	A gozinta chain for n is a sequence {1,a,b,..., n } where each element properly divides the next. For example, there are eight distinct gozinta chains for 12: {1,12}, {1,2,12}, {1,2,4,12}, {1,2,6,12}, {1,3,12}, {1,3,6,12}, {1,4,12} and {1,6,12}. Let S( n ) be the sum of all numbers, k , not exceeding n , which have 252 distinct gozinta chains. You are given S(10 6 )=8462952 and S(10 12 )=623291998881978. Find S(10 36 ), giving the last nine digits of your answer. \href{https://projecteuler.net/problem=606}{Zadanie 606}
\end{zadanie}
\begin{zadanie}
	Frodo and Sam need to travel 100 leagues due East from point A to point B. On normal terrain, they can cover 10 leagues per day, and so the journey would take 10 days. However, their path is crossed by a long marsh which runs exactly South-West to North-East, and walking through the marsh will slow them down. The marsh is 50 leagues wide at all points, and the mid-point of AB is located in the middle of the marsh. A map of the region is shown in the diagram below: The marsh consists of 5 distinct regions, each 10 leagues across, as shown by the shading in the map. The strip closest to point A is relatively light marsh, and can be crossed at a speed of 9 leagues per day. However, each strip becomes progressively harder to navigate, the speeds going down to 8, 7, 6 and finally 5 leagues per day for the final region of marsh, before it ends and the terrain becomes easier again, with the speed going back to 10 leagues per day. If Frodo and Sam were to head directly East for point B, they would travel exactly 100 leagues, and the journey would take approximately 13.4738 days. However, this time can be shortened if they deviate from the direct path. Find the shortest possible time required to travel from point A to B, and give your answer in days, rounded to 10 decimal places. \href{https://projecteuler.net/problem=607}{Zadanie 607}
\end{zadanie}
\begin{zadanie}
	Let $D(m,n)=\displaystyle\sum\_{d|m}\sum\_{k=1}^n\sigma\_{\small 0}(kd)$ where $d$ runs through all divisors of $m$ and $\sigma\_{\small 0}(n)$ is the number of divisors of $n$. You are given $D(3!,10^2)=3398$ and $D(4!,10^6)=268882292$. Find $D(200!,10^{12}) \text{ mod } (10^9 + 7)$. \href{https://projecteuler.net/problem=608}{Zadanie 608}
\end{zadanie}
\begin{zadanie}
	For every $n \ge 1$ the prime-counting function $\pi(n)$ is equal to the number of primes
	not exceeding $n$. E.g. $\pi(6)=3$ and $\pi(100)=25$. We say that a sequence of integers $u = (u\_0,\cdots,u\_m)$ is a $\pi$ sequence if For $u\_0=10$ there are three distinct $\pi$ sequences: (10,4), (10,4,2) and (10,4,2,1). Let $c(u)$ be the number of elements of $u$ that are not prime. Let $p(n,k)$ be the number of $\pi$ sequences $u$ for which $u\_0\le n$ and $c(u)=k$. Let $P(n)$ be the product of all $p(n,k)$ that are larger than 0. You are given: P(10)=3$\times$8$\times$9$\times$3=648 and P(100)=31038676032. Find $P(10^8)$. Give your answer modulo 1000000007. \href{https://projecteuler.net/problem=609}{Zadanie 609}
\end{zadanie}
\begin{zadanie}
	A random generator produces a sequence of symbols drawn from the set { an I , an V , an X , an L , an C , an D , an M , an # }. Each item in the sequence is determined by selecting one of these symbols at random, independently of the other items in the sequence. At each step, the seven letters are equally likely to be selected, with probability 14% each, but the an # symbol only has a 2\% chance of selection. an I an V an X an L an C an D an M an # an # We write down the sequence of letters from left to right as they are generated, and we stop at the first occurrence of the an # symbol (without writing it). However, we stipulate that what we have written down must always (when non-empty) be a valid Roman numeral representation in minimal form. If appending the next letter would contravene this then we simply skip it and try again with the next symbol generated. an # Please take careful note of About... Roman Numerals for the definitive rules for this problem on what constitutes a "valid Roman numeral representation" and "minimal form". For example, the (only) sequence that represents 49 is ily:'courier new', monospace;">XLIX . The subtractive combination ily:'courier new', monospace;">IL is invalid because of rule (ii), while ily:'courier new', monospace;">XXXXIX is valid but not minimal. The rules do not place any restriction on the number of occurrences of ily:'courier new', monospace;">M , so all integers have a valid representation. These are the same rules as were used in Problem 89 , and members are invited to solve that problem first. Find the expected value of the number represented by what we have written down when we stop. (If nothing is written down then count that as zero.) Give your answer rounded to 8 places after the decimal point. \href{https://projecteuler.net/problem=610}{Zadanie 610}
\end{zadanie}
\begin{zadanie}
	Peter moves in a hallway with N +1 doors consecutively numbered from 0 through N . All doors are initially closed. Peter starts in front of door 0, and repeatedly performs the following steps: We call an action any sequence of those steps. Peter never performs the exact same action twice, and makes sure to perform all possible actions that don't bring him past the last door. Let F ( N ) be the number of doors that are open after Peter has performed all possible actions. You are given that F (5)=1, F (100)=27, F (1000)=233 and F (10 6 )=112168. Find F (10 12 ). \href{https://projecteuler.net/problem=611}{Zadanie 611}
\end{zadanie}
\begin{zadanie}
	Let's call two numbers friend numbers if their representation in base 10 has at least one common digit. E.g. 1123 and 3981 are friend numbers. Let $f(n)$ be the number of pairs $(p,q)$ with $1\le p \lt q \lt n$ such that $p$ and $q$ are friend numbers. $f(100)=1539$. Find $f(10^{18})$ mod $1000267129$. \href{https://projecteuler.net/problem=612}{Zadanie 612}
\end{zadanie}
\begin{zadanie}
	Dave is doing his homework on the balcony and, preparing a presentation about Pythagorean triangles, has just cut out a triangle with side lengths 30cm, 40cm and 50cm from some cardboard, when a gust of wind blows the triangle down into the garden. Another gust blows a small ant straight onto this triangle. The poor ant is completely disoriented and starts to crawl straight ahead in random direction in order to get back into the grass. Assuming that all possible positions of the ant within the triangle and all possible directions of moving on are equiprobable, what is the probability that the ant leaves the triangle along its longest side? Give your answer rounded to 10 digits after the decimal point. \href{https://projecteuler.net/problem=613}{Zadanie 613}
\end{zadanie}
\begin{zadanie}
	An integer partition of a number $n$ is a way of writing $n$ as a sum of positive integers. Partitions that differ only by the order of their summands are considered the same. We call an integer partition special if 1) all its summands are distinct, and 2) all its even summands are also divisible by 4. For example, the special partitions of $10$ are: \[10 = 1+4+5=3+7=1+9\]
	The number $10$ admits many more integer partitions (a total of 42), but only those three are special. Let be $P(n)$ the number of special integer partitions of $n$. You are given that $P(1) = 1$, $P(2) = 0$, $P(3) = 1$, $P(6) = 1$, $P(10)=3$, $P(100) = 37076$ and $P(1000)=3699177285485660336$. Find $\displaystyle \sum\_{i=1}^{10^7}{P(i)}$. Give the result modulo $10^9+7$. \href{https://projecteuler.net/problem=614}{Zadanie 614}
\end{zadanie}
\begin{zadanie}
	Consider the natural numbers having at least 5 prime factors, which don't have to be distinct. Sorting these numbers by size gives a list which starts with: So, for example, the fifth number with at least 5 prime factors is 80. Find the millionth number with at least one million prime factors. Give your answer modulo 123454321. \href{https://projecteuler.net/problem=615}{Zadanie 615}
\end{zadanie}
\begin{zadanie}
	Alice plays the following game, she starts with a list of integers $L$ and on each step she can either: For example starting from the list $L=\{8\}$, Alice can remove $8$ and add $2$ and $3$ resulting in $L=\{2,3\}$ in a first step. Then she can obtain $L=\{9\}$ in a second step. Note that the same integer is allowed to appear multiple times in the list. An integer $n>1$ is said to be creative if for any integer $m>1$ Alice can obtain a list that contains $m$ starting from $L=\{n\}$. Find the sum of all creative integers less than or equal to $10^{12}$. \href{https://projecteuler.net/problem=616}{Zadanie 616}
\end{zadanie}
\begin{zadanie}
	For two integers $n,e > 1$, we define a $(n,e)$- MPS (Mirror Power Sequence) to be an infinite sequence of integers $(a\_i)\_{i\ge 0}$ such that for all $i\ge 0$, $a\_{i+1} = min(a\_i^e,n-a\_i^e)$ and $a\_i > 1$. Examples of such sequences are the two $(18,2)$-MPS sequences made of alternating $2$ and $4$. Note that even though such a sequence is uniquely determined by $n,e$ and $a\_0$, for most values such a sequence does not exist. For example, no $(n,e)$-MPS exists for $n < 6$. Define $C(n)$ to be the number of $(n,e)$-MPS for some $e$, and $\displaystyle D(N) = \sum\_{n=2}^N {C(n)}$. You are given that $D(10) = 2$, $D(100) = 21$, $D(1000) = 69$, $D(10^6) = 1303$ and $D(10^{12}) = 1014800$. Find $D(10^{18})$. \href{https://projecteuler.net/problem=617}{Zadanie 617}
\end{zadanie}
\begin{zadanie}
	Consider the numbers 15, 16 and 18: $15=3\times 5$ and $3+5=8$. $16 = 2\times 2\times 2\times 2$ and $2+2+2+2=8$. $18 = 2\times 3\times 3$ and $2+3+3=8$. 15, 16 and 18 are the only numbers that have 8 as sum of the prime factors (counted with multiplicity). We define $S(k)$ to be the sum of all numbers $n$ where the sum of the prime factors (with multiplicity) of $n$ is $k$. Hence $S(8) = 15+16+18 = 49$. Other examples: $S(1) = 0$, $S(2) = 2$, $S(3) = 3$, $S(5) = 5 + 6 = 11$. The Fibonacci sequence is $F\_1 = 1, F\_2 = 1, F\_3 = 2, F\_4 = 3, F\_5 = 5$, .... Find the last nine digits of $\displaystyle\sum\_{k=2}^{24}{S(F\_k)}$. \href{https://projecteuler.net/problem=618}{Zadanie 618}
\end{zadanie}
\begin{zadanie}
	For a set of positive integers $\{a, a+1, a+2, \dots , b\}$, let $C(a,b)$ be the number of non-empty subsets in which the product of all elements is a perfect square. For example $C(5,10)=3$, since the products of all elements of $\{5, 8, 10\}$, $\{5, 8, 9, 10\}$ and $\{9\}$ are perfect squares, and no other subsets of $\{5, 6, 7, 8, 9, 10\}$ have this property. You are given that $C(40,55) =15$, and $C(1000,1234) \text{ mod } 1000000007=975523611$. Find $C(1000000,1234567) \text{ mod } 1000000007$. \href{https://projecteuler.net/problem=619}{Zadanie 619}
\end{zadanie}
\begin{zadanie}
	A circle $C$ of circumference $c$ centimetres has a smaller circle $S$ of circumference $s$ centimetres lying off-centre within it. Four other distinct circles, which we call "planets", with circumferences $p$, $p$, $q$, $q$ centimetres respectively ($p<q$), are inscribed within $C$ but outside $S$, with each planet touching both $C$ and $S$ tangentially. The planets are permitted to overlap one another, but the boundaries of $S$ and $C$ must be at least 1cm apart at their closest point. Now suppose that these circles are actually gears with perfectly meshing teeth at a pitch of 1cm. $C$ is an internal gear with teeth on the inside. We require that $c$, $s$, $p$, $q$ are all integers (as they are the numbers of teeth), and we further stipulate that any gear must have at least 5 teeth. Note that "perfectly meshing" means that as the gears rotate, the ratio between their angular velocities remains constant, and the teeth of one gear perfectly align with the groves of the other gear and vice versa. Only for certain gear sizes and positions will it be possible for $S$ and $C$ each to mesh perfectly with all the planets. Arrangements where not all gears mesh perfectly are not valid. Define $g(c,s,p,q)$ to be the number of such gear arrangements for given values of $c$, $s$, $p$, $q$: it turns out that this is finite as only certain discrete arrangements are possible satisfying the above conditions. For example, $g(16,5,5,6)=9$. Here is one such arrangement: Let $G(n) = \sum\_{s+p+q\le n} g(s+p+q,s,p,q)$ where the sum only includes cases with $p<q$, $p\ge 5$, and $s\ge 5$, all integers. You are given that $G(16)=9$ and $G(20)=205$. Find $G(500)$. \href{https://projecteuler.net/problem=620}{Zadanie 620}
\end{zadanie}
\begin{zadanie}
	Gauss famously proved that every positive integer can be expressed as the sum of three triangular numbers (including 0 as the lowest triangular number). In fact most numbers can be expressed as a sum of three triangular numbers in several ways. Let $G(n)$ be the number of ways of expressing $n$ as the sum of three triangular numbers, regarding different arrangements of the terms of the sum as distinct. For example, $G(9) = 7$, as 9 can be expressed as: 3+3+3, 0+3+6, 0+6+3, 3+0+6, 3+6+0, 6+0+3, 6+3+0. You are given $G(1000) = 78$ and $G(10^6) = 2106$. Find $G(17 526 \times 10^9)$. \href{https://projecteuler.net/problem=621}{Zadanie 621}
\end{zadanie}
\begin{zadanie}
	A riffle shuffle is executed as follows: a deck of cards is split into two equal halves, with the top half taken in the left hand and the bottom half taken in the right hand. Next, the cards are interleaved exactly, with the top card in the right half inserted just after the top card in the left half, the 2nd card in the right half just after the 2nd card in the left half, etc. (Note that this process preserves the location of the top and bottom card of the deck) Let $s(n)$ be the minimum number of consecutive riffle shuffles needed to restore a deck of size $n$ to its original configuration, where $n$ is a positive even number. Amazingly, a standard deck of $52$ cards will first return to its original configuration after only $8$ perfect shuffles, so $s(52) = 8$. It can be verified that a deck of $86$ cards will also return to its original configuration after exactly $8$ shuffles, and the sum of all values of $n$ that satisfy $s(n) = 8$ is $412$. Find the sum of all values of n that satisfy $s(n) = 60$. \href{https://projecteuler.net/problem=622}{Zadanie 622}
\end{zadanie}
\begin{zadanie}
	The lambda-calculus is a universal model of computation at the core of functional programming languages. It is based on lambda-terms , a minimal programming language featuring only function definitions, function calls and variables. Lambda-terms are built according to the following rules: A lambda-term $T$ is said to be closed if for all variables $x$, all occurrences of $x$ within $T$ are contained within some abstraction $(\lambda x. M)$ in $T$. The smallest such abstraction is said to bind the occurrence of the variable $x$. In other words, a lambda-term is closed if all its variables are bound to parameters of enclosing functions definitions. For example, the term $(\lambda x. x)$ is closed, while the term $(\lambda x. (x y))$ is not because $y$ is not bound. Also, we can rename variables as long as no binding abstraction changes. This means that $(\lambda x. x)$ and $(\lambda y. y)$ should be considered equivalent since we merely renamed a parameter. Two terms equivalent modulo such renaming are called $\alpha$-equivalent . Note that $(\lambda x. (\lambda y. (x y)))$ and $(\lambda x. (\lambda x. (x x)))$ are not $\alpha$-equivalent, since the abstraction binding the first variable was the outer one and becomes the inner one. However, $(\lambda x. (\lambda y. (x y)))$ and $(\lambda y. (\lambda x. (y x)))$ are $\alpha$-equivalent. The following table regroups the lambda-terms that can be written with at most $15$ symbols, symbols being parenthesis, $\lambda$, dot and variables. Let be $\Lambda(n)$ the number of distinct closed lambda-terms that can be written using at most $n$ symbols, where terms that are $\alpha$-equivalent to one another should be counted only once. You are given that $\Lambda(6) = 1$, $\Lambda(9) = 2$, $\Lambda(15) = 20$ and $\Lambda(35) = 3166438$. Find $\Lambda(2000)$. Give the answer modulo $1\,000\,000\,007$. \href{https://projecteuler.net/problem=623}{Zadanie 623}
\end{zadanie}
\begin{zadanie}
	An unbiased coin is tossed repeatedly until two consecutive heads are obtained. Suppose these occur on the $(M-1)$th and $M$th toss. Let $P(n)$ be the probability that $M$ is divisible by $n$. For example, the outcomes HH, HTHH, and THTTHH all count towards $P(2)$, but THH and HTTHH do not. You are given that $P(2) =\frac 3 5$ and $P(3)=\frac 9 {31}$. Indeed, it can be shown that $P(n)$ is always a rational number. For a prime $p$ and a fully reduced fraction $\frac a b$, define $Q(\frac a b,p)$ to be the smallest positive $q$ for which $a \equiv b q \pmod{p}$. For example $Q(P(2), 109) = Q(\frac 3 5, 109) = 66$, because $5 \cdot 66 = 330 \equiv 3 \pmod{109}$ and 66 is the smallest positive such number. Similarly $Q(P(3),109) = 46$. Find $Q(P(10^{18}),1\,000\,000\,009)$. \href{https://projecteuler.net/problem=624}{Zadanie 624}
\end{zadanie}
\begin{zadanie}
	$G(N)=\sum\_{j=1}^N\sum\_{i=1}^j \text{gcd}(i,j)$. You are given: $G(10)=122$. Find $G(10^{11})$. Give your answer modulo 998244353 \href{https://projecteuler.net/problem=625}{Zadanie 625}
\end{zadanie}
\begin{zadanie}
	A binary matrix is a matrix consisting entirely of 0s and 1s. Consider the following transformations that can be performed on a binary matrix: Two binary matrices $A$ and $B$ will be considered equivalent if there is a sequence of such transformations that when applied to $A$ yields $B$. For example, the following two matrices are equivalent: via the sequence of two transformations "Flip all elements in column 3" followed by "Swap rows 1 and 2". Define $c(n)$ to be the maximum number of $n\times n$ binary matrices that can be found such that no two are equivalent. For example, $c(3)=3$. You are also given that $c(5)=39$ and $c(8)=656108$. Find $c(20)$, and give your answer modulo $1\,001\,001\,011$. \href{https://projecteuler.net/problem=626}{Zadanie 626}
\end{zadanie}
\begin{zadanie}
	Consider the set $S$ of all possible products of $n$ positive integers not exceeding $m$, that is $S=\{ x\_1x\_2\dots x\_n \, | \, 1 \le x\_1, x\_2, ..., x\_n \le m \}$. Let $F(m,n)$ be the number of the distinct elements of the set $S$. For example, $F(9, 2) = 36$ and $F(30,2)=308$. Find $F(30, 10001)\text{ mod }1\,000\,000\,007$. \href{https://projecteuler.net/problem=627}{Zadanie 627}
\end{zadanie}
\begin{zadanie}
	A position in chess is an (orientated) arrangement of chess pieces placed on a chessboard of given size. In the following, we consider all positions in which $n$ pawns are placed on a $n \times n$ board in such a way, that there is a single pawn in every row and every column. We call such a position an open position, if a rook, starting at the (empty) lower left corner and using only moves towards the right or upwards, can reach the upper right corner without moving onto any field occupied by a pawn. Let $f(n)$ be the number of open positions for a $n \times n$ chessboard. For example, $f(3)=2$, illustrated by the two open positions for a $3 \times 3$ chessboard below. You are also given $f(5)=70$. Find $f(10^8)$ modulo $1\,008\,691\,207$. \href{https://projecteuler.net/problem=628}{Zadanie 628}
\end{zadanie}
\begin{zadanie}
	Alice and Bob are playing a modified game of Nim called Scatterstone Nim, with Alice going first, alternating turns with Bob. The game begins with an arbitrary set of stone piles with a total number of stones equal to $n$. During a player's turn, he/she must pick a pile having at least $2$ stones and perform a split operation, dividing the pile into an arbitrary set of $p$ non-empty, arbitrarily-sized piles where $2 \leq p \leq k$ for some fixed constant $k$. For example, a pile of size $4$ can be split into $\{1, 3\}$ or $\{2, 2\}$, or $\{1, 1, 2\}$ if $k = 3$ and in addition $\{1, 1, 1, 1\}$ if $k = 4$. If no valid move is possible on a given turn, then the other player wins the game. A winning position is defined as a set of stone piles where a player can ultimately ensure victory no matter what the other player does. Let $f(n,k)$ be the number of winning positions for Alice on her first turn, given parameters $n$ and $k$. For example, $f(5, 2) = 3$ with winning positions $\{1, 1, 1, 2\}, \{1, 4\}, \{2, 3\}$. In contrast, $f(5, 3) = 5$ with winning positions $\{1, 1, 1, 2\}, \{1, 1, 3\}, \{1, 4\}, \{2, 3\}, \{5\}$. Let $g(n)$ be the sum of $f(n,k)$ over all $2 \leq k \leq n$. For example, $g(7)=66$ and $g(10)=291$. Find $g(200)$ mod $(10^9 + 7)$. \href{https://projecteuler.net/problem=629}{Zadanie 629}
\end{zadanie}
\begin{zadanie}
	Given a set, $L$, of unique lines, let $M(L)$ be the number of lines in the set and let $S(L)$ be the sum over every line of the number of times that line is crossed by another line in the set. For example, two sets of three lines are shown below: In both cases M(L) is 3 and S(L) is 6: each of the three lines is crossed by two other lines. Note that even if the lines cross at a single point, all of the separate crossings of lines are counted. Consider points ($T\_{2k-1}$,$T\_{2k}$), for integer $k>= 1$, generated in the following way: $S\_0 = 290797$ $S\_{n+1} = {S\_n}^2 \:\: \rm{mod} \:\: 50515093$ $T\_n = (S\_n \:\: \rm{mod} \:\: 2000) - 1000$ For example, the first three points are: (527,144), (-488,732), (-454,-947). Given the first $n$ points generated in this manner, let $L\_n$ be the set of unique lines that can be formed by joining each point with every other point, the lines being extended indefinitely in both directions. We can then define $M(L\_n)$ and $S(L\_n)$ as described above. For example, $M(L\_3) = 3$ and $S(L\_3) = 6$. Also $M(L\_{100}) = 4948$ and $S(L\_{100}) = 24477690$. Find $S(L\_{2500})$. \href{https://projecteuler.net/problem=630}{Zadanie 630}
\end{zadanie}
\begin{zadanie}
	Let $(p\_1 p\_2 \ldots p\_k)$ denote the permutation of the set ${1, ..., k}$ that maps $p\_i\mapsto i$. Define the length of the permutation to be $k$; note that the empty permutation $()$ has length zero. Define an occurrence of a permutation $p=(p\_1 p\_2 \ldots p\_k)$ in a permutation $P=(P\_1 P\_2 \ldots P\_n)$ to be a sequence $1\leq t\_1 < t\_2 < \cdots < t\_k \leq n$ such that $p\_i < p\_j$ if and only if $P\_{t\_i} < P\_{t\_j}$ for all $i,j \in \{1, ..., k\}$. For example, $(1243)$ occurs twice in the permutation $(314625)$: once as the 1st, 3rd, 4th and 6th elements $(3\,\,46\,\,5)$, and once as the 2nd, 3rd, 4th and 6th elements $(\,\,146\,\,5)$. Let $f(n, m)$ be the number of permutations $P$ of length at most $n$ such that there is no occurrence of the permutation $1243$ in $P$ and there are at most $m$ occurrences of the permutation $21$ in $P$. For example, $f(2,0) = 3$, with the permutations $()$, $(1)$, $(1,2)$ but not $(2,1)$. You are also given that $f(4, 5) = 32$ and $f(10, 25) = 294\,400$. Find $f(10^{18}, 40)$ modulo $1\,000\,000\,007$. \href{https://projecteuler.net/problem=631}{Zadanie 631}
\end{zadanie}
\begin{zadanie}
	For an integer $n$, we define the square prime factors of $n$ to be the primes whose square divides $n$. For example, the square prime factors of $1500=2^2 \times 3 \times 5^3$ are $2$ and $5$. Let $C\_k(N)$ be the number of integers between $1$ and $N$ inclusive with exactly $k$ square prime factors. You are given some values of $C\_k(N)$ in the table below. Find the product of all non-zero $C\_k(10^{16})$. Give the result reduced modulo $1\,000\,000\,007$. \href{https://projecteuler.net/problem=632}{Zadanie 632}
\end{zadanie}
\begin{zadanie}
	For an integer $n$, we define the square prime factors of $n$ to be the primes whose square divides $n$. For example, the square prime factors of $1500=2^2 \times 3 \times 5^3$ are $2$ and $5$. Let $C\_k(N)$ be the number of integers between $1$ and $N$ inclusive with exactly $k$ square prime factors. It can be shown that with growing $N$ the ratio $\frac{C\_k(N)}{N}$ gets arbitrarily close to a constant $c\_{k}^{\infty}$, as suggested by the table below. \href{https://projecteuler.net/problem=633}{Zadanie 633}
\end{zadanie}
\begin{zadanie}
	Define $F(n)$ to be the number of integers $x$\leq$n$ that can be written in the form $x=a^2b^3$, where $a$ and $b$ are integers not necessarily different and both greater than 1. For example, $32=2^2\times 2^3$ and $72=3^2\times 2^3$ are the only two integers less than 100 that can be written in this form. Hence, $F(100)=2$. Further you are given $F(2\times 10^4)=130$ and $F(3\times 10^6)=2014$. Find $F(9\times 10^{18})$. \href{https://projecteuler.net/problem=634}{Zadanie 634}
\end{zadanie}
\begin{zadanie}
	Let $A\_q(n)$ be the number of subsets, $B$, of the set $\{1, 2, ..., q \cdot n\}$ that satisfy two conditions: 1) $B$ has exactly $n$ elements; 2) the sum of the elements of $B$ is divisible by $n$. E.g. $A\_2(5)=52$ and $A\_3(5)=603$. Find $S\_2(10^8)+S\_3(10^8)$. Give your answer modulo $1\,000\,000\,009$. \href{https://projecteuler.net/problem=635}{Zadanie 635}
\end{zadanie}
\begin{zadanie}
	Consider writing a natural number as product of powers of natural numbers with given exponents, additionally requiring different base numbers for each power. For example, $256$ can be written as a product of a square and a fourth power in three ways such that the base numbers are different. That is, $256=1^2\times 4^4=4^2\times 2^4=16^2\times 1^4$ Though $4^2$ and $2^4$ are both equal, we are concerned only about the base numbers in this problem. Note that permutations are not considered distinct, for example $16^2\times 1^4$ and $1^4 \times 16^2$ are considered to be the same. Similarly, $10!$ can be written as a product of one natural number, two squares and three cubes in two ways ($10!=42\times5^2\times4^2\times3^3\times2^3\times1^3=21\times5^2\times2^2\times4^3\times3^3\times1^3$) whereas $20!$ can be given the same representation in $41680$ ways. Let $F(n)$ denote the number of ways in which $n$ can be written as a product of one natural number, two squares, three cubes and four fourth powers. You are given that $F(25!)=4933$, $F(100!) \bmod 1\,000\,000\,007=693\,952\,493$, and $F(1\,000!) \bmod 1\,000\,000\,007=6\,364\,496$. Find $F(1\,000\,000!) \bmod 1\,000\,000\,007$. \href{https://projecteuler.net/problem=636}{Zadanie 636}
\end{zadanie}
\begin{zadanie}
	Given any positive integer $n$, we can construct a new integer by inserting plus signs between some of the digits of the base $B$ representation of $n$, and then carrying out the additions. For example, from $n=123\_{10}$ ($n$ in base 10) we can construct the four base 10 integers $123\_{10}$, $1+23=24\_{10}$, $12+3=15\_{10}$ and $1+2+3=6\_{10}$ Let $f(n,B)$ be the smallest number of steps needed to arrive at a single-digit number in base $B$. For example, $f(7,10)=0$ and $f(123,10)=1$. Let $g(n,B\_1,B\_2)$ be the sum of the positive integers $i$ not exceeding $n$ such that $f(i,B\_1)=f(i,B\_2)$. You are given $g(100,10,3)=3302$. Find $g(10^7,10,3)$ \href{https://projecteuler.net/problem=637}{Zadanie 637}
\end{zadanie}
\begin{zadanie}
	Define $G(P\_{a,b},k)=k^{A(P\_{a,b})}$. Let $C(a,b,k)$ equal the sum of $G(P\_{a,b},k)$ over all valid paths in a $a\times b$ lattice grid. You are given that \href{https://projecteuler.net/problem=638}{Zadanie 638}
\end{zadanie}
\begin{zadanie}
	A multiplicative function $f(x)$ is a function over positive integers satisfying $f(1)=1$ and $f(a b)=f(a) f(b)$ for any two coprime positive integers $a$ and $b$. For integer $k$ let $f\_k(n)$ be a multiplicative function additionally satisfying $f\_k(p^e)=p^k$ for any prime $p$ and any integer $e>0$. For example, $f\_1(2)=2$, $f\_1(4)=2$, $f\_1(18)=6$ and $f\_2(18)=36$. Let $\displaystyle S\_k(n)=\sum\_{i=1}^{n} f\_k(i)$.
	For example, $S\_1(10)=41$, $S\_1(100)=3512$, $S\_2(100)=208090$, $S\_1(10000)=35252550$ and $\displaystyle \sum\_{k=1}^{3} S\_k(10^{8}) \equiv 338787512 \pmod{ 1\,000\,000\,007}$. Find $\displaystyle \sum\_{k=1}^{50} S\_k(10^{12}) \bmod 1\,000\,000\,007$. \href{https://projecteuler.net/problem=639}{Zadanie 639}
\end{zadanie}
\begin{zadanie}
	Bob plays a single-player game of chance using two standard 6-sided dice and twelve cards numbered 1 to 12. When the game starts, all cards are placed face up on a table. Each turn, Bob rolls both dice, getting numbers $x$ and $y$ respectively, each in the range 1,...,6. He must choose amongst three options: turn over card $x$, card $y$, or card $x+y$. (If the chosen card is already face down, it is turned to face up, and vice versa.) If Bob manages to have all twelve cards face down at the same time, he wins. Alice plays a similar game, except that instead of dice she uses two fair coins, counting heads as 2 and tails as 1, and that she uses four cards instead of twelve. Alice finds that, with the optimal strategy for her game, the expected number of turns taken until she wins is approximately 5.673651. Assuming that Bob plays with an optimal strategy, what is the expected number of turns taken until he wins? Give your answer rounded to 6 places after the decimal point. \href{https://projecteuler.net/problem=640}{Zadanie 640}
\end{zadanie}
\begin{zadanie}
	Consider a row of $n$ dice all showing 1. First turn every second die,$ (2,4,6,\ldots)$, so that the number showing is increased by 1. Then turn every third die. The sixth die will now show a 3. Then turn every fourth die and so on until every $n$th die (only the last die) is turned. If the die to be turned is showing a 6 then it is changed to show a 1. Let $f(n)$ be the number of dice that are showing a 1 when the process finishes. You are given $f(100)=2$ and $f(10^8) = 69$. Find $f(10^{36})$. \href{https://projecteuler.net/problem=641}{Zadanie 641}
\end{zadanie}
\begin{zadanie}
	Let $f(n)$ be the largest prime factor of $n$ and $\displaystyle F(n) = \sum\_{i=2}^{n}f(i)$. For example $F(10)=32$, $F(100)=1915$ and $F(10000)=10118280$. Find $F(201820182018)$. Give your answer modulus $10^9$. \href{https://projecteuler.net/problem=642}{Zadanie 642}
\end{zadanie}
\begin{zadanie}
	Two positive integers $a$ and $b$ are 2-friendly when $\gcd(a,b) = 2^t, t>0$. For example, 24 and 40 are 2-friendly because $\gcd(24,40) = 8 = 2^3$ while 24 and 36 are not because $\gcd(24,36) = 12 = 2^2\cdot 3$ not a power of 2. Let $f(n)$ be the number of pairs, $(p,q)$, of positive integers with $1\le p\lt q\le n$ such that $p$ and $q$ are 2-friendly. You are given $f(10^2) = 1031$ and $f(10^6) = 321418433$ modulo $1\,000\,000\,007$. Find $f(10^{11})$ modulo $1\,000\,000\,007$. \href{https://projecteuler.net/problem=643}{Zadanie 643}
\end{zadanie}
\begin{zadanie}
	Sam and Tom are trying a game of (partially) covering a given line segment of length $L$ by taking turns in placing unit squares onto the line segment. As illustrated below, the squares may be positioned in two different ways, either "straight" by placing the midpoints of two opposite sides on the line segment, or "diagonal" by placing two opposite corners on the line segment. Newly placed squares may touch other squares, but are not allowed to overlap any other square laid down before. The player who is able to place the last unit square onto the line segment wins. With Sam starting each game by placing the first square, they quickly realise that Sam can easily win every time by placing the first square in the middle of the line segment, making the game boring. Therefore they decide to randomise Sam's first move, by first tossing a fair coin to determine whether the square will be placed straight or diagonal onto the line segment and then choosing the actual position on the line segment randomly with all possible positions being equally likely. Sam's gain of the game is defined to be 0 if he loses the game and $L$ if he wins. Assuming optimal play of both players after Sam's initial move, you can see that Sam's expected gain, called $e(L)$, is only dependent on the length of the line segment. For example, if $L=2$, Sam will win with a probability of 1, so $e(2)= 2$. Choosing $L=4$, the winning probability will be 0.33333333 for the straight case and 0.22654092 for the diagonal case, leading to $e(4)=1.11974851$ (rounded to 8 digits after the decimal point each). Being interested in the optimal value of $L$ for Sam, let's define $f(a,b)$ to be the maximum of $e(L)$ for some $L \in [a,b]$. You are given $f(2,10)=2.61969775$, being reached for $L= 7.82842712$, and $f(10,20)=
		5.99374121$ (rounded to 8 digits each). Find $f(200,500)$, rounded to 8 digits after the decimal point. \href{https://projecteuler.net/problem=644}{Zadanie 644}
\end{zadanie}
\begin{zadanie}
	On planet J, a year lasts for $D$ days. Holidays are defined by the two following rules. Initially there are no holidays. Let $E(D)$ be the expected number of Emperors to reign before all the days of the year are holidays, assuming that their birthdays are independent and uniformly distributed throughout the $D$ days of the year. You are given $E(2)=1$, $E(5)=31/6$, $E(365)\approx 1174.3501$. Find $E(10000)$. Give your answer rounded to 4 digits after the decimal point. \href{https://projecteuler.net/problem=645}{Zadanie 645}
\end{zadanie}
\begin{zadanie}
	Let $n$ be a natural number and $p\_1^{\alpha\_1}\cdot p\_2^{\alpha\_2}...p\_k^{\alpha\_k}$ its prime factorisation. Define the Liouville function $\lambda(n)$ as $\lambda(n) = (-1)^{\sum\limits\_{i=1}^{k}\alpha\_i}$. (i.e. $-1$ if the sum of the exponents $\alpha\_i$ is odd and $1$ if the sum of the exponents is even. ) Let $S(n,L,H)$ be the sum $\lambda(d) \cdot d$ over all divisors $d$ of $n$ for which $L \leq d \leq H$. You are given: $\bullet\, S(10! , 100, 1000) = 1457$ $\bullet\, S(15!, 10^3, 10^5) = -107974$ $\bullet\, S(30!,10^8, 10^{12}) = 9766732243224$. Find $S(70!,10^{20}, 10^{60})$ and give your answer modulo $1\,000\,000\,007$. \href{https://projecteuler.net/problem=646}{Zadanie 646}
\end{zadanie}
\begin{zadanie}
	It is possible to find positive integers $A$ and $B$ such that given any triangular number, $T\_n$, then $AT\_n +B$ is always a triangular number. We define $F\_3(N)$ to be the sum of $(A+B)$ over all such possible pairs $(A,B)$ with $\max(A,B)\le N$. For example $F\_3(100) = 184$. Polygonal numbers are generalisations of triangular numbers. Polygonal numbers with parameter $k$ we call $k$-gonal numbers. The formula for the $n$th $k$-gonal number is $\frac 12n\big(n(k-2)+4-k\big)$ where $n \ge 1$. For example when $k = 3$ we get $\frac 12n(n+1)$ the formula for triangular numbers. The statement above is true for pentagonal, heptagonal and in fact any $k$-gonal number with $k$ odd. For example when $k=5$ we get the pentagonal numbers and we can find positive integers $A$ and $B$ such that given any pentagonal number, $P\_n$, then $AP\_n+B$ is always a pentagonal number. We define $F\_5(N)$ to be the sum of $(A+B)$ over all such possible pairs $(A,B)$ with $\max(A,B)\le N$. Similarly we define $F\_k(N)$ for odd $k$. You are given $\sum\_{k} F\_k(10^3) = 14993$ where the sum is over all odd $k = 3,5,7,\ldots$. Find $\sum\_{k} F\_k(10^{12})$ where the sum is over all odd $k = 3,5,7,\ldots$ \href{https://projecteuler.net/problem=647}{Zadanie 647}
\end{zadanie}
\begin{zadanie}
	For some fixed $\rho \in [0, 1]$, we begin a sum $s$ at $0$ and repeatedly apply a process: With probability $\rho$, we add $1$ to $s$, otherwise we add $2$ to $s$. The process ends when either $s$ is a perfect square or $s$ exceeds $10^{18}$, whichever occurs first. For example, if $s$ goes through $0, 2, 3, 5, 7, 9$, the process ends at $s=9$, and two squares $1$ and $4$ were skipped over. Let $f(\rho)$ be the expected number of perfect squares skipped over when the process finishes. It can be shown that the power series for $f(\rho)$ is $\sum\_{k=0}^\infty a\_k \rho^k$ for a suitable (unique) choice of coefficients $a\_k$. Some of the first few coefficients are $a\_0=1$, $a\_1=0$, $a\_5=-18$, $a\_{10}=45176$. Let $F(n) = \sum\_{k=0}^n a\_k$. You are given that $F(10) = 53964$ and $F(50) \equiv 842418857 \pmod{10^9}$. Find $F(1000)$, and give your answer modulo $10^9$. \href{https://projecteuler.net/problem=648}{Zadanie 648}
\end{zadanie}
\begin{zadanie}
	The game may start with any arrangement of $c$ coins in squares on the board. It is possible at any time for more than one coin to occupy the same square on the board at the same time. The coins are distinguishable, so swapping two coins gives a different arrangement if (and only if) they are on different squares. On a given turn, the player must choose a coin and move it either left or up $2$, $3$, $5$, or $7$ spaces in a single direction. The only restriction is that the coin cannot move off the edge of the board. The game ends when a player is unable to make a valid move, thereby granting the other player the victory. Assuming that Alice goes first and that both players are playing optimally, let $M(n, c)$ be the number of possible starting arrangements for which Alice can ensure her victory, given a board of size $n$ by $n$ with $c$ distinct coins. For example, $M(3, 1) = 4$, $M(3, 2) = 40$, and $M(9, 3) = 450304$. What are the last $9$ digits of $M(10\,000\,019, 100)$? \href{https://projecteuler.net/problem=649}{Zadanie 649}
\end{zadanie}
\begin{zadanie}
	Let $B(n) = \displaystyle \prod\_{k=0}^n {n \choose k}$, a product of binomial coefficients. For example, $B(5) = {5 \choose 0} \times {5 \choose 1} \times {5 \choose 2} \times {5 \choose 3} \times {5 \choose 4} \times {5 \choose 5} = 1 \times 5 \times 10 \times 10 \times 5 \times 1 = 2500$. Let $D(n) = \displaystyle \sum\_{d|B(n)} d$, the sum of the divisors of $B(n)$. For example, the divisors of B(5) are 1, 2, 4, 5, 10, 20, 25, 50, 100, 125, 250, 500, 625, 1250 and 2500, so D(5) = 1 + 2 + 4 + 5 + 10 + 20 + 25 + 50 + 100 + 125 + 250 + 500 + 625 + 1250 + 2500 = 5467. Let $S(n) = \displaystyle \sum\_{k=1}^n D(k)$. You are given $S(5) = 5736$, $S(10) = 141740594713218418$ and $S(100)$ mod $1\,000\,000\,007 = 332792866$. Find $S(20\,000)$ mod $1\,000\,000\,007$. \href{https://projecteuler.net/problem=650}{Zadanie 650}
\end{zadanie}
\begin{zadanie}
	An infinitely long cylinder has its curved surface fully covered with different coloured but otherwise identical rectangular stickers, without overlapping. The stickers are aligned with the cylinder, so two of their edges are parallel with the cylinder's axis, with four stickers meeting at each corner. Let $a>0$ and suppose that the colouring is periodic along the cylinder, with the pattern repeating every $a$ stickers. (The period is allowed to be any divisor of $a$.) Let $b$ be the number of stickers that fit round the circumference of the cylinder. Let $f(m, a, b)$ be the number of different such periodic patterns that use exactly $m$ distinct colours of stickers. Translations along the axis, reflections in any plane, rotations in any axis, (or combinations of such operations) applied to a pattern are to be counted as the same as the original pattern. You are given that $f(2, 2, 3) = 11$, $f(3, 2, 3) = 56$, and $f(2, 3, 4) = 156$.
	Furthermore, $f(8, 13, 21) \equiv 49718354 \pmod{1\,000\,000\,007}$,
	and $f(13, 144, 233) \equiv 907081451 \pmod{1\,000\,000\,007}$. Find $\sum\_{i=4}^{40} f(i, F\_{i-1}, F\_i) \bmod 1\,000\,000\,007$, where $F\_i$ are the Fibonacci numbers starting at $F\_0=0$, $F\_1=1$. \href{https://projecteuler.net/problem=651}{Zadanie 651}
\end{zadanie}
\begin{zadanie}
	Consider the values of $\log\_2(8)$, $\text{log}\_4(64)$ and $\text{log}\_3(27)$. All three are equal to $3$. Generally, the function $f(m,n)=\text{log}\_m(n)$ over integers $m,n \ge 2$ has the property that $f(m\_1,n\_1)=f(m\_2,n\_2)$ if We call a function $g(m,n)$ over integers $m,n \ge 2$ proto-logarithmic if Let $D(N)$ be the number of distinct values that any proto-logarithmic function $g(m,n)$ attains over $2\le m, n\le N$. For example, $D(5)=13$, $D(10)=69$, $D(100)=9607$ and $D(10000)=99959605$. Find $D(10^{18})$, and give the last 9 digits as answer. Note: According to the four exponentials conjecture the function $\text{log}\_m(n)$ is proto-logarithmic . While this conjecture is yet unproven in general, $\text{log}\_m(n)$ can be used to calculate $D(N)$ for small values of $N$. \href{https://projecteuler.net/problem=652}{Zadanie 652}
\end{zadanie}
\begin{zadanie}
	Consider a horizontal frictionless tube with length $L$ millimetres, and a diameter of 20 millimetres. The east end of the tube is open, while the west end is sealed. The tube contains $N$ marbles of diameter 20 millimetres at designated starting locations, each one initially moving either westward or eastward with common speed $v$. Since there are marbles moving in opposite directions, there are bound to be some collisions. We assume that the collisions are perfectly elastic, so both marbles involved instantly change direction and continue with speed $v$ away from the collision site. Similarly, if the west-most marble collides with the sealed end of the tube, it instantly changes direction and continues eastward at speed $v$. On the other hand, once a marble reaches the unsealed east end, it exits the tube and has no further interaction with the remaining marbles. To obtain the starting positions and initial directions, we use the pseudo-random sequence $r\_j$ defined by: $r\_1 = 6\,563\,116$ $r\_{j+1} = r\_j^2 \bmod 32\,745\,673$ The west-most marble is initially positioned with a gap of $(r\_1 \bmod 1000) + 1$ millimetres between it and the sealed end of the tube, measured from the west-most point of the surface of the marble. Then, for $2\le j\le N$, counting from the west, the gap between the $(j-1)$th and $j$th marbles, as measured from their closest points, is given by $(r\_j \bmod 1000) + 1$ millimetres.
	Furthermore, the $j$th marble is initially moving eastward if $r\_j \le 10\,000\,000$, and westward if $r\_j > 10\,000\,000$. For example, with $N=3$, the sequence specifies gaps of 117, 432, and 173 millimetres. The marbles' centres are therefore 127, 579, and 772 millimetres from the sealed west end of the tube. The west-most marble initially moves eastward, while the other two initially move westward. Under this setup, and with a five metre tube ($L=5000$), it turns out that the middle (second) marble travels 5519 millimetres before its centre reaches the east-most end of the tube. Let $d(L, N, j)$ be the distance in millimetres that the $j$th marble travels before its centre reaches the eastern end of the tube. So $d(5000, 3, 2) = 5519$. You are also given that $d(10\,000, 11, 6) = 11\,780$ and $d(100\,000, 101, 51) = 114\,101$. Find $d(1\,000\,000\,000, 1\,000\,001, 500\,001)$. \href{https://projecteuler.net/problem=653}{Zadanie 653}
\end{zadanie}
\begin{zadanie}
	Let $T(n, m)$ be the number of $m$-tuples of positive integers such that the sum of any two neighbouring elements of the tuple is $\le n$. For example, $T(3, 4)=8$, via the following eight $4$-tuples: $(1, 1, 1, 1)$ $(1, 1, 1, 2)$ $(1, 1, 2, 1)$ $(1, 2, 1, 1)$ $(1, 2, 1, 2)$ $(2, 1, 1, 1)$ $(2, 1, 1, 2)$ $(2, 1, 2, 1)$ You are also given that $T(5, 5)=246$, $T(10, 10^{2}) \equiv 862820094 \pmod{1\,000\,000\,007}$ and $T(10^2, 10) \equiv 782136797 \pmod{1\,000\,000\,007}$. Find $T(5000, 10^{12}) \bmod 1\,000\,000\,007$. \href{https://projecteuler.net/problem=654}{Zadanie 654}
\end{zadanie}
\begin{zadanie}
	The numbers $545$, $5\,995$ and $15\,151$ are the three smallest palindromes divisible by $109$. There are nine palindromes less than $100\,000$ which are divisible by $109$. How many palindromes less than $10^{32}$ are divisible by $10\,000\,019\,$ ? \href{https://projecteuler.net/problem=655}{Zadanie 655}
\end{zadanie}
\begin{zadanie}
	Given an irrational number $\alpha$, let $S\_\alpha(n)$ be the sequence $S\_\alpha(n)=\lfloor {\alpha \cdot n} \rfloor - \lfloor {\alpha \cdot (n-1)} \rfloor$ for $n \ge 1$. ($\lfloor ... \rfloor$ is the floor-function.) It can be proven that for any irrational $\alpha$ there exist infinitely many values of $n$ such that the subsequence $ \{S\_\alpha(1),S\_\alpha(2)...S\_\alpha(n) \} $ is palindromic. The first 20 values of $n$ that give a palindromic subsequence for $\alpha = \sqrt{31}$ are:
	1, 3, 5, 7, 44, 81, 118, 273, 3158, 9201, 15244, 21287, 133765, 246243, 358721, 829920, 9600319, 27971037, 46341755, 64712473. Let $H\_g(\alpha)$ be the sum of the first $g$ values of $n$ for which the corresponding subsequence is palindromic. So $H\_{20}(\sqrt{31})=150243655$. Let $T=\{2,3,5,6,7,8,10,...,1000\}$ be the set of positive integers, not exceeding 1000, excluding perfect squares. Calculate the sum of $H\_{100}(\sqrt \beta)$ for $\beta \in T$. Give the last 15 digits of your answer. \href{https://projecteuler.net/problem=656}{Zadanie 656}
\end{zadanie}
\begin{zadanie}
	In the context of formal languages , any finite sequence of letters of a given alphabet $\Sigma$ is called a word over $\Sigma$. We call a word incomplete if it does not contain every letter of $\Sigma$. For example, using the alphabet $\Sigma=\{ a, b, c\}$, '$ab$', '$abab$' and '$\,$' (the empty word) are incomplete words over $\Sigma$, while '$abac$' is a complete word over $\Sigma$. Given an alphabet $\Sigma$ of $\alpha$ letters, we define $I(\alpha,n)$ to be the number of incomplete words over $\Sigma$ with a length not exceeding $n$. For example, $I(3,0)=1$, $I(3,2)=13$ and $I(3,4)=79$. Find $I(10^7,10^{12})$. Give your answer modulo $1\,000\,000\,007$. \href{https://projecteuler.net/problem=657}{Zadanie 657}
\end{zadanie}
\begin{zadanie}
	In the context of formal languages , any finite sequence of letters of a given alphabet $\Sigma$ is called a word over $\Sigma$. We call a word incomplete if it does not contain every letter of $\Sigma$. For example, using the alphabet $\Sigma=\{ a, b, c\}$, '$ab$', '$abab$' and '$\,$' (the empty word) are incomplete words over $\Sigma$, while '$abac$' is a complete word over $\Sigma$. Given an alphabet $\Sigma$ of $\alpha$ letters, we define $I(\alpha,n)$ to be the number of incomplete words over $\Sigma$ with a length not exceeding $n$. For example, $I(3,0)=1$, $I(3,2)=13$ and $I(3,4)=79$. Let $\displaystyle S(k,n)=\sum\_{\alpha=1}^k I(\alpha,n)$. For example, $S(4,4)=406$, $S(8,8)=27902680$ and $S (10,100) \equiv 983602076 \text { mod } 1\,000\,000\,007$. Find $S(10^7,10^{12})$. Give your answer modulo $1\,000\,000\,007$. \href{https://projecteuler.net/problem=658}{Zadanie 658}
\end{zadanie}
\begin{zadanie}
	Consider the sequence $n^2+3$ with $n \ge 1$. If we write down the first terms of this sequence we get: $4, 7, 12, 19, 28, 39, 52, 67, 84, 103, 124, 147, 172, 199, 228, 259, 292, 327, 364,$... . We see that the terms for $n=6$ and $n=7$ ($39$ and $52$) are both divisible by $13$. In fact $13$ is the largest prime dividing any two successive terms of this sequence. Let $P(k)$ be the largest prime that divides any two successive terms of the sequence $n^2+k^2$. Find the last 18 digits of $\displaystyle \sum\_{k=1}^{10\,000\,000} P(k)$. \href{https://projecteuler.net/problem=659}{Zadanie 659}
\end{zadanie}
\begin{zadanie}
	We call an integer sided triangle $n$-pandigital if it contains one angle of 120 degrees and, when the sides of the triangle are written in base $n$, together they use all $n$ digits of that base exactly once. For example, the triangle (217, 248, 403) is 9-pandigital because it contains one angle of 120 degrees and the sides written in base 9 are $261\_9, 305\_9, 487\_9$ using each of the 9 digits of that base once. Find the sum of the largest sides of all n-pandigital triangles with $9 \le n \le 18$. \href{https://projecteuler.net/problem=660}{Zadanie 660}
\end{zadanie}
\begin{zadanie}
	Two friends $A$ and $B$ are great fans of Chess. They both enjoy playing the game, but after each game the player who lost the game would like to continue (to get back at his opponent) and the player who won would prefer to stop (to finish on a high). So they come up with a plan. After every game, they would toss a (biased) coin with probability $p$ of Heads (and hence probability $1-p$ of Tails). If they get Tails, they will continue with the next game. Otherwise they end the match. Also, after every game the players make a note of who is leading in the match. Let $p\_A$ denote the probability of $A$ winning a game and $p\_B$ the probability of $B$ winning a game. Accordingly $1-p\_A-p\_B$ is the probability that a game ends in a draw. Let $\mathbb{E}\_A(p\_A,p\_B,p)$ denote the expected number of times $A$ was leading in the match. For example, $\mathbb{E}\_A(0.25,0.25,0.5)\approx 0.585786$ and $\mathbb{E}\_A(0.47,0.48,0.001)\approx 377.471736$, both rounded to six places after the decimal point. Let $\displaystyle H(n)=\sum\_{k=3}^n \mathbb{E}\_A\left(\frac 1 {\sqrt{k+3}},\frac 1 {\sqrt{k+3}}+\frac 1 {k^2},\frac 1 {k^3}\right)$ For example $H(3) \approx 6.8345$, rounded to 4 digits after the decimal point. Find $H(50)$, rounded to 4 digits after the decimal point. \href{https://projecteuler.net/problem=661}{Zadanie 661}
\end{zadanie}
\begin{zadanie}
	Alice walks on a lattice grid. She can step from one lattice point $A (a,b)$ to another $B (a+x,b+y)$ providing distance $AB = \sqrt{x^2+y^2}$ is a Fibonacci number $\{1,2,3,5,8,13,\ldots\}$ and $x\ge 0,$ $y\ge 0$. In the lattice grid below Alice can step from the blue point to any of the red points. Let $F(W,H)$ be the number of paths Alice can take from $(0,0)$ to $(W,H)$. You are given $F(3,4) = 278$ and $F(10,10) = 215846462$. Find $F(10\,000,10\,000) \bmod 1\,000\,000\,007$. \href{https://projecteuler.net/problem=662}{Zadanie 662}
\end{zadanie}
\begin{zadanie}
	Let $t\_k$ be the tribonacci numbers defined as: $\quad t\_0 = t\_1 = 0$; $\quad t\_2 = 1$; $\quad t\_k = t\_{k-1} + t\_{k-2} + t\_{k-3} \quad \text{ for } k \ge 3$. For a given integer $n$, let $A\_n$ be an array of length $n$ (indexed from 0 to $n-1$), that is initially filled with zeros. The array is changed iteratively by replacing $A\_n[(t\_{2 i-2} \text{ mod } n)]$ with $A\_n[(t\_{2 i-2} \text{ mod } n)]+2 (t\_{2 i-1} \text{ mod } n)-n+1$ in each step $i$. After each step $i$, define $M\_n(i)$ to be $\displaystyle \max\{\sum\_{j=p}^q A\_n[j]: 0\le p\le q<n\}$, the maximal sum of any contiguous subarray of $A\_n$. The first 6 steps for $n=5$ are illustrated below: Initial state: $\, A\_5=\{0,0,0,0,0\}$ Step 1: $\quad \Rightarrow A\_5=\{-4,0,0,0,0\}$ , $M\_5(1)=0$ Step 2: $\quad \Rightarrow A\_5=\{-4, -2, 0, 0, 0\}$ , $M\_5(2)=0$ Step 3: $\quad \Rightarrow A\_5=\{-4, -2, 4, 0, 0\}$ , $M\_5(3)=4$ Step 4: $\quad \Rightarrow A\_5=\{-4, -2, 6, 0, 0\}$ , $M\_5(4)=6$ Step 5: $\quad \Rightarrow A\_5=\{-4, -2, 6, 0, 4\}$ , $M\_5(5)=10$ Step 6: $\quad \Rightarrow A\_5=\{-4, 2, 6, 0, 4\}$ , $M\_5(6)=12$ Let $\displaystyle S(n,l)=\sum\_{i=1}^l M\_n(i)$. Thus $S(5,6)=32$. You are given $S(5,100)=2416$, $S(14,100)=3881$ and $S(107,1000)=1618572$. Find $S(10\,000\,003,10\,200\,000)-S(10\,000\,003,10\,000\,000)$. \href{https://projecteuler.net/problem=663}{Zadanie 663}
\end{zadanie}
\begin{zadanie}
	Peter is playing a solitaire game on an infinite checkerboard, each square of which can hold an unlimited number of tokens. Each move of the game consists of the following steps: The board is marked with a line called the dividing line . Initially, every square to the left of the dividing line contains a token, and every square to the right of the dividing line is empty: Peter's goal is to get a token as far as possible to the right in a finite number of moves. However, he quickly finds out that, even with his infinite supply of tokens, he cannot move a token more than four squares beyond the dividing line. Peter then considers starting configurations with larger supplies of tokens: each square in the $d$th column to the left of the dividing line starts with $d^n$ tokens instead of 1. This is illustrated below for $n=1$: Let $F(n)$ be the maximum number of squares Peter can move a token beyond the dividing line. For example, $F(0)=4$.
	You are also given that $F(1)=6$, $F(2)=9$, $F(3)=13$, $F(11)=58$ and $F(123)=1173$. Find $F(1234567)$. \href{https://projecteuler.net/problem=664}{Zadanie 664}
\end{zadanie}
\begin{zadanie}
	Two players play a game with two piles of stones. On his or her turn, a player chooses a positive integer $n$ and does one of the following: The player who removes the last stone wins. We denote by $(n,m)$ the position in which the piles have $n$ and $m$ stones remaining. Note that $(n,m)$ is considered to be the same position as $(m,n)$. Then, for example, if the position is $(2,6)$, the next player may reach the following positions: $(0,2)$, $(0,4)$, $(0,5)$, $(0,6)$, $(1,2)$, $(1,4)$, $(1,5)$, $(1,6)$, $(2,2)$, $(2,3)$, $(2,4)$, $(2,5)$ A position is a losing position if the player to move next cannot force a win. For example, $(1,3)$, $(2,6)$, $(4,5)$ are the first few losing positions. Let $f(M)$ be the sum of $n+m$ for all losing positions $(n,m)$ with $n\le m$ and $n+m \le M$. For example, $f(10) = 21$, by considering the losing positions $(1,3)$, $(2,6)$, $(4,5)$. You are given that $f(100) = 1164$ and $f(1000) = 117002$. Find $f(10^7)$. \href{https://projecteuler.net/problem=665}{Zadanie 665}
\end{zadanie}
\begin{zadanie}
	If a population starts with a single bacterium of type $\alpha$, then it can be shown that there is a 0.07243802 probability that the population will eventually die out, and a 0.92756198 probability that the population will last forever. These probabilities are given rounded to 8 decimal places. Now consider another species of bacteria, $S\_{k,m}$ (where $k$ and $m$ are positive integers), which occurs in $k$ different types $\alpha\_i$ for $0\le i< k$. The rules governing this species' lifecycle involve the sequence $r\_n$ defined by: Every minute, for each $i$, each bacterium $A$ of type $\alpha\_i$ will independently choose an integer $j$ uniformly at random in the range $0\le j<m$. What it then does depends on $q = r\_{im+j} \bmod 5$: In fact, our original species was none other than $S\_{2,2}$, with $\alpha=\alpha\_0$ and $\beta=\alpha\_1$. Let $P\_{k,m}$ be the probability that a population of species $S\_{k,m}$, starting with a single bacterium of type $\alpha\_0$, will eventually die out. So $P\_{2,2} = 0.07243802$. You are also given that $P\_{4,3} = 0.18554021$ and $P\_{10,5} = 0.53466253$, all rounded to 8 decimal places. Find $P\_{500,10}$, and give your answer rounded to 8 decimal places. \href{https://projecteuler.net/problem=666}{Zadanie 666}
\end{zadanie}
\begin{zadanie}
	After buying a Gerver Sofa from the Moving Sofa Company , Jack wants to buy a matching cocktail table from the same company. Most important for him is that the table can be pushed through his L-shaped corridor into the living room without having to be lifted from its table legs. Unfortunately, the simple square model offered to him is too small for him, so he asks for a bigger model. He is offered the new pentagonal model illustrated below: Note, while the shape and size can be ordered individually, due to the production process, all edges of the pentagonal table have to have the same length. Given optimal form and size, what is the biggest pentagonal cocktail table (in terms of area) that Jack can buy that still fits through his unit wide L-shaped corridor? Give your answer rounded to 10 digits after the decimal point (if Jack had choosen the square model instead the answer would have been 1.0000000000). \href{https://projecteuler.net/problem=667}{Zadanie 667}
\end{zadanie}
\begin{zadanie}
	A positive integer is called square root smooth if all of its prime factors are strictly less than its square root. Including the number $1$, there are $29$ square root smooth numbers not exceeding $100$. How many square root smooth numbers are there not exceeding $10\,000\,000\,000$? \href{https://projecteuler.net/problem=668}{Zadanie 668}
\end{zadanie}
\begin{zadanie}
	The Knights of the Order of Fibonacci are preparing a grand feast for their king. There are $n$ knights, and each knight is assigned a distinct number from 1 to $n$. When the knights sit down at the roundtable for their feast, they follow a peculiar seating rule: two knights can only sit next to each other if their respective numbers sum to a Fibonacci number. When the $n$ knights all try to sit down around a circular table with $n$ chairs, they are unable to find a suitable seating arrangement for any $n>2$ despite their best efforts. Just when they are about to give up, they remember that the king will sit on his throne at the table as well. Suppose there are $n=7$ knights and 7 chairs at the roundtable, in addition to the kings throne. After some trial and error, they come up with the following seating arrangement (K represents the king): Notice that the sums $4+1$, $1+7$, $7+6$, $6+2$, $2+3$, and $3+5$ are all Fibonacci numbers, as required. It should also be mentioned that the king always prefers an arrangement where the knight to the his left has a smaller number than the knight to his right. With this additional rule, the above arrangement is unique for $n=7$, and the knight sitting in the 3rd chair from the kings left is knight number 7. Later, several new knights are appointed to the Order, giving 34 knights and chairs in addition to the king's throne. The knights eventually determine that there is a unique seating arrangement for $n=34$ satisfying the above rules, and this time knight number 30 is sitting in the 3rd chair from the king's left. Now suppose there are $n=99\,194\,853\,094\,755\,497$ knights and the same number of chairs at the roundtable (not including the kings throne). After great trials and tribulations, they are finally able to find the unique seating arrangement for this value of $n$ that satisfies the above rules. Find the number of the knight sitting in the $10\,000\,000\,000\,000\,000$th chair from the kings left. \href{https://projecteuler.net/problem=669}{Zadanie 669}
\end{zadanie}
\begin{zadanie}
	A certain type of tile comes in three different sizes - 1$\times$1, 1$\times$2, and 1$\times$3 - and in four different colours: blue, green, red and yellow. There is an unlimited number of tiles available in each combination of size and colour. These are used to tile a $2\times n$ rectangle, where $n$ is a positive integer, subject to the following conditions: For example, the following is an acceptable tiling of a $2\times 12$ rectangle: but the following is not an acceptable tiling, because it violates the "no four corners meeting at a point" rule: Let $F(n)$ be the number of ways the $2\times n$ rectangle can be tiled subject to these rules. Where reflecting horizontally or vertically would give a different tiling, these tilings are to be counted separately. For example, $F(2) = 120$, $F(5) = 45876$, and $F(100)\equiv 53275818 \pmod{1\,000\,004\,321}$. Find $F(10^{16}) \bmod 1\,000\,004\,321$. \href{https://projecteuler.net/problem=670}{Zadanie 670}
\end{zadanie}

\end{comment}

\section{Bibliografia}
\begin{thebibliography}{9}

	\bibitem{python100} 100+ Python challenging programming exercises
	\bibitem{logia} Konkurs informatyczny LOGIA.
	\bibitem{oig} Olimpiada Informatyczna Gimnazjalistów
	\bibitem{oi} Olimpiada Informatyczna
\end{thebibliography}


\end{document}